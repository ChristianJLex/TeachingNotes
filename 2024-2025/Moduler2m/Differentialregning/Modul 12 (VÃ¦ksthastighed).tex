\begin{center}
\Huge
Vækst og væksthastighed
\end{center}

\section*{Relation mellem vækst og differentialregning}
\stepcounter{section}

\begin{defn}
Har vi en differentiabel funktion $f(t)$, som funktion af tid $t$, så kalder vi $f'(t)$ for væksthastigheden til tidspunktet $t$. 
\end{defn}
\begin{exa}
En accelererende bil kører vinkelret væk fra os, og dens afstand fra os kan beskrives ved funktionen \begin{align*}
D(t) = 3\cdot t^2 + 5,
\end{align*}
hvor $t$ betegner tid i sekunder og $D(t)$ betegner afstanden fra os i $m$ efter $t$ sekunder. Væksthastigheden af bilen til tiden $t$ er 
\begin{align*}
D'(t) = 6t.
\end{align*}
Efter to sekunder bevæger bilen sig med $D'(2) = 12 m/s.$ Vi kan også bestemme accelerationen af bilen som væksthastigheden af væksthastigheden:
\begin{align*}
D''(t) = 6,
\end{align*}
altså er accelerationen af bilen $6m/s^2$. Dette betyder, at hastigheden af bilen vokser med $6m/s$ hvert sekund.  
Hvis vi skal finde ud af, hvor langt bilen er væk efter $4$ sekunder bestemmer vi
\begin{align*}
D(4) = 3\cdot 4^2 +5 = 53,
\end{align*}
altså er bilen $53 m$ væk. 
Hvis vi vil bestemme hastigheden efter $4$ sekunder, så får vi 
\begin{align*}
D'(4) = 6\cdot 4 = 24, 
\end{align*}
altså er hastigheden efter $4$ sekunder givet ved $24 m/s$.
\end{exa}
\section*{Opgave 1}
Antallet af bakterier i en bakteriekoloni kan bestemmes ved
\begin{align*}
N(t) = 0.7\cdot e^{0.3t},
\end{align*}
hvor t betegner tid i timer og $N(t)$ er antal bakterier i mio. 
\begin{enumerate}
\item Hvor mange bakterier er der efter $7$ timer? Hvad med 14?
\item Hvad er bakterievæksten efter et døgn? Forklar bakterievæksten med ord.
\item Bliver antallet af bakterier ved med at stige? Er det realistisk? 
\item Bliver væksten ved med at stige? Er det realistisk?
\end{enumerate}
\section*{Opgave 2}
Vægten i kg af et radioaktivt stof til efter $t$ timer kan beskrives ved
\begin{align*}
V(t) =10\cdot e^{-0.16t}.
\end{align*}
\begin{enumerate}
\item Hvad er vægten af stoffet efter $0$ timer? Hvad med efter $10$ timer?
\item Hvad er væksten af stoffet efter en uge? Beskriv væksten med ord.
\end{enumerate}
\section*{Opgave 3}
Den faldne afstand for et objekt i frit fald kan approksimeres ved funktionen $A$ givet ved
\begin{align*}
A(t) = 4.91 \cdot t^2, 
\end{align*}
hvor $t$ er tiden i sekunder og $A$ er den faldne længde i $m$.
\begin{enumerate}
\item Hvor langt er objektet faldet efter 10 sekunder?
\item Med hvilken hastighed falder objektet efter 10 sekunder?
\item Hvad er accelerationen af faldet?
\end{enumerate}
