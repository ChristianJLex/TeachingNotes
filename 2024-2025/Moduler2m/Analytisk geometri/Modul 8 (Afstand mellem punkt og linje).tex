
\begin{center}
\Huge
Afstand mellem linje og punkt
\end{center}

\section*{Afstand mellem linje og punkt}
\stepcounter{section}

Vi kan bruge projektioner af vektorer til at bestemme afstanden mellem et punkt $P$ og en linje $l$ i planen. Det er ikke umiddelbart klart ud fra formlen at det er projektioner, vi bruger, men det vil ses af beviset. Afstanden findes ved følgende sætning. 

\begin{setn}[Afstand mellem punkt og linje]
Har vi et punkt $Q(x_1,y_1)$ og en linje $l$ givet ved ligningen
\begin{align*}
ax+by+c = 0, 
\end{align*}
så kan vi bestemme afstanden mellem $l$ og $Q$ ved
\begin{align*}
\textnormal{dist}(Q,l) = \frac{|ax_1+by_1+c|}{\sqrt{a^2+b^2}}
\end{align*}
\end{setn}
\begin{proof}
Vi vælger et punkt  $P(x_0,y_0)$ på $l$. Da $l$ har ligningen
\begin{align*}
ax+by+c = 0, 
\end{align*}
må der gælde, at 
\begin{align*}
ax_0+by_0+c = 0, 
\end{align*}
og derfor har vi et udtryk for $c$ givet ved
\begin{align*}
c = -ax_0-by_0.
\end{align*}
Vektoren $\vv{PQ}$ er givet ved
\begin{align*}
\vv{PQ} = \begin{pmatrix}
x_1 - x_0 \\ y_1 - y_0.
\end{pmatrix}
\end{align*}
Vi har en normalvektor til $l$ givet ved 
\begin{align*}
\vv{n}=
\begin{pmatrix}
a \\ b
\end{pmatrix}
\end{align*}
Vi bemærker nu, at længden af projektionen $\vv{PQ_{\vv{n}}}$ må være afstanden fra $l$ til $Q$. Vi bestemmer derfor længden af denne projektion:
\begin{align*}
\left|\vv{PQ_{\vv{n}}}\right| &= \frac{|\vv{PQ_{\vv{n}}}\cdot \vv{n}|}{|\vv{n}|} \\ 
&= \frac{\left|\begin{pmatrix}
x_1-x_0 \\ y_1 - y_0
\end{pmatrix}\cdot \begin{pmatrix}
a \\ b
\end{pmatrix} \right|}{\left|\begin{pmatrix}
a \\ b
\end{pmatrix}\right|}\\
&=\frac{|a(x_1-x_0)+b(y_1-y_0)|}{\sqrt{a^2+b^2}}\\
&= \frac{|ax_1+by_1 -ax_0-by_0|}{\sqrt{a^2+b^2}}\\
&= \frac{|ax_1+by_1+c|}{\sqrt{a^2+b^2}}.
\end{align*}
\end{proof}
Argumentationen i beviset er beskrevet på Figur \ref{fig:afstand}
\begin{figure}[H]
	\centering
	\resizebox{0.7\textwidth}{0.7\textwidth}{
	\begin{tikzpicture}
		\begin{axis}[axis lines = middle, xmin = -1, ymin = -1, xmax = 4, ymax = 4]
			\addplot[color = teal, thick] {0.5*x+0.5 };
			\draw[-{Stealth[scale = 1.4]}, thick, teal]
			 (axis cs: 2,1.5) -- (axis cs:1,3.5);
			 \node at (axis cs: 1+0.4,3.5) {$\vv{n}$};
			
			\node at (axis cs:0.5,0.75-0.3) {$P$};
			\node at (axis cs: 1.25-0.4,3) {$Q$};
			\draw[-{Stealth[scale = 1.4]}, thick, teal]
			(axis cs: 2,1.5) -- (axis cs: 1.25,3);
			
			\draw[-{Stealth[scale= 1.4]}, thick, color = olive]
			(axis cs: 0.5,0.75) -- (axis cs: 1.25,3); 
			\draw[-{Stealth[scale= 1.4]}, thick, color = olive]
			(axis cs: 0.5+1.5,0.75+0.75) -- (axis cs: 1.25+1.5,3+0.75); 
			\draw[thick, dashed, color = teal]
			 (axis cs: 1.25+1.5,3+0.75) -- (axis cs: 1.25,3);
			\node at (axis cs: 1.85,2.5) {$\vv{PQ}_{\vv{n}}$};
			\filldraw[purple] (axis cs:0.5,0.75) circle (2pt);
			\filldraw[purple] (axis cs: 1.25,3) circle (2pt);
		\end{axis}
	\end{tikzpicture}
	}
	\caption{Afstand fra punkt til linje}
	\label{fig:afstand}
\end{figure}

\begin{exa}
Vi skal bestemme afstanden fra punktet $P(1,1)$ til linjen med ligningen $4(x-1) + 3(y+1) = 0$. Vi starter med at hæve parenteserne i ligningen:
\begin{align*}
4(x-1) + 3(y+1) = 0 \Leftrightarrow 4x+3y-1 = 0.
\end{align*}
Vi kan nu bruge formlen for afstand mellem punkt og linje og får:
\begin{align*}
\textnormal{dist}(P,l) &= \frac{|4\cdot 1 + 3\cdot 1-1|}{\sqrt{3^2+4^2}}\\
&=\frac{6}{5},
\end{align*}
hvilket er afstanden fra punktet $P$ til linjen $l$. 
\end{exa}

\begin{exa}
Vi skal bestemme $k$, så punktet $P(4,k)$ og linjen $l$ med ligningen 
\begin{align*}
2x-4y-3 = 0
\end{align*}
har afstand 1. Vi bruger afstandsformlen og får
\begin{align*}
\textnormal{dist}(P,l) = \frac{|2\cdot 4-4\cdot k-3|}{\sqrt{2^2+4^2}|} = \frac{|5-4k|}{\sqrt{20}} =  1.
\end{align*}
Vi løser denne ligning og får, at $k \approx 0.63$. 
\end{exa}



\section*{Opgave 1}
Tegn en linje med den korteste afstand mellem følgende linjer og punkter. Giv et overslag over længden og bestem derefter afstanden ved at bruge afstandsformlen. 
\begin{center}
\resizebox{0.45\textwidth}{0.45\textwidth}{
\begin{tikzpicture}
	\begin{axis}[axis lines = middle, 
	xmin = -0.5, xmax = 4.5, ymin = -0.5, ymax = 4.5, 
	grid]
		\addplot[thick, color = blue!40] {x+1};
		\filldraw[purple]  (axis cs:2,1) circle (2pt);
		\draw[-{Stealth[scale = 1.4]}, thick, color = purple] (axis cs: 2,3) --(axis cs:4,1);
	\end{axis}
\end{tikzpicture}
}
\resizebox{0.45\textwidth}{0.45\textwidth}{
\begin{tikzpicture}
	\begin{axis}[axis lines = middle, 
	xmin = -0.5, xmax = 4.5, ymin = -0.5, ymax = 4.5, 
	grid]
			\addplot[thick, color = blue!40] {-x+2};
		\filldraw[purple]  (axis cs:0,0) circle (2pt);
		\draw[-{Stealth[scale = 1.4]}, thick, color = purple] (axis cs: 1,1) --(axis cs:2,2);
	\end{axis}
\end{tikzpicture}
}
\end{center}
\begin{center}
\resizebox{0.45\textwidth}{0.45\textwidth}{
\begin{tikzpicture}
	\begin{axis}[axis lines = middle, 
	xmin = -1.5, xmax = 4.5, ymin = -4.5, ymax = 1.5, 
	grid]
		\addplot[thick, color = blue!40] {0.5*x-3};
		\filldraw[purple]  (axis cs:-1,-1) circle (2pt);
		\draw[-{Stealth[scale = 1.4]}, thick, color = purple] 
		(axis cs: 2,-2) --(axis cs:3,-4);
	\end{axis}
\end{tikzpicture}
}
\resizebox{0.45\textwidth}{0.45\textwidth}{
\begin{tikzpicture}
	\begin{axis}[axis lines = middle, 
	xmin = -3.5, xmax = 3.5, ymin = -3.5, ymax = 3.5, 
	grid]
		\addplot[thick, color = blue!40] {-1/3*x+2};
		\filldraw[purple]  (axis cs:-3,-2) circle (2pt);
		\draw[-{Stealth[scale = 1.4]}, thick, color = purple] 
		(axis cs: 3,1) --(axis cs:2,-2); 
	\end{axis}
\end{tikzpicture}
}
\end{center}
\newpage

\section*{Opgave 2}
\begin{enumerate}[label=\roman*)]
\item Bestem $k$, så afstanden mellem linjen givet ved 
\begin{align*}
x-5y+10 = 0
\end{align*}
og punktet $P(k,1)$ har afstand $4$. Start med at tegne det i Geogebra.
\item Bestem $b$, så $l$ med ligningen
\begin{align*}
y = 2x+b,
\end{align*}
 og punktet $P(1,1)$ har afstand 5. Start med at tegne i Geogebra.
\end{enumerate}

\section*{Opgave 3}
Prøv at læse og forstå beviset.