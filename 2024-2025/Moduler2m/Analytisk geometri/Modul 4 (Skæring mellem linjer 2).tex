
\begin{center}
\Huge
Skæring mellem linjer 2
\end{center}

Det er også muligt at bestemme skæringspunkter mellem to linjer, hvis de er givet ved linjens parameterfremstilling. 
\section*{Skæring givet parameterfremstilling}
\stepcounter{section}
Tilsvarende kan vi også bestemme et skæringspunkt mellem to linjer $l$ og $m$, hvis deres parameterfremstilling er givet. 
\begin{exa}
	Lad $l$ og $m$ være linjer med følgende parametriseringer:
	\begin{align*}
		&l: \ 
		\begin{pmatrix}
			x \\ y
		\end{pmatrix} = 
		\begin{pmatrix}
			4 \\ 4
		\end{pmatrix} + t
		\begin{pmatrix}
			2 \\ -2
		\end{pmatrix},\\
		&m: \ 
		\begin{pmatrix}
			x \\ y 
		\end{pmatrix} = 
		\begin{pmatrix}
			-4 \\ 0
		\end{pmatrix} + t
		\begin{pmatrix}
			5 \\ 1
		\end{pmatrix}.
	\end{align*}
	Vi skal bestemme skæringen mellem disse linjer. Vi sætter dem derfor lig hinanden:
	\begin{align*}
		\begin{pmatrix}
			4 \\ 4
		\end{pmatrix} + t_1
		\begin{pmatrix}
			2 \\ -2
		\end{pmatrix} =
		\begin{pmatrix}
			-4 \\ 0
		\end{pmatrix} + t_2
		\begin{pmatrix}
			5 \\ -1
		\end{pmatrix}.
	\end{align*}
	Dette giver os to lineære ligninger med to ubekendte, som vi enten kan løse med 
	substitution eller ved lige store koefficienters metode. Vi bruger substitutionsmetoden. 
	Ligningerne lyder:
	\begin{align*}
		4 + 2t_1 = -4+5t_2
	\end{align*} og
	\begin{align*}
		4-2t_1 = t_2.
	\end{align*}
	Anden ligning indsættes i første ligning:
	\begin{align*}
		&4 + 2t_1 = -4+5t_2\\
		\Leftrightarrow \ &4 + 2t_1 = -4+5(4-2t_1)\\
		\Leftrightarrow \ &4+2t_1 = -4+20-10t_1\\
		\Leftrightarrow \ &8+2t_1 = 20-10t_1\\
		\Leftrightarrow \ &12t_1 = 12\\
		\Leftrightarrow \ &t_1 = 1.
	\end{align*}
	Dette indsættes i første parameterfremstilling:
	\begin{align*}
		\begin{pmatrix}
			x \\ y
		\end{pmatrix} = 
		\begin{pmatrix}
			4 \\ 4 
		\end{pmatrix} 
		+ 1
		\begin{pmatrix}
			2 \\ -2
		\end{pmatrix} = 
		\begin{pmatrix}
			6 \\ 2
		\end{pmatrix}.
	\end{align*}
	Derfor skærer linjerne $l$ og $m$ hinanden i punktet $(6,2)$. 
\end{exa}


\section*{Opgave 1}
\begin{enumerate}[label=\roman*)]

\item Bestem skæringen mellem linjerne $l$ og $m$, der har følgende parameterfremstillinger henholdsvist:
\begin{align*}
\begin{pmatrix}
x \\ y
\end{pmatrix}
= 
\begin{pmatrix}
1 \\ -1
\end{pmatrix}
+
t
\begin{pmatrix}
2 \\ 1
\end{pmatrix}
\end{align*}
og 
\begin{align*}
\begin{pmatrix}
x \\ y
\end{pmatrix}
= 
\begin{pmatrix}
-3 \\ 3
\end{pmatrix}
+
t
\begin{pmatrix}
-2 \\ 1
\end{pmatrix}
\end{align*}

\item Bestem skæringen mellem linjerne $l$ og $m$, der har følgende parameterfremstillinger henholdsvist:
\begin{align*}
\begin{pmatrix}
x \\ y
\end{pmatrix}
= 
\begin{pmatrix}
5 \\ 2
\end{pmatrix}
+
t
\begin{pmatrix}
3 \\ 4
\end{pmatrix}
\end{align*}
og 
\begin{align*}
\begin{pmatrix}
x \\ y
\end{pmatrix}
= 
\begin{pmatrix}
-1 \\ 6
\end{pmatrix}
+
t
\begin{pmatrix}
-3 \\ 6
\end{pmatrix}
\end{align*}

	\item Bestem skæringspunktet mellem følgende to linjer. 
	\begin{align*}
 		\begin{pmatrix}
 			x \\ y
 		\end{pmatrix}
 		=
 		\begin{pmatrix}
 			0 \\ 1
 		\end{pmatrix}
 		+
 		t
 		\begin{pmatrix}
 			-4 \\ 2
 		\end{pmatrix}
 	\end{align*}  
 	og
 	\begin{align*}
 		\begin{pmatrix}
 			 x \\ y
 		\end{pmatrix}
 		=
 		\begin{pmatrix}
 			5 \\ 2
 		\end{pmatrix}
 		+
 		t
 		\begin{pmatrix}
 			3 \\ 2
 		\end{pmatrix}
 	\end{align*}
 	
 	
\end{enumerate}

\section*{Opgave 2}
\begin{enumerate}[label=\roman*)]
	\item En linje $l$ går gennem punkterne $(1,1)$ og $(2,3)$. Bestem en parameterfremstillinger for 
	$l$. 
	\item En linje $m$ går gennem punkterne $(-2,-4)$ og $(3,5)$. Bestem en ligning for $m$. 
\end{enumerate}

\section*{Opgave 3}
\begin{enumerate}[label=\roman*)]
	\item En linje $l$ har parameterfremstillingen 
	\begin{align*}
		\begin{pmatrix}
			x \\ y
		\end{pmatrix} = 
		\begin{pmatrix}
			1 \\ -2
		\end{pmatrix} + t
		\begin{pmatrix}
			2 \\ 1
		\end{pmatrix}
	\end{align*}
	og en anden linje $m$ har ligningen
	\begin{align*}
		(x-3) + 7(y+2) = 0
	\end{align*}
	Bestem skæringspunktet mellem $l$ og $m$. 
	\item En linje $l$ har parameterfremstillingen 
	\begin{align*}
		\begin{pmatrix}
			x \\ y
		\end{pmatrix} = 
		\begin{pmatrix}
			0 \\ 5
		\end{pmatrix} + t
		\begin{pmatrix}
			5 \\ 3
		\end{pmatrix}
	\end{align*}
	og en anden linje $m$ har ligningen
	\begin{align*}
		-5(x+3) + 2(y-2) = 0
	\end{align*}
	Bestem skæringspunktet mellem $l$ og $m$. 
\end{enumerate}
