
\begin{center}
\Huge
Cirklens ligning
\end{center}

\section*{Cirklens ligning}
\stepcounter{section}
Præcist som det er tilfældet for linjen kan vi bestemme en ligning for en cirkel. 
\begin{setn}[Cirklens ligning]
	Cirklen $C$ med centrum i $(x_0,y_0)$ og radius $r$ har ligningen
	\begin{align*}
		(x-x_0)^2 + (y-y_0)^2 = r^2.
	\end{align*}
\end{setn}
\begin{proof}
	Lad $(x,y)$ være et punkt på cirklen $C$, og lad $(x_0,y_0)$ være cirklens centrum. Så er
	forbindelsesvektoren $\vv{v}$ mellem $(x,y)$ og $(x_0,y_0)$ givet ved
	\begin{align*}
		\vv{v} = 
		\begin{pmatrix}
			x-x_0 \\
			y-y_0
		\end{pmatrix}.
	\end{align*}
	Da radius for $C$ er $r$, så skal længden af $\vv{v}$ være $r$. Dette medfører
	\begin{align*}
		r = |\vv{v}| = \sqrt{(x-x_0)^2+(y-y_0)^2}. 
	\end{align*}
	Vi opløfter nu begge sider af lighedstegnet i anden og får
	\begin{align*}
		r^2 = \left(\sqrt{(x-x_0)^2+(y-y_0)^2}\right)^2 = (x-x_0)^2 + (y-y_0)^2.
	\end{align*}
\end{proof}

\begin{figure}[H]
	\centering
	\begin{tikzpicture}
		\begin{axis}[
			axis lines = middle, 
			xmin = -2, xmax = 8,
			ymin = -2, ymax = 8,
			x = 1cm, y = 1cm,
			ticks = none, 
			xlabel = $x$, ylabel = $y$,
			]
			\draw[-{Stealth[scale = 1.5]},thick, color = olive] (axis cs: 4,4) -- (axis cs: 6.58,5.53);
			\filldraw[color = teal] (axis cs: 4,4) circle (2pt);
			\filldraw[color = teal] (axis cs: 6.58,5.53) circle (2pt);
			\node[color = teal] at (axis cs: 4,3.5) {$(x_0,y_0)$};
			\node[color = teal] at (axis cs: 7,6) {$(x,y)$};
			\draw[decorate,decoration={brace,amplitude=5pt,raise=0.9ex}, color = olive, thick] (axis cs: 4,4) 
			-- (axis cs: 6.58,5.53) node[midway, color = olive, xshift = -1.8ex, yshift = 2.5ex] {$r$};
			\draw[color = teal, thick] (axis cs: 4,4)	 circle (3cm);	
		\end{axis}
	\end{tikzpicture}
	\caption{Cirkel med centrum i $(x_0,y_0)$ og radius $r$.}
\end{figure}

\begin{exa}
	En cirkel har centrum i $(3,2)$ og radius $2$. Cirklens ligning er så
	\begin{align*}
		(x-3)^2 + (y-2)^2 = 4.
	\end{align*}
\end{exa}

\begin{exa}
	En cirkel har ligningen 
	\begin{align}\label{eq:1}
		(x-1)^2 + (y-4)^2 = 16. 
	\end{align}
	Vi skal afgøre, om punktet $(5,4)$ ligger på cirklen. Vi indsætter derfor dette
	i \eqref{eq:1} og får
	\begin{align*}
		(5-1)^2 + (4-4)^2 = 4^2=16,
	\end{align*}
	og da denne lighed er korrekt, så er punktet på cirklen. 
\end{exa}

\section*{Opgave 1}

\begin{enumerate}[label=\roman*)]
	\item Bestem cirklens ligning for cirklen, der har centrum i $(1,2)$
	 og radius $r = 1$.
	\item Bestem cirklens ligning for cirklen, der har centrum i $(0,0)$
	og radius $r = 1$.
	\item Bestem cirklens ligning for cirklen, der har centrum i $(-4,2)$
	og radius $r = 3$.
	\item Bestem cirklens ligning for cirklen, der har centrum i $(\sqrt{2},-\sqrt{3})$
	og radius $r = \sqrt{5}$.
	\item Bestem cirklens ligning for cirklen, der har centrum i $(\frac{1}{3},-\frac{2}{7})$
	og radius $r = \frac{2}{3}$.
\end{enumerate}

\section*{Opgave 2}

\begin{enumerate}[label=\roman*)]
	\item Afgør, om punktet $(1,1)$ ligger på cirklen med ligningen
	\begin{align*}
		(x-1)^2 +(y-2)^2 = 1 
	\end{align*}
	\item Afgør, om punktet $(2,4)$ ligger på cirklen med ligningen
	\begin{align*}
		(x+1)^2 + (y-5)^2 = 49
	\end{align*}
	\item Afgør, om punktet $(1,6)$ ligger på cirklen med ligningen 
	\begin{align*}
		x^2-2x+y^2-4y=11
	\end{align*}
	\item Afgør, om punktet $(-2,2)$ ligger på cirklen med ligningen
	\begin{align*}
		x^2+8x+y^2-8y=32
	\end{align*}
\end{enumerate}

\section*{Opgave 3}
\begin{enumerate}[label=\roman*)]
	\item En cirkel $C$ har centrum i $(-2,-3)$ og radius $r=4$. Afgør, om punktet $(-2,-2)$ 
	ligger på $C$.
	\item En cirkel $C$ har centrum i $(-5,4)$ og radius $r=6$. Afgør, om punktet $(-5,10)$ 
	ligger på $C$.
\end{enumerate}

\section*{Opgave 4}
\begin{enumerate}[label=\roman*)]
	\item En cirkel $C$ er givet ved ligningen
	\begin{align*}
		(x+1)^2+(y-t)^2 = 4.
	\end{align*}
	Bestem $t$, så punktet $(1,2)$ ligger på $C$.
	\item En cirkel $C$ er givet ved ligningen
	\begin{align*}
		(x-4)^2 + (y-4)^2 = r^2.
	\end{align*}
	Bestem $r$, så punktet $(4,8)$ ligger på $C$.
	\item En cirkel $C$ er givet ved ligningen
	\begin{align*}
		x^2-10x+y^2-2y=-1.
	\end{align*}
	Bestem $k$, så punktet $(k,1)$ ligger på $C$.
\end{enumerate}