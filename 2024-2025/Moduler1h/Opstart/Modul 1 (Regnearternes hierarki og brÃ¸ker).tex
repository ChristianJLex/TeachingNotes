\begin{center}
\Huge
Regnearternes hierarki og brøker
\end{center}
\stepcounter{section}

Vi starter med at huske os selv på regnearternes hierarki også kaldet operatorpræcedens, der fortæller os i hvilken rækkefølge, vi skal anvende operatorerne i et givent regnestykke. Altså i hvilken rækkefølge, vi skal lægge sammen, gange, dividere, tage potenser osv. 
\begin{defn}[Regnearternes hierarki]
I en udregning anvender vi operatorerne i følgende rækkefølge:
\begin{enumerate}[label=\roman*)]
\item Parentes. Betegnes med $()$. (Alt, der står i parentes udregnes først efter rækkefølgen bestemt for de resterende operatorer).
\item Fakultet. Betegnes med $!$. Vi husker på, at for $n\in \mathbb{N}$ er $n!$ defineret som 
\begin{align*}
n! = \begin{cases}
1 \ &\textnormal{ for }n=0,\\
n(n-1)(n-2)\cdots 2\cdot 1 \ &\textnormal{ for }n>0.
\end{cases}
\end{align*}
\item Potenser og rødder. Et tal $a$ i $n$'te potens og $n$'te rod betegnes med henholdsvist $a^n$ og $\sqrt[n]{a}$.
\item Multiplikation og division. Betegnes med henholdsvist $\cdot$ og $/$.
\item Addition og subtraktion. Betegnes med henholdsvist $+$ og $-$.
\end{enumerate}
\end{defn} 
\begin{exa}
Lad os betragte regnestykket
\begin{align}\label{eq:exa1}
7+10-\underbrace{(5-2\cdot \frac{3}{6}+3!^2)}_{\textnormal{Parentes 1}}+\underbrace{(7-9)}_{\textnormal{Parentes 2}}\cdot 4.
\end{align}
Vi starter med at udregne Parentes 1. Vi følger regnearternes hierarki:
\begin{align*}
(5-2\cdot \frac{3}{6}+3!^2) &\overset{\textnormal{ii)}}{=} (5-2\cdot \frac{3}{6}+6^2)\\
							&\overset{\textnormal{iii)}}{=} (5-2\cdot \frac{3}{6}+36)\\
							&\overset{\textnormal{iv)}}{=} (5-1+36)\\
							&\overset{\textnormal{v)}}{=} (40).
\end{align*}
Og Parentes 2 tilsvarende:
\begin{align*}
(7-9) &\overset{\textnormal{v)}}{=} (-2).
\end{align*}
Disse indsættes nu i \eqref{eq:exa1}, og vi anvender igen regnearternes hierarki til at udregne:
\begin{align*}
7+10+\underbrace{40}_{\textnormal{Par. 1}} + \underbrace{(-2)}_{\textnormal{Par. 2}}\cdot 4 &\overset{\textnormal{iv)}}{=} 7+10+40 -8 \\
&\overset{\textnormal{v)}}{=} 39.
\end{align*}
\end{exa}


\section*{Brøkregneregler}
\stepcounter{section}

Vi skal i dag arbejde med brøkregneregler.
\begin{enumerate}[label=\roman*)]
\item Addition og subtraktion af brøker. For to brøker $\frac{a}{b}$ og $\frac{c}{d}$ er summen og differensen givet ved
\begin{align*}
\frac{a}{b} \pm \frac{c}{d} = \frac{ad\pm cb}{bd}.
\end{align*}
\item Multiplikation af brøker. Produktet mellem $\frac{a}{b}$ og $\frac{c}{d}$ er givet ved
\begin{align*}
\frac{a}{b}\cdot \frac{c}{d} = \frac{ac}{bd}.
\end{align*}
Specielt er produktet mellem en brøk $\frac{a}{b}$ og et tal $c$ givet ved
\begin{align*}
c\cdot \frac{a}{b} = \frac{ca}{b}.
\end{align*}
\item Division af brøker. Forholdet mellem to brøker $\frac{a}{b}$ og $\frac{c}{d}$ er givet ved
\begin{align*}
\frac{\frac{a}{b}}{\frac{c}{d}} = \frac{ad}{bc}.
\end{align*}
(Vi ganger med den omvendte brøk.) Specielt er en brøk $\frac{a}{b}$ divideret med et tal $c$ givet ved
\begin{align*}
\frac{\frac{a}{b}}{c} = \frac{a}{bc},
\end{align*}
og et tal $c$ divideret med en brøk $\frac{a}{b}$ givet ved
\begin{align*}
\frac{c}{\frac{a}{b}} = \frac{cb}{a}.
\end{align*}
\item Brøker og potenser/rødder. En brøk $\frac{a}{b}$ opløftet i et tal $c$ er givet ved 
\begin{align*}
\left(\frac{a}{b}\right)^{c} = \frac{a^c}{b^c}.
\end{align*}
$n$'te roden af en brøk $\frac{a}{b}$ er givet ved 
\begin{align*}
\sqrt[n]{\frac{a}{b}} = \frac{\sqrt[n]{a}}{\sqrt[n]{b}}.
\end{align*}
\end{enumerate}
\begin{exa}
Lad os se på et eksempel, hvor vi anvender nogle af disse regler:
\begin{align*}
\frac{\frac{2}{5}+\frac{5}{7}}{\frac{10}{3}} \overset{\textnormal{i)}}{=} \frac{\frac{14+25}{35}}{\frac{10}{3}} = \frac{\frac{39}{35}}{\frac{10}{3}} \overset{\textnormal{iii)}}{=} \frac{39\cdot 3}{35 \cdot 10} = \frac{117}{350}.
\end{align*}
\end{exa}

\subsection*{Opgave 1}
Udregn følgende. 
\begin{align*}
	&1) \ 4(2+7)   &&2) \   \frac{6}{3}\cdot 7+3\frac{10}{2}  \\
	&3) \ 2!^3   &&4) \  (2+4)^2   \\
	&5) \  \frac{12}{4} +9 &&6) \  -5^2+9-\frac{14}{7}\cdot(-2)   \\
	&7) \ (-5)^3+\frac{24}{2+2}    &&8) \  (1+3!)^2   \\
	&9) \  (2+3)^3  &&10) \  \sqrt{3^2+4^2}   \\
	&11) \ \sqrt{(-6)^2+(-8)^2}   &&12) \  \frac{(1+1+1)!+6^2 + (2^2-6\cdot(-2))}{\sqrt{4^2+20}}   \\
\end{align*}

\subsection*{Opgave 2}

\begin{enumerate}[label=\roman*)]
	\item Udregn $(-2)(-3)$
	\item Udregn $(-5)2$.
	\item Udregn $(-3)(-0.5)(-6)(-1)$
	\item Udregn $(3)(-3)(-3)(-1)$
	\item For hvilke heltal $n>0$ gælder det, at $(-2)^n$ bliver et positivt tal?
	\item For hvilke heltal $n>0$ gælder det, at $(-2)^n$ bliver et negativt tal?
\end{enumerate}



\subsection*{Opgave 3}
Forkort følgende udtryk så meget som muligt.
\begin{align*}
	&1) \  (a+a)b  &&2) \  \sqrt{a^2}   \\
	&3) \  (a-b)a-a^2+ab+c  &&4) \ (\sqrt[7]{a+b})^7    \\
\end{align*}


\subsection*{Opgave 4}
Løs følgende ligninger.	
\begin{align*}
	&1) \  2x = 4  &&2) \ (-5+2)x+3!x = 2x   \\
	&3) \  5x+2x = 21  &&4) \ \sqrt{x+2} = 2   \\
	&5) \  (x-4)! = 24  &&6) \  2^x = 8  \\
	&7) \  3+4\cdot 2x = 11x  &&8) \ x^2 = 2x   \\
	&9) \  \frac{x^2}{3} = x  &&10) \  \frac{x+x}{2} = 1  \\	
\end{align*}


\subsection*{Opgave 5}
Udregn følgende brøker (forkort så meget som muligt). Brug Maple til at tjekke jeres svar
\begin{align*}
	&1)\  \frac{6}{7} + \frac{3}{1}            &&2)\ \frac{7}{22} + \frac{9}{10}\\
 	&3)\ \frac{4}{5} + \frac{3}{2}      &&4)\ \frac{1}{2} - \frac{3}{4}\\
 	&5)\  2\frac{2}{3} + \frac{7+4}{2}              &&6)\ \frac{4}{\frac{5}{7}}\\
 	&7)\  \frac{\frac{2}{3}}{6} - 2           &&8)\ \frac{-7+\frac{2}{6}}{8} + \frac{9}{5}\\
	&9)\ \frac{\frac{4}{3}}{\frac{2}{3}} &&10)\ \frac{\frac{10}{3}-\frac{2}{4}}{\frac{4}{3}+\frac{5}{3}} \\
	&11)\  \frac{\frac{1}{2}-\frac{7}{10}}{\frac{2}{5}+\frac{11}{3}} -  \frac{\frac{22}{3}+ \frac{-23}{6}}{\frac{1}{2}}        &&12)\ \sqrt{\frac{16}{25}} + \left(\frac{3}{2+4}\right)^2\\
	&13)\ \sqrt{\frac{\frac{100}{36}}{\frac{25}{49}}} && 14) \   \left(\frac{\sqrt{\frac{5}{7}}}{\sqrt{\frac{2}{9}}}\right)^2 + \sqrt[3]{\frac{\left(\frac{2+5}{3+11}\right)^3}{\left( \frac{5}{7-6}\right)^3}}
\end{align*}

\subsection*{Opgave 6}
Forkort følgende brøker så meget som muligt. 
\begin{align*}
	&1) \ \frac{ab}{a}   &2) \  \frac{a+b}{b} + \frac{a-c}{a}    \\
	&3) \ \frac{(a+b)b-ab}{b}   &4) \  \left(\frac{a}{b}\right)^2 - \frac{a^3}{b^3}   \\
\end{align*}

\subsection*{Opgave 7}
Løs følgende ligninger.
\begin{align*}
	&1) \ \frac{x}{9} = \frac{4}{x}   &&2) \ \frac{x}{4+\frac{2}{5}} = 2     \\
	&3) \ \sqrt{\frac{x}{4}} = \frac{\frac{2}{5}+\frac{16}{10}}{\frac{1}{2}}    &&4) \ \frac{\frac{1}{4}}{\frac{1}{8}} + \frac{x+\frac{4}{2}x}{2} = \frac{7}{\frac{1}{2}}    \\
\end{align*}