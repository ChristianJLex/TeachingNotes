\begin{center}
\Huge
Vækst
\end{center}
Vi lægger vores vekstforløb ud med et gensyn med procentregning. Dette bliver brugbart, når vi næste gang begynder på eksponentiel vækst, som er den første væksttype, vi skal arbejde med i dette forløb. 
\section*{Procentregning}
\stepcounter{section}

Procentregning handler om forhold mellem tal. Hvis vi eksempelvis vil bestemme hvor stor en del af en klasse, der er højere en $180cm$, så udregner vi forholdet 
\begin{align*}
	\frac{\textnormal{Antal personer højere end 180cm}}{\textnormal{Det samlede antal personer}}.
\end{align*}
 Hvis vi vil have det i procent, så ganger vi med $100$ (da procent betyder pr. hundrede), og får
\begin{align*}
	\frac{\textnormal{Antal personer højere end 180cm}}{\textnormal{Det samlede antal personer}}\cdot 100.
\end{align*}
Lad os sige, at 6 personer i 1.m er højere end 180. Inklusiv undertegnede er vi i alt 26 personer. Det vil altså sige, at procentdelen af personer i 1.m, der er højere end 180cm er 
\begin{align*}
\frac{6}{26}\cdot 100 = 23.08\%.
\end{align*}
\section*{Procentregneregler}
\stepcounter{section}
Vi har følgende procentregneregler
\begin{regel}[Procentregneregel 1]
Skal vi bestemme $p\ \%$ af en størrelse $S$, så bestemmes den ved 
\begin{align*}
\frac{p}{100}\cdot S.
\end{align*} 
\end{regel}
\begin{exa}
Vi skal bestemme $p=10\%$ af $S=200$. Vi bruger Procentregneregel 1 og bestemmer
\begin{align*}
\frac{10}{100}\cdot 200 = 20. 
\end{align*} 
Normalt vil vi omskrive brøken $10/100$ til $0.1$, så regnestykket lyder
\begin{align*}
0.1\cdot 200 = 20.
\end{align*} 
\end{exa}
\begin{regel}[Procentregneregel 2]
Vi bestemmer den procentvise andel en størrelse $S$ udgør af en størrelse $T$ ved 
\begin{align*}
\frac{S}{T}\cdot 100
\end{align*}
\end{regel}
\begin{regel}[Procentregneregel 3]
Vi øger en størrelse $S$ med $p$ procent ved at bestemme
\begin{align*}
S\cdot(1+\frac{p}{100}). 
\end{align*}
$(1+\frac{p}{100})$ er det, der kaldes fremskrivningsfaktoren. 
Tilsvarende mindsker vi en størrelse $S$ med $p$ procent ved at bestemme
\begin{align*}
S \cdot (1-\frac{p}{100}).
\end{align*}
\end{regel}
\begin{exa}
Vi skal øge 130 med $15 \%$. Vi bestemmer derfor 
\begin{align*}
130 (1+\frac{15}{100}) = 130\cdot 1.15 = 149.5.
\end{align*}
Vi skal mindske $149.5$ med $15 \%$. Vi bestemmer
\begin{align*}
149.5(1-\frac{15}{100}) = 149.5\cdot 0.85 = 127.075.
\end{align*}
\end{exa}
Det er klart nemmest altid at bruge fremskrivningsfaktoren, når vi skal øge med en bestemt procentdel. 

\begin{regel}[Procentregneregel 4]
Hvis vi skal bestemme den procentvise ændring af en størrelse $S$ til en størrelse $T$, så beregnes dette 
\begin{align*}
\frac{T-S}{S}\cdot 100.
\end{align*}
\end{regel}
\begin{exa}
Prisen på 1000 liter fyringsolie var d. 24 august 2021 $9318.5$ kr. D. 24 november koster 1000 liter fyringsolie $10328.5$ kr. Vi bestemmer den procentvise ændring
\begin{align*}
\frac{10328.5-9318,5}{9318.5}\cdot 100 = 10.83\%. 
\end{align*}
\end{exa}

\subsection*{Opgave 1}

\begin{enumerate}[label=\roman*)]
	\item Bestem $15\%$ af 100.
	\item Bestem $30\%$ af 400.
	\item Bestem $60\%$ af 20.
	\item Bestem $75\%$ af 60.
\end{enumerate}

\subsection*{Opgave 2}

\begin{enumerate}[label=\roman*)]
	\item Hvor mange $\% $ udgør 16 af 100?
	\item Hvor mange $\% $ udgør 7 af 50?
	\item Hvor mange $\% $ udgør 3 af 30?
	\item Hvor mange $\% $ udgør 6 af 1000?
\end{enumerate}

\subsection*{Opgave 3}

\begin{enumerate}[label=\roman*)]
	\item Forøg 100 med $60\%$.
	\item Forøg 50 med $10\%$.
	\item Formindst 40 med $25\%$.
	\item Formindst 250 med $50\%$.
\end{enumerate}

\subsection*{Opgave 4}

\begin{enumerate}[label=\roman*)]
	\item Bestem den procentvise forskel fra $50$ til $100$.
	\item Bestem den procentvise forskel fra $20$ til $30$.
	\item Bestem den procentvise forskel fra $40$ til $10$.
	\item Bestem den procentvise forskel fra $100$ til $350$.
\end{enumerate}

\subsection*{Opgave 5}
\begin{enumerate}[label=\roman*)]
\item Bestem $10\%$, $50\%$ og $150\%$ af $300$.
\item Bestem $93\%$, $107\%$ og $1.2\%$ af 45.
\item Hvor mange procent er $7$, $11.5$ og $13$ af $100$?
\item Hvor stor en procentdel er $500$ af $100000$?
\item Øg 50 med $80\%$. 
\item Øg 20 med $12,5\%$.
\item Gør 166 65$\%$ mindre
\item Gør 10005 99$\%$ mindre.
\end{enumerate}
\subsection*{Opgave 6}
\begin{enumerate}[label=\roman*)]
\item Der er udsalg, og priserne på en bestemt trøje er gået fra 300kr til $170$kr. Hvad er det procentvise fald i prisen?
\item En mand er gået op i vægt. Han vejede tidligere for et år siden $92$kg og vejer nu $165$kg. Hvor stor er hans procentvise stigning i vægt?
\item  To huse er  hhv. 5 og 6 meter høje. Hvor stor procentvis forskel er der på deres højder? (Både det lille hus i forhold til det store, og det store i forhold til det lille.)
\end{enumerate}

\subsection*{Opgave 7}
\begin{enumerate}[label=\roman*)]
	\item Du sætter 9500kr ind på en investeringskonto, der lover et årligt afkast på 10$\%$. Hvor meget forventer du,
	 at der er på kontoen efter et år? Hvad med to år? Hvad med 10 år?
	\item Hvor stor en procentvis stigning vil du have opnået efter 6 år, hvis du sætter et vilkårligt beløb ind til at
	starte med?
	\item Efter 15 år finder du ud af, at det totale procentvise afkast efter 15 år er $300\%$. Er det mere eller mindre 
	end hvad udbyderen af kontoen lovede?
\end{enumerate}

\subsection*{Opgave 8}
Du indtager et bestemt lægemiddel, og har umiddelbart efter indtagelsen af lægemiddelet en blodkoncentration 
	på $234.12 \mu g / l$. Hver gang der går en time, så vil koncentrationen af lægemiddelet aftage med 4.3$\%$. 
\begin{enumerate}[label=\roman*)]
	\item Hvad er blodkoncentrationen efter en time? 
	\item Hvad er blodkoncentrationen efter 7 timer?
	\item Vi antager, at lægemiddelet kan betragtes som nedbrudt, når blodkoncentrationen af lægemiddelet er mindre end
	 $1\mu g/l$. Hvornår kan lægemiddelet betragtes som nedbrudt?
\end{enumerate}

\subsection*{Opgave 9}
Lad os sige, at vi har en væksttype, der kan beskrives ved en funktion $f$. Vi vil gerne bestemme en forskrift for denne væksttype. Vi betegner vores begyndelsesværdi med $b$, og vi ønsker at øge denne værdi med $p\%$ hver gang en bestemt tidsperiode er forløbet. Lad os uden tab af generalitet antage, at $b$ skal øges med $p\%$ en gang i timen. 
\begin{enumerate}[label=\roman*)]
	\item Hvad skal vi gange vores begyndelsesværdi med for at øge den med $p\%$? Vi kalder dette tal for $a$.
	\item Når der er forløbet to timer, skal vi igen gange vores begyndelsesværdi med $a$. Hvordan noterer vi,
	at der ganget to gange med $a$? Hvad med tre?
	\item Vi skal have lavet en forskrift for $f$, der beskriver udviklingen af væksten. Hvis vi betegner 
	den forløbne tid med $t$, kan I så komme frem til en forskrift for $f$, der beskriver denne væksttype?
\end{enumerate}
