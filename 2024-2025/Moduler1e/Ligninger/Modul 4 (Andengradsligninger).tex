\begin{center}
\Huge
	Andengradsligninger
\end{center}

\section*{Diskriminanten}
\stepcounter{section}

Når vi skal løse andengradsligninger, skal vi bruge det, vi kalder for \textit{diskriminanten}. Vi definerer den som følgende.
\begin{defn}[Diskriminanten]
	For en andengradsligning på formen
	\begin{align*}
		ax^2 + bx + c = 0
	\end{align*}
	defineres diskriminanten som
	\begin{align*}
		d = b^2 - 4ac.
	\end{align*}
\end{defn}
Diskriminanten har fået sit navn, fordi den diskriminerer mellem antallet af løsninger for andengradsligningen. For diskriminanten gælder følgende sætning.
\begin{setn}
	For andengradsligningen
	\begin{align*}
		ax^2 + bx + c = 0
	\end{align*}
	gælder der for diskriminanten $d$, at 
	\begin{itemize}
		\item[$\cdot$] hvis $d < 0$, så har ligningen nul løsninger,
		\item[$\cdot$] hvis $d = 0$, så har ligningen netop én løsning, og 
		\item[$\cdot$] hvis $d > 0$, så har ligningen to løsninger. 
	\end{itemize}
\end{setn}
Vi skal senere se et bevis for dette. 
\begin{exa}
	Vi betragter andengradsligningen
	\begin{align*}
		2x^2 - 4x + 6 = 0.
	\end{align*}
	Denne har $a = 2$, $b = -4$ og $c = 6$. Vi udregner diskriminanten og får
	\begin{align*}
		d = (-4)^2 - 4\cdot 2 \cdot 6 = 16 - 48 = -32, 
	\end{align*}
	og andengradsligningen har derfor nul løsninger. 
	Tilsvarende kan vi finde diskriminanten for andengradsligningen
	\begin{align*}
		-3x^2 + 6x - 3 = 0.
	\end{align*}
	Her bliver diskriminanten
	\begin{align*}
		d = 6^2 - 4\cdot (-3)\cdot (-3) = 36 - 36 = 0,
	\end{align*}
	så denne ligning har netop én løsning. 
	Til slut kan vi finde diskriminanten for andengradsligningen
	\begin{align*}
		x^2 + 5x - 2 = 0,
	\end{align*}
	der lyder
	\begin{align*}
		d = 5^2 - 4\cdot 1 \cdot (-2) = 25 + 8 = 33,
	\end{align*}
	og denne andengradsligning har derfor to løsninger. 
\end{exa}

\section*{Løsning af andengradsligninger}
\stepcounter{section}
Vi har set, at vi kan anvende kvadratsætninger baglæns og nulreglen for at løse andengradsligninger. Dette kan i princippet gøres med alle andengradsligninger, men er ofte besværligt at gennemskue. Vi har derfor en formel, der kan bruges for at løse andengradsligninger. Denne kaldes for \textit{diskriminantformlen} eller \textit{løsningsformlen for andengradsligninger}.
\begin{setn}[Diskriminantformlen]
	Lad en andengradsligning være givet ved
	\begin{align*}
		ax^2 + bx + c,
	\end{align*}
	hvor $a \neq 0$. Hvis $d \geq 0$, kan løsningerne til ligningen findes som
	\begin{align*}
		x_1 = \frac{-b+ \sqrt{d}}{2a},
	\end{align*}
	og
	\begin{align*}
		x_2 = \frac{-b - \sqrt{d}}{2a},
	\end{align*}
	hvor $d = b^2 - 4ac$. 
\end{setn}
\begin{proof}
	Vi betragter ligningen
	\begin{align*}
		ax^2 + bx + c = 0.
	\end{align*}
	Vi multiplicerer ligningen med $4a$ og får
	\begin{align*}
		4a\cdot x^2 + 4a\cdot bx + 4a \cdot c = 4a\cdot 0 \ \Leftrightarrow \ 4a^2x^2 + 4abx + 4ac = 0.
	\end{align*}
	Vi lægger diskriminanten $d = b^2 - 4ac$ til på begge sider af lighedstegnet
	\begin{align*}
		4a^2x^2 + 4abx + 4ac = 0 \ &\Leftrightarrow \ 4a^2x^2 + 4abx + 4ac + b^2 - 4ac = b^2 - 4ac \\
		&\Leftrightarrow \ 4a^2x^2 + 4abx + b^2 = b^2 - 4ac \\
		&\Leftrightarrow \ (2ax)^2 + 4axb + b^2 = d\\
	\end{align*}
	Vi indser nu, at vi kan bruge kvadratsætningen
	\begin{align*}
		(\alpha + \beta)^2 = \alpha^2 + \beta^2 + 2\alpha\beta
	\end{align*}
	på dette, hvor $\alpha^2 = (2ax)^2$, $\beta^2 = b^2$ og $2\alpha\beta = 2\cdot 2ax \cdot b$. 
	Vi kan derfor skrive
	\begin{align*}
		(2ax)^2 + 4axb + b^2 = d \ &\Leftrightarrow \ (2ax + b)^2 = d \\
	\end{align*}
	Vi ønsker at isolere $x$ i dette udtryk. Vi tager derfor kvadratroden på begge sider af lighedstegnet og
	får
	\begin{align*}
		2ax + b = \pm \sqrt{d},
	\end{align*}
	hvor vi husker på, at vi skal have både den positive og negative løsning til kvadratet.
	Afslutningsvist får vi så
	\begin{align*}
		2ax + b = \pm \sqrt{d} \ &\Leftrightarrow	\ 2ax =-b \pm \sqrt{d}\\
		&\Leftrightarrow \ x = \frac{-b \pm \sqrt{d}}{2a}.
	\end{align*}
\end{proof}
\begin{exa}
	Vi betragter andengradsligningen 
	\begin{align*}
		x^2 - x - 2 = 0,
	\end{align*}
	og vi ønsker at bestemme løsningerne. Vi bestemmer først diskriminanten som
	\begin{align*}
		d = (-1)^2 - 4\cdot 1 \cdot (-2) = 1 + 8  = 9.
	\end{align*}
	Der er derfor to løsninger. Vi sætter dette ind i løsningsformlen. 
	\begin{align*}
		x_1 = \frac{-(-1)+\sqrt{9}}{2\cdot 1} = \frac{1+3}{2} = 2
	\end{align*}
	og 
	\begin{align*}
		x_2 = \frac{1-3}{2} = -1.
	\end{align*}
	Vi vinder altså løsningerne $x_1 = 2$ og $x_2 = -1$. 
\end{exa}
\subsection*{Opgave 1}

Bestem diskriminanten $d$ for følgende andengradspolynomier og afgør antallet af løsninger. 
\begin{align*}
	&1) \ 2x^2 + 4x - 2 = 0    &   &2) \ -x^2 + 5x + 7 = 0    \\
	&3) \ x^2 + 4 + 7x = 0    &   &4) \ 5x^2 = -9x + 8    \\
	&5) \ 6x^2 + \sqrt{2}x - 2 = 0    &   &6) \ 3x^2 - 4x = 1    \\
	&7) \ \sqrt{3}x^2 -3x + \sqrt{3} = 0    &   &8) \ \frac{2}{5}x^2 - \frac{6}{5}x + \frac{1}{6} = 0    \\
\end{align*}

\subsection*{Opgave 2}
Bestem løsningen til følgende andengradsligninger, hvis den eksisterer, ved at bruge diskriminantformlen.
\begin{align*}
	&1) \ x^2 - 11x + 10 = 0 &  &2) \ 4x^2 + 4x + 1 = 0\\
	&3) \ -3x^2 + 6x - 6 = 0 &  &4) \ x^2 - 6x + 5 = 0\\
	&5) \ 3x^2 + 3x - 18 = 0 &  &6) \ x^2 - 10x + 24 = 0\\
	&7) \ 4x^2 - 11x + 6 = 0  &  &8) \ 2x^2 - 50 = 0\\
	&9) \ 4x^2 - 12x + 9 = 0 &  &10) \ 9x^2 - 18x + 9 = 0\\	
 \end{align*}

