
\begin{center}
\Huge
Den binomialfordelte stokastiske variabel
\end{center}
\section*{Den binomialfordelte stokastiske variabel}
\stepcounter{section}

En \textit{binomialfordelt stokastisk variabel} er en stokastisk variabel, der beskriver sandsynligheden for et bestemt antal succeser i en række Bernoulli-eksperimenter. Dette kunne være sandsynligheden for at få krone to gange i syv forsøg eller at få to sekser i fem kast med en terning. 

Vi definerer den binomialfordelte stokastiske variabel lidt anderledes og motiverer herefter definitionen. Først defineres dog, hvad der menes, når vi siger \textit{sandsynlighedsfunktion} i forbindelse med stokastiske variable. 
\begin{defn}[Sandsynlighedsfunktion]
	Lad $X$ være en stokastisk variabel. Sandsynlighedsfunktionen for $X$ defineres da til at være funktion $f$, der opfylder,  at
	\begin{align*}
		f(x) = P(X = x).
	\end{align*}
\end{defn}
\begin{defn}[Binomialfordelt stokastisk variabel]
	\label{def:bin}
	En stokastisk variabel $X$ siges at være \textit{binomialfordelt} med antalsparameter $n$ og sandsynlighedsparameter $p$ hvis sandsynlighedsfunktionen for $X$ er givet ved
	\begin{align*}
		f(r) = P(X = r) = \textnormal{K}(n,r)\cdot p^r\cdot (1-p)^{n-r}.
	\end{align*}
	Vi skriver, at $X \sim \textnormal{B}(n,p)$. 
\end{defn}

Vi vil nu motivere definitionen. Betragt situationen, hvor vi har udført $n$ stokastiske eksperimenter, der hver har sandsynlighed $p$ for succes. Desuden er de første $r$ eksperimenter succeser og resten er fiaskoer. Sandsynligheden for at få $r$ succeser i streg er (da eksperimenterne er uafhængige)
\begin{align*}
	\underbrace{p\cdot p\cdots p}_{\textnormal{r gange}} = p^r
\end{align*}
Sandsynligheden for at få $n-r$ fiaskoer i streg er
\begin{align*}
	\underbrace{(1-p)\cdot (1-p)\cdots (1-p)}_{\textnormal{n-r gange}} = (1-p)^{n-r}.
\end{align*}
Skal vi først have succeserne \textit{og} derefter fiaskoerne er sandsynligheden for dette produktet af de to tidligere sandsynligheder.
\begin{align*}
	p^r \cdot (1-p)^{n-r}
\end{align*}
Vi er næsten i mål. Vi mangler blot følgende betragtning; vi er ligeglade med, hvornår vi får vores succeser. Vi skal derfor bestemme antallet af måder, vi kan placere vores succeser i vores $n$ forsøg. Dette må tilsvare $\textnormal{K}(n,r)$, da vi skal udtrække $r$ pladser til vores succeser blandt de $n$ forsøg. I alt må sandsynligheden for at få vores $r$ succeser derfor være
\begin{align*}
	\textnormal{K}(n,r) \cdot p^r \cdot (1-p)^{n-r},
\end{align*}
da enhver placering af succeserne må have samme sandsynlighed.

Vi betragter et eksempel på denne argumentation.
\begin{exa}
	\label{exa:bin}
	Vi skal bestemme sandsynligheden for at slå 2 seksere i 5 forsøg. Sandsynligheden for at de to første slag er seksere er 
	\begin{align*}
		\frac{1}{6}\cdot \frac{1}{6} = \frac{1}{36}
	\end{align*}	
	Sandsynligheden for, at de resterende slag ikke er seksere er
	\begin{align*}
		\frac{5}{6}\cdot \frac{5}{6} \cdot \frac{5}{6} \approx 0.58.
	\end{align*}
	Sandsynligheden for kombinationen af succeser SSFFF er derfor
	\begin{align*}
		\frac{1}{2}\cdot \frac{1}{2}\cdot \frac{5}{6}\cdot \frac{5}{6} \cdot  \frac{5}{6} \approx \frac{1}{36} \cdot 0.58 \approx 0.016.
	\end{align*}
	Sandsynligheden for andre kombinationer af succeser (eksempelvis SFSFF) må alle have denne sandsynlighed. Antallet af sådanne kombinationer er
	\begin{align*}
		\textnormal{K}(5,2) = \frac{5!}{(5-3)!\cdot 2!} = 10.
	\end{align*}
	Sandsynligheden for at få netop 2 seksere i fem forsøg er derfor
	\begin{align*}
		P(X = 2) &= K(5,2)\left(\frac{1}{6} \right)^2\left(1 - \frac{1}{6}\right)^3 \\
				 &= K(5,2)\left(\frac{1}{6} \right)^2 \left(\frac{5}{6}\right)^3 \\
				 &\approx 10\cdot \frac{1}{36} \cdot 0.58 \\
				 &\approx 10 \cdot 0.016 \\
				 &= 0.16, 
	\end{align*}
	så sandsynligheden for at få netop 2 seksere i 5 forsøg er altså omtrent 16$\%$.
\end{exa}

Det vil ofte være besværligt at bruge sandsynlighedsfunktionen for binomialfordelingen i hånden. I Maple hedder sandsynlighedsfunktionen for binomialfordelingen for \texttt{binpdf} (for binomial probability distribution function), og den bruges ved at skrive
\begin{align*}
	\texttt{binpdf(n,p,x)},
\end{align*}
hvor \texttt{n} er antalsparameteren, \texttt{p} er sandsynlighedsparameteren og \texttt{x} er udfaldet, du ønsker at bestemme sandsynligheden for.
\begin{exa}
	Vi ønsker i Maple at bestemme sandsynligheden for at få 2 seksere i 5 forsøg. Vi skriver i Maple
	\begin{align*}
		&\texttt{restart}\\
		&\texttt{with(Gym):}\\
		&\texttt{binpdf(5,$\frac{\texttt{1}}{\texttt{6}}$,2})
	\end{align*}
	og får outputtet 0.16.
\end{exa}

\subsection*{Fordelingsfunktion}

Vi er ofte interesserede i sandsynligheden for, om vi har fået mindre end et bestemt antal succeser. Derfor introduceres \textit{fordelingsfunktionen}.
\begin{defn}[Fordelingsfunktion]
	Fordelingsfunktionen for en stokastisk variabel $X$ er en funktion $F$, der opfylder, at
	\begin{align*}
		F(x) = P(X \leq x).
	\end{align*}
\end{defn}

\begin{exa}
	Vi skal bestemme sandsynligheden for at slå mindre end eller lig 2 seksere i 5 forsøg med en terning. Vi skal derfor for den binomialfordelte stokastiske variabel $X \sim 		
	\textnormal{B}\left(5,\frac{1}{6}\right)$ bestemme
	\begin{align*}
		P(X \leq 2).
	\end{align*}
	Dette gøres ved at bestemme 
	\begin{align*}
		P(X = 0) + P(X = 1) + P(X = 2) \approx 0.40 + 0.40 + 0.16 = 0.96,
	\end{align*}
	men kan også gøres i Maple ved at bruge funktionen \texttt{bincdf} (for binomial cumulative distribution function).
	\begin{align*}
		&\texttt{restart}\\
		&\texttt{with(Gym):}\\
		&\texttt{bincdf$\left(\texttt{5,$\frac{\texttt{1}}{\texttt{6}}$,2}\right)$}
	\end{align*}
	og vi får outputtet 0.96.
\end{exa}



\subsection*{Opgave 1}
\begin{enumerate}[label=\roman*)]
	\item Hvad er sandsynligheden for at slå nøjagtigt fem seksere på seks slag med en terning?
	\item Hvad er sandsynligheden for at få mindst 3 seksere?
	\item Hvad er sandsynligheden for at få mindre en 4 nøjagtigt 2 gange?
\end{enumerate}
\subsection*{Opgave 2}
Et lægemiddel bruges til en bestemt behandling, og gives til 10 personer. Sandsynligheden for helbredelse er 20$\%$. 
\begin{enumerate}[label=\roman*)]
	\item Hvad er sandsynligheden for, at to personer helbredes?
	\item Hvad er sandsynligheden for, at alle personer helbredes?
	\item Hvad er sandsynligheden for, at mindst 4 helbredes? 
\end{enumerate}

\subsection*{Opgave 3}
I en by har 15$\%$ stemt på partiet "liste Q". 100 Personer udvælges tilfældigt og spørges "har du stemt liste Q?"
\begin{enumerate}[label=\roman*)]
	\item Hvad er sandsynligheden for, at netop 15 personer svarer ja?
	\item Hvad er sandsynligheden for, at ingen svarer ja?
	\item Bestem sandsynligheden for, at mere end 20 svarer ja. 
\end{enumerate}

\subsection*{Opgave 4}
Til kurset "Introduktion til matematisk analyse" på Husum Universitet dumper 50$\%$ af deltagerne. 20 procent af dem, der har dumpet vil lyve, når de bliver spurgt, om de har dumpet kurset. 15 personer udvælges tilfældigt og spørges, om de er dumpet. 
\begin{enumerate}[label=\roman*)]
	\item Hvad er sandsynligheden for, at 3 personer svarer, at de er dumpet?
	\item Bestem sandsynligheden for, at mere end 5 personer svarer, at de er dumpet.
	\item Bestem sandsynligheden for, at mellem 2 og 4 personer svarer, at de er dumpet. 
\end{enumerate}