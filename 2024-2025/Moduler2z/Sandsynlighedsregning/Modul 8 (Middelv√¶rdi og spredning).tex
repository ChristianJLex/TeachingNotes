\begin{center}
\Huge
Middelværdi og spredning
\end{center}
\section*{Middelværdi og spredning}
\stepcounter{section}

Har vi en stokastisk variabel $X$, så kunne vi godt tænke os at beskrive det mest sandsynlige udfald af den stokastiske variabel. Betragter vi $1000$ slag med en terning, så vil det gennemsnitlige slag være summen af alle de slåede slag delt med $1000$. Lad os se på et eksempel:
\begin{exa}
Vi slår $1000$ gange med en terning og får følgende resultater
\begin{center}
\begin{tabular}{c|c|c|c|c|c}
1 & 2 & 3 & 4 & 5 & 6 \\
\hline 
157& 162& 165& 176& 173& 167
\end{tabular}
\end{center}
Vi vil så bestemme det gennemsnitlige slag som
\[
\frac{1\cdot 157 + 2\cdot 162 + 3\cdot 165 + 4 \cdot 176 + 5\cdot 173 + 6\cdot 167}{1000} = 3.445.
\]
Vi har derfor eksperimentelt bestemt den forventede værdi, når vi slår med en terning. 
\end{exa}
Vi vil bruge mere eller mindre det samme princip, når vi definerer \textit{middelværdien for en stokastisk variabel}.
\begin{defn}[Middelværdi]
Lad $X$ være en (diskret) stokastisk variabel med værdimængde $\{x_1,x_2,\hdots,x_n\}$ og fordelingsfunktion $P$. Så defineres middelværdien $\mathbb{E}(X)$ for en stokastisk variabel som
\begin{align*}
\mathbb{E}(X) = \mu &= x_1 \cdot P(X = x_1) + x_2 \cdot P(X=x_2) + \cdots + x_n \cdot P(X=x_n)\\
&= \sum_{i=1}^{n} x_i P(X=x_i).
\end{align*} 
\end{defn}
Middelværdien giver os et indblik i, hvad den stokastiske variabel mest sandsynligt antager af værdi, men den siger ikke noget om, hvor meget værdierne, variablen kan antage varierer omkring middelværdien $\mu$. Vi vil definere spredningen som den "gennemsnitlige variation fra middelværdien". Derfor ville det give mening at definere spredningen som 
\begin{align*}
\sigma = \mathbb{E}(X-\mu), 
\end{align*}
men dette vil tage højde for, at $X-\mu$ nogle gange er negativ og nogle gange er positiv. Vi vil bare gerne vide, hvor langt $X$ ligger fra $\mu$, så derfor definerer vi spredningen på følgende vis:
\begin{defn}[Spredning]
Lad $X$ være en (diskret) stokastisk variabel med værdimængde $\{x_1,x_2,\hdots,x_n\}$ og fordelingsfunktion $P$. Så defineres \textit{spredningen} for $X$ som
\begin{align*}
\sigma &= \sqrt{E[(X-\mu)^2]}\\
&= \sqrt{(x_1-\mu)^2\cdot P(X=x_1) + \cdots + (x_n-\mu)^2 \cdot P(X=x_n)}\\
\end{align*}
 
\end{defn}

\begin{exa}
Lad os fortsætte eksemplet med terningekastet. Vi vil bestemme middelværdien $\mathbb{E}(X)$ for den stokastiske variabel $X$, der beskriver et terningekast. Dette bestemmes ifølge definitionen som
\begin{align*}
\mathbb{E}(X) = 1\cdot \frac{1}{6} + 2\cdot \frac{1}{6}+3\cdot \frac{1}{6}+4\cdot \frac{1}{6}+5\cdot \frac{1}{6}+6\cdot \frac{1}{6} = \frac{21}{6} = 3.5,
\end{align*}
hvilket er tæt på det vi eksperimentelt havde bestemt gennemsnittet til at være. 
Spredningen er så
\begin{align*}
\sigma^2 &= (1-3.5)^2 \cdot \frac{1}{6} + (2-3.5)^2 \cdot \frac{1}{6} + (3-3.5)^2 \cdot \frac{1}{6}\\
 &+ (4-3.5)^2 \cdot \frac{1}{6} + (5-3.5)^2 \cdot \frac{1}{6} +(6-3.5)^2 \cdot \frac{1}{6} \approx 2.92
\end{align*}
Spredningen er derfor 
\begin{align*}
\sigma \approx \sqrt{2.92} \approx 1.71.
\end{align*}
Det er svært at give en god fortolkning af spredningen ud over at det er den gennemsnitlige afstand et slag vil have fra middelværdien. 
\end{exa}

Hvis vi kender sandsynlighedsparametren $p$ og antalsparametren $n$ for en binomialfordelt stokastisk variabel, så er det muligt at bestemme middelværdien og variansen.
\begin{setn}
Lad $X$ være en binomialfordelt stokastisk variabel med antalsparameter $n$ og sandsynlighedsparameter $p$. Så er middelværdien $\mu$ og spredningen $\sigma$ for $X$ givet ved
\begin{align*}
\mathbb{E}(X) = \mu = n\cdot p,
\end{align*} 
og 
\[
 \sigma = \sqrt{n\cdot p\cdot (1-p)}
\]
\end{setn}



\subsection*{Opgave 1}
\stepcounter{section}
\begin{enumerate}[label=\roman*)]
\item Slå plat og krone, og lad $X=1$, hvis udfaldet er plat og $X=0$ ellers. Hvad er middelværdien og spredningen for $X$?
\item Slå plat og krone to gange og lad $X$ være antallet af plat. Hvad er middelværdien og spredningen af $X$?
\item Kast en 12-sidet terning og lad $X$ være antallet af øjne. Hvad er middelværdien og spredningen af $X$?

\end{enumerate}

\subsection*{Opgave 2}

Fordelingen for en stokastisk variabel $X$ kan beskrives ved følgende tabel.

\begin{center}
	\begin{tabular}{c|c|c|c|c|c|c}
		$x$ & -5 & 2 & 3 & 4 & 7 & 11 \\
		\hline
		$P(X = x)$ & 0.1 & 0.2 & 0.1 & 0.4 & 0.05 & 0.15
	\end{tabular}
\end{center}

\begin{enumerate}[label=\roman*)]
	\item Bestem middelværdien for $X$.
	\item Bestem spredningen for $X$.
\end{enumerate}

\subsection*{Opgave 3}
\stepcounter{section}
\begin{enumerate}[label=\roman*)]
\item Lad $X$ være en Bernoulli-fordelt stokastisk variabel med sandsynlighedsparameter $p$. Hvad er middelværdien og spredningen af $X$?
\item Forklar med ord, hvorfor det giver god mening, at middelværdien for en binomialfordelt stokastisk variabel er $n\cdot p$. 
\end{enumerate}

\subsection*{Opgave 4}
\begin{enumerate}[label=\roman*)]
\item Et lægemiddel helbreder 15$\%$ af behandlede personer. Lad $X$ beskrive antallet af helbredte personer, når vi prøver at helbrede $10000$ personer. Hvad er middelværdien og spredningen af $X$?
\item Vi slår fem gange med en terning, og lader $X$ beskrive antallet af gange, vi slår mere end 4. Hvad er middelværdien og spredningen for $X$?
\end{enumerate}
