
\begin{center}
\Huge
Afstand mellem linje og punkt
\end{center}

\section*{Afstand mellem linje og punkt}
\stepcounter{section}

Vi kan bruge projektioner af vektorer til at bestemme afstanden mellem et punkt $P$ og en linje $l$ i planen. Det er ikke umiddelbart klart ud fra formlen at det er projektioner, vi bruger, men det vil ses af beviset. Afstanden findes ved følgende sætning. 

\begin{setn}[Afstand mellem punkt og linje]
Har vi et punkt $Q(x_1,y_1)$ og en linje $l$ givet ved ligningen
\begin{align*}
ax+by+c = 0, 
\end{align*}
så kan vi bestemme afstanden mellem $l$ og $Q$ ved
\begin{align*}
\textnormal{dist}(Q,l) = \frac{|ax_1+by_1+c|}{\sqrt{a^2+b^2}}
\end{align*}
\end{setn}
\begin{figure}[H]
	\centering
	\resizebox{0.7\textwidth}{0.7\textwidth}{
	\begin{tikzpicture}
		\begin{axis}[axis lines = middle, xmin = -1, ymin = -1, xmax = 4, ymax = 4]
			\addplot[color = teal, thick] {0.5*x+0.5 };
			\draw[-{Stealth[scale = 1.4]}, thick, teal]
			 (axis cs: 2,1.5) -- (axis cs:1,3.5);
			 \node at (axis cs: 1+0.4,3.5) {$\vv{n}$};
			
			\node at (axis cs:0.5,0.75-0.3) {$P$};
			\node at (axis cs: 1.25-0.4,3) {$Q$};
			\draw[-{Stealth[scale = 1.4]}, thick, teal]
			(axis cs: 2,1.5) -- (axis cs: 1.25,3);
			
			\draw[-{Stealth[scale= 1.4]}, thick, color = olive]
			(axis cs: 0.5,0.75) -- (axis cs: 1.25,3); 
			\draw[-{Stealth[scale= 1.4]}, thick, color = olive]
			(axis cs: 0.5+1.5,0.75+0.75) -- (axis cs: 1.25+1.5,3+0.75); 
			\draw[thick, dashed, color = teal]
			 (axis cs: 1.25+1.5,3+0.75) -- (axis cs: 1.25,3);
			\node at (axis cs: 1.85,2.5) {$\vv{PQ}_{\vv{n}}$};
			\filldraw[purple] (axis cs:0.5,0.75) circle (2pt);
			\filldraw[purple] (axis cs: 1.25,3) circle (2pt);
		\end{axis}
	\end{tikzpicture}
	}
	\caption{Afstand fra punkt til linje}
	\label{fig:afstand}
\end{figure}
\subsection*{Bevis}
Vi skal bevise afstandsformlen. Vi bruger Figur \ref{fig:afstand} som skabelon
\begin{enumerate}[label=\roman*)]
	\item Tegn en linje $l$ i et koordinatsystem som på Figur \ref{fig:afstand}.
	\item Tegn et punkt $Q(x_1,y_1)$ i koordinatsystemet. Det må ikke ligge på $l$.
	\item Tegn et punkt $P(x_0,y_0)$ på $l$. Du bestemmer selv hvor. 
	\item Tegn en vektor $\vv{PQ}$ fra $P$ til $Q$. 
	\item Tegn en normalvektor $\vv{n}$ til $l$, der går gennem $Q$
	\item Tegn en repræsentant for den samme vektor som $\vv{PQ}$. Denne skal starte samme sted som $\vv{n}$. 
	\item Tegn projektionen $\vv{PQ_{\vv{n}}}$.
	\item Overbevis dig selv om, at afstanden mellem $l$ og $Q$ må være lig længden af $\vv{PQ_{\vv{n}}}$.
	\item Overbevis dig selv om, at $\vv{n}$ har koordinaterne $\vv{n}=\begin{pmatrix}
	a \\ b
	\end{pmatrix}$.
	\item Opskriv koordinaterne til $\vv{PQ}$. (Vink: hvordan bestemmer vi en forbindelsesvektors koordinater?)
	\item Vi skal have bestemt længden af $\vv{PQ_{\vv{n}}}$. Denne er givet ved
	\begin{align*}
		|\vv{PQ_{n}}| = \frac{|\vv{PQ}\cdot \vv{n}|}{|\vv{n}|}.
	\end{align*}
	Opskriv prikproduktet og længden af vektoren i denne brøk.
	\item Hæv parentesen i tælleren af din brøk.
	\item Vi er nu næsten færdige. Isolér $c$ i linjens ligning $ax+by+c=0$. 
	\item Indsæt punktet $P(x_0,y_0)$ i ligningen. 
	\item Sammenlign udtrykket for $c$ med tælleren i din brøk og lav en substitution. 
	\item Sammenlign dit resultat med sætningen. 
\end{enumerate}