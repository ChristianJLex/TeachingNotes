
\begin{center}
\Huge
Projektioner af vektorer
\end{center}

\section*{Projektioner}
\stepcounter{section}

Har vi to vektorer $\vv{a}$ og $\vv{b}$, så kan vi være interesserede i at bestemme den vektor $\vv{a_{\vv{b}}}$, der peger i samme retning som $\vv{b}$ og som i en forstand er så tæt på $\vv{a}$ som muligt. Vi kalder i et sådant tilfælde vektoren $\vv{a_{\vv{b}}}$ for projektionen af $\vv{a}$ på $\vv{b}$. Vi skriver også 
\begin{align*}
\textnormal{proj}_{\vv{b}}(\vv{a}) = \vv{a}_{\vv{b}}.
\end{align*}
\begin{figure}[H]
	\centering
	\begin{tikzpicture}
		\begin{axis}[axis lines = middle,
		xlabel = $x$,
		ylabel = $y$,
		ticks = none, 
		xmin = -1, xmax = 6,
		ymin = -1, ymax = 6,
		]
			\draw[-{Stealth[scale = 1.3]}, color = blue!40, thick] (axis cs: 0,0) -- (axis cs: 5,1);
			\draw[-{Stealth[scale = 1.3]}, color = red!40, thick] (axis cs: 0,0) -- (axis cs: 3,3);
			\draw[dashed, thick, color = red!40] (axis cs:  45/13, 9/13) -- (axis cs: 3,3);
			\draw[dashed, red!40, thick] (axis cs: 45/13-0.4-0.4*0.2, 9/13-0.4*0.2+0.4) -- (axis cs: 45/13-0.4, 9/13-0.4*0.2);
			\draw[dashed, red!40, thick] (axis cs: 45/13-0.4-0.4*0.2, 9/13-0.4*0.2+0.4) -- (axis cs: 45/13-0.4*0.2, 9/13+0.4);
			\node[color = blue!40] at (axis cs: 5,1-0.5) {$\vv{b}$};
			\node[color = red!40] at (axis cs: 3,3+0.5) {$\vv{a}$};
			\node[color = purple] at (axis cs:  45/13, 9/13-0.4) {$\vv{a_{\vv{b}}}$};
			\draw[-{Stealth[scale = 1.3]}, color = purple, thick] (axis cs: 0,0) -- (axis cs:  45/13, 9/13);
		\end{axis}
	\end{tikzpicture}
	\caption{Projektion af vektor på vektor.}
	\label{fig:projektion}
\end{figure}

Vi starter med at vise, hvordan vi finder projektionen af en vektor på en anden vektor. 
\begin{setn}[Projektionssætningen]
For to vektorer $\vv{a}$ og $\vv{b}$ er projektionen af $\vv{a}$ på $\vv{b}$, som vi betegner
\begin{align*}
\textnormal{proj}_{\vv{b}}(\vv{a}) = \vv{a}_{\vv{b}}, 
\end{align*}
givet ved
\begin{align*}
\textnormal{proj}_{\vv{b}}(\vv{a}) = \frac{\vv{a}\cdot \vv{b}}{|\vv{b}|^2} \vv{b}.
\end{align*}
Længden af  $\vv{a}_{\vv{b}}$ er givet ved
\begin{align*}
\left|\vv{a}_{\vv{b}}\right| = \frac{|\vv{a}\cdot \vv{b}|}{|\vv{b}|}
\end{align*}
\end{setn}
\subsection*{Bevis for projektionssætningen}
Vi skal gennem en række små opgaver bevise projektionssætningen. 
\begin{enumerate}[label=\roman*)]
	\item Tegn en skitse af Figur \ref{fig:projektion}.
	\item Tegn en normalvektor $\vv{n}$ til $\vv{b}$, der går langs den stiplede linje på Figur \ref{fig:projektion} op til $\vv{a}$.
\end{enumerate}
Vi skal bruge tre kendsgerninger, som I skal overbevise jer selv om sandheden af.
\begin{enumerate}[label=\roman*)]
	\setcounter{enumi}{2}
	\item Overbevis jer selv om, at 
	\begin{align}\label{eq:1}
		\vv{a_{\vv{b}}} + \vv{n} = \vv{a}.
	\end{align}	 
	\item Overbevis jer selv om, at
	\begin{align}\label{eq:2}
		\vv{a_{\vv{b}}} = k \vv{b}
	\end{align}	  
	for et tal $k$.
	\item Overbevis jer selv om, at
	\begin{align}\label{eq:3}
		\vv{n} \cdot \vv{b} = 0.
	\end{align}	
	\item Isolér $\vv{n}$ i \eqref{eq:1}.
	\item Indsæt dette udtryk for $\vv{n}$ i \eqref{eq:3} (vink: husk parentes).
	\item Prik $\vv{b}$ ind i parentesen (Hæv parentesen). 
	\item Indsæt udtrykket for $\vv{a_{\vv{b}}}$ fra \eqref{eq:2} i udtrykket.
	\item Isolér $k$ i udtrykket. 
	\item Udnyt, at $\vv{b}\cdot\vv{b} = \big|\vv{b}\big|^2$ til at omskrive udtrykket.
	\item Indsæt dette udtryk for $k$ i \eqref{eq:2}.
	\item Sammenlign med projektionssætningen.
\end{enumerate}

Ekstraudfordring: Bestem længden af projektionsvektoren og bevis anden del af projektionssætningen.