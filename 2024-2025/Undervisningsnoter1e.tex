\documentclass[12pt]{article}
\input{preamble}

\pagestyle{fancy}
\fancyhf{}

\rhead{Nørre Gymnasium\\
1.e
}


%Husk at rette modul og dato!
\lhead{Matematik A\\
Vækst
}
\chead{16. januar 2025
}

\cfoot{Side \thepage \hspace{1pt} af \pageref{LastPage}}

\begin{document}

%Udfyld afsnit herunder og lav til egen Latex-fil

%Kopier følgende til overskrift:

%\begin{center}
%\Huge
%Aflevering 1
%\end{center}
%\section*{Opgave 1}
%\stepcounter{section}
%Overskrift



\begin{center}
\Huge
Forberedelse til prøve	
\end{center}

\subsection*{Opgave 1 (Uden hjælpemidler)}

\begin{enumerate}[label=\roman*)]
	\item Bestem fremskrivningsfaktoren og begyndelsesværdien for eksponentialfunktionen $f$ givet ved
	\begin{align*}
		f(x) = 7\cdot 1.3^x
	\end{align*}
	\item Bestem vækstraten for eksponentialfunktionen $g$ givet ved
	\begin{align*}
		g(x) = 0.7\cdot 0.9^x
	\end{align*}
	og beskriv, hvad denne fortæller om væksten for $g$. 
\end{enumerate}


\subsection*{Opgave 2 (Med hjælpemidler)}

I en by er der i år 2000 $750\TS 000$ personer. Antallet af personer stiger med 3$\%$ om året.
\begin{enumerate}[label = \roman*)]
	\item Opskriv en model, der beskriver antallet af personer til $t$ år efter år 2000.
	\item Hvor mange mennesker er der i byen efter 10 år?
	\item Hvornår vil der være $1\TS 000\TS 000$ personer i byen?
\end{enumerate}

\subsection*{Opgave 3 (Uden hjælpemidler)}

For en funktion $f$ givet ved 
\begin{align*}
	f(x) = b\cdot a^x
\end{align*}
gælder det, at $f(1) = 7$ og $f(4) = 56$.
\begin{enumerate}[label=\roman*)]
	\item Bestem tallene $a$ og $b$. 
	\item Bestem tallet $f(2)$.
\end{enumerate}

\subsection*{Opgave 4 (Uden hjælpemidler)}

Udregn følgende.
\begin{align*}
	&1) \   \ln(1)  &  &2) \ \log_{10}(100\TS 000)\\
	&3) \ \log_3(81)  &  &4) \ \log_{2}(512) 
\end{align*}

\subsection*{Opgave 5 (Uden hjælpemidler)}

Isolér $x$ i følgende ligninger
\begin{align*}
	&1) \ 2^{3x + 10} = 16 &&2) \ \log_5(4x + 105) = 3 \\
\end{align*}
\subsection*{Opgave 6 (Med hjælpemidler)}







\end{document}
