\begin{center}
	\LARGE
	Uden hjælpemidler
\end{center}
\begin{opgavetekst}{Opgave 1}
	Bestem skæringen mellem linjerne givet ved følgende ligninger ved brug af både substitutionsmetoden og 
	lige store koefficienters metode
\end{opgavetekst}
\begin{delopgave}{}{1}
	\begin{align*}
		2(x + 1) - 4y &= 0,\\
		6x - 2y &= 4.
	\end{align*}	 
\end{delopgave}
\begin{delopgave}{}{2}
	\begin{align*}
		2x + 3y &= 12, \\
		3x + 2y &= 13.
	\end{align*}
\end{delopgave}
\begin{opgavetekst}{Opgave 2}
	Løs følgende lineære ligninger
\end{opgavetekst}
\begin{delopgave}{}{1}
	\begin{align*}
		\frac{15x + 20}{-1 + 2x} = 5
	\end{align*}
\end{delopgave}
\begin{delopgave}{}{2}
	\begin{align*}
		\frac{x-1}{x+2} = \frac{2x-4}{2x+6}
	\end{align*}
\end{delopgave}

\newpage

\begin{opgavetekst}{Opgave 3}
\end{opgavetekst}
\begin{delopgave}{}{1}
	Brug nulreglen til at løse følgende ligninger
	\begin{align*}
		&1) \ (x-2)(2x-6) = 0  & &2) \ 4x^2 - 16x = 0  \\
		&3) \ 4^{x^2+12x} = 1  & &4) \ \log_2(2x^2 - 50x + 32) = 5  \\
	\end{align*}
\end{delopgave}
\begin{opgavetekst}{Opgave 4}
\end{opgavetekst}
\begin{delopgave}{}{1}
	Bestem antallet af løsninger for følgende andengradsligninger. 
	\begin{align*}
		&1) \ x^2 + 5x - 10 = 0 & &2) \ 3x^2 - 6x + 3 = 0 \\
		&3) \ -10x^2 +2x - 1 & &4) \ 5x^2 = -7x + 11 \\
	\end{align*}
\end{delopgave}
\begin{opgavetekst}{Opgave 5}
\end{opgavetekst}
\begin{delopgave}{}{1}
	Bestem løsningerne til følgende andengradsligninger ved brug af diskriminantformlen.
	\begin{align*}
		&1) \ x^2 + 2x - 8 = 0 & &2) \ 2x^2-4x-30 = 0 \\
		&3) \ x^2 + 10x  = -24 & &4) \ (x-2)^2+ 8 = 24 \\
	\end{align*}
\end{delopgave}
