\documentclass[12pt,x11names,a4paper]{article}
\input{preamble}


\newgeometry{margin=2cm}

\pagestyle{fancy}
\fancyhf{}

\rhead{Nørre Gymnasium\\2.m
}
\cfoot{Side \thepage \hspace{1pt} af \pageref{LastPage}}

%Husk at rette modul og dato!
\lhead{Aflevering 2\\ Matematik B
}
\chead{20. september
}

\begin{document}

%\includepdf[pages=-]{Forsider/aarsprove_1v.pdf}
\savegeometry{art}

\begin{titlepage}
\newgeometry{margin=0pt}

\begin{minipage}{0.27\textwidth}

\begin{tikzpicture}[overlay]
\fill[top color = NorregGroen!40, bottom color = NorregGroen] (6,10) rectangle (-10,-30);
\end{tikzpicture}
\end{minipage}
\begin{minipage}{0.73\textwidth}
\begin{center}
\phantom{h} \vspace{1cm}\\
\hspace{4cm}
\includegraphics[scale = 1]{Billeder/Norreg.png} \\
\phantom{h} \vspace{5cm}\\
\rule{0.7\textwidth}{0.3mm}\\
\phantom{h}\\
{\fontsize{50}{60}\selectfont Aflevering 2}\\
\phantom{h}\\
\rule{0.7\textwidth}{0.3mm}\\
\Large 2024\\
\Large 2.m Ma

\end{center}
\end{minipage}
\end{titlepage}
\loadgeometry{art}

%Udfyld afsnit herunder og lav til egen Latex-fil

%Kopier følgende til overskrift:

%\begin{center}
%\Huge
%Aflevering 1
%\end{center}
%\section*{Opgave 1}
%\stepcounter{section}
\begin{center}
%Opgavesætter er delt i to dele:\\
%Delprøve 1 kun med den centralt udmeldte formelsamling.\\
%Delprøve 2 med alle hjælpemidler.
\end{center}

\section*{Krav til formidling af din besvarelse}

Ved bedømmelse af helhedsindtrykket af besvarelsen af de enkelte opgaver lægges særlig vægt på følgende fire punkter:
\begin{itemize}
\item[$\cdot$] \textbf{Redegørelse og dokumentation for metode} \\
Besvarelsen skal indeholde en redegørelse for den anvendte løsningsstragegi med dokumentation i form af et passende antal mellemregninger \textit{eller} matematiske forklaringer på metoden, når et matematisk værktøjsprogram anvendes.
\item[$\cdot$] \textbf{Figurer, grafer og andre illustrationer} \\
Besvarelsen skal indeholde hensigtsmæssig brug af figurer, grafer og andre illustrationer, og der skal være tydelige henvisninger til brug af disse i den forklarende tekst.
\item[$\cdot$] \textbf{Notation og layout}\\
Besvarelsen skal i overensstemmelse med god matematisk skik opstilles med hensigtsmæssig brug af symbolsprog, og med en redegørelse for den matematiske notation, der indføres og anvendes, og som ikke kan henføres stil standardviden.
\item[$\cdot$] \textbf{Formidling og forklaring}\\
Besvarelsen af rene matematikopgaver skal indeholde en angivelse af givne oplysninger og korte forklaringer knyttet til den anvendte løsningsstrategi beskrevet med brug af almindelig matematisk notation. 

Besvarelsen af opgaver, der omhandler matematiske modeller, skal indeholde en kort præsentation af modellens kontekst, herunder betydning af modellens parametre. De enkelte delspørgsmål skal afsluttes med en præcis konklusion præsenteret i et klart sprog i relation til konteksten.
\end{itemize}

\newpage

\begin{center}
\LARGE
Delprøve uden hjælpemidler 
\end{center}
\stepcounter{section}

\begin{opgavetekst}{Opgave 1}
	En funktion $f$ er givet ved 
	\begin{align*}
		f(x) = 3x^2+2x+1
	\end{align*}
\end{opgavetekst}
	\begin{delopgave}{}{1}
		Bestem $f(7)$.
	\end{delopgave}
	\begin{delopgave}{}{2}
		Bestem det punkt, hvor grafen for $f$ skærer $y$-aksen.
	\end{delopgave}
\begin{opgavetekst}{Opgave 2}
	Punkterne $A(0,1)$, $B(-4,2)$ og  $C(5,3)$ er givet. 
\end{opgavetekst}
	\begin{delopgave}{}{1}
		Bestem vektoren $\vv{BC}$.
	\end{delopgave}
	\begin{delopgave}{}{2}
		Bestem en parameterfremstilling for linjen $l$, der går gennem $A$ og $B$.
	\end{delopgave}
	\begin{delopgave}{}{3}
		Bestem en ligning for linjen $m$, der går gennem $A$ og $C$. 
	\end{delopgave}
	\begin{delopgave}{}{4}
		Bestem skæringspunktet mellem $l$ og $m$. 
	\end{delopgave}
\begin{opgavetekst}{Opgave 3}
	En linje $l$ er givet ved ligningen 
	\begin{align*}
		2(x-1) + 3(y+4) = 0
	\end{align*}
\end{opgavetekst}
\begin{delopgave}{}{1}
	Bestem en ligning for linjen $m$, der er orthogonal med $l$ og som skærer gennem $P(-2,-4)$.
\end{delopgave}
\begin{opgavetekst}{Opgave 4}
	En ligning er givet ved
	\begin{align*}
		2(s+K) = 24s
	\end{align*}
\end{opgavetekst}
\begin{delopgave}{}{1}
	Isolér $K$ i ligningen. 
\end{delopgave}
\begin{opgavetekst}{Opgave 5}
	To punkter $P(1,2)$ og $Q(3,32)$ er givet. Funktionen $f$ givet ved
	\begin{align*}
		f(x) = b\cdot a^x
	\end{align*}
	skærer gennem $P$ og $Q$.
\end{opgavetekst}
\begin{delopgave}{}{1}
	Bestem konstanterne $a$ og $b$
\end{delopgave}
\begin{delopgave}{}{2}
	Udregn funktionsværdien $f(2)$. 
\end{delopgave}

\newpage
\begin{center}
\LARGE
Delprøve med hjælpemidler 
\end{center}
\stepcounter{section}

\begin{opgavetekst}{Opgave 6}
	I en bakteriekoloni kan bakterieantallet $B$ i de første 24 timer beskrives ved en eksponentiel sammenhæng. Et datasæt hvori bakterieantallet $B$ til tiden $t$ er givet i 
	Tab. \ref{tab:bakterie}.
	\begin{table}[H]
		\centering
		\begin{tabular}{c|c|c|c|c|c|c|c}
		$t$ (timer) &1 & 2 & 3 & 4 & 5 & 6 & 7 \\
		\hline
		$B$ (bakterier i mio.) & 19.1 & 22.6 & 29.1 & 32.9 & 44.0 & 50.4 & 65.1
		\end{tabular}
		\caption{Antallet af bakterier $(B)$ i mio. som funktion af tiden $(t)$ i timer. }
		\label{tab:bakterie}
	\end{table}\phantom{h}
\end{opgavetekst}
\begin{delopgave}{}{1}
	Brug datasættet fra Tab. \ref{tab:bakterie} til at bestemme en forskrift for $B$.
\end{delopgave}
\begin{delopgave}{}{2}
	Bestem, hvor mange bakterier, der er efter 10 timer. 
\end{delopgave}
\begin{delopgave}{}{3}
	Afgør, hvornår antallet af bakterier overstiger 1 mia.
\end{delopgave}
\begin{opgavetekst}{Opgave 7}
	To linjer $l$ og $m$ er givet ved følgende ligning og parameterfremstilling henholdsvist:
	\begin{align*}
		&l: \ 2(x-5)-6(y+7) = 0 \\
		&m: \ 
		\begin{pmatrix}
			x \\ y
		\end{pmatrix}=
		\begin{pmatrix}
			1 \\ 1
		\end{pmatrix} + t
		\begin{pmatrix}
			-2 \\ -3
		\end{pmatrix}
	\end{align*}
\end{opgavetekst}
\begin{delopgave}{}{1}
	Bestem en retningsvektor for $l$ og $m$ og bestem derefter vinklen mellem linjerne. 
\end{delopgave}
\begin{delopgave}{}{2}
	Projicér retningsvektoren for $l$ ned på retningsvektoren for $m$.  
\end{delopgave}
\begin{delopgave}{}{3}
	Bestem skæringspunktet mellem $l$ og $m$.
\end{delopgave}
\begin{opgavetekst}{Opgave 9}
	Et polynomium $f$ er givet ved
	\begin{align*}
		f(x) = x^5+3x^4+x^2-4.
	\end{align*}
\end{opgavetekst}
\begin{delopgave}{}{1}
	Bestem graden for $f$ og brug denne til at bestemme det maksimale antal rødder for $f$. 
\end{delopgave}
\begin{delopgave}{}{2}
	Tegn grafen for $f$ på intervallet $[-4,2]$.
\end{delopgave}
\begin{delopgave}{}{3}
	Bestem rødderne for $f$ på intervallet $[-4,2]$.
\end{delopgave}

\end{document}




