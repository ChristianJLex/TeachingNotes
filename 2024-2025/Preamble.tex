%Preamble

\usepackage{comment}
\usepackage[utf8]{inputenc}
\usepackage{amsmath}
\usepackage{amssymb}
\usepackage{amsthm}
\usepackage{enumitem}
\usepackage{mathrsfs}
\usepackage{mathtools}
\usepackage{esvect}
\usepackage{xfp}
\usepackage{xparse}
\usepackage{array}
\usepackage{changepage}
\usepackage{pdfpages}
\usepackage[parfill]{parskip}
\usepackage{anyfontsize}
\usepackage{alphalph}

\usepackage{algorithm}
\usepackage{algorithmic}
\usepackage{matlab-prettifier}
\usepackage{textcomp}
\usepackage{listings}
\usepackage{xcolor}
\usepackage{colortbl}
\usepackage{eso-pic}
\usepackage[margin=1.35in]{geometry}
\usepackage{float}
\usepackage[danish]{babel}
\usepackage{tikz}
\usetikzlibrary{matrix,calc}
\usetikzlibrary{arrows}
\usetikzlibrary{arrows.meta}
\usetikzlibrary{decorations.pathreplacing}
\usetikzlibrary{decorations.pathreplacing,calligraphy}
\usetikzlibrary{patterns}
\usetikzlibrary{angles, quotes}
\usepackage{csquotes}
\usepackage{pdfpages}
\usepackage{fancyhdr}
\usepackage{lastpage}
\usepackage{pgfplots}
\usepackage{pgffor}

%Ændre følgende, hvis du vil have en anden bredde på pgfplots
\pgfplotsset{width=10cm,compat = 1.9}




\usepackage{caption}
\usepackage{subcaption}
\makeatletter
% \def\@seccntformat#1{%
%   \expandafter\ifx\csname c@#1\endcsname\c@section\else
%   \csname the#1\endcsname\quad
%   \fi}
 \makeatother
\counterwithin*{equation}{section}
\numberwithin{equation}{section}



\newtheorem{setn}{Sætning}[section]
\newtheorem{prop}[setn]{Proposition}
\newtheorem{lem}[setn]{Lemma}
\newtheorem{cor}[setn]{Korollar}

\newcommand\Tr{\textnormal{Tr}}
\newcommand\N{\textnormal{N}}
\newcommand{\suco}[2]{\left.#1\right|_{\mathbb{F}_{#2}}}
\renewcommand\qedsymbol{$\blacksquare$}
\theoremstyle{definition}
\newtheorem{defn}[setn]{Definition}
\newtheorem{exa}[setn]{Eksempel}
\newtheorem{regel}[setn]{Regel}
\newcommand\Ker{\textnormal{Ker}\hspace{0.5mm}}
\renewcommand\Im{\textnormal{Im}\hspace{0.5mm}}


\usepackage{graphicx}
\usepackage[backend=biber]{biblatex}
\addbibresource{biblio.bib}

\renewcommand{\mod}[1]{\ (\textnormal{mod }{#1})}
\usepackage{hyperref, bookmark}
\usepackage{booktabs}
\usepackage{calc}
\usepackage[nottoc,numbib]{tocbibind} 
\usepackage{multirow,bigdelim}

\renewcommand{\arraystretch}{1.3}
\newcommand\intd{\textnormal{d}}
\newcommand\e{\textnormal{e}}




\newenvironment{opgavetekst}[1]
{
	\begin{minipage}[t]{0.17\textwidth}
	 	\textbf{ #1}
	\end{minipage}
	\begin{minipage}[t]{0.85\textwidth}
		\rule{\textwidth}{0.3mm} \\
}
{ 
\\	\end{minipage}
}

\newenvironment{meretekst}
{
	\begin{minipage}[t]{0.15\textwidth}
	 	\phantom{h}
	\end{minipage}
	\begin{minipage}[t]{0.85\textwidth}
}
{ 
\\	\end{minipage}
}

\newenvironment{delopgave}[2]
{
	\begin{minipage}[t]{0.15\textwidth}
		\hspace{0.3cm} #1
	\end{minipage}
	\begin{minipage}[t]{0.85\textwidth}
	\begin{minipage}[t]{0.3cm}
		\symbol{\numexpr96+ #2})
	\end{minipage}
	\begin{minipage}[t]{\textwidth -0.3cm} \ 
}{
\end{minipage}
\phantom{h}\\
\end{minipage}
}


%\newcommand{opgavelinje}[1]{
%\begin{minipage}[t]{0.15\textwidth}
%\textbf{#1}
%\end{minipage} 
%\begin{minipage}[t]{\textwidth}
%\rule{0.81\textwidth}{0.3mm}
%\end{minipage}
%}




%Farver!
\definecolor{NorregGroen}{RGB}{0,105,78}


