\begin{center}
\Huge
Forberedelse til prøve
\end{center}

\section*{Mængder}

\subsubsection*{Uden hjælpemidler}

\subsection*{Niveau 1}

Bestem følgende mængder.
\begin{align*}
	&a) \  \{1,2,3,4\} \cup \{2,4,6,9\}     &&b)  \  \{2,5,9,a,b\} \cup \{9,b,3,5\}    \\
	&c) \  \{99,100\} \backslash \{97,98,99,100,101\}     &&d)  \  \{9,8,7,4\} \times \{p,q\}    \\	
\end{align*}

\subsection*{Niveau 2}
\begin{enumerate}[label=\roman*)]
	\item Bestem følgende mængder
	\begin{align*}
		&a) \  (\{\textnormal{banan},\textnormal{pære}\} \cup \{6,10\}) \cup (\{6\} \cap \{\textnormal{citron},11\})   \\ 
		&b)  \  \{(11,12),(11,13),(17,13),(10,10)\} \cap (\{10,11\} \times \{12,13\})    \\
	\end{align*}
	
	\item For mængden $A = \{2,5,6,9\}$ afgør da hvilken af følgende mængder, der ikke er lig $A$. (Der er kun én).
	\begin{align*}
		&B = \{1,2,\hdots,10\} \setminus \{1,3,4,7,8,10\}\\
		&C = \{2,5,6,9\} \cup \emptyset \\
		&D = \{x \in \mathbb{Z} \mid 0 < x < 10 \textnormal{ og 2 eller 3 går op i }x \} \\
		&E = (\{2,5\} \cup \{6,7,9\}) \backslash \{7,10,12,13\} 
	\end{align*}	 
	\item Opskriv elementerne i mængden $\{x \in \mathbb{Z} \mid x^2 < 20 \}$.
	\item Opskriv elementerne i mængden $\{x \in \mathbb{Z} \mid 0 < x \leq 7 \} $.
\end{enumerate}

\newpage
\subsection*{Niveau 3}
\begin{enumerate}[label=\roman*)]
	\item For mængden $A = \{2,4,6,8\}$ afgør da hvilke af følgende mængder, der er lig $A$. (Der kan være flere).
	\begin{align*}
		&B = \{2,2,2,4,6,6,6,8,8\} \\
		&C = \{x \in \mathbb{Z} \mid 0 < x < 10 \textnormal{ og 2 går ikke op i }x \}\\
		&D = \{x \in \mathbb{Z} \mid x \textnormal{ er et positivt lige tal mindre end 10}\} \\
		&E= \{x \in \mathbb{Z} \mid 0 < x < 10 \} \backslash \{x \mid x \textnormal{ er et ulige tal}\}
	\end{align*}	
	\item For mængden $A = \{1,2,4,6,7,9,10,11\}$ Opskriv da følgende mængder
	\begin{align*}
		&1) \ \{(a,b) \in A^2 \mid a/b \in \mathbb{Z}\}  & &2) \   \{(a,b) \in A^2 \mid a + b \textnormal{ er et lige tal}\} \\
		&3) \ \{(a,b) \in A^2 \mid a/b \in \mathbb{Z}\}  & &4) \   \{(a,b) \in A^2 \mid a + b \textnormal{ er et lige tal}\}
	\end{align*}
	\item For hvert af følgende udsagn afgør da, om de er sande eller falske 
	\begin{align*}
		&a) \ x \in \emptyset   &&b) \ \{x\} \in \{x\} \\
		&c) \ x \in \{x\}      &&d) \  \{x\} \subseteq \{x\} \\
		&c) \ \emptyset \in \{x\}      &&d) \  \emptyset \subseteq \{x\} \\
		&c) \ \{x\} \cup \emptyset = \{x\}      &&d) \  \{x\} \in \{\{x\}\} \\
	\end{align*}
\end{enumerate}

\newpage

\section*{Kombinatorik}

\subsubsection*{Uden hjælpemidler}

\subsection*{Niveau 1}

\begin{enumerate}[label=\roman*)]
	\item På hvor mange måder kan vi arrangere en følge af elementerne i mængden $\{a,b,c,d,e\}$, hvis rækkefølgen har betydning?
	\item På hvor mange måder kan vi konstruere en delmængde på to elementer af mængden $\{a,b,c,d,e\}$?
	\item På hvor mange forskellige måder kan du vælge en is, hvis du skal have tre forskellige kugler, og der er 6 forskellige smage?
	\item Bestem tallet $\textnormal{K}(6,2)$.
	\item Bestem tallet $\textnormal{K}(200,1)$.
	\item I en klasse på 28 personer, på hvor mange måder kan man så lave netværksgrupper på 4 personer? (Du behøver ikke udregne svaret)
	\item Hvis der i hver netværksgruppe skal være en ordstyrer, referant, formand og kagebager, på hvor mange måder kan man så lave netværksgrupper? (Du behøver ikke udregne svaret)
\end{enumerate}

\subsection*{Niveau 2}

\subsubsection*{Med hjælpemidler}

\subsubsection*{Opgave 1}
Du er på restaurant og skal vælge en menu blandt 14 forretter, 7 hovedretter og 4 desserter.
\begin{enumerate}[label=\roman*)]
	\item På hvor mange måder kan du vælge en forret, en hovedret og en dessert, hvis du skal have alle tre dele?
	\item Hvis du kun har råd til enten en hovedret \textit{eller} en dessert, på hvor mange måder kan du så sammensætte dit måltid?
	\item Hvor mange muligheder har du, hvis du kun har råd til én ret. 
\end{enumerate}

\subsubsection*{Opgave 2}
En mand har i sit klædeskab 7 skjorter, 5 par bukser og 3 jakker.
\begin{enumerate}[label=\roman*)]
	\item Han har en meget lille kuffert og kan kun vælge én beklædningsgenstand. På hvor mange måder kan dette gøres?
	\item Han beslutter sig for at købe en større kuffert og vil tage 3 skjorter, 2 par bukser og én jakke med på ferie. På hvor mange måder kan dette gøres?
	\item Da han skal afsted har han så travlt, at han ikke tjekker hvilke beklædningsgenstande han pakker. Han pakker 6 genstande i alt. På hvor mange måder kan dette gøres?
\end{enumerate}

\subsubsection*{Opgave 3}
Fire svenskere ved navn Agneta, Annifrid, Benny og Björn lavede i 1972 et band. De brugte deres initialer til at lave deres bandnavn.
\begin{enumerate}[label=\roman*)]
	\item På hvor mange forskellige måder kunne de vælge deres bandnavn?
	\item Lad os sige, at Björn var vild med surstrømning, og at bandet derfor besluttede sig for at erstatte ham med en fyr ved navn Henning. På hvor mange måder kan de så lave et bandnavn?
\end{enumerate}

\newpage

\section*{Sandsynlighedsregning}

\subsubsection*{Uden hjælpemidler}

\subsection*{Niveau 1}

\subsection*{Opgave 1}

En stokastisk variabel $X$ har følgende fordeling.
\begin{table}[H]
	\centering
	\begin{tabular}{c|c|c|c|c|c|c}
		$x$ & 1 & 4 & 5 & 7 & 11 & 13 \\
		\hline
		$P(X = x)$ & 0.1 & 0.5 & $P(X = 5)$ & 0.05 & 0.1 & 0.2
	\end{tabular}
\end{table}

\begin{enumerate}[label=\roman*)]
	\item Bestem $P(X = 1)$.
	\item Bestem $P(X = 5)$.
	\item Bestem $P(X > 8)$.
\end{enumerate}

\subsection*{Opgave 2}

Opskriv udfaldsrummene for følgende stokastiske eksperimenter.
\begin{align*}
	&a) \ \textnormal{slag med 6-sidet terning}     &&b) \ \textnormal{To kast med mønt}     \\
	&c) \ \textnormal{Nedbørstype i morgen}   &&d) \ \textnormal{Fraværende elever i 2.x}     \\
	&e) \ \textnormal{Udfald af prøve}    &&f) \ \textnormal{Første solskinsdag i 2024}       \\
\end{align*}


\subsection*{Niveau 2}
\subsubsection*{Med hjælpemidler}

\subsection*{Opgave 1}
En terning kastes 10 gange.
\begin{enumerate}[label=\roman*)]
	\item Hvad er sandsynligheden for at få netop 4 seksere?
	\item Hvad er sandsynligheden for at få 4 eller færre seksere?
	\item Hvad er sandsynligheden for at få mere end 7 seksere?
\end{enumerate}

\subsection*{Opgave 2}

Et lægemiddel helbreder 70 procent af de patienter, det anvendes på. Det anvendes på 200 personer
\begin{enumerate}[label=\roman*)]
	\item Hvad er sandsynligheden for at netop 140 personer helbredes?
	\item Hvad er sandsynligheden for at færre end 100 helbredes?
	\item Hvad er sandsynligheden for at mellem 120 og 160 helbredes?
\end{enumerate}

\subsection*{Opgave 3}
På en skole er der 900 elever, hvoraf 700 er piger. I 2.b er 23 ud af 28 drenge.
\begin{enumerate}[label=\roman*)]
	\item Bestem sandsynligheden for, at en tilfældigt udvalgt person er en dreng givet at personen går i 2.b
	\item Bestem sandsynligheden for, at en tilfældigt udvalgt person går i 2.b givet at personen er en dreng. 
\end{enumerate}

\subsection*{Niveau 3}

\begin{enumerate}[label=\roman*)]
	\item En person har to børn, hvoraf én er en dreng født på en tirsdag. Hvad er sandsynligheden for, at det andet barn er en dreng? (Det er ikke 0.5).
\end{enumerate}


\newpage

\section*{Middelværdi og spredning}

\subsubsection*{Med hjælpemidler}
\subsection*{Niveau 2}

\begin{enumerate}[label=\roman*)]
	\item En mønt lander på krone 70 procent af tiden og plat ellers. Lad $X = 1$, hvis resultatet er krone og $X = 0 $ ellers. Bestem middelværdien og spredningen for denne stokastiske variabel. 
	\item Kast en 20-sidet terning og lad $X$ betegne antallet af øjne. Hvad er middelværdien og spredningen for $X$?
\end{enumerate}

En stokastisk variabel $X$ har følgende fordeling.
\begin{table}[H]
	\centering
	\begin{tabular}{c|c|c|c|c|c|c}
		$x$ & 1 & 4 & 5 & 7 & 11 & 13 \\
		\hline
		$P(X = x)$ & 0.1 & 0.5 & 0.05 & 0.05 & 0.1 & 0.2
	\end{tabular}
\end{table}

\begin{enumerate}[label=\roman*)]
	\item Bestem middelværdien for $X$.
	\item Bestem spredningen for $X$.
\end{enumerate}

\subsection*{Niveau 3}

En stokastisk variabel $X$ har følgende fordeling
\begin{table}[H]
	\centering
	\begin{tabular}{c|c|c|c|c|c|c}
		$x$ & -2 & 3 & $x_3$ & 6 & 9 & 11 \\
		\hline
		$P(X = x)$ & 0.2  & 0.2 & 0.1 & 0.2 & 0.15 & 0.15
	\end{tabular}
\end{table}

\begin{enumerate}[label=\roman*)]
	\item Bestem $x_3$, så middelværdien for $X$ bliver 0.
	\item Bestem $x_3$, så spredningen for $X$ bliver så lille som mulig. 
\end{enumerate}

\newpage

\section*{Binomialtest og konfidensinterval}

\section*{Niveau 2}

\subsection*{Opgave 1}
En sjusket producent af chokoladejulekalendre sælger en julekalender med lys chokolade og en julekalender med mørk chokolade. Han holder ikke styr på hvor meget han sælger af hver, og han ved derfor
ikke, hvor meget han skal købe af forskellige ingredienser. Sidste år var andelen af solgte chokoladekalendre med lys chokolade 71.2$\%$. 

Han laver en stikprøve af solgte chokoladekalendre og opdager, at 351 ud af 522 chokoladekalendre er med lys chokolade.
\begin{enumerate}[label=\roman*)]
	\item Opstil en nulhypotese, der kan bruges til at afgøre, om andelen af chokoladekalende med lys chokolade har ændret sig
	\item Opstil en binomialmodel, der beskriver andelet af kalendre med lys chokolade givet at nulhypotesen er sand og brug denne model til at bestemme $p$-værdien for andelen af chokoladekalendre med lys chokolade. Skal vi forkaste nulhypotesen med et signifikansniveau på 5$\%$?
	\item Bestem den kritiske mængde for eksperimentet.
\end{enumerate} 

\subsection*{Opgave 2}
En butik har et klientel, der består af $30\%$ mænd. 
\begin{enumerate}[label=\roman*)]
\item Opstil en binomialmodel, der beskriver antallet af mænd i en stikprøve på $50$ personer.
\item Hvad er sandsynligheden for, at 12 af personerne i stikprøven er mænd?
\item En dag havde butikken $220$ kunder. 79 af disse var mænd. Opstil en nulhypotese, der beskriver undersøgelsen, og brug en binomialtest med et signifikansniveau på $5\%$ til at undersøge nulhypotesen. 
\end{enumerate}


\subsection*{Opgave 3}
Laver vi krydsninger med en bestemt type hvide og violette blomster siger Mendels love, at $25\%$ af blomsterne vil være hvide og $75\%$ af blomsterne vil være violette. I et krydsningsforsøg har vi 705 violette blomster og 224 hvide blomster. 
\begin{enumerate}[label=\roman*)]
\item Opstil en nulhypotese, der kan benyttes til at afgøre, om blomsternes farve følger Mendels love
\item Bestem de forventede antal violette og hvide blomster, hvis nulhypotesen er sand
\item Benyt binomialtest for at afgøre, om vi skal forkaste nulhypotesen. Signifikansniveauet er som altid $5\%$.
\item Hvad er den kritiske mængde for dette eksperiment?
\end{enumerate}

\subsection*{Opgave 4}
I en skov er fordelingen af løvtræer og nåletræer fordelt omtrent ligeligt. En gruppe biologistuderende antager, at spættearten \textit{den store flagspætte} er ligeglad med, om den laver bo i et udgået løvtræ eller et udgået nåletræ. I skoven finder de 201 spætteboer i nåletræer og 167 i løvtræer. 
\begin{enumerate}[label=\roman*)]
\item Bestem en nulhypotese og opstil en binomialmodel, som beskriver antallet af flagspætter, der har bo i et løvtræ.
\item Lav binomialtest og bestem den kritiske mængde for antallet af bo i løvtræer.
\item Brug binomialtesten til at afgøre, om vi accepterer eller forkaster nulhypotesen med et signifikansniveau på $5\%$.

\end{enumerate}

\section*{Niveau 3}

\subsection*{Opgave 5}
Meyer er et terningespil, hvor hver spiller skiftes til at slå med to terninger. Det bedste slag er Meyer, der er en toer og en etter. Du spiller Meyer med en gruppe af dine venner, og du synes, at en af dem får Meyer lidt for ofte. Du tæller, at han ud af 93 slag har fået Meyer 11 gange.
\begin{enumerate}[label=\roman*)]
\item Hvad er sandsynligheden for at slå Meyer med to terninger?
\item Hvilken fordeling vil du forvente, at antallet af Meyere i 93 slag vil følge? Hvad er parametrene for fordelingen?
\item Opstil en nulhypotese, der undersøger din formodning.
\item Undersøg din nulhypotese med en passende test. 
\item Du spiller med 10 venner. Kan du med et signifikansniveau på $5\%$ sige, at ingen af disse bør få 11 eller flere gange Meyer?
\end{enumerate}

\subsection*{Opgave 6}
En medicinalvirksomhed med dårlig moral tester et nyt præparat op mod placebo. De giver $500$ personer placebo, og det viser sig, at $15\%$ af disse personer opnår fremgang i henseende A. De tester nu deres nye præparat på en tilsvarende gruppe af $500$ personer, og oplever, at $90$ personer opnår fremgang i henseende $A$. 
\begin{enumerate}[label=\roman*)]
\item Hvis præparatet hverken er bedre eller værre end placebo, hvor mange forventer vi så at helbrede?
\item Under nulhypotesen, at præparatet ikke er bedre end placebo, test så med binomialtest om denne nulhypotese kan forkastes med et signifikansniveau på $5\%$.
\end{enumerate}
I virksomheden har de glemt at slette deres indbyrdes sms'er, og det kommer frem, at virksomheden udover at teste for henseende A også har testet for forbedring inden for 26 andre henseender uden at kunne se en effekt. Det kaldes en type I fejl at forkaste en sand nulhypotese, og med et signifikansniveau på $5\%$ er sandsynligheden for at lave en type I fejl generelt mindre end $5\%$. 
\begin{enumerate}[label=\roman*)]
\setcounter{enumi}{2}
\item Hvad er sandsynligheden for at lave en type I fejl, hvis man tester for forbedring i 27 forskellige henseender op mod placebo hver med et signifikansniveau på $5\%$?

\end{enumerate}