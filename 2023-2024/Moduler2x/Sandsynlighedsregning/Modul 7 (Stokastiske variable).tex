\begin{center}
\Huge
Stokastiske variable
\end{center}
\section*{Stokastiske variable}
\stepcounter{section}

Vi har tidligere set på sandsynligheder i stil med 
\begin{align*}
P(\{\textit{En mønt lander på krone}\}),
\end{align*}
men det ville være smart, hvis vi i udgangspunktet havde kvantificeret alle udfaldene, så vi i stedet for at opskrive hændelser, så blot kunne opskrive et tal. Til dette vil vi definere begrebet stokastisk variabel.
\begin{defn}
	En stokastisk variabel er en funktion $X$, der til hvert element $u$ i et udfaldsrum $U$ afbilder $u$ over i et tal.  
\end{defn}
En stokastisk variabel vil typisk betegnes med $X$.
\begin{exa}
Vi kaster en mønt to gange og lader $X$ betegne antallet af gange, vi slår plat. Værdimængden (de værdier $X$ kan antage) er $\{0,1,2\}$ og vi har følgende sandsynligheder for udfaldene af $X$:
\begin{align*}
P(X=0) = \frac{1}{4}, \ P(X=1) = \frac{1}{4}, \ P(X=2) = \frac{1}{2}. 
\end{align*}
Som vi tidligere har nævnt, så kaldes disse sandsynligheder for \textit{fordelingen} af $X$.
\end{exa}
\begin{exa}
Vi lader $X_m$ være den totale levetid for en amerikansk mand og $X_k$ er levetiden for en amerikansk kvinde. Så gælder der, at værdimængderne for de to variable er $]0,\infty[$ og
\begin{align*}
P(X_m\leq 30) \approx 0.16
\end{align*} 
samt
\begin{align*}
P(X_k\leq 30) \approx 0.11.
\end{align*}
\end{exa}
\begin{exa}
Vi slår med en terning og lader $X$ være antallet af øjne på terningen. Fordelingen af $X$ vil så være givet ved
\[P(X=1) = P(X=2) = \cdots = P(X=6) = \frac{1}{6}.\]
\end{exa}

En stokastisk variabel vi allerede har stiftet bekendtskab med kaldes for \textit{den Bernoullifordelte stokastiske variabel}.
\begin{defn}[Bernoullifordelt stokastisk variabel]
En stokastisk variabel $X$ kaldes en Bernoullifordelt stokastisk variabel, hvis den har to udfald (ofte betegnet $0$ og 1). Vi betegner $X=1$ som "succes" og $X=0$ som "fiasko". Sandsynligheden $p= P(X=1)$ kaldes for sandsynlighedsparametren og vi skriver, at $X \sim \textnormal{Ber}(p)$. 
\end{defn}

\begin{exa}
	Vi kaster en mønt og observerer resultatet. Den stokastiske variabel
	\begin{align*}
		X = \begin{cases}
			1 \ &\textnormal{ hvis resultatet er krone}, \\
			0 \ &\textnormal{ ellers.}
		\end{cases}
	\end{align*}
	er en Bernoullifordelt stokastisk variabel med sandsynlighedsparameter $p = 0.5$.
\end{exa}

\subsection*{Opgave 1}
For følgende stokastiske variable bestem da deres værdimængder.
\begin{enumerate}[label=\roman*)]
	\item $X$ er antallet af seksere i 4 slag med en terning.
	\item $X$ er regnvejrsdage i 2024.
	\item $X$ er øjensummen ved 2 slag med en 20-sidet terning.
	\item $X$ er antal lottovindere i 2.x ved næste trækning.
\end{enumerate}

\subsection*{Opgave 2}
\begin{enumerate}[label=\roman*)]
	\item Hvad er sandsynlighedsparameteren for den Bernoullifordelte stokastiske variabel, der beskriver et slag på mere end 2 med en terning.
	\item Hvad er sandsynlighedsparameteren for den Bernoullifordelte stokastiske variabel, der beskriver om du vinder på grøn i roulette?
\end{enumerate}

\subsection*{Opgave 3}

Fordelingen for en stokastisk variabel $X$ er beskrevet ved tabellen
\begin{center}
	\begin{tabular}{c|c|c|c|c|c}
		$x$ & 1 & 2 & 3 & 4 & 5 \\
		\hline
		$P(X = x)$ & 0.1 & $P(X = 2)$ & 0.25 & 0.25 & 0.05 
	\end{tabular}
\end{center}
\begin{enumerate}[label=\roman*)]
	\item Bestem $P(X = 2)$.
	\item Bestem $P(X \leq 3)$.
	\item Bestem $P(X \in \{1,5\})$.
\end{enumerate}

\subsection*{Opgave 4}
I denne opgave skal du selv finde på en stokastisk variabel, der beskriver det stokastiske eksperiment. Du skal altså selv finde på, hvad den stokastiske variabel skal måle. \\

1) Du kaster med en terning, og udfaldsrummet er $U = \{1,2,3,4,5,6\}$. 

\begin{enumerate}[label=\roman*)]
	\item Find selv på en stokastisk variabel, der tildeler et tal til hvert element i $U$.
	\item Bestem værdimængden af din stokastiske variabel.
	\item Bestem fordelingen af din stokastiske variabel. 
\end{enumerate}

2) Du kaster med en mønt tre gange og udfaldsrummet er
\begin{align*}
	U = \{(p,p,p),(p,p,k),(p,k,p),(p,k,k),(k,p,p),(k,p,k),(k,k,p),(k,k,k)\}.
\end{align*}

\begin{enumerate}[label=\roman*)]
	\item Find selv på en stokastisk variabel, der tildeler et tal til hvert element i $U$.
	\item Bestem værdimængden af din stokastiske variabel.
	\item Bestem fordelingen af din stokastiske variabel. 
\end{enumerate}
