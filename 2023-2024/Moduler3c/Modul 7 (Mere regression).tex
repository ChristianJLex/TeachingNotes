
\subsection*{Opgave 1}
I \href{https://github.com/ChristianJLex/TeachingNotes/raw/master/2023-2024/Data og lign/Loendata.xlsx}{\color{blue!60} dette datasæt} fremgår løndata for 200 medarbejdere hos en virksomhed. Det antages, at sammenhængen mellem løn og anciennitet er lineær.

\begin{enumerate}[label=\roman*)]
	\item Lav lineær regression på datasættet
	\item Brug residualerne for den lineære regression til at afgøre, om sammenhængen mellem anciennitet og løn er 
	lineær
	\item Bestem den forventede løn efter 10 år.
	\item Bestem sandsynligheden for, at lønnen efter 10 år er mellem 39000 og 41000 månedligt.
	\item \textbf{Svær:} Bestem et $95\%$-konfidensinterval for lønnen efter 5 år.
\end{enumerate}  

\subsection*{Opgave 2}

En gruppe af tomatentutiaster har en antagelse om, at en bestemt sort af tomatplanter bliver ved med at vokse i august. De har derfor målt vægten af deres tomatplanter i løbet af august måned. Vægten af planterne fremgår af \href{https://github.com/ChristianJLex/TeachingNotes/raw/master/2023-2024/Data og lign/TomatVaegt.xlsx}{\color{blue!60} dette datasæt.}

\begin{enumerate}[label=\roman*)]
	\item Lav lineær regression på datasættet.
	\item Brug residualerne for regression til at afgøre, om sammenhængen mellem forløbne dage og vægt er lineær
	\item Bestem et $95\%$-konfidensinterval for hældningen $a$, og brug dette til at afgøre, om vi kan konkludere, at 
	planterne stadig vokser.
\end{enumerate}

\subsection*{Opgave 3}

En virksomhed producerer slikposer i størrelserne 50, 100, 150, 200 og 250 gram. De har målt variationen på posens forventede vægt og den reelle vægt. Resultatet af deres undersøgelse fremgår af \href{https://github.com/ChristianJLex/TeachingNotes/raw/master/2023-2024/Data og lign/VariationSlikpose.xlsx}{\color{blue!60} dette regneark}.


\begin{enumerate}[label = \roman*) ]
	\item Lav lineær regression på datasættet.
	\item Brug residualerne for regressionen til at afgøre, om sammenhængen mellem den forventede vægt og variationen 
	fra den forventede vægt er lineær.
	\item Bestem et $95\%$-konfidensinterval for hældningen af den lineære sammenhæng.
	\item Tegn tre lineære funktioner i samme koordinatsystem; én for hver af $a$-værdierne:
	\begin{itemize}
		\item[$\cdot$] Den nedre grænse for konfidensintervallet,
		\item[$\cdot$] $a$-værdien, som regressionen giver,
		\item[$\cdot$] Den øvre grænse for konfidensintervallet.
	\end{itemize}
	\item \textbf{Svær:} Bestem sandsynligheden for, at en pose, der bør veje 200 gram har en vægt på mellem 180g og
	 220g
\end{enumerate}
