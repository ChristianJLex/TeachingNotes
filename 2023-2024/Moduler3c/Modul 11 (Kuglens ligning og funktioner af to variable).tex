\begin{center}
	\huge Kuglens ligning
\end{center}

\subsection*{Opgave 1}
\begin{enumerate}[label=\roman*)]
	\item En kugle $K$ har centrum i $(0,0,0)$ og radius $1$. Bestem ligningen for $K$. 
	\item En kugle $K$ har centrum i $(-2,4,8)$ og radius $5$. Bestem ligningen for $K$.
	\item En kugle $K$ har centrum i $(-1,-2,5)$ og radius $3$. Bestem ligningen for $K$.
	\item En kugle $K$ har centrum i $(0.5, -11, 13)$ og radius $\sqrt{7}$. Bestem ligningen for $K$. 
\end{enumerate}

\subsection*{Opgave 2}
\begin{enumerate}[label=\roman*)]
	\item En kugle $K$ har ligningen 
	\begin{align*}
		x^2 + y^2 + z^2 - 4x - 10y + 6z  = -22
	\end{align*}.
	Afgør, om punkterne $(6,5,-3)$ og $(1,3,-4)$ ligger på kuglen.
	\item En kugle $K$ har ligningen 
	\begin{align*}
		x^2 + y^2 + z^2 + 2x - 2y + 4z = 3
	\end{align*}	 
	Afgør, om punkterne $(-3, -1, -3)$ og $(2,-2,0)$ ligger på kuglen.
\end{enumerate}

\subsection*{Opgave 3}
Vektoren 
\begin{align*}
	\begin{pmatrix}
		2 \\ -2 \\ 1
	\end{pmatrix}
\end{align*}
går fra centrum af kuglen $K$ ud til kuglens overflade. Centrum af kuglen er i punktet $(-5,2,-10)$.
\begin{enumerate}[label=\roman*)]
	\item Bestem ligningen for $K$.
	\item Afgør, om punktet $(-5,-1,10)$ ligger på kuglen.
\end{enumerate}
\newpage
\subsection*{Opgave 4}
I følgende opgave skal I kvadratkomplettere. Fremgangsmåden er nøjagtigt som når i kvadratkompletterer cirklens ligning.
\begin{enumerate}[label=\roman*)]
	\item En kugle $K$ har ligningen 
	\begin{align*}
		x^2 -4x+y^2+4y+z^2-8z=1.
	\end{align*}
	Bestem centrum og radius for $K$.
	\item En kugle $K$ har ligningen 
	\begin{align*}
		x^2-6x+y^2-6y+z^2-10z=21.
	\end{align*}
	Bestem centrum og radius for $K$.
\end{enumerate}

\subsection*{Opgave 5}
Et \textit{tangentplan} til en kugle er et plan, der rører kuglen i nétop et punkt. Der tilhører altså et tangentplan til hvert punkt på en kugle.
En kugle $K$ har ligningen 
\begin{align*}
	(x-4)^2 + (y-1)^2 + (z+3) ^2 = 5
\end{align*}
og punktet $(2,1,-2)$ ligger på kuglen.
\begin{enumerate}[label=\roman*)]
	\item Bestem en ligning for tangentplanen til $K$ i punktet $(2,1,-2)$.
\end{enumerate}


\begin{center}
	\huge Funktioner af to variable
\end{center}

\subsection*{Opgave 7}
En funktion $f$ er givet ved
	\begin{align*}
		f(x,y) = x^2+2y.
	\end{align*}
\begin{enumerate}[label=\roman*)]
	\item Bestem tallene $f(2,4)$, $f(-3,3)$ og $f(3,-3)$.
	\item Prøv at gætte dig frem til en løsning til ligningen $f(x,y) = 3$
	\item (Svær) Kan du gennemskue en strategi til at finde heltalsløsninger til ligningen $f(x,y) = 4$?
\end{enumerate}
\subsection*{Opgave 8}
En funktion $f$ er givet ved 
\begin{align*}
	f(x,y) = \sqrt{x} + 3y^2.
\end{align*}
\begin{enumerate}[label=\roman*)]
	\item Hvilke af punkterne $(4,1,5)$, $(1,2,3)$ og $(2,1,\sqrt{2}+3)$ ligger på grafen for $f$?
\end{enumerate}
\subsection*{Opgave 9}
\begin{enumerate}[label=\roman*)]
	\item Et plan $L$ går gennem punktet $(-5,4,2)$ og har vektoren $\vv{n}$ givet ved
	\begin{align*}
		\vv{n} = 
		\begin{pmatrix}
			6 \\ 3 \\ -1
		\end{pmatrix}.
	\end{align*}
	Bestem en funktion $f$, der har $L$ som graf. Brug denne funktion til at afgøre, om punktet $(1,1,1)$ ligger på $L$.
	\item Et plan $L$ går gennem punktet $(2,4,8)$ og har normalvektoren $\vv{n}$ givet ved
	\begin{align*}
		\vv{n} = 
		\begin{pmatrix}
			3 \\ -4 \\ -3
		\end{pmatrix}.
	\end{align*}
	Bestem en funktion $f$, der har $L$ som graf. Brug denne funktion til at afgøre, om punktet $(0,1,4)$ ligger på $L$.
\end{enumerate}

\subsection*{Opgave 10}
\begin{enumerate}[label=\roman*)]
	\item En kugle $K$ med centrum i origo har radius $7$. Bestem en funktion $f$, der har den øvre halvkugle af $K$ som graf for funktionen. Brug denne funktion til at afgøre, 
	om punktet $(0,0,9)$ ligger på kuglen. 
	\item En kugle $K$ med centrum i origo har radius $\sqrt{3}$. Bestem en funktion $f$, der har den nedre halvkugle af $K$ som graf for funktionen. Brug denne funktion til at 
	afgøre, 	om punktet $(-1,1,-\sqrt{3})$ ligger på kuglen. 
\end{enumerate}

\subsection*{Opgave 11}
\begin{enumerate}[label=\roman*)]
	\item Vis, at den øvre halvkugle for en kugle med centrum i $(x_0,y_0,z_0)$ og radius $r$ kan beskrives ved grafen for funktionen $f$ givet ved
	\begin{align*}
		f(x,y) = \sqrt{r^2-x^2-x_0^2+2xx_0-y^2-y_0^2+2yy_0} +z_0
	\end{align*}
	\item Brug dette til at bestemme en funktion, hvis graf er den øvre halvkugle for en kugle med centrum i $(1,4,3)$ og radius $2$.
\end{enumerate}
