
\subsection*{Opgave 1}

En radioaktiv kilde placeres i varierende afstand fra en Geiger-tæller, der måler radioaktivitet. I følge afstandskvadratloven vil antallet af aktiveringer per sekund $A$ kunne beskrives ved sammenhængen
\begin{align*}
	A(x) = \frac{1}{x^2}\cdot k,
\end{align*}
hvor $x$ er afstanden fra kilden i meter, og $k$ er en passende proportionalitetskonstant. Resultatet af forsøget fremgår af \href{https://github.com/ChristianJLex/TeachingNotes/raw/master/2023-2024/Data og lign/GeigerAktiveringer.xlsx}{\color{blue!60} dette datasæt.}

\begin{enumerate}[label=\roman*)]
	\item Lav en passende datatransformation og lav derefter lineær regression på datasættet.
	\item Hvor mange aktiveringer forventer vi, hvis afstanden er 4 meter?
	\item Hvad skal afstanden være, hvis antallet af aktiveringer skal være på 3 per sekund?
\end{enumerate}

\subsection*{Opgave 2}

Et objekt kastes ud fra en højde på 100 meter. Højden $H$ over jorden som funktion af den forløbne tid i sekunder $x$ antages at kunne beskrives ved en sammenhæng af typen
\begin{align*}
	H(x) = -ax^2 + b.
\end{align*}
Højden og tiden kan findes i \href{https://github.com/ChristianJLex/TeachingNotes/raw/master/2023-2024/Data og lign/Fritfald.xlsx}{\color{blue!60} dette datasæt.}
\begin{enumerate}[label=\roman*)]
	\item Lav en passende datatransformation, og lav derefter lineær regression på datasættet
	\item Hvornår rammer objektet jorden?
	\item Hvad er objektets hastighed, når det rammer jorden?
\end{enumerate}


\subsection*{Opgave 3}

I \href{https://github.com/ChristianJLex/TeachingNotes/raw/master/2023-2024/Data%20og%20lign/Bakteriedata.xlsx}{\color{blue!60}dette datasæt} fremgår antallet af bakterier i mia. samt den forløbne tid i timer.
Det antages, at antallet af bakterier $B$ og den forløbne tid $x$ kan beskrives ved en sammenhæng af typen
\begin{align*}
	\ln(B(x)) = ax + b
\end{align*}
\begin{enumerate}[label=\roman*)]
	\item Lav en passende datatransformation, og lav derefter lineær regression på datasættet
	\item Hvor mange bakterier vil der være efter 100 timer?
	\item Brug modellen til at afgøre, hvornår antallet af bakterier vil være på 700 mia.
\end{enumerate}

\subsection*{Opgave 4}

Man har for en bestemt biltype målt den effekt (i hk (hestekræfter)) det kræves at køre ved bestemte hastigheder. Resultatet af undersøgelsen kan ses i \href{https://github.com/ChristianJLex/TeachingNotes/raw/master/2023-2024/Data og lign/hk.xlsx}{\color{blue!60}dette datasæt.}

Det antages, at effekten $E$ som funktion af hastigheden $x$ (i km/t) kan beskrives ved en sammenhæng af typen
\begin{align*}
	\ln(E(x)) = a\ln(x) + b
\end{align*}

\begin{enumerate}[label=\roman*)]
	\item Lav en passende datatransformation og lav derefter lineær regression på datasættet.
	\item Bestem den effekt det kræves for at køre 200km/t
	\item Hvor stærkt kører man, hvis man har en konstant effekt på 600hk?
\end{enumerate}

\subsection*{Opgave 5}

Det viser sig, at man også kan lave lineær regression på eksponentiel data ved en passende variabeltransformation. En eksponentialfunktion er som bekendt en funktion af typen
\begin{align*}
	f(x) = b\cdot a^x
\end{align*}
\begin{enumerate}[label=\roman*)]
	\item Anvend $\ln(x)$ på begge sider af lighedstegnet af eksponentialfunktionen
	\item Brug logaritmeregneregler til at lave en lineær sammenhæng mellem $\ln(f(x))$ og $x$. 
	\item Bestem hældningen og begyndelsesværdien for denne lineære sammenhæng.
	\item Prøv at lave eksponentiel regression på datasættet fra Opgave 3 og tag $\ln(x)$ af begyndelsesværdien og fremskrivningsfaktoren for denne regression. Sammenlign disse tal med den lineære regressions hældning og begyndelsesværdi fra Opgave 3.
\end{enumerate}

\subsection*{Opgave 6}
Det viser sig også, at man kan lave lineær regression på potensdata. Vi husker os selv på, at en potensfunktion er en funktion af typen
\begin{align*}
	g(x) = b\cdot x^a.
\end{align*}

\begin{enumerate}[label=\roman*)]
	\item Anvend $\ln(x)$ på begge sider af lighedstegnes af potensfunktionen
	\item Brug logaritmeregneregler til at lave en lineær sammenhæng mellem $\ln(g(x))$ og $\ln(x)$.
	\item Bestem hældningen og begyndelsesværdien for denne lineære sammenhæng.
	\item Prøv at lave potensregression på datasættet fra Opgave 4 og anvend $\ln(x)$ på $b$-værdien for denne regression. Sammenlign disse tal med den lineære regressions hældning og begyndelsesværdi fra Opgave 4.
\end{enumerate}


