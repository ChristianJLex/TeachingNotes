
\subsection*{Opgave 1}
\href{https://github.com/ChristianJLex/TeachingNotes/raw/master/2022-2023/Data%20og%20lign/IQdata.xlsx}{\color{blue!60}Dette datasæt} beskriver IQ for 213 værnepligtige. 
\begin{enumerate}[label=\roman*)]
	\item Bestem middelværdi og spredningen for IQ af de værnepligtige.
	\item Afgør, om dataen kan antages at være normalfordelt. 
	\item Bestem sandsynligheden for at have en IQ på under 110.
	\item Bestem det tal $x$, så 99$\%$ af befolkningen har en IQ på mere end $x$. 
\end{enumerate}

\subsection*{Opgave 2}
173 personer har fået målt deres temperatur. Resultatet er i \href{https://github.com/ChristianJLex/TeachingNotes/raw/master/2022-2023/Data%20og%20lign/Temperaturdata.xlsx}{\color{blue!60} dette datasæt}.
\begin{enumerate}[label=\roman*)]
	\item Bestem middelværdi og spredningen for temperaturen.
	\item Afgør, om dataen kan antages at være normalfordelt. 
	\item Bestem sandsynligheden for at have en temperatur på under 40 grader. 
	\item Bestem sandsynligheden for at have en temperatur mellem 35 og 38 grader.  
\end{enumerate}


\subsection*{Opgave 3}
501 kvinder har målt deres højde. Resultatet kan findes \href{https://github.com/ChristianJLex/TeachingNotes/raw/master/2022-2023/Data%20og%20lign/Hojdedata.xlsx}{\color{blue!60}her}.
\begin{enumerate}[label=\roman*)]
	\item Bestem middelværdi og spredningsestimatet for højden.
	\item Afgør, om dataen kan antages at være normalfordelt.
	\item Bestem sandsynligheden for, at en kvinde er mindre end 150 cm
	\item Bestem et $95\%$-konfidensinterval for den rigtige middelværdi $\mu$. 
\end{enumerate}

\subsection*{Opgave 4}

En slikproducent har lavet en stikprøve på 170 af sine poser med vingummibamser. I posen fremgår det, at den indeholder 130g vingummibamser. Indholdet af posernes vægt (i gram) fremgår af \href{https://github.com/ChristianJLex/TeachingNotes/raw/master/2023-2024/Data%20og%20lign/Vingummibamser.xlsx}{\color{blue!60} dette datasæt}. 

\begin{enumerate}[label=\roman*)]
	\item Bestem middelværdien og spredningen for vægten af vingummibamserne.
	\item Afgør, om vægten af posernes indhold kan antages at være normalfordelt.
	\item Hvad er sandsynligheden for at en pose vejer mindre end 130g?
	\item Hvilken $z$-værdi tilsvarer udfaldet 140?
\end{enumerate}
Producenten har et ønske om, at sandsynligheden for at få en pose med mindre end 130g vingummi er under 1 promille. De kan ikke ændre på spredningen
\begin{enumerate}[label=\roman*)]
	\setcounter{enumi}{3}
	\item Hvad skal middelværdien være på poserne for at sandsynligheden for at en pose vejer under 130g er 1 promille?
\end{enumerate}

\subsection*{Opgave 5}
En producent af mobiltelefoner har henover et år målt, hvor mange telefoner de har solgt ugentligt. Dette fremgår af \href{https://github.com/ChristianJLex/TeachingNotes/raw/master/2023-2024/Data og lign/Mobiltelefoner.xlsx}{\color{blue!60} dette datasæt}.

\begin{enumerate}[label=\roman*)]
	\item Afgør, om antallet af mobiltelefoner kan antages at være tilnærmelsesvist normalfordelt
\end{enumerate}
Producenten af telefoner kan ugentligt producere 8500, og sidste uges usolgte telefoner bliver automatisk sendt videre til en grossist. De har derfor ikke noget lager. 
\begin{enumerate}[label=\roman*)]
	\setcounter{enumi}{1}
	\item Hvad er sandsynligheden for, at de på en given uge ikke kan producere nok telefoner?
	\item Hvor mange telefoner skal de producere, hvis sandsynligheden for ikke at have produceret nok telefoner skal være under 1$\%$?
\end{enumerate}

