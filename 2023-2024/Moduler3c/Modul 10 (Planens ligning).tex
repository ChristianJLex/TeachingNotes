
\subsection*{Opgave 1}
\begin{enumerate}[label=\roman*)]
	\item På et plan $L$ ligger punktet $P(5,4,-2)$ og det har 
	\begin{align*}
		\vv{n} = 
		\begin{pmatrix}
			7 \\ 2 \\ -4
		\end{pmatrix} 
	\end{align*}
	som normalvektor. Bestem en ligning for $L$. 
	\item På et plan $L$ ligger punktet $P(1,10,5)$ og det har 
	\begin{align*}
		\vv{n} = 
		\begin{pmatrix}
			-11 \\ -12 \\ 13
		\end{pmatrix} 
	\end{align*}
	som normalvektor. Bestem en ligning for $L$.
	\item På en plat $L$ ligger punktet $P(1,10,5)$, og det har 
	\begin{align*}
		\vv{n} =
		\begin{pmatrix}
			\frac{1}{2} \\ \frac{2}{3} \\ \frac{3}{4}
		\end{pmatrix}
	\end{align*}
	som normalvektor. Besten en ligning for $L$.
	\item På et plan $L$ ligger punktet $P(-4,-5,12)$, og det har 
	\begin{align*}
		\vv{n} =
		\begin{pmatrix}
			0.5 \\ -2 \\ -1.5
		\end{pmatrix}
	\end{align*}
	som normalvektor. Besten en ligning for $L$.
\end{enumerate}

\subsection*{Opgave 2}
\begin{enumerate}[label=\roman*)]
	\item Et plan $L$ har ligningen
	\begin{align*}
		2(x-2) + 3(y+3) +5(z-1) = 0.
	\end{align*}
	Afgør om punkterne $(1,1,1)$ og $(1,6,-4)$ ligger på $L$. 
	\item Et plan $L$ har normalvektor 
	\begin{align*}
		\vv{n} =
		\begin{pmatrix}
			1 \\ 2 \\ 3
		\end{pmatrix}
	\end{align*}	 
	og punktet $(2,3,5)$ ligger på $L$. 
	Afgør, om punkterne $(-12,1,1)$ og $(-7,-3,2)$ ligger på $L$.
	\begin{align*}
		z=0.
	\end{align*}
	Afgør, om punkterne $(10000,4,2)$ og $(\pi, e,0)$ ligger på $L$.
\end{enumerate}

\subsection*{Opgave 3}
For et plan $L$ gælder det, at vektorerne 
\begin{align*}
	\begin{pmatrix}
		2 \\ 0 \\ 0
	\end{pmatrix}
	\textnormal{ og } 
	\begin{pmatrix}
		0 \\ 4 \\ 0
	\end{pmatrix}
\end{align*}
 er parallelle med planen. Desuden går planen gennem punktet $(2,4,8)$.


\subsection*{Opgave 4}
Det bør være ligegyldigt hvilken normalvektor, vi vælger, når vi konstruerer planens ligning. Vektorerne
\begin{align*}
	\vv{a} = 
	\begin{pmatrix}
		2 \\ 4 \\ 6
	\end{pmatrix} 
	\textnormal{ og }
	\vv{b} = 
	\begin{pmatrix}
		-1 \\ -2 \\ -3
	\end{pmatrix}.
\end{align*}
er parallelle, og derfor må de være normalvektorer til de samme planer.
\begin{enumerate}[label=\roman*)]
	\item Bestem en ligning for det plan, der har $\vv{a}$ som normalvektor og som går gennem punktet $(-3,4,-2)$.
	\item Bestem en ligning for det plan, der har $\vv{b}$ som normalvektor og som går gennem punktet $(-3,4,-2)$.
	\item Undersøg, om de to planer har samme ligning ved at omskrive den ene ligning til den anden. 
\end{enumerate}

\subsection*{Opgave 5}
Det bør også være ligegyldigt hvilket punkt, vi vælger på ligningen. Vi betragter derfor et plan med normalvektoren
\begin{align*}
	\vv{n} = 
	\begin{pmatrix}
		1 \\ 1 \\ -2
	\end{pmatrix},
\end{align*}
og som går gennem punktet $P(2,1,3)$.

\begin{enumerate}[label=\roman*)]
	\item Bestem en ligning for det plan $L$, der har $\vv{n}$ som normalvektor, og hvor $P$ ligger på planen. 
	\item Undersøg, om punktet  $Q(2,-1,-4)$ ligger på $L$.
	\item Bestem en ligning for det plan $M$, der har $\vv{n}$ som normalvektor, og hvor $Q$ ligger på planen.
	\item Undersøg, om $L$ og $M$ er det samme plan. 
\end{enumerate}
