\documentclass[12pt,x11names,a4paper]{article}
\input{preamble}


\newgeometry{margin=2cm}

\pagestyle{fancy}
\fancyhf{}

\rhead{Nørre Gymnasium\\2.x
}
\cfoot{Side \thepage \hspace{1pt} af \pageref{LastPage}}

%Husk at rette modul og dato!
\lhead{Sandsynlighedsregning \\ Matematik A
}
\chead{28. november 2023
}

\begin{document}

%\includepdf[pages=-]{Forsider/aarsprove_1v.pdf}
\savegeometry{art}

\begin{titlepage}
\newgeometry{margin=0pt}

\begin{minipage}{0.27\textwidth}

\begin{tikzpicture}[overlay]
\fill[top color = NorregGroen!40, bottom color = NorregGroen] (6,10) rectangle (-10,-30);
\end{tikzpicture}
\end{minipage}
\begin{minipage}{0.73\textwidth}
\begin{center}
\phantom{h} \vspace{1cm}\\
\hspace{4cm}
\includegraphics[scale = 1]{Billeder/Norreg.png} \\
\phantom{h} \vspace{5cm}\\
\rule{0.7\textwidth}{0.3mm}\\
\phantom{h}\\
{\fontsize{50}{60}\selectfont Matematik-\\aflevering}\\
\phantom{h}\\
\rule{0.7\textwidth}{0.3mm}\\
\Large 2023\\
\Large 2.x Ma

\end{center}
\end{minipage}
\end{titlepage}
\loadgeometry{art}

%Udfyld afsnit herunder og lav til egen Latex-fil

%Kopier følgende til overskrift:

%\begin{center}
%\Huge
%Aflevering 1
%\end{center}
%\section*{Opgave 1}
%\stepcounter{section}
\begin{center}
%Opgavesætter er delt i to dele:\\
%Delprøve 1 kun med den centralt udmeldte formelsamling.\\
%Delprøve 2 med alle hjælpemidler.
\end{center}

\section*{Krav til formidling af din besvarelse}

Ved bedømmelse af helhedsindtrykket af besvarelsen af de enkelte opgaver lægges særlig vægt på følgende fire punkter:
\begin{itemize}
\item[$\cdot$] \textbf{Redegørelse og dokumentation for metode} \\
Besvarelsen skal indeholde en redegørelse for den anvendte løsningsstragegi med dokumentation i form af et passende antal mellemregninger \textit{eller} matematiske forklaringer på metoden, når et matematisk værktøjsprogram anvendes.
\item[$\cdot$] \textbf{Figurer, grafer og andre illustrationer} \\
Besvarelsen skal indeholde hensigtsmæssig brug af figurer, grafer og andre illustrationer, og der skal være tydelige henvisninger til brug af disse i den forklarende tekst.
\item[$\cdot$] \textbf{Notation og layout}\\
Besvarelsen skal i overensstemmelse med god matematisk skik opstilles med hensigtsmæssig brug af symbolsprog, og med en redegørelse for den matematiske notation, der indføres og anvendes, og som ikke kan henføres stil standardviden.
\item[$\cdot$] \textbf{Formidling og forklaring}\\
Besvarelsen af rene matematikopgaver skal indeholde en angivelse af givne oplysninger og korte forklaringer knyttet til den anvendte løsningsstrategi beskrevet med brug af almindelig matematisk notation. 

Besvarelsen af opgaver, der omhandler matematiske modeller, skal indeholde en kort præsentation af modellens kontekst, herunder betydning af modellens parametre. De enkelte delspørgsmål skal afsluttes med en præcis konklusion præsenteret i et klart sprog i relation til konteksten.
\end{itemize}

\newpage



\begin{opgavetekst}{Opgave 1}
	To mængder $A$ og $B$ er givet ved
	\begin{align*}
		A &= \{1,2,3,4,5\},  \\
		B &= \{2,4,5,6,a,b\}.
	\end{align*}
\end{opgavetekst}
\begin{delopgave}{}{1}
	Bestem $A \cup B$, $A \cap B$ og $A \backslash B$.
\end{delopgave}
\begin{delopgave}{}{2}
	For to vilkårlige mængder $S$ og $T$ tegn et Venn-diagram, der viser mængden
	\begin{align*}
		(S \cup T) \backslash (S \cap T).
	\end{align*}
\end{delopgave}

\begin{delopgave}{}{3}
	Bestem mængden 
	\begin{align*}
		C = (A \cup B) \backslash (A \cap B)
	\end{align*}
\end{delopgave}


\begin{opgavetekst}{Opgave 2}
	I en klasse på 28 elever skal tre personer vælges til at gøre rent.
\end{opgavetekst}
\begin{delopgave}{}{1}
	Bestem antallet af måder, de tre personer kan udvælges, hvis den første person skal feje 
	gulvet, den næste skal tørre tavlen af og den sidste skal vaske tavlen.
\end{delopgave}
\begin{delopgave}{}{2}
	Bestem antallet af måder de kan vælges, hvis de alle skal feje gulvet
\end{delopgave}

\begin{opgavetekst}{Opgave 3}
	Du er til fest og alle får serveret et glas champagne. Alle skåler med alle og du hører 780
	klir.
\end{opgavetekst}
\begin{delopgave}{}{1}
	Bestem antallet af gæster til festen. 
\end{delopgave}

\newpage

\begin{opgavetekst}{Opgave 4}
	Du kaster med en særlig mønt der lander på højkant 20$\%$ af gangene, på plat $40\%$ af 
	gangene og på krone $40\%$ af gangene. Du kaster med mønten to gange og observerer 
	resultatet.
\end{opgavetekst}
\begin{delopgave}{}{1}
	Bestem udfaldsrummet for dette eksperiment
\end{delopgave}
\begin{meretekst}
	En stokastisk variabel $X$ tæller antallet af plat, du har slået med mønten
\end{meretekst}

\begin{delopgave}{}{2}
	Bestem værdimængden for $X$.
\end{delopgave}
\begin{delopgave}{}{3}
	Bestem fordelingen for $X$.
\end{delopgave}

\begin{opgavetekst}{Opgave 5}
	En stokastisk variabel $X$ har følgende fordeling.
	\begin{center}
		\vspace{0.5cm}
		\begin{tabular}{c|c|c|c|c|c|c}
			$x$ & 1 & 2 & 3 & 4 & 5 & 6 \\
			\hline
			$P(X = x)$ & 0.1 & 0.1 & 0.4 & 0.05 & 0.05 & $P(X = 6)$
		\end{tabular}
	\end{center}
	\phantom{h}
\end{opgavetekst}

\begin{delopgave}{}{1}
	Bestem $P(X = 6)$.
\end{delopgave}
\begin{delopgave}{}{2}
	Bestem $P(X > 3)$.
\end{delopgave}
\begin{delopgave}{}{3}
	Bestem $P(X \in \{2,4,6\})$.
\end{delopgave}

\begin{opgavetekst}{Opgave 6}
\end{opgavetekst}
\begin{delopgave}{}{1}
	Bestem sandsynligheden for, at en tilfældig familie med tre børn har netop én datter
\end{delopgave}
\begin{delopgave}{}{2}
	Bestem sandsynligheden for, at en pige med to søskende ingen søstre har. 
\end{delopgave}
\end{document}



