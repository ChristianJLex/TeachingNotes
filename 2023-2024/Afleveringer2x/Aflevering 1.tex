
\begin{opgavetekst}{Opgave 1}
	To mængder $A$ og $B$ er givet ved
	\begin{align*}
		A &= \{1,2,3,4,5\},  \\
		B &= \{2,4,5,6,a,b\}.
	\end{align*}
\end{opgavetekst}
\begin{delopgave}{}{1}
	Bestem $A \cup B$, $A \cap B$ og $A \backslash B$.
\end{delopgave}
\begin{delopgave}{}{2}
	For to vilkårlige mængder $S$ og $T$ tegn et Venn-diagram, der viser mængden
	\begin{align*}
		(S \cup T) \backslash (S \cap T).
	\end{align*}
\end{delopgave}

\begin{delopgave}{}{3}
	Bestem mængden 
	\begin{align*}
		C = (A \cup B) \backslash (A \cap B)
	\end{align*}
\end{delopgave}


\begin{opgavetekst}{Opgave 2}
	I en klasse på 28 elever skal tre personer vælges til at gøre rent.
\end{opgavetekst}
\begin{delopgave}{}{1}
	Bestem antallet af måder, de tre personer kan udvælges, hvis den første person skal feje 
	gulvet, den næste skal tørre tavlen af og den sidste skal vaske tavlen.
\end{delopgave}
\begin{delopgave}{}{2}
	Bestem antallet af måder de kan vælges, hvis de alle skal feje gulvet
\end{delopgave}

\begin{opgavetekst}{Opgave 3}
	Du er til fest og alle får serveret et glas champagne. Alle skåler med alle og du hører 780
	klir.
\end{opgavetekst}
\begin{delopgave}{}{1}
	Bestem antallet af gæster til festen. 
\end{delopgave}

\newpage

\begin{opgavetekst}{Opgave 4}
	Du kaster med en særlig mønt der lander på højkant 20$\%$ af gangene, på plat $40\%$ af 
	gangene og på krone $40\%$ af gangene. Du kaster med mønten to gange og observerer 
	resultatet.
\end{opgavetekst}
\begin{delopgave}{}{1}
	Bestem udfaldsrummet for dette eksperiment
\end{delopgave}
\begin{meretekst}
	En stokastisk variabel $X$ tæller antallet af plat, du har slået med mønten
\end{meretekst}

\begin{delopgave}{}{2}
	Bestem værdimængden for $X$.
\end{delopgave}
\begin{delopgave}{}{3}
	Bestem fordelingen for $X$.
\end{delopgave}

\begin{opgavetekst}{Opgave 5}
	En stokastisk variabel $X$ har følgende fordeling.
	\begin{center}
		\vspace{0.5cm}
		\begin{tabular}{c|c|c|c|c|c|c}
			$x$ & 1 & 2 & 3 & 4 & 5 & 6 \\
			\hline
			$P(X = x)$ & 0.1 & 0.1 & 0.4 & 0.05 & 0.05 & $P(X = 6)$
		\end{tabular}
	\end{center}
	\phantom{h}
\end{opgavetekst}

\begin{delopgave}{}{1}
	Bestem $P(X = 6)$.
\end{delopgave}
\begin{delopgave}{}{2}
	Bestem $P(X > 3)$.
\end{delopgave}
\begin{delopgave}{}{3}
	Bestem $P(X \in \{2,4,6\})$.
\end{delopgave}

\begin{opgavetekst}{Opgave 6}
\end{opgavetekst}
\begin{delopgave}{}{1}
	Bestem sandsynligheden for, at en tilfældig familie med tre børn har netop én datter
\end{delopgave}
\begin{delopgave}{}{2}
	Bestem sandsynligheden for, at en pige med to søskende ingen søstre har. 
\end{delopgave}
