\begin{center}
\Huge
Prikprodukter
\end{center}


\section*{Prikprodukt}
\stepcounter{section}

Vi vil ikke definere nogen måde at gange vektorer sammen. Vi vil i stedet definere det såkaldte prikprodukt. Dette kan blandt andet bruges til at bestemme vinklen mellem to vektorer. 
\begin{defn}
Lad $\vv{v}$ og $\vv{w}$ være defineret som
\begin{align*}
\vv{v} = \begin{pmatrix}
v_1\\ v_2
\end{pmatrix}, \textnormal{ og }\vv{w} = \begin{pmatrix}w_1 \\ w_2
\end{pmatrix}.
\end{align*}
Så defineres \textit{prikproduktet} eller \textit{skalarproduktet} mellem $\vv{v}$ og $\vv{w}$ som
\begin{align*}
v\cdot w = v_1w_1 + v_2w_2.
\end{align*}
Dette skrives også til tider $\langle \vv{v}, \vv{w} \rangle$. 
\end{defn}
\begin{exa}
Lad $\vv{v} = \begin{pmatrix}
2 \\ 2
\end{pmatrix}$ og lad $\vv{w} = \begin{pmatrix}
-2 \\ 5
\end{pmatrix}$. Så kan vi bestemme prikproduktet mellem $\vv{v}$ og $\vv{w}$ som
\begin{align*}
\vv{v} \cdot \vv{w} = 2\cdot(-2)+2\cdot 5 = 6.
\end{align*}
\end{exa}

\section*{Vinkler mellem vektorer}
\stepcounter{section}

Prikproduktet relaterer sig til vinklen mellem vektorer. Vi skal senere se, hvordan vi kan bestemme vinklen mellem to vektorer. Vinklen mellem to vektorer $\vv{v}$ og $\vv{w}$ noteres som $\angle(\vv{v},\vv{w})$. På Figur \ref{fig:vinkel} er vinklen mellem to vektorer illustreret
\begin{figure}[H]
	\centering
	\begin{tikzpicture}
		\begin{axis}[
			axis lines = center, 
			xmin = -1, xmax = 5, 
			ymin = -1, ymax = 5, 
			x = 1.5cm, y = 1.5cm,
			ticks = none, 
			xlabel = $x$, ylabel = $y$
		]
			\draw[-{Stealth[scale=1.5]}, thick, color = teal] (axis cs: 1,1) -- (axis cs: 3,4);
			\draw[-{Stealth[scale=1.5]}, thick, color = teal] (axis cs: 1,1) -- (axis cs: 4,0);
			\draw[thick,teal] ([shift=(-18.43:0.7cm)] axis cs: 1,1) arc (-18.43:56.31:0.7cm);
			\node[color = teal] at (axis cs: 1.6,1.2) {$\theta$};
			\node[color = teal] at (axis cs: 2.5,0.7) {$\vv{v}$};
			\node[color = teal] at (axis cs: 1.7,2.5) {$\vv{w}$};			
		\end{axis}
	\end{tikzpicture}
	\caption{Vinkel mellem to vektorer}
	\label{fig:vinkel}
\end{figure}

Hvis vinklen mellem to vektorer $\vv{v}$ og $\vv{w}$ er $90^\circ$, altså at 
$\angle(\vv{v},\vv{w}) = 90^\circ$, så siger vi, at vektorerne er vinkelrette eller \textit{orthogonale}. Vi skriver da $\vv{w} \perp \vv{v}$.

\begin{setn}\label{setn:1}
To vektorer $\vv{v}$ og $\vv{w}$ er orthogonale hvis og kun hvis $\vv{v}\cdot \vv{w} = 0$.
\end{setn}
Vi vil senere se mere præcist hvordan sammenhængen mellem vinklen mellem vektorer og prikproduktet er. 
\begin{exa}
Vi vil afgøre, om $\vv{v} = \begin{pmatrix}
2\\2
\end{pmatrix}$ og $\vv{w} = \begin{pmatrix}
-1\\1
\end{pmatrix}$ er orthogonale. Vi bestemmer derfor prikproduktet
\begin{align*}
\vv{v}\cdot \vv{w} = 2\cdot(-1)+2\cdot 1 = 0. 
\end{align*}
Derfor ved vi, at vinklen mellem de to vektorer er $0^\circ$, og at de derfor er orthogonale eller vinkelrette.
\end{exa}


\subsection*{Opgave 1}
\begin{enumerate}[label=\roman*)]
\item Bestem prikproduktet mellem følgende vektorer
\begin{align*}
&1) \ \begin{pmatrix}1 \\ 0\end{pmatrix} \textnormal{ og } \begin{pmatrix}0 \\ 1\end{pmatrix}    &&2) \  \begin{pmatrix} 4 \\  5\end{pmatrix} \textnormal{ og } \begin{pmatrix} -2  \\ 6\end{pmatrix}    \\
&3) \ \begin{pmatrix}12 \\ 15\end{pmatrix} \textnormal{ og } \begin{pmatrix}-3 \\ 14 \end{pmatrix}   &&4) \ \begin{pmatrix} -2 \\ -5\end{pmatrix} \textnormal{ og } \begin{pmatrix} 0.5 \\ -12 \end{pmatrix}     \\
\end{align*}
\end{enumerate}

\subsection*{Opgave 2}

Afgør hvilke af følgende par af vektorer, der er orthogonale

\begin{align*}
	&1) \ \begin{pmatrix}
		2 \\ 5
	\end{pmatrix} 
	\textnormal{ og }
	\begin{pmatrix}
		-10 \\ 4
	\end{pmatrix} 
	&
	&2) \ \begin{pmatrix}
		4 \\ 11
	\end{pmatrix} 
	\textnormal{ og }
	\begin{pmatrix}
		-12 \\ 3
	\end{pmatrix} 
	\\
	&3) \ \begin{pmatrix}
		-0.5 \\ 3
	\end{pmatrix} 
	\textnormal{ og }
	\begin{pmatrix}
		12 \\ 0
	\end{pmatrix} 
	&
	&4) \ \begin{pmatrix}
		\frac{1}{4} \\ -8
	\end{pmatrix} 
	\textnormal{ og }
	\begin{pmatrix}
		16 \\ \frac{1}{2}
	\end{pmatrix} 
\end{align*}

\subsection*{Opgave 3}
Bestem prikproduktet mellem følgende vektorer.
\begin{center}
\resizebox{0.45\textwidth}{!}{
\begin{tikzpicture}
	\begin{axis}[
		axis lines = center, 
		xmin = -1.5, xmax = 6.5, 
		ymin = -1.5, ymax = 6.5,
		grid,
		x = 1cm, y = 1cm,
		xtick = {-1,0,...,5,6}, ytick = {-1,0,...,5,6},
		xlabel = $x$, ylabel = $y$	
	]
		\draw[-{Stealth[scale=1.5]}, thick, color = teal] (axis cs: 0,0) -- (axis cs: 4,5);
		\draw[-{Stealth[scale=1.5]}, thick, color = teal] (axis cs: 1,5) -- (axis cs: -1,-1);
	\end{axis}
\end{tikzpicture}
}
\resizebox{0.45\textwidth}{!}{
\begin{tikzpicture}
	\begin{axis}[
		axis lines = center, 
		xmin = -1.5, xmax = 6.5, 
		ymin = -1.5, ymax = 6.5,
		grid,
		x = 1cm, y = 1cm,
		xtick = {-1,0,...,5,6}, ytick = {-1,0,...,5,6},
		xlabel = $x$, ylabel = $y$	
	]
		\draw[-{Stealth[scale=1.5]}, thick, color = teal] (axis cs: 1,1) -- (axis cs: 6,-1);
		\draw[-{Stealth[scale=1.5]}, thick, color = teal] (axis cs: 1,1) -- (axis cs: 3,6);
	\end{axis}
\end{tikzpicture}
}
\end{center}

\subsection*{Opgave 4}
Afgør hvilke af følgende par af vektorer, der er orthogonale

\begin{center}
\resizebox{0.45\textwidth}{!}{
\begin{tikzpicture}
	\begin{axis}[
		axis lines = center, 
		xmin = -4.5, xmax = 2.5, 
		ymin = -4.5, ymax = 2.5,
		grid,
		x = 1cm, y = 1cm,
		xtick = {-4,-3,...,2,3}, ytick = {-4,-3,...,2,3},
		xlabel = $x$, ylabel = $y$	
	]
		\draw[-{Stealth[scale=1.5]}, thick, color = teal] (axis cs: -3,-2) -- (axis cs: 2,-4);
		\draw[-{Stealth[scale=1.5]}, thick, color = teal] (axis cs: -2,2) -- (axis cs: -4,-4);
	\end{axis}
\end{tikzpicture}
}
\resizebox{0.45\textwidth}{!}{
\begin{tikzpicture}
	\begin{axis}[
		axis lines = center, 
		xmin = -1.5, xmax = 6.5, 
		ymin = -1.5, ymax = 6.5,
		grid,
		x = 1cm, y = 1cm,
		xtick = {-1,0,...,5,6}, ytick = {-1,0,...,5,6},
		xlabel = $x$, ylabel = $y$	
	]
		\draw[-{Stealth[scale=1.5]}, thick, color = teal] (axis cs: 4,4) -- (axis cs: 6,6);
		\draw[-{Stealth[scale=1.5]}, thick, color = teal] (axis cs: 5,0) -- (axis cs: 0,5);
	\end{axis}
\end{tikzpicture}
}

\end{center}

\subsection*{Opgave 5}

Fire punkter er givet ved $A(2,2)$, $B(5,4)$, $C(-2,7)$ og $D(-7,-3)$.
\begin{enumerate}[label=\roman*)]
	\item Bestem $\vv{AB}\cdot \vv{CD}$
	\item Afgør om $\vv{AC}$ og $\vv{DB}$ er orthogonale
\end{enumerate}

\subsection*{Opgave 6}

To vektorer er givet ved
\begin{align*}
	\vv{a} = 
	\begin{pmatrix}
		x^2 - 2 \\
		1	
	\end{pmatrix}
	\textnormal{ og }
	\vv{b} = 
	\begin{pmatrix}
		1 \\
		x
	\end{pmatrix}
\end{align*}
\begin{enumerate}[label = \roman*)]
	\item Bestem $x$, så $\vv{a}$ og $\vv{b}$ er orthogonale.
\end{enumerate}
