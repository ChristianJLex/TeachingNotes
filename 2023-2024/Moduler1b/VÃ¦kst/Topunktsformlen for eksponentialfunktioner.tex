\begin{center}
\Huge
Eksponentialfunktion gennen to punkter. 
\end{center}
\section*{Topunktsformlen for ekspontialfunktioner}
\stepcounter{section}
Har vi to punkter $(x_1,y_1)$ og $(x_2,y_2)$, så kan vi finde den entydige rette linje med ligning $y = ax + b$, der skærer gennem disse punkter ved brug af topunktsformlen. 
Vi kan tilsvarende finde en entydig eksponentialfunktion, der skærer gennem to punkter.
Vi starter med at huske på, hvordan en eksponentialfunktion er defineret
\begin{defn}[Eksponentialfunktion]
	Lad $a,b>0$. Så kaldes en funktion $f$ givet ved
	\begin{align*}
		f(x)=b\cdot a^x
	\end{align*}
	for en \textit{eksponentialfunktion}. Tallet $b$ kaldes for \textit{begyndelsesværdien} og tallet $a$ kaldes for \textit{fremskrivningsfaktoren}.
\end{defn}

\begin{setn}
Givet to punkter $(x_1,y_1)$ og $(x_2,y_2)$ er der en entydig eksponentialfunktion $f$ givet ved
\begin{align*}
f(x) = b\cdot a^x,
\end{align*}
hvis graf går gennem disse punkter. Fremskrivningsfaktoren $a$ er givet ved
\begin{align*}
a=\sqrt[(x_2-x_1)]{\frac{y_2}{y_1}}.
\end{align*}
Skæringen med $y$-aksen $b$ er givet ved
\begin{align*}
b = \frac{y_1}{a^{x_1}}.
\end{align*}
\end{setn}
\begin{proof}
Vi skal bestemme en eksponentiel funktion
\begin{align*}
f(x) = b\cdot a^x, 
\end{align*}
der går gennem punkterne $(x_1,y_1)$ og $(x_2,y_2)$. Vi må derfor have, at $y_1 = b\cdot a^{x_1}$ og $y_2 = b\cdot a^{x_2}$. Vi finder nu forholdet mellem $y_2$ og $y_1$ som
\begin{align*}
\frac{y_2}{y_1} &= \frac{b\cdot a^{x_2}}{b\cdot a^{x_1}}\\
				&= \frac{a^{x_2}}{a^{x_1}}\\
				&=a^{x_2-x_1}.
\end{align*}
Vi tager nu $x_2-x_1$'te roden på begge sider af lighedstegnet.
\begin{align*}
\sqrt[(x_2-x_1)]{\frac{y_2}{y_1}} = \sqrt[(x_2-x_1)]{a^{x_2-x_1}} = a,
\end{align*}
og vi har altså bestemt $a$, siden $a>0$. Da vi ved, at $y_1 = b\cdot a^{x_1}$, så får vi ved at dividere igennem med $a^{x_1}$, at 
\begin{align*}
\frac{y_1}{a^{x_1}} = \frac{b\cdot a^{x_1}}{a^{x_1}} = b, 
\end{align*}
og beviset er færdigt. 
\end{proof}
Bemærk, at vi gerne vil have, at punkterne ligger over x-aksen.
\begin{exa}
Vi skal bestemme den eksponentialfunktion, der går gennem $(1,2)$ og $(3,8)$. Vi bruger topunktsformlen for eksponentialfunktioner, og får
\begin{align*}
a = \sqrt[3-1]{\frac{8}{2}} = \sqrt[2]{4} = 2, 
\end{align*}
og 
\begin{align*}
b = \frac{2}{2^{1}} = 1.
\end{align*}
Den eksponentialfunktion, der går gennem disse punkter er derfor 
\begin{align*}
f(x) = 1\cdot 2^x = 2^x.
\end{align*}
\end{exa}

\subsection*{Opgave 1 (Uden Maple)}
Bestem de eksponentialfunktioner, der går gennem følgende par af punkter.
\begin{align*}
&1) \ (0,3), \ (1,6)  &&2) \ (1,3), \ (3,27)      \\
&3) \ (1,2), \  (3,8)   &&4) \  (0,4), \ (3,32)    
\end{align*}

\subsection*{Opgave 2 (med Maple)}
Vi har efter 3 år 6750kr på en konto og efter 7 år 7433kr på samme konto. Beløbet på kontoen kan beskrives ved eksponentiel vækst
\begin{enumerate}[label=\roman*)]
	\item Bestem den eksponentialfunktion, der beskriver beløbet på kontoen efter $x$ år.
	\item Afgør, hvornår meget der står på kontoen efter 12 år.
	\item Hvornår står der 10.000kr på kontoen?
\end{enumerate}

\subsection*{Opgave 3 (Med Maple)}

En eksponentialfunktion $h$ opfylder, at $h(1) = 4$ og $h(3)  = 20$. 
\begin{enumerate}[label = \roman*)]
	\item Bestem forskriften for $h$.
	\item Bestem $h(2)$.
\end{enumerate}

\subsection*{Opgave 4 (Med Maple)}
På Figur \ref{fig:graf} ses grafen for en eksponentialfunktion $f$.
\begin{figure}[H]
	\centering
	\begin{tikzpicture}
		\begin{axis}[
			axis lines = middle, 
			xmin = -1, xmax = 6,
			xlabel = $x$, ylabel = $y$,
			xtick = {2,5}, ytick = {7,15}
			]
			\addplot[thick, color = teal, samples = 100, domain = -1:6] {4.2115*1.2892^x};
			\node[circle, fill = purple, inner sep = 1.5pt] at (axis cs: 2,7) {};
			\node[circle, fill = purple, inner sep = 1.5pt] at (axis cs: 5,15) {};
			\draw[dashed, color = purple] (axis cs: 2,0) -- (axis cs: 2,7);
			\draw[dashed, color = purple] (axis cs: 2,7) -- (axis cs: 0,7);
			\draw[dashed, color = purple] (axis cs: 5,0) -- (axis cs: 5,15);
			\draw[dashed, color = purple] (axis cs: 5,15) -- (axis cs: 0,15);			
			\node[color = teal] at (axis cs: 3,8) {$f$};
		\end{axis}
	\end{tikzpicture}
	\caption{Graf for eksponentialfunktionen $f$.}
	\label{fig:graf}
\end{figure}

\begin{enumerate}[label=\roman*)]
	\item Bestem forskriften for $f$.
	\item Bestem $f(4)$.
	\item Løs ligningen $f(x)=20$.
\end{enumerate}


\subsection*{Opgave 5 (med Maple)}
Vi har placeret en radioaktiv isotop på en vægt. Den vejer til start 1 kg. Efter 10 dage vejer den 0.997kg. Henfaldet af isotopen antages at kunne beskrives ved eksponentiel vækst.
\begin{enumerate}[label=\roman*)]
	\item  Bestem den eksponentialfunktion, der beskriver vægten af den radioaktive isotop.
	\item Hvor meget vejer isotopen efter en måned?
	\item Hvornår er vægten af isotopen halveret?
\end{enumerate}

\subsection*{Opgave 6 (med Maple)}
En bakteriekoloni vokser med 5$\%$ hver time. Efter 8 timer er der i kolonien 1.3 mia bakterier. 
\begin{enumerate}[label=\roman*)]
	\item Bestem forskriften på den eksponentialfunktion, der beskriver antallet af bakterier i kolonien. 
	\item Hvad var begyndelsesværdien for eksponentialfunktionen?
	\item Hvornår overstiger antallet af bakterier 3 mia.?
\end{enumerate}

\subsection*{Opgave 7 (med Maple)}
\begin{enumerate}[label=\roman*)]
	\item En eksponentialfunktion er givet ved
	\begin{align*}
		f(x) = b \cdot 1.3^x.
	\end{align*}
	Udnyt, at $f(2) = 7$ til at bestemme $b$.
	\item En eksponentialfunktion er givet ved
	\begin{align*}
		g(x) = 0.743\cdot a^x
	\end{align*}
	Udnyt, at $g(0.1) = 1.236$ til at bestemme $a$.  
\end{enumerate}

