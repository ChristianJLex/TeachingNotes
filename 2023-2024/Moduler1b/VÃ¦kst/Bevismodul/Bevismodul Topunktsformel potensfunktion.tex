
\begin{center}
	\begin{tabular}{|p{0.5\textwidth}|p{0.5\textwidth}|}
		\hline
		Vi indsætter punkterne $(x_1,y_1)$ og $(x_2,y_2)$ i forskriften for potensfunktionen.
		\vspace{0.8cm}
		& 	
		\begin{center}		
		\begin{tabular}{c} 
			$f(x_1) = b \cdot x_1^a = y_1$ \\
			$f(x_2) = b \cdot x_2^a = y_2$  
		\end{tabular}
		\end{center}				  	 
		\\
		\hline
		Vi dividerer udtrykket for $y_2$ med udtrykket for $y_1$.\vspace{1.3cm}
		& 
		$$
			\frac{y_2}{y_1} = \frac{b\cdot x_2^a}{b\cdot x_1^a} 	
		$$
		\\
		\hline
		Da $b$ er ganget på udtrykket både i tælleren og i nævneren, lader vi $b$ gå ud.
		\vspace{1.3cm}
		&
		$$
		\frac{y_2}{y_1} = \frac{x_2^a}{x_1^a}
		$$
		\\
		\hline
		Vi anvender regnereglen $\left(\frac{x}{y}\right)^a = \frac{x^a}{y^a}$.\vspace{1.7cm}
		&
		$$
			\frac{y_2}{y_1} = \left(\frac{x_2}{x_1}\right)^a
		$$
		\\
		\hline
		Vi tager logaritmen på begge sider af lighedstegnet. \vspace{1.3cm}
		&
		$$ 
			\log\left(\frac{y_2}{y_1} \right) 
			= \log\left( \left(\frac{x_2}{x_1}\right)^a\right)
		$$
		\\
		\hline
		Vi anvender logaritmeregnereglen $\log(x^a) = a\log(x)$
		\vspace{1.3cm}
		&
		$$ 
			\log\left(\frac{y_2}{y_1} \right) 
			= a\log\left(\frac{x_2}{x_1}\right)
		$$
		\\
		\hline
		Vi isolerer $a$ ved at dividere med $\log\left(\frac{x_2}{x_1}\right)$ på begge
		sider af lighedstegnet
		&
		$$
			\frac{\log\left(\frac{y_2}{y_1}\right)}{\log\left(\frac{x_2}{x_1}\right)} = a
		$$
		\\
		\hline
		Vi anvender regnereglen 
		$$
		\log\left(\frac{a}{b}\right) = \log(a) - \log(b)
		$$
		&
		$$
		\frac{\log(y_2)-\log(y_1)}{\log(x_2)-\log(x_1)} = a
		$$
		\\
		\hline
		Vi er nu færdige
		&
		$\blacksquare$\\
		\hline
	\end{tabular}
\end{center}