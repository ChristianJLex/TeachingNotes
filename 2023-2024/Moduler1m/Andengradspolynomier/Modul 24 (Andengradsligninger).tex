
\begin{center}
\Huge
Andengradsligninger
\end{center}
\stepcounter{section}
Når vi bestemmer rødder til andengradspolynomier, så løser vi netop ligningen 
\begin{align*}
	ax^2 + bx + c = 0,
\end{align*}
som er det, vi kalder en andengradsligning. Så det at bestemme rødder til andengradspolynomier
tilsvarer altså at løse en andengradsligning. Vi kalder derfor også rodformlen for andengradspolynomier for løsningsformlen for andengradsligninger eller blot \textit{diskriminantformlen}

\begin{exa}
	Vi skal løse ligningen 
	\begin{align*}
		x^2-5x = 6.
	\end{align*}
	Vi samler først alle led på den ene side af lighedstegnet og får
	\begin{align*}
		x^2 + 5x - 6 = 0.
	\end{align*}
	Denne ligning tilsvarer at finde rødderne for polynomiet
	\begin{align*}
		f(x) = x^2 + 5x - 6,
	\end{align*}
	og vi bruger derfor rodformlen til at bestemme løsningerne.
	Diskriminanten er
	\begin{align*}
		d &= 5^2 - 4 \cdot 1 \cdot (-6) \\
		  &= 25 + 24 \\
		  &= 49,
	\end{align*}
	og der er derfor to løsninger. Løsningerne er
	\begin{align*}
		x &= \frac{-5 \pm \sqrt{49}}{2\cdot 1} \\
		 &= \frac{-5 \pm 7}{2},
	\end{align*}
	og løsningerne er derfor 
	\begin{align*}
		x = \frac{-5 + 7}{2} = 1
	\end{align*}
	og
	\begin{align*}
		x = \frac{-5 -7}{2} = -6.
	\end{align*}
\end{exa}

\subsection*{Opgave 1}
Bestem løsningerne for følgende andengradsligninger, hvis der er nogle. 

\begin{align*}
	&1) \ 3 x^2 - 3 x - 6 = 0  &&2) \ 3 x^2 + 6 x  = -3   \\
	&3) \ x^2 = -6 x -9 &&4) \ x^2 + 4x = -4   \\
	&5) \ 3x^2 = 3        &&6) \ x^2-3x+2 = 0 \\
    &7) \ 2x^2+4x = 6    &&8) \ x^2 = 6 x - 9   \\
\end{align*}
