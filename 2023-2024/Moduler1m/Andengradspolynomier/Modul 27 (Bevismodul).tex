
\begin{center}
\Huge
Rødder for andengradspolynomier
\end{center}

\section*{Rødder for andengradspolynomier}
\stepcounter{section}

Hvis vi skal bestemme rødderne for et andengradspolynomium, så skal vi bruge løsningsformlen for andengradsligninger, som vi også til tider kalder for diskriminantformlen. Denne er givet ved følgende. 
\begin{setn}[Løsningsformlen for andengradsligninger]
	Rødderne for et andengradspolynomium $f$ givet ved
	\begin{align*}
		f(x) = ax^2+bx+c
	\end{align*}
	er givet ved
	\begin{align*}
		x = \frac{-b \pm \sqrt{d}}{2a},
	\end{align*}
	hvor vi antager, at diskriminanten $d = b^2 - 4ac$ er ikke-negativ.
\end{setn}
\begin{proof}
	Vi skal løse ligningen 
	\begin{align}
		\label{eq:1}
		ax^2+bx+c = 0.
	\end{align}
	Før vi går i gang med at løse ligningen, så vil vi gerne udregne et hjælperesultat. Vi vil 		gerne hæve parentesen i 
	\begin{align}
		\label{eq:2}
		(2ax+b)^2,
	\end{align}
	og til dette skal vi bruge kvadratsætningen
	\begin{align*}
		(a+b)^2 = a^2+b^2+2ab.
	\end{align*}
	Vi anvender denne kvadratsætning på \eqref{eq:2} og får
	\begin{align}
		\label{eq:3}
		(2ax + b)^2 = 4a^2x^2+b^2+4axb.
	\end{align}
	Dette skal vi bruge senere i beviset. Vi vender nu tilbage til ligningen \eqref{eq:1} og 			ganger begge sider af lighedstegnet med $4a$.
	\begin{align*}
		4a^2x^2 + 4abx + 4ac = 0
	\end{align*}
	Vi lægger nu diskriminanten $d = b^2 - 4ac$ til på begge sider af lighedstegnet.
	\begin{align*}
		4a^2x^2+4abx + b^2 = b^2 - 4ac.
	\end{align*}
	Vi skal nu anvende \eqref{eq:2} på venstresiden og får
	\begin{align*}
		(2ax + b)^2 = d.
	\end{align*}
	Da $d$ er ikke-negativ, så kan vi tage kvadratroden på begge sider af lighedstegnet, og vi 
	får
	\begin{align*}
		2ax + b = \pm \sqrt{d}
	\end{align*}
	Vi kan nu isolere $x$ og får
	\begin{align*}
		2ax + b = \pm \sqrt{d} &\Leftrightarrow 2ax = -b \pm \sqrt{d} \\
		&\Leftrightarrow \frac{-b \pm \sqrt{d}}{2a}.
	\end{align*}
	Med dette er vi færdige.
\end{proof}
