\begin{center}
\Huge
Parenteser og kvadratsætninger
\end{center}
\section*{Parenteser}
\stepcounter{section}
Vi vil se på, hvordan to parenteser multipliceres. Dette vises nok nemmest ved en række eksempler.
\begin{exa}
Vi vil multiplicere $(a+b)$ med $(c+d)$. Dette gøres ved at gange hvert led i parentes 1 ind på hvert led i parentes 2 og så lægge de 4 resulterende produkter til sammen til sidst:
\begin{align*}
(a+b)(c+d) = ac+ad+bc+bd.
\end{align*}
Har vi flere led, gøres det fuldstændigt tilsvarende:
\begin{align*}
(a+b+c)(d+e) = ad+ae+bd+be+cd+ce,
\end{align*}
eller 
\begin{align*}
(a+b+c)(d+e+f) = ad+ae+af+bd+be+bf+cd+ce+cf.
\end{align*}
Vores konstanter kan også have koefficienter:
\begin{align*}
(2a+3b)(-2x-y) = -4ax-2ay-6bx-3y.
\end{align*}
\end{exa}
\section*{Kvadratsætninger}
Hvis vi har en sum af to led kvadreret, kan vi bruge de såkaldte kvadratsætninger til at bestemme resultatet. Vi vil udlede kvadratsætningerne ved brug af regnereglerne for produkter af parenteser.
\begin{setn}
For alle tal $a,b$ har vi
\begin{enumerate}[label=\roman*)]
\item $(a+b)^2 = a^2+b^2+2ab,$
\item $(a-b)^2 = a^2+b^2-2ab$ og 
\item $(a+b)(a-b) = a^2-b^2$.
\end{enumerate}
\end{setn}
\begin{proof}
Det er blot at regne:

\begin{enumerate}[label=$\roman*)$]
\item $(a+b)^2 = (a+b)(a+b)=a^2+ab+ba+b^2=a^2+b^2+2ab$,
\item $(a-b)^2 = (a-b)(a-b) = a^2-ab-ba+b^2 = a^2+b^2-2ab$, og
\item $(a+b)(a-b) = a^2-ab+ba+-b^2 = a^2-b^2$ 
\end{enumerate}
\end{proof}

\begin{exa}
Vi vil vise, hvordan man anvender kvadratsætningerne. For \\$(7x+3y)^2$ har vi
\begin{align*}
(7x+3y)^2 \overset{i)}{=} (7x)^2+(3y)^2+2\cdot 7x\cdot 3y = 49x^2+9y^2+42xy.
\end{align*}
For $(3a-2y)^2$ har vi
\begin{align*}
(3a-2y)^2 \overset{ii)}{=} (3a)^2+(2y)^2 -2\cdot 3a\cdot 2y = 9a^2 + 4y^2-6ay.
\end{align*}
For $(4c+2x)(2c-2x)$ har vi
\begin{align*}
(4c+2x)(4c-2x) \overset{iii)}{=} (4c)^2-(2x)^2 = 16c^2-4x^2.
\end{align*}
\end{exa}
Det er også muligt at regne den anden vej:
\begin{exa}
For $4x^2+9y^2+12xy$ har vi tilsvarende
\begin{align*}
4x^2+9y^2+12xy = (2x+3y)^2.
\end{align*}
For $4a^2-16b^2$ har vi
\begin{align*}
   4a^2-16b^2 = (2a+4b)(2a-4b).
\end{align*}
\end{exa}

\section*{Opgave 1}
Reducér følgende udtryk mest muligt:

\begin{align*}
	&1) \  a+b+2a+b+a-b  &2) \  14m+7n-(10m-3n)     \\
	&3) \  4(a+5b) + 7(3a+b)   &4) \  a(b + 2) - 4      \\
	&5) \ a(6-b) - b(6-a)   &6) \ (a+b)^2 -a(a+2b)       \\
	&7) \ (x+2y)(x-2y)    &8) \ y(2x-y)+(y-x)^2      \\
\end{align*}
\section*{Opgave 2}
Udregn følgende produkter af parenteser:
\begin{align*}
&1) \ (a+b)(c+d+e)  &&2) \ (\sqrt{2}+\sqrt{3})(\sqrt{2}+3)  \\
 &3) \ (4+ab)(2+3f)   &&4) \ (\frac{2}{3} + \frac{6}{7})(\frac{1}{2} - 3)  \\
 &5) \ (2+x^2)(x+y)  &&6) \ (3x^2+x^3)(x^4+x)  \\
 &7) \ (x^{\frac{3}{2}}+y^{\frac{1}{2}})(x^{3}+y^2)  &&8) \ (1x+2y+3z)(4a-5b-6c)  \\
 &9) \ (a+2b)(c+d)   &&10) \ (\sqrt{4}b+\sqrt{3}c)(\sqrt{4}b-\sqrt{3}c)  \\
\end{align*}
\section*{Opgave 3}
Brug kvadratsætningerne til at udregne følgende:
\begin{align*}
&1)  \ (a^2+b^2)^2  &&2) \  (2a+3b)^2 \\
&3) \ (\sqrt{2}-3)^2  &&4) \  (b+7x)^2 \\
&5) \ (x^{\frac{1}{2}}+y^{\frac{1}{2}})^2   &&6) \ (10a+3t)(10a-3t)   \\
&7) \ (x^{\frac{1}{3}}+y^{\frac{1}{3}})(x^{\frac{1}{3}}-y^{\frac{1}{3}})  &&8) \ (\frac{3}{13}-\frac{2}{3}y)^2  \\
&9) \ (7-yx)(7+yx)   &&10) \ (3+2)^2   \\
\end{align*}
\section*{Opgave 4}
Brug kvadratsætningerne til at "regne baglæns".
\begin{align*}
&1) \  a^2+b^2+2ab  &&2) \  9a^2-9b^2   \\
&3) \  2a^2+3b^2-2\sqrt{6}ab  &&4) \  16a^2+49b^2+56ab   \\
&5) \   100x^2 - 25y^2  &&6) \   4x^2+3y^2+4\sqrt{3}xy   \\
&7) \  4x^4+16x^8+16x^6  &&8) \ x^{16}-y^{3}     
\end{align*}