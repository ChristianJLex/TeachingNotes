
\begin{center}
\Huge
Rødder for andengradspolynomier 2
\end{center}

\section*{Rødder for andengradspolynomier}
\stepcounter{section}

Hvis vi skal bestemme rødderne for et andengradspolynomium, så skal vi bruge løsningsformlen for andengradsligninger, som vi også til tider kalder for diskriminantformlen. Denne er givet ved følgende. 
\begin{setn}[Løsningsformlen for andengradsligninger]
	Rødderne for et andengradspolynomium $f$ givet ved
	\begin{align*}
		f(x) = ax^2+bx+c
	\end{align*}
	er givet ved
	\begin{align*}
		x = \frac{-b \pm \sqrt{d}}{2a},
	\end{align*}
	hvor vi antager, at diskriminanten $d = b^2 - 4ac$ er ikke-negativ.
\end{setn}
Grunden til, at vi kalder rodformlen for andengradspolynomier for løsningsformlen for andengradsligninger er, at en andengradsligning netop er en ligning, hvor løsningen er en rod til et andengradspolynomium. Vi vil bevise formlen senere, men vil for nu blot fokusere på at anvende den. 
\begin{exa}
	Et polynomium $f$ er givet ved
	\begin{align*}
		f(x) = x^2 - x - 2,
	\end{align*}
	og vi ønsker at bestemme rødderne. Vi indsætter koefficienterne i løsningsformlen og får
	\begin{align*}
		x &= \frac{-(-1) \pm \sqrt{(-1)^2-4\cdot 1 \cdot (-2)}}{2\cdot 1}\\
		&= \frac{1\pm \sqrt{9}}{2}\\
		&= \frac{1\pm 3}{2}.
	\end{align*}
	Rødderne er derfor givet ved $x = 2$ og $x = -1$. For at verificere løsningen tegner vi grafen for $f$. Denne kan ses af Fig. \ref{fig:rødder}
	\begin{figure}[H]
		\centering
		\begin{tikzpicture}
			\begin{axis}[
				axis lines = middle, 
				xmin = -2, xmax = 3,
				ymin = -3, ymax = 4
				]	
				\addplot[color = teal, samples = 100, thick] {x^2-x-2};
			\end{axis}
		\end{tikzpicture}
		\caption{Parabel for $f$}
		\label{fig:rødder}
	\end{figure}
	Det ser på Figur \ref{fig:rødder} ud til at rødderne er fundet korrekt.
\end{exa}
\subsection*{Opgave 1}
Bestem rødderne for følgende polynomier ved at bruge løsningsformlen for andengradsligninger. Verificér dit svar ved at tegne graferne i Maple. 

\begin{align*}
&1) \ x^2 + x - 2    &&2) \ x^2 - x - 6  \\
&3) \ x^2 - 4 x + 4    &&4) \ 2x^2-8x+4  \\
&5) \ x^2 + 5 x + 6   &&6) \ x^2 + 2 x - 8 \\
&7) \ 3x^2-3   &&8) \ x^2-3x+2  \\
&9) \ 2x^2+4x - 6    &&10) \ x^2 - 6 x + 9   \\
&11) \ 3 x^2 - 3 x - 6  &&12) \ 3 x^2 + 6 x + 3    \\
&13) \ x^2 + 6 x + 9 &&14) \ x^2 + 4x + 4   \\
\end{align*}

