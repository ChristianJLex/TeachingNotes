
\begin{center}
\Huge
Den naturlige logaritme
\end{center}
\stepcounter{section}

\section*{Den naturlige logaritme}

Vi så sidst på titalslogaritmen $\log$, der opfyldte at 
\begin{align*}
	\log(10^x) = x,
\end{align*}
og
\begin{align*}
	10^{\log(x)} = x.
\end{align*}
Vi skal i dag arbejde med den naturlige logaritme $\ln$, der kan bruges i forbindelse med \textit{den naturlige eksponentialfunktion }
\begin{align*}
	f(x) = \textnormal{e}^x,
\end{align*}
hvor $\textnormal{e} \approx 2.72$ er \textit{Eulers tal}. Vi kommer til at arbejde med Eulers tal igen senere. 

\begin{defn}[Den naturlige logaritme]
	Den naturlige logaritme $\ln$ er funktionen, der opfylder at 
	\begin{align*}
		\ln(\e^x) = x,
	\end{align*}
	og 
	\begin{align*}
		\e^{\ln(x)} = x.
	\end{align*}
\end{defn}

\begin{exa}
	Det gælder eksempelvis, at 
	\begin{align*}
		\ln(\e^{7}) = 7,
	\end{align*}
	eller 
	\begin{align*}
		\e^{\ln(3)} = 3.
	\end{align*}
\end{exa}
Som for titalslogaritmen gælder der tilsvarende nogle regneregler
\begin{setn}[Logaritmeregneregler]
	For $a,b>0$ gælder der, at
	\begin{enumerate}[label=\roman*)]
		\item $\ln(a\cdot b) = \ln(a)+ \ln(b)$,
		\item $\ln\left(\frac{a}{b}\right) = \ln(a)-\ln(b)$,
		\item $\ln(a^x) = x\ln(a).$
	\end{enumerate}		
\end{setn}
\begin{proof}
	Vi vil løbende udnytte, at $\ln(\e^a) = a$ og $\e^{\ln(a)} = a$. Vi betragter udtrykkene.
	\\
	i)
	\begin{align*}
		\ln(a\cdot b) &= \ln(\e^{\log(a)}\e^{\ln(b)}) \\
		&= \ln(\e^{\ln(a)+\ln(b)})\\
		&\ln(a) + \ln(b).
	\end{align*}
	ii)
	\begin{align*}
		\ln\left(\frac{a}{b}\right) &= \ln\left(\frac{\e^a}{\e^b}\right)\\
		&= \ln(\e^{\ln(a)-\ln(b)})\\
		&= \ln(a)-\ln(b).
	\end{align*}
	iii)
	\begin{align*}
		\ln(a^x) &= \ln\left( \left(\e^{\ln(a)}\right)^x\right)\\
		&= \ln\left(\e^{\ln(a)x}\right)\\
		&= x\ln(a),
    \end{align*}		
    og vi er færdige med beviset. 
\end{proof}


\subsection*{Opgave 1}
En tabel med funktionsværdier for $10^x$ er givet.
\begin{table}[H]
	\centering
	\begin{tabular}{c|c|c|c|c|c|c|c|c|c|c}
		$x$ & -4 & -3 & -2 & -1 & 0 & 1 & 2 & 3 & 4 & 5 \\
		\hline
		$\e^x$ & 0.018 & 0.049 & 0.135& 0.368 & 1 & 2.718 & 7.389 & 20.085 & 54.598 & 148.413
	\end{tabular}
\end{table}
Brug tabellen til at bestemme følgende.
\begin{align*}
	&1) \ \ln(20.085)     &&2) \ \ln(54.598)    \\
	&3) \ \ln(1)     &&4) \  \ln(0.018)		\\
	&5) \ \ln(0.049) &&6) \ \ln(148.413)    
\end{align*}


\subsection*{Opgave 2}
Bestem følgende udtryk 
\begin{align*}
	&1) \ \ln(\e^3)    &&2) \ \ln(\e^17)   \\  
	&3) \ \ln(\e^{\sqrt{4}})   &&4) \ \ln(\e)     \\  
	&5) \ \ln(1)   &&6) \ \ln(\e^{-0.157})   \\  
\end{align*}


\subsection*{Opgave 3}
Vi har set på logaritmerne $\ln$ og $\log$, der er den omvendte funktion til $\e^x$ og $10^x$ henholdsvist. For hver eksponentialfunktion $a^x$ findes en tilsvarende logaritme
$\log_a$. Der gælder eksempelvist at
\begin{align*}
	\log_5(5^3) = 3
\end{align*}
eller 
\begin{align*}
	\log_2(8) = 3.
\end{align*}
Vi skal altså for $\log_a(x) = y$ finde det tal $y$, som vi skal opløfte $a$ i for at få $x$.  \\

Brug dette til at bestemme følgende.
\begin{align*}
	&1) \  \log_{10}(1000)      &2) \  \log_5(25)       \\
	&3) \  \log_2(16)       &4) \ \log_3(9)         \\
	&5) \  \log_3(27)      &6) \  \log_7(1)      \\
	&7) \  \log_8(1)      &8) \  \log_2(1024)      \\
	&9) \  \log_{10}(1000000)     &10) \  \log_4(64)        \\
\end{align*}


\subsection*{Opgave 4}

Aflevering