\begin{center}
\Huge
Opgaver om eksponentialfunktioner
\end{center}

\subsection*{Opgave 1}

\begin{enumerate}[label=\roman*)]
	\item Forøg 20 med $10\%$.
	\item Formindsk 100 med $50\%$.
	\item Bestem $80\%$ af 300.
	\item Hvor mange $\% $ udgør 6 af 60?
	\item Bestem den procentvise forskel fra $200$ til $250$.
\end{enumerate}

\subsection*{Opgave 2}

\begin{enumerate}[label=\roman*)]
	\item En eksponentialfunktion $f$ er givet ved
	\begin{align*}
		f(x) = 2\cdot 1.54^x.
	\end{align*}
	Bestem fremskrivningsfaktoren og vækstraten for $f$. Afgør desuden hvor mange procent $f$ stiger/falder med, hvis $x$ øges med 1.
	\item En eksponentialfunktion $f$ er givet ved
	\begin{align*}
		f(x) = 100\cdot 0.96^x.
	\end{align*}
	Bestem fremskrivningsfaktoren og vækstraten for $f$. Afgør desuden hvor mange procent $f$ stiger/falder med, hvis $x$ øges med 1.
	\item En eksponentialfunktion $f$ er givet ved
	\begin{align*}
		f(x) = \sqrt{2}\cdot 3^x.
	\end{align*}
	Bestem fremskrivningsfaktoren og vækstraten for $f$. Afgør desuden hvor mange procent $f$ stiger/falder med, hvis $x$ øges med 1.
\end{enumerate}


\subsection*{Opgave 3}
a) En eksponentialfunktion $f$ er givet ved
\begin{align*}
	f(x) = 11\cdot 1.97^x.
\end{align*}
\begin{enumerate}[label=\roman*)]
	\item Bestem $f(10)$.
	\item Bestem $f(-2)$.
\end{enumerate}

b) En eksponentialfunktion $g$ har begyndelsesværdi $2$ og fremskrivningsfaktor $0.11$.
\begin{enumerate}[label=\roman*)]
	\item Opskriv forskriften for $g$. 
	\item Bestem $g(3)$. 
\end{enumerate}

\subsection*{Opgave 4}

Følgende grafer for eksponentialfunktioner er givet.
\begin{enumerate}[label=\roman*)]
	\item Bestem $b$ for eksponentialfunktionerne
	\item Afgør, om $a$ er større eller mindre end 1.
\end{enumerate}
\begin{center}
	\resizebox{0.4\textwidth}{!}{
	\begin{tikzpicture}
		\begin{axis}[axis lines = middle, xmin = -1, xmax = 3,
		ytick = {1,2,...,10},
		ymin = -1,
		xlabel = $x$, ylabel = $y$]
			\addplot[teal, thick, domain = -1:4, samples = 200] {3*1.5^x};
		\end{axis}
	\end{tikzpicture}
	}
	\resizebox{0.4\textwidth}{!}{
	\begin{tikzpicture}
		\begin{axis}[axis lines = middle, xmin = -1, xmax = 4,
		ytick = {1,2,...,10},
		ymin = -1,
		xlabel = $x$, ylabel = $y$]
			\addplot[teal, thick, domain = -1:4, samples = 200] {6 * 0.6^x};
		\end{axis}
	\end{tikzpicture}
	}
	\resizebox{0.4\textwidth}{!}{
	\begin{tikzpicture}
		\begin{axis}[axis lines = middle, xmin = -0.2, xmax = 4,
		ytick = {1,2,...,12},
		ymin = -1,
		xlabel = $x$, ylabel = $y$]
			\addplot[teal, thick, domain = -0.2:4, samples = 200] {9 * 0.2^x};
		\end{axis}
	\end{tikzpicture}
	}
	\resizebox{0.4\textwidth}{!}{
	\begin{tikzpicture}
		\begin{axis}[axis lines = middle, xmin = -1, xmax = 3,
		ytick = {1,2,...,16},
		ymin = -1,
		xlabel = $x$, ylabel = $y$]
			\addplot[teal, thick, domain = -1:3, samples = 200] {2 * 2^x};
		\end{axis}
	\end{tikzpicture}
	}
\end{center}


\subsection*{Opgave 5}

Bestem forskriften for de eksponentialfunktioner, der går gennem følgende par af punkter.
\begin{align*}
&1) \ (0,3), \ (1,6)  &&2) \ (1,3), \ (3,27)      \\
&3) \ (1,2), \  (3,8)   &&4) \  (2,8), \ (5,64)    
\end{align*}


\subsection*{Opgave 6}

To biologistuderende har brugt absorbansen af en væske til at måle antallet af bakterier i væsken. De antager, at sammenhængen mellem den forløbne tid (i minutter) og antallet af bakterier (i mia.) kan beskrives ved en eksponentiel sammenhæng. Deres data kan findes \href{https://github.com/ChristianJLex/TeachingNotes/raw/master/2022-2023/Data%20og%20lign/Bakteriedata.xlsx}{\color{blue!60} her}.

\begin{enumerate}[label=\roman*)]
	\item Lav eksponentiel regression og lineær regression på datasættet.
	\item Lav en kvalitativ vurdering af de to modeller. Hvilken virker til at være bedst?
	\item Sammenlign forklaringsgraderne for de to modeller. Hvilken en er bedst? Tror du, at forklaringsgrader kan bruges til at sammenligne forskellige regressionsmodeller?
	\item Brug din valgte model til at afgøre, hvornår der vil være 1 bio. bakterier i væsken. 
\end{enumerate}

\subsection*{Opgave 7}

Bestem følgende.
\begin{align*}
	&1) \  \log_{10}(1000)      &2) \  \log_5(25)       \\
	&3) \  \log_2(16)       &4) \ \log_3(9)         \\
	&5) \  \log_3(27)      &6) \  \log_7(1)      \\
	&7) \  \log_8(1)      &8) \  \log_2(1024)      \\
	&9) \  \log_{10}(1000000)     &10) \  \log_4(64)        \\
\end{align*}

\subsection*{Opgave 8}
Bestem fordoblingskonstanten eller halveringskonstanten for følgende eksponentialfunktioner
\begin{align*}
	&1) \ 7\cdot 1.5^x   &&2) \ 15\cdot 1.2^x   \\
	&3) \ 10.5\cdot 1.04^x   &&4) \ 1.02\cdot 0.97^x   \\
	&5) \ 0.95\cdot 1.95^x   &&6) \ 22.9\cdot 15^x   \\		
\end{align*}

\subsection*{Opgave 9}
Tetrahydrocannabinol (THC) er det primære psykoaktive stof i hampplanten. Koncentrationen af stoffet efter indtagelse hos en bestemt person kan ses af Tabel \ref{tab:cannabis}.

\begin{table}[H]
	\centering
	\rowcolors{2}{gray!25}{white}
	\begin{tabular}{cc}
		\textbf{Tid i timer} & \textbf{Koncentration i mg/L} \\

		0 & 0.30249 \\

		8 & 0.24512 \\
 
		16 &  0.20132 \\

		24 & 0.16371 \\

		32 & 0.13303\\

		40 & 0.11069\\

		48 & 0.08757\\
		56 & 0.07317\\

		64 & 0.05748
	\end{tabular}
	\caption{Koncentration af tetrahydrocannabiol (THC)}
	\label{tab:cannabis}
\end{table}

Det antages at sammenhængen mellem antal forløbne timer $x$ og blodkoncentrationen af THC $f(x)$ er givet ved en sammenhæng af typen
\begin{align*}
	f(x) = b \cdot a^x
\end{align*}
\begin{enumerate}[label=\roman*)]
	\item Lav regression på observationerne fra Tabel \ref{tab:cannabis} og bestem en forskrift for $f$.
	\item Bestem halveringskonstanten for $f$.
\end{enumerate}
Et stof eller lægemiddel vil betragtes som helt metaboliseret efter fem halveringstider.

\begin{enumerate}[label=\roman*)]
	\setcounter{enumi}{2}
	\item Hvor længe skal der gå, før vi kan sige, at der ikke er mere THC i kroppen på personen?
\end{enumerate}
