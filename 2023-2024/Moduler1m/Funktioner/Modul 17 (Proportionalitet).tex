\begin{center}
	\Huge
	Proportionalitet
\end{center}
\section*{Ligefrem proportionalitet}
\stepcounter{section}

\begin{defn}[Ligefrem proportionalitet]
	To variable $x$ og $y$ siges at være \textit{ligefrem proportionale} eller blot proportionale, hvis $y = a\cdot x$. Konstanten $a$ kaldes for \textit{proportionalitetsfaktoren} eller
	\textit{proportionalitetskonstanten}.
\end{defn}

\begin{exa}
	\label{exa:slik}
	Prisen på 100g bland-selv-slik er 12 kr. Prisen på bland-selv-slik i kr er dermed proportional med vægten i gram. Proportionalitetsfaktoren er 0.12. Derfor kan prisen $P$ opskrives som funktion af 		vægten $x$ som
 	\begin{align*}
 		P(x) = 0.12 \cdot x
 	\end{align*}
\end{exa}

\begin{exa}
	Hvis vi har en sammenhæng mellem $x$ og $y$ givet ved
	\begin{align*}
		\frac{y}{x} = 2,
	\end{align*}
	så vil $x$ og $y$ være proportionale med proportionalitetsfaktoren $2$, da vi kan omskrive ligningen til
	\begin{align*}
		y = 2x.
	\end{align*}
\end{exa}

\section*{Omvendt proportionalitet}
\stepcounter{section}

\begin{defn}[Omvendt proportionalitet]
	To variable $x$ og $y$ siges at være \textit{omvendt proportionale}, hvis $y\cdot x = a$. 
\end{defn}
Vi bemærker, at vi i tilfældet af, at $x$ og $y$ er omvendt proportionale kan skrive
\begin{align*}
	y = a \cdot \frac{1}{x}.
\end{align*}
Da $\frac{1}{x} = x^{-1}$, så er omvendt proportionalitet faktisk et særtilfælde af en potenssammenhæng, da
\begin{align*}
	y &= a \cdot \frac{1}{x} \\
	&= a \cdot x^{-1},
\end{align*}
hvor $b$-værdien er lig $a$, og $a$-værdien er lig $-1$ for omvendt proportionalitet som en potenssammenhæng. 

\begin{exa}
	Sammenhængen mellem $y$ og $x$ givet ved
	\begin{align*}
		y\cdot x = 10
	\end{align*}
	er omvendt proportional.
\end{exa}

\begin{exa}
	Sammenhængen mellem $y$ og $x$ givet ved
	\begin{align*}
		y = -4 \cdot \frac{1}{x}
	\end{align*}
	er omvendt proportional, da vi kan omskrive den til
	\begin{align*}
		y\cdot x = -4.
	\end{align*}
\end{exa}


Hvilke af følgende variabelsammenhænge er proportionale, omvendt proportionale eller ingen af delene
\subsection*{Opgave 1}

\begin{align*}
	&1) \ y = 3x   &&2) \  7 = -7.32x \\
	&3) \ \frac{y}{x} = 1.2  &&4) y\cdot x = -1 \\
	&5) \  y = \frac{1}{x} &&6) \  x\cdot y = 27    \\
	&7) \ \frac{y}{x} = 27 &&8) \  y=x^2   \\
	&9) \ 2y = 3x &&10) \ 10x^3 = x^2y    \\
\end{align*}

\subsection*{Opgave 2}
For følgende beskrivelser opskriv da en sammenhæng mellem de beskrevne variable lig Eksempel \ref{exa:slik}.
\begin{enumerate}[label=\roman*)]
	\item Prisen på tyggegummi er proportional med antallet af købte pakker. Proportionalitetskonstanten er 12.
	\item Den tid, det tager at køre 20km i en bil er omvendt proportional med den kørte hastighed.
	\item Den mængde mel, du kan købe for 100 kroner er omvendt proportional med kiloprisen på melet. 
\end{enumerate}

\subsection*{Opgave 3}
For en person på 80kg er BMI (body-mass index) omvendt proportional med højden i meter $h$ i anden. Dette kan skrives som
\begin{align*}
	\textnormal{BMI} \cdot h^2 = 80
\end{align*}
\begin{enumerate}[label=\roman*)]
	\item Hvad er BMI for en person på 1.7 meter med denne vægt?
	\item Hvad er BMI for en person på 2.0 meter med denne vægt?
	\item Hvor høj er man, hvis man har en BMI på 25?
\end{enumerate}

\subsection*{Opgave 4}
For en bil er bremselængden $f$ (i meter) proportional med hastigheden $x$ (i m/s) i anden. Proportionalitetskonstanten er 0.01.

\begin{enumerate}[label=\roman*)]
	\item Opstil en sammenhæng mellem $f$ og $x$.
	\item Bestem bremselængden for en bil, der kører 30m/s
	\item Hvor stærkt kører en bil, der har en bremselængde på 100m?
\end{enumerate}
