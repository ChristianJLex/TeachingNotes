
\begin{center}
	\Huge
	Prikprodukt og vinkel mellem vektorer
\end{center}

\section*{Vinkel og prikprodukt i Maple}
\stepcounter{section}

Vi lægger ud med et eksempel, der viser hvordan vi definerer en vektor i Maple samt hvordan vi laver prikprodukt og bestemmer vinkler mellem vektorer.

\begin{exa}
	Vi får givet vektorerne $\vv{u}$ og $\vv{v}$ givet ved
	\begin{align*}
		&\vv{u} = 
		\begin{pmatrix} 
			2 \\ 5
		\end{pmatrix} & &\textnormal{og} & & 
		\vv{v} = 
		\begin{pmatrix}
			-4 \\ 3
		\end{pmatrix}. 
	\end{align*}
	Skal vi definere vektorerne i Maple, så skal vi skrive
	\begin{align*}
		&\texttt{with(Gym):} \\
		&\texttt{$\vv{\texttt{u}}$ := <2 , 5>} \\
		&\texttt{$\vv{\texttt{v}}$ := <-4 , 3>}
	\end{align*}
	Ønsker vi at lægge to vektorer sammen, skriver vi
	\begin{align*}
		\vv{\texttt{u}} + \vv{\texttt{v}}
	\end{align*}
	Ønsker vi at trække dem fra hinanden, skriver vi
	\begin{align*}
		\vv{\texttt{u}} - \vv{\texttt{v}}
	\end{align*}
	Og ønsker vi at skalere dem med et tal $k$, skriver vi
	\begin{align*}
		\texttt{k$\vv{\texttt{u}}$}
	\end{align*}
	Skal vi bestemme prikproduktet mellem de to vektorer, så skal vi skrive
	\begin{align*}
		\texttt{dotP($\vv{\texttt{u}},\vv{\texttt{v}}$)} 
	\end{align*}
	Ønsker vi at bestemme vinklen mellem vektorerne, så skal vi skrive
	\begin{align*}
		\texttt{vinkel($\vv{\texttt{u}},\vv{\texttt{v}}$)}
	\end{align*}
	Ønsker vi til slut at bestemme længden af en vektor, skal vi skrive
	\begin{align*}
		\texttt{len($\vv{\texttt{u}}$)}
	\end{align*}
\end{exa}

Når Maple bestemmer vinklen mellem to vektorer, så bruger den følgende resultat.
\begin{setn}[Vinkel mellem vektorer]
	For to vektorer $\vv{u}$ og $\vv{v}$ gælder der, at 
	\begin{align*}
		\frac{\vv{u}\cdot \vv{v}}{|\vv{u}||\vv{v}|} = \cos(\theta),
	\end{align*}
	hvor $\theta$ er vinklen mellem vektorerne $\vv{u}$ og $\vv{v}$.
\end{setn}
Vi behøver dog ikke bruge denne sætning, da vi blot kan bruge \texttt{vinkel}-kommandoen i Maple.


\subsection*{Opgave 1}


\subsection*{Opgave 7}
Fire vektorer er givet ved
\begin{align*}
	&\vv{a} = 
	\begin{pmatrix}
		4 \\ -3
	\end{pmatrix} 
	&
	&\vv{b} = 
	\begin{pmatrix}
		-2 \\ 5
	\end{pmatrix}
	\\
	&\vv{c} = 
	\begin{pmatrix}
		6 \\ 10
	\end{pmatrix}
	&
	&\vv{d} = 
	\begin{pmatrix}
		-11 \\ -7
	\end{pmatrix}
\end{align*}

\begin{enumerate}[label=\roman*)]
	\item Bestem $4\vv{a}$.
	\item Bestem $\vv{a} + \vv{b}$.
	\item Bestem $2\vv{b} - \vv{c}$.
	\item Bestem $\vv{a}+\vv{b}+\vv{c}+\vv{d}$.
	\item Bestem $-2\vv{d} - \vv{a}$.
	\item Bestem $\vv{a}-\vv{b}+\vv{c}-3\vv{d}$.
\end{enumerate}

\subsection*{Opgave 2}

Fire punkter er givet ved $A(1,3)$, $B(-5,6)$, $C(-4,1)$ og $D(12,-7)$.
\begin{enumerate}[label = \roman*)]
	\item Bestem koordinaterne til vektorerne $\vv{AB}$, $\vv{BC}$ og $\vv{DA}$.
	\item Bestem længden af vektorerne $\vv{AB}$, $\vv{BC}$ og $\vv{DA}$
	\item Bestem $\vv{BA} + \vv{CD}$.
	\item Bestem $|2\vv{CA} - 3\vv{BC} + \vv{DB}|$.
\end{enumerate}

\subsection*{Opgave 3}

\begin{enumerate}[label=\roman*)]
\item Bestem prikproduktet og vinklen mellem følgende vektorer
\begin{align*}
&1) \ \begin{pmatrix}1 \\ 0\end{pmatrix} \textnormal{ og } \begin{pmatrix}0 \\ 1\end{pmatrix}    &&2) \  \begin{pmatrix} 4 \\  5\end{pmatrix} \textnormal{ og } \begin{pmatrix} -2  \\ 6\end{pmatrix}    \\
&3) \ \begin{pmatrix}12 \\ 15\end{pmatrix} \textnormal{ og } \begin{pmatrix}-3 \\ 14 \end{pmatrix}   &&4) \ \begin{pmatrix} -2 \\ -5\end{pmatrix} \textnormal{ og } \begin{pmatrix} 0.5 \\ -12 \end{pmatrix}     \\
\end{align*}
\end{enumerate}

\subsection*{Opgave 4}

Afgør hvilke af følgende par af vektorer, der er orthogonale

\begin{center}
\resizebox{0.45\textwidth}{!}{
\begin{tikzpicture}
	\begin{axis}[
		axis lines = center, 
		xmin = -4.5, xmax = 2.5, 
		ymin = -4.5, ymax = 2.5,
		grid,
		x = 1cm, y = 1cm,
		xtick = {-4,-3,...,2,3}, ytick = {-4,-3,...,2,3},
		xlabel = $x$, ylabel = $y$	
	]
		\draw[-{Stealth[scale=1.5]}, thick, color = teal] (axis cs: -3,-2) -- (axis cs: 2,-4);
		\draw[-{Stealth[scale=1.5]}, thick, color = teal] (axis cs: -2,2) -- (axis cs: -4,-4);
	\end{axis}
\end{tikzpicture}
}
\resizebox{0.45\textwidth}{!}{
\begin{tikzpicture}
	\begin{axis}[
		axis lines = center, 
		xmin = -1.5, xmax = 6.5, 
		ymin = -1.5, ymax = 6.5,
		grid,
		x = 1cm, y = 1cm,
		xtick = {-1,0,...,5,6}, ytick = {-1,0,...,5,6},
		xlabel = $x$, ylabel = $y$	
	]
		\draw[-{Stealth[scale=1.5]}, thick, color = teal] (axis cs: 4,4) -- (axis cs: 6,6);
		\draw[-{Stealth[scale=1.5]}, thick, color = teal] (axis cs: 5,0) -- (axis cs: 0,5);
	\end{axis}
\end{tikzpicture}
}

\end{center}

\subsection*{Opgave 5}

Vi får givet følgende to vektorer
\begin{align*}
	&\vv{u} =
	\begin{pmatrix}
		t-1 \\ 2
	\end{pmatrix} & &\textnormal{og} &
	&\vv{v} = 
	\begin{pmatrix}
		3 \\ -t + 3
	\end{pmatrix}
\end{align*}
\begin{enumerate}[label=\roman*)]
	\item Bestem prikproduktet mellem $\vv{u}$ og $\vv{v}$, hvis $t = 4$.
	\item Bestem $t$ så $\vv{u}$ og $\vv{v}$ er orthogonale. 
\end{enumerate}

\subsection*{Opgave 6}


To vektorer er givet ved
\begin{align*}
	&\vv{a} = 
	\begin{pmatrix}
		x^2 - 2 \\
		1	
	\end{pmatrix}& 
	&\textnormal{ og } &
	&\vv{b} = 
	\begin{pmatrix}
		1 \\
		x
	\end{pmatrix}
\end{align*}
\begin{enumerate}[label = \roman*)]
	\item Bestem $x$, så $\vv{a}$ og $\vv{b}$ er orthogonale.
\end{enumerate}