
\begin{center}
\Huge
Vektorer
\end{center}

\subsection*{Opgave 1}
En vektor er en pil indlagt i et koordinatsystem, og vi skal i denne opgave arbejde med, hvordan man kan repræsentere vektorer og anvende deres længder, når man skal bestemme afstande på et kort. 

Vi skal i denne opgave kunne
\begin{itemize}
	\item[$\cdot$] Tegne en vektor i GeoGebra ud fra to punkter,
	\item[$\cdot$] Bestemme længden af en vektor i GeoGebra,
	\item[$\cdot$] Fortolke koordinaterne til en vektor,
	\item[$\cdot$] Tegne en vektor ved brug af vektorens koordinater.
\end{itemize}

Skal vi tegne en vektor i GeoGebra, skal vi lave to punkter i koordinatsystemet. Lad os kalde disse \texttt{A} og \texttt{B}. Vi skal derefter for at lave vektoren, der går fra \texttt{A} til \texttt{B} skrive
\begin{align*}
	\texttt{v = Vektor(A,B)}
\end{align*}
Ønsker vi nu at bestemme længden af vektoren, så kan vi skrive 
\begin{align*}
	\texttt{Længde(v)}
\end{align*}



Denne opgave tager udgangspunkt i GeoGebra-filen med et kort over Husum. Vi lægger ud med at prøve at gennemskue, hvad koordinaterne i en vektor beskriver.

\begin{enumerate}[label = \roman*)]
	\item Tegn en vektor $\vv{v}$ fra krydset mellem Mørkhøjvej og Frederiksundsvej til Nørre.
	\item Bestem længden af $\vv{v}$.
	\item I algebravinduet til venstre kan vi se koordinaterne for $\vv{v}$. Diskutér i jeres gruppe hvad koordinaterne for vektoren beskriver. 
\end{enumerate}

\newpage
\subsection*{Opgave 2}

Vi skal løbe en tur. Vi starter og slutter på Nørre og skal forbi følgende steder i Brønshøj-Husum.
\begin{itemize}
	\item[$\cdot$] Husum torv
	\item[$\cdot$] Brønshøj torv
	\item[$\cdot$] Lidl Tingbjerg
	\item[$\cdot$] Føtex Husum
	\item[$\cdot$] Tingbjerg Kirke
	\item[$\cdot$] Emdrup Kirke
\end{itemize}
Det er desuden et krav, at vi løber mindst 1000 meter langs volden og 1000 meter langs Utterslev mose. Ruten må heller ikke krydse sig selv. 

\begin{enumerate}[label = \roman*)]
	\item Indtegn en løberute, der opfylder disse betingelser ved brug af vektorer i GeoGebra
\end{enumerate}
Vi skal finde ud af, hvor lang til vi vil bruge på at løbe ruten. Vi har tidligere løbet fra BIG i Herlev til Brønshøj torv på 25 minutter, og vi forventer at kunne løbe turen med samme hastighed.
\begin{enumerate}[label = \roman*)]
	\setcounter{enumi}{1}
	\item Afgør, hvor længe det vil tage at løbe ruten.
\end{enumerate}

\subsection*{Opgave 3}
Vi modtager en rutevejledning. Vi får at vide, at vi skal starte på Husum Torv og herefter følge ruten lagt af følgende vektorer
\begin{align*}
	&\vv{a} = \begin{pmatrix}
		295.98 \\
		-138.38
	\end{pmatrix}
	&
	&\vv{b} = \begin{pmatrix}
		134.54 \\
		-868.72
	\end{pmatrix}
	&
	&\vv{c} = \begin{pmatrix}
		269.07\\
		80.72
	\end{pmatrix}
	&
	&\vv{d} = \begin{pmatrix}
		399.77 \\
		622.71
	\end{pmatrix}
		\\
	&\vv{e} = \begin{pmatrix}
		192.20 \\
		-126.85
	\end{pmatrix}
	&
	&\vv{f} = \begin{pmatrix}
		59.62 \\
		345.35
	\end{pmatrix}
	&
	&\vv{g} = \begin{pmatrix}
		124.88 \\
		158.20
	\end{pmatrix}
	&
	&\vv{h} = \begin{pmatrix}
		184.51\\
		-11.53
	\end{pmatrix}
\end{align*}

\begin{enumerate}[label=\roman*)]
	\item Lav ruten i GeoGebra og afgør, hvor vi ender.
	\item Hvor lang tid vil det tage at gå ruten, hvis du går med en hastighed på 5km/t?
\end{enumerate}
