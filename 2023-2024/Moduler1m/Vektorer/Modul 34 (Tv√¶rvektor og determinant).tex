\begin{center}
	\Huge
	Tværvektorer og determinanter
\end{center}

\section*{Tværvektor}
\stepcounter{section}

Ønsker vi at bestemme en vektor, der står vinkelret på en vektor, kan vi konstruere den vektor, vi kalder for tværvektoren.

\begin{defn}[Tværvektor]
	For en vektor $\vv{v}$ med koordinaterne
	\begin{align*}
		\vv{v} = 
		\begin{pmatrix}
			v_1 \\ v_2
		\end{pmatrix}
	\end{align*}
	defineret \textit{tværvektoren} til $\vv{v}$ som
	\begin{align*}
		\hat{\vv{v}} = 
		\begin{pmatrix}
			-v_2 \\ v_1
		\end{pmatrix}.
	\end{align*}
\end{defn}	

På Figur \ref{fig:tværvektor} kan vi se, tværvektoren for en vektor.

\begin{figure}[H]
	\centering
	\begin{tikzpicture}
		\begin{axis}[
			axis lines = center, 
			xmin = -1, xmax = 6.5, 
			ymin = -1, ymax = 6.5, 
			x = 1.5cm, y = 1.5cm,
			ticks = none, 
			xlabel = $x$, ylabel = $y$
		]
			\draw[-{Stealth[scale=1.5]}, thick, color = teal] (axis cs: 4,1) -- (axis cs: 6,4) node[midway, xshift = 7pt, yshift = -7pt] {$\vv{v}$};
			\draw[-{Stealth[scale=1.5]}, thick, color = teal] (axis cs: 4,1) -- (axis cs: 1,3) node[midway,xshift = 7pt, yshift = 7pt] {$\hat{\vv{v}}$};

		\end{axis}
	\end{tikzpicture}
	\caption{En vektor og dens tværvektor}
	\label{fig:tværvektor}
\end{figure}

\begin{exa}
	Vektoren
	\begin{align*}
		\begin{pmatrix}
			4 \\ 2
		\end{pmatrix}
	\end{align*}
	har vektoren 
	\begin{align*}
		\begin{pmatrix}
			-2 \\ 4
		\end{pmatrix}
	\end{align*}
	som tværvektor. 
\end{exa}

\section*{Determinanter}
\stepcounter{section}

\begin{defn}[Determinanter]
	For to vektorer $\vv{u}$ og $\vv{v}$ givet ved
	\begin{align*}
		&\vv{u} =
		\begin{pmatrix}
			u_1 \\ u_2
		\end{pmatrix}& &\textnormal{og} &
		&\vv{v} = 
		\begin{pmatrix}
			v_1 \\ v_2
		\end{pmatrix}
	\end{align*}
	 defineres \textit{determinanten} mellem $\vv{u}$ og $\vv{v}$ som
	\begin{align*}
		\det(\vv{u},\vv{v}) = u_1v_2 - u_2v_1.
	\end{align*}
	Dette skrives også til tider
	\begin{align*}
		\det(\vv{u},\vv{v}) = 
		\left|
		\begin{array}{cc}
			u_1 & v_1 \\
			u_2 & v_2
		\end{array}
		\right|
	\end{align*}
\end{defn}

\begin{exa}
	For vektorerne
	\begin{align*}
		&\vv{u} =
		\begin{pmatrix}
			2 \\ 1
		\end{pmatrix}& &\textnormal{og} &
		&\vv{v} = 
		\begin{pmatrix}
			-3 \\ 5
		\end{pmatrix}
	\end{align*}	 
	er determinanten givet ved
	\begin{align*}
		\det(\vv{u},\vv{v}) = 2\cdot 5 - 1\cdot(-3) = 10+3 = 13
	\end{align*}
\end{exa}

\begin{setn}[Areal og determinant]
	For to vektorer $\vv{u}$ og $\vv{v}$ gælder der, at arealet af deres 
	udspændende parallelogram er lig $|\det(\vv{u},\vv{v})|$.
\end{setn}

På Figur \ref{fig:determinant} kan vi se arealet, som determinanten tilsvarer

\begin{figure}[H]
	\centering
	\begin{tikzpicture}
		\begin{axis}[
			axis lines = center, 
			xmin = -1, xmax = 5, 
			ymin = -1, ymax = 5, 
			x = 1.5cm, y = 1.5cm,
			ticks = none, 
			xlabel = $x$, ylabel = $y$
		]
			\draw[-{Stealth[scale=1.5]}, thick, color = teal] (axis cs: 0,0) -- (axis cs: 2,1) node[midway, xshift = 7pt, yshift = -7pt] {$\vv{u}$};
			\draw[-{Stealth[scale=1.5]}, thick, color = teal] (axis cs: 0,0) -- (axis cs: 1,3) node[midway,xshift = 7pt, yshift = -7pt] {$\vv{v}$};
			\fill[olive,nearly transparent] (axis cs: 0,0) -- (axis cs: 2,1) -- (axis cs:3,4) -- (axis cs: 1,3) -- cycle;
			\node[color = teal] at (axis cs:1.5,2) {$|\det(\vv{u},\vv{v})|$};

		\end{axis}
	\end{tikzpicture}
	\caption{Determinant som areal}
	\label{fig:determinant}
\end{figure}

Skal vi bestemme determinanten i Maple, så skal vi skrive
\begin{align*}
	&\texttt{with(Gym):}\\
	&\texttt{$\vv{\texttt{u}}$ := <u$_{1}$,u$_{2}$>}\\
	&\texttt{$\vv{\texttt{v}}$ := <v$_{1}$,v$_{2}$>}\\
	&\texttt{det($\vv{\texttt{u}}$,$\vv{\texttt{v}}$)}
\end{align*}

\newpage
\subsection*{Opgave 1}

Bestem tværvektoren for følgende vektorer 
\begin{align*}
	&a) \ 
	\begin{pmatrix}
		2 \\ 5
	\end{pmatrix}		
	&
    &b) \ 
    \begin{pmatrix}
    		-7 \\ 11
    \end{pmatrix}   \\
    &c) \
    \begin{pmatrix}
    		3 \\ -0.5
    \end{pmatrix}
    &
    &d) \
    \begin{pmatrix}
    		-\sqrt{3} \\ -27
    \end{pmatrix}
\end{align*}

\subsection*{Opgave 2}

Bestem determinanten for følgende par af vektorer

\begin{align*}
	&a) \	
	\begin{pmatrix}
		3 \\ -2
	\end{pmatrix}
	\textnormal{ og }
	\begin{pmatrix}
		5 \\ 1
	\end{pmatrix}
	&
	&b) \ 
	\begin{pmatrix}
		6 \\ 10
	\end{pmatrix}
	\textnormal{ og }
	\begin{pmatrix}
		4 \\ -7
	\end{pmatrix}
\end{align*}

\subsection*{Opgave 3}
Udregn følgende
\begin{align*}
	&a) \ 
	\left|
	\begin{array}{cc}
		5 & 7 \\
		-2 & -5
	\end{array}
	\right|
	&
	&b) \
	\left|
	\begin{array}{cc}
		-9 & 4 \\
		4 & -6
	\end{array}
	\right|
\end{align*}

\newpage
\subsection*{Opgave 4}

Beregn arealet af det skraverede område på følgende figur.

\begin{center}
	\begin{tikzpicture}
		\begin{axis}[
			axis lines = center, 
			xmin = -3, xmax = 7, 
			ymin = -3, ymax = 7, 
			x = 1.5cm, y = 1.5cm,
			grid,
			xlabel = $x$, ylabel = $y$
		]
			\draw[-{Stealth[scale=1.5]}, thick, color = teal] (axis cs: -2,-1) -- (axis cs: 2,-2) node[midway, xshift = -7pt, yshift = -7pt] {$\vv{u}$};
			\draw[-{Stealth[scale=1.5]}, thick, color = teal] (axis cs: -2,-1) -- (axis cs: -1,6) node[midway,xshift = -7pt, yshift = 7pt] {$\vv{v}$};
			\fill[olive,nearly transparent] (axis cs: -2,-1) -- (axis cs: 2,-2) -- (axis cs:3,5) -- (axis cs: -1,6) -- cycle;
		\end{axis}
	\end{tikzpicture}
\end{center}

\subsection*{Opgave 5}
Vektorerne $\vv{a}$ og $\vv{b}$ givet ved
\begin{align*}
	&\vv{a} =	
	\begin{pmatrix}
		2 \\ -4
	\end{pmatrix}
	&
	&\textnormal{og}
	&
	&\vv{b} =
	\begin{pmatrix}
		4 \\ -8
	\end{pmatrix}
\end{align*}
er parallelle, da $2\vv{a} = \vv{b}$.

\begin{enumerate}[label=\roman*)]
	\item Bestem $\det(\vv{a},\vv{b})$. Hvorfor giver resultatet god mening? Tænk på den geometriske fortolkning af determinanten
	\item Hvordan kan vi undersøge, om to vektorer er parallelle?
\end{enumerate} 


\subsection*{Opgave 6}

To vektorer $\vv{u}$ og $\vv{v}$ er givet ved
\begin{align*}
		&\vv{u} =
		\begin{pmatrix}
			t+2 \\ 5
		\end{pmatrix}
		& 
		&\textnormal{og} 
		&
		&\vv{v} =
		\begin{pmatrix}
			-t+3 \\ 4
		\end{pmatrix}
\end{align*}	
\begin{enumerate}[label=\roman*)]
	\item Bestem $\det(\vv{u},\vv{v})$, hvis $t = 6$
	\item Bestem $t$, så arealet af vektorernes udspændende parallelogram er 20.
\end{enumerate}

\subsection*{Opgave 7}
\begin{enumerate}[label=\roman*)]
	\item Bevis, at en vektor $\vv{v}$ og dens tværvektor $\hat{\vv{v}}$ er orthogonale
	\item Bevis, at der for to vektorer $\vv{u}$ og $\vv{v}$ givet ved
	\begin{align*}
		&\vv{u} =
		\begin{pmatrix}
			u_1 \\ u_2
		\end{pmatrix}
		& 
		&\textnormal{og} 
		&
		&\vv{v} =
		\begin{pmatrix}
			v_1 \\ v_2
		\end{pmatrix}
	\end{align*}
	gælder, at
	\begin{align*}
		\det(\vv{u},\vv{v}) = \hat{\vv{u}}\cdot \vv{v}.
	\end{align*}
\end{enumerate}