\documentclass[12pt,x11names,a4paper]{article}
\input{preamble}


\newgeometry{margin=2cm}

\pagestyle{fancy}
\fancyhf{}

\rhead{Nørre Gymnasium\\1.e
}
\cfoot{Side \thepage \hspace{1pt} af \pageref{LastPage}}

%Husk at rette modul og dato!
\lhead{Aflevering 2 \\ Matematik A
}
\chead{9. december 2022
}

\begin{document}

%\includepdf[pages=-]{Forsider/aarsprove_1v.pdf}
\savegeometry{art}

\begin{titlepage}
\newgeometry{margin=0pt}

\begin{minipage}{0.27\textwidth}

\begin{tikzpicture}[overlay]
\fill[top color = NorregGroen!40, bottom color = NorregGroen] (6,10) rectangle (-10,-30);
\end{tikzpicture}
\end{minipage}
\begin{minipage}{0.73\textwidth}
\begin{center}
\phantom{h} \vspace{1cm}\\
\hspace{4cm}
\includegraphics[scale = 1]{Billeder/Norreg.png} \\
\phantom{h} \vspace{5cm}\\
\rule{0.7\textwidth}{0.3mm}\\
\phantom{h}\\
{\fontsize{50}{60}\selectfont Matematik-\\aflevering}\\
\phantom{h}\\
\rule{0.7\textwidth}{0.3mm}\\
\Large 2022\\
\Large 1.e MA

\end{center}
\end{minipage}
\end{titlepage}
\loadgeometry{art}

%Udfyld afsnit herunder og lav til egen Latex-fil

%Kopier følgende til overskrift:

%\begin{center}
%\Huge
%Aflevering 1
%\end{center}
%\section*{Opgave 1}
%\stepcounter{section}
\begin{center}
Opgavesætter er delt i to dele:\\
Delprøve 1 kun med den centralt udmeldte formelsamling.\\
Delprøve 2 med alle hjælpemidler.
\end{center}

\section*{Krav til formidling af din besvarelse}

Ved bedømmelse af helhedsindtrykket af besvarelsen af de enkelte opgaver lægges særlig vægt på følgende fire punkter:
\begin{itemize}
\item[$\cdot$] \textbf{Redegørelse og dokumentation for metode} \\
Besvarelsen skal indeholde en redegørelse for den anvendte løsningsstragegi med dokumentation i form af et passende antal mellemregninger \textit{eller} matematiske forklaringer på metoden, når et matematisk værktøjsprogram anvendes.
\item[$\cdot$] \textbf{Figurer, grafer og andre illustrationer} \\
Besvarelsen skal indeholde hensigtsmæssig brug af figurer, grafer og andre illustrationer, og der skal være tydelige henvisninger til brug af disse i den forklarende tekst.
\item[$\cdot$] \textbf{Notation og layout}\\
Besvarelsen skal i overensstemmelse med god matematisk skik opstilles med hensigtsmæssig brug af symbolsprog, og med en redegørelse for den matematiske notation, der indføres og anvendes, og som ikke kan henføres stil standardviden.
\item[$\cdot$] \textbf{Formidling og forklaring}\\
Besvarelsen af rene matematikopgaver skal indeholde en angivelse af givne oplysninger og korte forklaringer knyttet til den anvendte løsningsstrategi beskrevet med brug af almindelig matematisk notation. 

Besvarelsen af opgaver, der omhandler matematiske modeller, skal indeholde en kort præsentation af modellens kontekst, herunder betydning af modellens parametre. De enkelte delspørgsmål skal afsluttes med en præcis konklusion præsenteret i et klart sprog i relation til konteksten.
\end{itemize}

\newpage


\begin{center}
\LARGE
Delprøve uden hjælpemidler
\end{center}
\stepcounter{section}
%%%%%%%%%%%%%%%%%%%%%%%%%%%%%%%%%%%%%%%%%%%%%%%%%%%%%%%%%%%%%%%%%%%%%%%
%							Ny Opgave!!!!!							%
%%%%%%%%%%%%%%%%%%%%%%%%%%%%%%%%%%%%%%%%%%%%%%%%%%%%%%%%%%%%%%%%%%%%%%%
\begin{opgavetekst}{Opgave 1}
	Forkort følgende brøker mest muligt
	\begin{align*}
		&1) \ 7\cdot \frac{4}{14}  &&2)\ 5\cdot \frac{3-10}{7}   \\
		&3) \ \frac{\frac{6}{5}}{\frac{12}{10}}  &&4)\ \frac{3}{4}+\frac{7}{11}   \\
		&5) \ \frac{10}{11}-\frac{11}{11}  &&6)\ \frac{2}{8}\cdot \frac{10}{16}   \\
	\end{align*}
\end{opgavetekst}
%%%%%%%%%%%%%%%%%%%%%%%%%%%%%%%%%%%%%%%%%%%%%%%%%%%%%%%%%%%%%%%%%%%%%%%
%							Ny Opgave!!!!!							%
%%%%%%%%%%%%%%%%%%%%%%%%%%%%%%%%%%%%%%%%%%%%%%%%%%%%%%%%%%%%%%%%%%%%%%%
\begin{opgavetekst}{Opgave 2}
	Udregn følgende potenser eller rødder:
	\begin{align*}
		&1) \ \left(\frac{2}{4}\right)^3  &&2) \  \sqrt[5]{3^{10}}  \\
		&3) \ 4^{\frac{1}{2}}  &&4) \ \sqrt{2}\cdot \sqrt{32}   \\
		&5) \ 7^{\frac{1}{2}}7^{-\frac{1}{2}}  &&6) \ \left(3^{6}\right)^\frac{1}{3}   \\
	\end{align*}
\end{opgavetekst}

%%%%%%%%%%%%%%%%%%%%%%%%%%%%%%%%%%%%%%%%%%%%%%%%%%%%%%%%%%%%%%%%%%%%%%%
%							Ny Opgave!!!!!							%
%%%%%%%%%%%%%%%%%%%%%%%%%%%%%%%%%%%%%%%%%%%%%%%%%%%%%%%%%%%%%%%%%%%%%%%
\begin{opgavetekst}{Opgave 3}
	Løs følgende ligningssystemer:
\end{opgavetekst}
\begin{delopgave}{}{1}
	\begin{align*}
		2x+4y &= 8,\\
		4x+9y &= 9.
	\end{align*}
\end{delopgave}
\begin{delopgave}{}{2}
	\begin{align*}
		-x+5y &= 10,\\
		3x-2y &= 9.
	\end{align*}
\end{delopgave}
%%%%%%%%%%%%%%%%%%%%%%%%%%%%%%%%%%%%%%%%%%%%%%%%%%%%%%%%%%%%%%%%%%%%%%%
%							Ny Opgave!!!!!							%
%%%%%%%%%%%%%%%%%%%%%%%%%%%%%%%%%%%%%%%%%%%%%%%%%%%%%%%%%%%%%%%%%%%%%%%
\begin{opgavetekst}{Opgave 4}
	Prisen på en bestemt jakke var mandag 1500kr, onsdag var prisen 1800kr og fredag var prisen 1200kr.
\end{opgavetekst}
	\begin{delopgave}{}{1}
		Hvor stor en del udgør prisen fredag af prisen mandag?
	\end{delopgave}
	\begin{delopgave}{}{2}
		Hvor mange procent er prisen faldet fra onsdag til fredag?
	\end{delopgave}

%%%%%%%%%%%%%%%%%%%%%%%%%%%%%%%%%%%%%%%%%%%%%%%%%%%%%%%%%%%%%%%%%%%%%%%
%							Ny Opgave!!!!!							%
%%%%%%%%%%%%%%%%%%%%%%%%%%%%%%%%%%%%%%%%%%%%%%%%%%%%%%%%%%%%%%%%%%%%%%%
\begin{opgavetekst}{Opgave 5}
	En virksomheds omsætning kan fra år 2000-2010 beskrives ved en eksponentialfunktion. Det oplyses, at virksomheden
	i år 2000 omsætter for 2.73 mia. Deres omsætning øges årligt med $7\%$. 
\end{opgavetekst}
\begin{delopgave}{}{1}
	Bestem beyndelsesværdien, fremskrivningsfaktoren og vækstraten for den eksponentialfunktion $f$, der beskriver 
	virksomhedens omsætning til år $x$ efter år 2000.
\end{delopgave}
\begin{delopgave}{}{2}
	Bestem en forskift for $f$ og opstil et udtryk, der viser, hvordan du vil afgøre, hvornår virksomhedens omsætning er
	på $3$ mia. (Du skal ikke løse udtrykket.)
\end{delopgave}
%%%%%%%%%%%%%%%%%%%%%%%%%%%%%%%%%%%%%%%%%%%%%%%%%%%%%%%%%%%%%%%%%%%%%%%
%							Ny Opgave!!!!!							%
%%%%%%%%%%%%%%%%%%%%%%%%%%%%%%%%%%%%%%%%%%%%%%%%%%%%%%%%%%%%%%%%%%%%%%%
\begin{opgavetekst}{Opgave 6}
	En eksponentialfunktion $f$ går gennem punkterne $P(2,8)$ og $Q(5,512)$. 
\end{opgavetekst}
\begin{delopgave}{}{1}
	Bestem forskriften for $f$. 
\end{delopgave}
\begin{delopgave}{}{2}
	Bestem $f(4)$.
\end{delopgave}

\end{document}



