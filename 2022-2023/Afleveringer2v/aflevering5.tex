\begin{center}
\LARGE
Delprøve uden hjælpemidler 
\end{center}
\stepcounter{section}
%%%%%%%%%%%%%%%%%%%%%%%%%%%%%%%%%%%%%%%%%%%%%%%%%%%%%%%%%%%%%%%%%%%%%%%
%							Ny Opgave!!!!!							%
%%%%%%%%%%%%%%%%%%%%%%%%%%%%%%%%%%%%%%%%%%%%%%%%%%%%%%%%%%%%%%%%%%%%%%%
\begin{opgavetekst}{Opgave 1}
	En stokastisk variabel $X$ har sandsynlighedsfunktion $P$ og udfaldsrum  \\
	$U = \{a,b,c,d,e\}$. Fordelingen for $X$ 
	kan ses af Tab. \ref{tab:fordeling}
	\begin{table}[H]
		\centering
		\begin{tabular}{c|c|c|c|c|c}
			$U$ & $a$ & $b$ &  $c$ & $d$ & $e$ \\
			\hline
			$P(X=x)$ & 0.5 & 0.1 & 0.05 & 0.05 & $P(X=e)$
		\end{tabular}
		\caption{Fordeling for den stokastiske variabel $X$. }
		\label{tab:fordeling}		
	\end{table}
	\phantom{h}
\end{opgavetekst}
	\begin{delopgave}{}{1}
		Bestem $P(X=e)$.
	\end{delopgave}

%%%%%%%%%%%%%%%%%%%%%%%%%%%%%%%%%%%%%%%%%%%%%%%%%%%%%%%%%%%%%%%%%%%%%%%
%							Ny Opgave!!!!!							%
%%%%%%%%%%%%%%%%%%%%%%%%%%%%%%%%%%%%%%%%%%%%%%%%%%%%%%%%%%%%%%%%%%%%%%%
\begin{opgavetekst}{Opgave 2}
	Et polynomium $f$ er givet ved
	\begin{align*}
		f(x) = x^3-9x^2+24x+7.
	\end{align*}
	Funktionen $f$ har ikke nogle vendetangenter.
\end{opgavetekst}
\begin{delopgave}{}{1}
	Bestem ekstremumspunkterne for $f$. 
\end{delopgave}

%%%%%%%%%%%%%%%%%%%%%%%%%%%%%%%%%%%%%%%%%%%%%%%%%%%%%%%%%%%%%%%%%%%%%%%
%							Ny Opgave!!!!!							%
%%%%%%%%%%%%%%%%%%%%%%%%%%%%%%%%%%%%%%%%%%%%%%%%%%%%%%%%%%%%%%%%%%%%%%%
\begin{opgavetekst}{Opgave 3}
	En linje $l$ går gennem punktet $(-1,-2)$ og har normalvektoren $\vv{n}$ givet ved
	\begin{align*}
		\vv{n} = 
		\begin{pmatrix}
			4 \\ 3
		\end{pmatrix}.
	\end{align*}	
\end{opgavetekst}
\begin{delopgave}{}{1}
	Bestem linjens ligning for $l$. 
\end{delopgave}
\begin{delopgave}{}{2}
	Afgør, om punktet $(2,7)$ ligger på linjen. 
\end{delopgave}
\newpage
%%%%%%%%%%%%%%%%%%%%%%%%%%%%%%%%%%%%%%%%%%%%%%%%%%%%%%%%%%%%%%%%%%%%%%%
%							Ny Opgave!!!!!							%
%%%%%%%%%%%%%%%%%%%%%%%%%%%%%%%%%%%%%%%%%%%%%%%%%%%%%%%%%%%%%%%%%%%%%%%
\begin{opgavetekst}{Opgave 4}
	En funktion er givet ved
	\begin{align*}
		f(x) = x^3\cdot 2\sqrt{x}
	\end{align*}
\end{opgavetekst}
\begin{delopgave}{}{1}
	Bestem $f'(x)$. 
\end{delopgave}


\newpage
\begin{center}
\LARGE
Delprøve med hjælpemidler 
\end{center}
\stepcounter{section}
%%%%%%%%%%%%%%%%%%%%%%%%%%%%%%%%%%%%%%%%%%%%%%%%%%%%%%%%%%%%%%%%%%%%%%%
%							Ny Opgave!!!!!							%
%%%%%%%%%%%%%%%%%%%%%%%%%%%%%%%%%%%%%%%%%%%%%%%%%%%%%%%%%%%%%%%%%%%%%%%
\begin{opgavetekst}{Opgave 5}
	En funktion $f$ er givet ved
	\begin{align*}
		f(x) = x^5 + 8x^4+7x^2+2x+1
	\end{align*}
\end{opgavetekst}
\begin{delopgave}{}{1}
	Bestem ligningen for tangenten i punktet $P(2,f(2))$.
\end{delopgave}
\begin{meretekst}
	I et andet punkt $Q$ er hældningen for tangenten til $f$ lig hældningen for tangenten i punktet $P$. 
\end{meretekst}
\begin{delopgave}{}{2}
	Bestem koordinaterne for punktet $Q$.
\end{delopgave}

%%%%%%%%%%%%%%%%%%%%%%%%%%%%%%%%%%%%%%%%%%%%%%%%%%%%%%%%%%%%%%%%%%%%%%%
%							Ny Opgave!!!!!							%
%%%%%%%%%%%%%%%%%%%%%%%%%%%%%%%%%%%%%%%%%%%%%%%%%%%%%%%%%%%%%%%%%%%%%%%

\begin{opgavetekst}{Opgave 6}
	\begin{center}
		\includegraphics[width = 0.7\textwidth]{Billeder/bygge.jpg}
	\end{center}
	1000 personer er i en meningsmåling blevet spurgt til deres holdning til et byggeri i lokalområdet. Af disse har
	351 personer tilkendegivet, at de er utilfredse med byggeriet. 
\end{opgavetekst}
\begin{delopgave}{}{1}
	Bestem et $95\%$-konfidensinterval for andelen af personer i området, der er utilfredse med byggeriet. 
\end{delopgave}
\begin{meretekst}
	Efter flere forsinkelser af byggeriet spørges 1000 nye personer igen om deres holdning til byggeriet. Denne gang 
	tilkendegiver 399 personer, at de er utilfredse med byggeriet. 
\end{meretekst}
\begin{delopgave}{}{2}
	Begrund på baggrund af dit $95\%$-konfidensinterval om holdningen til byggeriet har ændret sig signifikant siden 
	den første meningsmåling. 
\end{delopgave}

%%%%%%%%%%%%%%%%%%%%%%%%%%%%%%%%%%%%%%%%%%%%%%%%%%%%%%%%%%%%%%%%%%%%%%%
%							Ny Opgave!!!!!							%
%%%%%%%%%%%%%%%%%%%%%%%%%%%%%%%%%%%%%%%%%%%%%%%%%%%%%%%%%%%%%%%%%%%%%%%

\begin{opgavetekst}{Opgave 7}
	En cirkel er givet ved ligningen 
	\begin{align*}
		x^2-4x+y^2-6y=23.
	\end{align*}
\end{opgavetekst}
\begin{delopgave}{}{1}
	Afgør, om punktet $P(1,1)$ ligger på cirklen. 
\end{delopgave}
\begin{delopgave}{}{2}
	Bestem centrum og radius for cirklen. 
\end{delopgave}


%%%%%%%%%%%%%%%%%%%%%%%%%%%%%%%%%%%%%%%%%%%%%%%%%%%%%%%%%%%%%%%%%%%%%%%
%							Ny Opgave!!!!!							%
%%%%%%%%%%%%%%%%%%%%%%%%%%%%%%%%%%%%%%%%%%%%%%%%%%%%%%%%%%%%%%%%%%%%%%%

\begin{opgavetekst}{Opgave 8}
	\begin{center}
		\includegraphics[width=0.7\textwidth]{Billeder/Bakterie.jpg}
	\end{center}
	I \href{https://github.com/ChristianJLex/TeachingNotes/raw/master/2022-2023/Data%20og%20lign/BakterierAfl5.xlsx}{\color{blue!60} dette datasæt} er sammenhængen mellem forløbet tid (i timer) og antallet af 	
	   bakterier (i mia.) i en opløsning givet. Det antages at antallet af bakterier $B$ (i mia) kan beskrives ved en 
	model af typen
	\begin{align*}
		B(t) = b\cdot a^t,
	\end{align*}
	hvor $t$ beskriver den forløbne tid (i timer).
\end{opgavetekst}

\begin{delopgave}{}{1}
	Brug datasættet til at bestemme tallene $a$ og $b$.
\end{delopgave}

\begin{delopgave}{}{2}
	Brug din model til at bestemme antallet af bakterier efter 130 timer. 
\end{delopgave}

\begin{meretekst}
	I en anden opløsning kan antallet af bakterier beskrives ved sammenhængen 
	\begin{align*}
		C(t) = 3.55 \cdot 1.006^t,
	\end{align*}
	hvor $C$ er antallet af bakterier (i mia.) og $t$ er den forløbne tid (i timer).
\end{meretekst}

\begin{delopgave}{}{3}
	Afgør, hvor mange bakterier der er i de to opløsninger, når der er lige mange bakterier i opløsningerne. 
\end{delopgave}

