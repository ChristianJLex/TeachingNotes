\begin{center}
\LARGE
Delprøve uden hjælpemidler
\end{center}
\stepcounter{section}
%%%%%%%%%%%%%%%%%%%%%%%%%%%%%%%%%%%%%%%%%%%%%%%%%%%%%%%%%%%%%%%%%%%%%%%
%							Ny Opgave!!!!!							%
%%%%%%%%%%%%%%%%%%%%%%%%%%%%%%%%%%%%%%%%%%%%%%%%%%%%%%%%%%%%%%%%%%%%%%%
\begin{opgavetekst}{Opgave 1}
	Forkort følgende brøker mest muligt
	\begin{align*}
		&1) \ 7\cdot \frac{4}{14}  &&2)\ 5\cdot \frac{3-10}{7}   \\
		&3) \ \frac{\frac{6}{5}}{\frac{12}{10}}  &&4)\ \frac{3}{4}+\frac{7}{11}   \\
		&5) \ \frac{10}{11}-\frac{11}{11}  &&6)\ \frac{2}{8}\cdot \frac{10}{16}   \\
	\end{align*}
\end{opgavetekst}
%%%%%%%%%%%%%%%%%%%%%%%%%%%%%%%%%%%%%%%%%%%%%%%%%%%%%%%%%%%%%%%%%%%%%%%
%							Ny Opgave!!!!!							%
%%%%%%%%%%%%%%%%%%%%%%%%%%%%%%%%%%%%%%%%%%%%%%%%%%%%%%%%%%%%%%%%%%%%%%%
\begin{opgavetekst}{Opgave 2}
	Udregn følgende potenser eller rødder:
	\begin{align*}
		&1) \ \left(\frac{2}{4}\right)^3  &&2) \  \sqrt[5]{3^{10}}  \\
		&3) \ 4^{\frac{1}{2}}  &&4) \ \sqrt{2}\cdot \sqrt{32}   \\
		&5) \ 7^{\frac{1}{2}}7^{-\frac{1}{2}}  &&6) \ \left(3^{6}\right)^\frac{1}{3}   \\
	\end{align*}
\end{opgavetekst}

%%%%%%%%%%%%%%%%%%%%%%%%%%%%%%%%%%%%%%%%%%%%%%%%%%%%%%%%%%%%%%%%%%%%%%%
%							Ny Opgave!!!!!							%
%%%%%%%%%%%%%%%%%%%%%%%%%%%%%%%%%%%%%%%%%%%%%%%%%%%%%%%%%%%%%%%%%%%%%%%
\begin{opgavetekst}{Opgave 3}
	Løs følgende ligningssystemer:
\end{opgavetekst}
\begin{delopgave}{}{1}
	\begin{align*}
		2x+4y &= 8,\\
		4x+9y &= 9.
	\end{align*}
\end{delopgave}
\begin{delopgave}{}{2}
	\begin{align*}
		-x+5y &= 10,\\
		3x-2y &= 9.
	\end{align*}
\end{delopgave}
%%%%%%%%%%%%%%%%%%%%%%%%%%%%%%%%%%%%%%%%%%%%%%%%%%%%%%%%%%%%%%%%%%%%%%%
%							Ny Opgave!!!!!							%
%%%%%%%%%%%%%%%%%%%%%%%%%%%%%%%%%%%%%%%%%%%%%%%%%%%%%%%%%%%%%%%%%%%%%%%
\begin{opgavetekst}{Opgave 4}
	Prisen på en bestemt jakke var mandag 1500kr, onsdag var prisen 1800kr og fredag var prisen 1200kr.
\end{opgavetekst}
	\begin{delopgave}{}{1}
		Hvor stor en del udgør prisen fredag af prisen mandag?
	\end{delopgave}
	\begin{delopgave}{}{2}
		Hvor mange procent er prisen faldet fra onsdag til fredag?
	\end{delopgave}

%%%%%%%%%%%%%%%%%%%%%%%%%%%%%%%%%%%%%%%%%%%%%%%%%%%%%%%%%%%%%%%%%%%%%%%
%							Ny Opgave!!!!!							%
%%%%%%%%%%%%%%%%%%%%%%%%%%%%%%%%%%%%%%%%%%%%%%%%%%%%%%%%%%%%%%%%%%%%%%%
\begin{opgavetekst}{Opgave 5}
	En virksomheds omsætning kan fra år 2000-2010 beskrives ved en eksponentialfunktion. Det oplyses, at virksomheden
	i år 2000 omsætter for 2.73 mia. Deres omsætning øges årligt med $7\%$. 
\end{opgavetekst}
\begin{delopgave}{}{1}
	Bestem beyndelsesværdien, fremskrivningsfaktoren og vækstraten for den eksponentialfunktion $f$, der beskriver 
	virksomhedens omsætning til år $x$ efter år 2000.
\end{delopgave}
\begin{delopgave}{}{2}
	Bestem en forskift for $f$ og opstil et udtryk, der viser, hvordan du vil afgøre, hvornår virksomhedens omsætning er
	på $3$ mia. (Du skal ikke løse udtrykket.)
\end{delopgave}
%%%%%%%%%%%%%%%%%%%%%%%%%%%%%%%%%%%%%%%%%%%%%%%%%%%%%%%%%%%%%%%%%%%%%%%
%							Ny Opgave!!!!!							%
%%%%%%%%%%%%%%%%%%%%%%%%%%%%%%%%%%%%%%%%%%%%%%%%%%%%%%%%%%%%%%%%%%%%%%%
\begin{opgavetekst}{Opgave 6}
	En eksponentialfunktion $f$ går gennem punkterne $P(2,8)$ og $Q(5,512)$. 
\end{opgavetekst}
\begin{delopgave}{}{1}
	Bestem forskriften for $f$. 
\end{delopgave}
\begin{delopgave}{}{2}
	Bestem $f(4)$.
\end{delopgave}
