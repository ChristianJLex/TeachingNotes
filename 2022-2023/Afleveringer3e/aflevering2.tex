\begin{center}
\LARGE
Delprøve uden hjælpemidler 
\end{center}
\stepcounter{section}

\begin{opgavetekst}{Opgave 1}
	Overfladen for en kugle $K$ kan beskrives ved ligningen
	\begin{align*}
		x^2-4x+y^2-8y+z^2-6z = -13.
	\end{align*}
\end{opgavetekst}
	\begin{delopgave}{}{1}
		Afgør, om punktet $P(2,4,7)$ ligger på $K$
	\end{delopgave}
	\begin{delopgave}{}{2}
		Bestem centrum $C(x_0,y_0,z_0)$ og radius $r$ for $K$
	\end{delopgave}
\begin{opgavetekst}{Opgave 2}
	En plan $L$ har normalvektoren $\vv{n}$ givet ved
	\begin{align*}
		\vv{n} = 
		\begin{pmatrix}
			5 \\ 4 \\ -2
		\end{pmatrix}
	\end{align*}
	og går gennem punktet $P(-7,2,3)$.
\end{opgavetekst}
	\begin{delopgave}{}{1}
		Bestem en ligning for $L$. 
	\end{delopgave}
	\begin{meretekst}
		$xy$-planen har som bekendt ligningen $z = 0$.
	\end{meretekst}
	\begin{delopgave}{}{2}
		Bestem ligningen for den linje $l$, der dannes af skæringen mellem $L$ og $xy$-planen.
	\end{delopgave}
	\begin{delopgave}{}{3}
		Afgør, om punktet $Q(1074,-51034,1)$ ligger på $l$.
	\end{delopgave}
	
\begin{opgavetekst}{Opgave 3}
\end{opgavetekst}
\begin{delopgave}{}{1}
	Løs følgende ubestemte integral:
	\begin{align*}
		\int \frac{3x+1}{3x^2+2x+1}\intd x
	\end{align*}
\end{delopgave}
\begin{opgavetekst}{Opgave 4}
\end{opgavetekst}
\begin{delopgave}{}{1}
	Differentier følgende funktion:
	\begin{align*}
		\ln(x)\cdot 3x^5
	\end{align*}
\end{delopgave}
\begin{delopgave}{}{2}
	Differentier følgende funktion:
	\begin{align*}
		\cos(x^7+\sqrt{x})
	\end{align*}
\end{delopgave}
\begin{opgavetekst}{Opgave 5}
\end{opgavetekst}
\begin{delopgave}{}{1}
	Bestem værdien af følgende bestemte integral:
	\begin{align*}
		\int_{-\pi}^{2\pi} \sin(x) + \cos(x) \intd x.
	\end{align*}
\end{delopgave}

\begin{opgavetekst}{Opgave 6}
	En funktion $f:\mathbb{R} \to \mathbb{R}$ er givet ved
	\begin{align*}
		f(x) = 2x+9.
	\end{align*}
	Funktionen $f$ er bijektiv. 
\end{opgavetekst}
\begin{delopgave}{}{1}
	Bestem en invers funktion $f^{-1}:\mathbb{R} \to \mathbb{R}$ til $f$. 
\end{delopgave}

\begin{delopgave}{}{2}
	Bestem urbilledet for $f$ til mængden $[-4,1]$. 
\end{delopgave}

\newpage
\begin{center}
\LARGE
Delprøve med hjælpemidler 
\end{center}
\stepcounter{section}

\begin{opgavetekst}{Opgave 7}
	I en bakteriekoloni kan bakterieantallet $B$ i de første 24 timer beskrives ved en eksponentiel sammenhæng. Et datasæt hvori bakterieantallet $B$ til tiden $t$ er givet i 
	Tab. \ref{tab:bakterie}.
	\begin{table}[H]
		\centering
		\begin{tabular}{c|c|c|c|c|c|c|c}
		$t$ (timer) &1 & 2 & 3 & 4 & 5 & 6 & 7 \\
		\hline
		$B$ (bakterier i mio.) & 19.1 & 22.6 & 29.1 & 32.9 & 44.0 & 50.4 & 65.1
		\end{tabular}
		\caption{Antallet af bakterier $(B)$ i mio. som funktion af tiden $(t)$ i timer. }
		\label{tab:bakterie}
	\end{table}\phantom{h}
\end{opgavetekst}
\begin{delopgave}{}{1}
	Brug datasættet fra Tab. \ref{tab:bakterie} til at bestemme en forskrift for $B(t)$.
\end{delopgave}
\begin{delopgave}{}{2}
	Afgør, hvornår antallet af bakterier overstiger 1 mia.
\end{delopgave}
\begin{delopgave}{}{3}
	Afgør, hvornår bakterievæksten overstiger 100 mio bakterier pr. time.  
\end{delopgave}
\begin{opgavetekst}{Opgave 8}
	En særlig vase udformes ved at dreje funktionen $f$ givet ved
	\begin{align*}
		f(x) = 0.1(x^5+3x^4+x^2+40)
	\end{align*}
	omkring $x$-aksen på intervallet $[-3.38992,b]$. 
\end{opgavetekst}
\begin{delopgave}{}{1}
	Tegn funktionen $f$ på intervallet $[-3.38992,3]$.
\end{delopgave}
\begin{delopgave}{}{2}
	Bestem rumfanget af vasen, hvis $b = 2$. 
\end{delopgave}
\begin{delopgave}{}{3}
	Bestem hvad $b$ skal være, hvis vasen skal have et rumfang på $700$.
\end{delopgave}