\begin{center}
\Huge
Funktioner
\end{center}

\section*{Definitionsmængde og dispositionsmængde}
\stepcounter{section}

Vi har indtil nu hovedsagligt betragtet funktioner som noget, vi kan beskrive med et funktionsudtryk à la $f(x) = 2x+1$ eller $g(x) = x^2$. Funktioner er mere generelle end det, og vi vil komme lidt nærmere på, hvad funktioner er for en størrelse. Vi vil arbejde med følgende "intuitive definition": For to mængder $A$ og $B$ definerer vi en funktion som en entydig tildeling af elementer i $B$ til ethvert element i $A$. Vi noterer i så fald funktionen $f$ som $f:A \to B$. 
\begin{defn}[Definitionsmængde og dispositionsmængde/codomæne]
Lad $A$ og $B$ være to mængder, og lad $f:A \to B$ være en funktion. Så kaldes $A$ for \textit{definitionsmængden/domænet for f} og $B$ kaldes for \textit{dispositionsmængden/codomænet for f}.
\end{defn}
\begin{exa}
Vi har funktionen $f:\mathbb{R} \to \mathbb{R}$ givet ved 
\begin{align*}
f(x) = x^2,
\end{align*}
som har $\mathbb{R}$ som både definitionsmængde og codomæne.
\end{exa}

\begin{exa}\label{exa:navne}
Lad mængden $N$ bestå af alle navne i 3.e. Vi definerer så funktionen $f:N\to \mathbb{N}$ som funktionen, der giver alderen på en person i 3.e. Vi vil så have eksempelvis
$$f(\textnormal{Lærer-Christian}) = 26.$$
\end{exa}
Vi kan også tænke på funktioner som diagrammer som set på Fig. \ref{fig:diag}.
\begin{figure}[H]
	\centering
	\begin{tikzpicture}
		\draw[] (0,0) circle (1cm);
		\draw[] (4,0) circle (1cm);
		\node at (0,1.3) {$A$};
		\node at (4,1.3) {$B$};
		\draw[-{Stealth[scale = 1.3]}] (1.3,0) --(2.7,0);
		\node at (2,0.3) {$f$};
	\end{tikzpicture}
	\caption{Funktionsdiagram for funktionen $f$.}
	\label{fig:diag}
\end{figure}

Codomænet for en funktion kan godt bestå af en større mængde en det, som vores funktion rent faktisk kan ``ramme''- det er ikke altid trivielt at bestemme, hvad en funktion helt præcist rammer i codomænet. Til at indkapsle denne idé introducerer vi begrebet \textit{værdimængde/billedmængde}.
\begin{defn}[Værdimængde]
For en funktion $f:A \to B$ kaldes mængden Vm$(f) = \{f(x) \mid x \in A\}$ for værdimængden for $f$. Den består af de elementer, $f$ kan "ramme".
\end{defn}

\begin{exa}

Funktionen $f: \mathbb{R} \to \mathbb{R}$ givet ved
\begin{align*}
f(x) = x^2
\end{align*}
har værdimængde Vm$(f) = \mathbb{R}_{\geq 0}$, da den kan ramme alle positive reelle tal. 
\end{exa}

\begin{exa}\label{exa:udd}
Lad $U$ bestå af mængden af alle lange videregående uddannelser, og lad funktion $l:U \to \mathbb{R}$ være funktionen, der tager en uddannelse og giver gennemsnitsbruttoindkomsten efter afsluttet uddannelse. Så vil funktionen give os eksempelvis
\begin{align*}
l(\textnormal{Samfundsfag}) = 660.500.
\end{align*}
Værdimængden for denne funktion vil være
\begin{align*}
\textnormal{Vm}(l) = [267.200,1.383.200].
\end{align*}

\end{exa}

Til slut har vi \textit{billedet} og \textit{urbilledet} for mængder.
\begin{defn}[Billede og urbillede]
Lad $f:A \to B$ være en funktion, og lad $K\subseteq A$ være en delmængde af $A$. Så kalder vi mængden
\begin{align*}
f(K) = \{f(x) \mid x \in K\}
\end{align*}
for billedet af $K$ under $f$. 
Lad tilsvarende $L\subseteq B$. Så kaldes mængden 
\begin{align*}
 \{x \in A \mid f(x) \in K\}
\end{align*}
for urbilledet af $L$ under $f$.
\end{defn}

\begin{exa}
Lad $K = [1,2]\subseteq \mathbb{R}$, og lad $f:\mathbb{R} \to \mathbb{R}$ være givet ved
\begin{align*}
f(x) = 2x+1.
\end{align*}
Så er billedet af $K$ under $f$ givet ved mængden 
\begin{align*}
f([1,2]) = [3,5].
\end{align*}
Lader vi tilsvarende $L = [-2,0] \subseteq \mathbb{R}$, så vil urbilledet af $L$ under $f$ være givet ved mængden
\begin{align*}
[-1.5, -0.5].
\end{align*}
\end{exa}

Et diagram, der illustrerer billedet af en mængde under en funktion kan ses på Fig. \ref{fig:im}.
\begin{figure}[H]
	\centering
	\begin{tikzpicture}
		\draw[] (0,0) circle (1.5cm);
		\draw[] (6,0) circle (1.5cm);
		\node at (0,1.8) {$A$};
		\node at (6,1.8) {$B$};
		\draw[-{Stealth[scale = 1.3]}] (1.8,0) --(4.2,0);
		\node at (3,0.3) {$f$};
		\draw[] (0,0.6) circle (0.7cm);
		\node at (0,0.6) {$L$};
		\draw[] (6,0.6) circle (0.7cm);
		\node at (6,0.6) {$f(L)$};
		\draw[-{Stealth[scale = 1.3]}] (0.6,1.2) to[out = 45] (5.2,1.1);
		\node at (3,2.4) {$f$};
	\end{tikzpicture}
	\caption{Funktionsdiagram for funktionen $f$.}
	\label{fig:im}
\end{figure}


\section*{Opgave 1}
\begin{enumerate}[label=\roman*)]
\item Bestem værdimængden for funktionen $f:\mathbb{R} \to \mathbb{R}$ givet ved
\begin{align*}
f(x) = x^3.
\end{align*}
\item Bestem værdimængde for funktionen $f:\mathbb{R} \to \mathbb{R}$ givet ved
\begin{align*}
	f(x) = |x|.
\end{align*}
\item Bestem værdimængden for funktionen $f:\mathbb{R} \to \mathbb{R}$ givet ved
\begin{align*}
f(x) = \lfloor x \rfloor,
\end{align*}
der runder alle tal ned til nærmeste heltal. 
\item Bestem værdimængden for funktionen $f:\mathbb{R}_{\geq 0} \to \mathbb{R}_{\geq 0}$ givet ved
\begin{align*}
	f(x) = \sqrt{x}.
\end{align*}
\item Bestem værdimængden for funktionen $f:\mathbb{R} \to \mathbb{R}$ givet ved
\begin{align*}
	f(x) = x^4.
\end{align*}
\item Bestem værdimængen for funktionen $f: \mathbb{R} \to \mathbb{R}$ givet ved
\begin{align*}
	f(x) = x^2+2x+1.
\end{align*}
\item Bestem værdimængden for funktionen $f$ fra Eksempel \ref{exa:navne}.
\end{enumerate}
\section*{Opgave 2}
Vi husker på, at det kartesiske produkt $A\times B$ af to mængder $A$ og $B$ er defineret ved
\begin{align*}
	A\times B = \{(a,b) \mid a\in A,\  b\in B\}.
\end{align*}
Bestem værdimængden for funktionen $f:\mathbb{Z}\times \mathbb{Z} \to \mathbb{R}$ givet ved
\begin{align*}
f(x,y) = \frac{x}{y}.
\end{align*}
\section*{Opgave 3}
\begin{enumerate}[label=\roman*)]
\item Lad $f:\mathbb{R} \to \mathbb{R}$ være givet ved
\begin{align*}
f(x) = x + 4.
\end{align*}
Bestem billedet af følgende delmængder af $\mathbb{R}$ under $f$:
\begin{align*}
&1) \ \{1,2,4\} &&2) \ [0,5]\\
&3) \ \emptyset &&4) \ \{20\}
\end{align*}
\item Lad $f: \mathbb{R} \to \mathbb{R}$ være givet ved 
\begin{align*}
f(x) = 2x^2-4x+1.
\end{align*}
Bestem billedet af følgende delmængder af $\mathbb{R}$ under $f$ (det kan være en fordel at plotte $f$):
\begin{align*}
&1)\  [3,8] &&2) \  \{1,\hdots, 8\};\\
&3)\  \emptyset &&4)\ [-2,2];
\end{align*}
\item Lad $f:\mathbb{R} \to \mathbb{Z}$ være givet ved
\begin{align*}
f(x) = \lceil x \rceil + 2.
\end{align*}
Bestem billedet af følgende delmængder af $\mathbb{R}$ under $f$:
\begin{align*}
&1) \ [-4,5]  &&2) \ \{1.1,1.2,1.3,1.7,1.85\}\\
&3) \ [0,1]  &&4) \ \{1,2,3,4\}\\
\end{align*}
\end{enumerate}

\section*{Opgave 4}
Lad $l$ være funktionen fra Eksempel \ref{exa:udd}. Bestem billedet af mængden 
\begin{align*}
\{\textnormal{Medicin, Tandlæge, Erhvervsøkonomi}\}.
\end{align*}
Lønstatistik kan findes \href{https://cepos.dk/artikler/se-listen-hvilke-uddannelser-giver-den-hoejeste-indkomst/}{\color{blue!60} her}.

\section*{Opgave 5}
\begin{enumerate}[label=\roman*)]
\item Lad $f: \mathbb{R} \to \mathbb{R}$ være givet ved 
\begin{align*}
	f(x) = x-5.
\end{align*}
Bestem urbilledet af følgende mængder under $f$:
\begin{align*}
	&1) \  [9,12] &&2) \ \{2\}\\
	&3) \ \{-4,-2,-1\} &&4) \ \emptyset
\end{align*}

\item Lad $f:\mathbb{R}_{\geq 0} \to \mathbb{R}_{\geq 0}$ være givet ved
\begin{align*}
	f(x) = \sqrt{x}.
\end{align*}
Bestem urbilledet af følgende mængder under $f$:
\begin{align*}
	&1) \ \{2\}  &&2) \  [0,4] \\
	&3) \ \{1,3,4,5,6\} &&4) \ [8,16] 
\end{align*}

\item Lad $f$ være funktionen fra Eksempel \ref{exa:navne}. Bestem urbilledet af følgende mængder:
\begin{align*}
	&1) \ [0,15]  &&2) \ \{18\}  \\
	&3) \ [17,30] &&4) \ \{17,19\} 
\end{align*}
\item Lad $l$ være funktionen fra Eksempel \ref{exa:udd}. Bestem urbilledet af mængden $[700.000,1.000.000]$.
\end{enumerate}