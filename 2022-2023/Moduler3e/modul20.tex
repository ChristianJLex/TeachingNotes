\begin{center}
\Huge
Separable differentialligninger
\end{center}


\section*{Separable differentialligninger}
\stepcounter{section}

Vi har tidligere set på differentialligningen
\begin{align*}
	\frac{dy}{dx} =xy,
\end{align*}
som vi påstod havde den fuldstændige løsning
\begin{align*}
y(x) = ce^{\frac{1}{2}x^2}.
\end{align*}
Denne differentialligning tilhører en stor klasse af differentialligninger kaldet \textit{separable differentialligninger}, og vi skal i dag se på, hvordan man går til denne differentialligningstype.
\begin{defn}[Separabel differentialligning]
	En differentialligning på formen
	\begin{align*}
		\frac{dy}{dx} = h(x)g(y)
 	\end{align*}		
 	for kontinuerte funktioner $g$ og $h$ kaldes for en separabel differentialligning. 
\end{defn}

\begin{setn}[Separation af variable]
	Lad $f$ og $g$ være kontinuerte funktioner, samt $g \neq 0$. 
	Så har differentialligningen
	\begin{align*}
		y' = h(x)g(y)
	\end{align*}
	den fuldstændige løsning $y = f(x)$, der opfylder, at 
	\begin{align*}
		\int \frac{1}{g(y)}dy = \int h(x) dx.
	\end{align*}
\end{setn}
\begin{proof}
	Vi antager, at $y = f(x)$ er en løsning til differentialligningen
	\begin{align*}
		y' = h(x)g(y).
	\end{align*}
	Vi definerer nu 
	\begin{align*}
		\tilde{g}(x) = \frac{1}{g(x)},
	\end{align*}
	så 
	\begin{align*}
		g(x) = \frac{1}{\tilde{g}(x)}.
	\end{align*}
	Vi får derfor
	\begin{align*}
		y' = h(x)g(y) \ &\Leftrightarrow \ f'(x) = \frac{h(x)}{\tilde{g}(f(x))}\\
					&\Leftrightarrow \tilde{g}(f(x))f'(x) = h(x). 
	\end{align*}
	Vi integrerer med hensyn til $x$ på begge sider af lighedstegnet.
	\begin{align}\label{eq:1}
		\int\tilde{g}(f(x))f'(x) dx = \int h(x) dx.
	\end{align}
	Vi integrerer nu venstresiden ved integration ved substitution og får
	\begin{align*}
		\int\tilde{g}(f(x))f'(x) dx = \tilde{G}(f(x)) +k
	\end{align*}
	hvor $\tilde{G}$ er en stamfunktion til $\tilde{g}$. Vi indsætter nu $y =f(x)$ og får
	\begin{align*}
		\int\tilde{g}(f(x))f'(x) dx &= \tilde{G}(f(x)) +k \\
		 &= \tilde{G}(y) + k \\
		&= \int \tilde{g}(y) dy \\
		&= \int \frac{1}{g(y)}dy.
	\end{align*}
	Dette er lig venstresiden i \eqref{eq:1}, så det indsættes og vi får
	\begin{align*}
		\int \frac{1}{g(y)}dy = \int h(x) dx,
	\end{align*}
	og vi er færdige.
\end{proof}

\begin{exa}
	Vi betragter differentialligningen
	\begin{align*}
		y' = xy
	\end{align*}
	for $y>0$. 
	Vi har så, at 
	\begin{align*}
		y' = h(x)g(y)
	\end{align*}
	for $g(x) =x $ og $h(y) = y$. Vi løser denne ved separation af variable og får
	\begin{align*}
	\int \frac{1}{g(y)}d(y) = \int h(x) dx  \ &\Leftrightarrow \ \int \frac{1}{y} dy = \int x dx\\
	&\Leftrightarrow \ \ln(y) = \frac{1}{2}x^2 + k \\
	& \Leftrightarrow \ y = e^{\frac{1}{2}x^2+k} = e^{k}e^{\frac{1}{2}x^2} = ce^{\frac{1}{2}x^2}.
	\end{align*}
	Vi har nu fundet den generelle løsning til differentialligningen 
	\begin{align*}
		y' = xy.
	\end{align*}
\end{exa}

\section*{Opgave 1}
\begin{enumerate}[label=\roman*)]
	\item Bestem en generel løsning til differentialligningen 
	\begin{align*}
		y' = 2xy
	\end{align*}
	\item  Bestem en generel løsning til differentialligningen 
	\begin{align*}
		y' = x^2y
	\end{align*}
	 \item Bestem en generel løsning til differentialligningen 
	\begin{align*}
		y' = \frac{y}{x}
	\end{align*}
	\item Bestem en generel løsning til differentialligningen 
	\begin{align*}
		y' = x^2e^{-y}
	\end{align*}
\end{enumerate}

\section*{Opgave 2}
\begin{enumerate}[label=\roman*)]
	\item Bestem en generel løsning til differentialligningen 
	\begin{align*}
		y' = 2x\cos(x^2)y.
	\end{align*}
	\item Bestem en generel løsning til differentialligningen 
	\begin{align*}
		y' = 4x(-y^2).
	\end{align*}
	\item Bestem en generel løsning til differentialligningen 
	\begin{align*}
		y' = 3x^2e^{x^3}y^2.
	\end{align*}
\end{enumerate}

\section*{Opgave 3}
\begin{enumerate}[label=\roman*)]
	\item Bestem den partikulære løsning til differentialligningen
	\begin{align*}
		y' = 2xe^y,
	\end{align*}
	der går gennem punktet $(0,1)$.
	\item Bestem den partikulære løsning til differentialligningen 
	\begin{align*}
		y' = (2x+2)sin(x^2+2x)y,
	\end{align*}
	der går gennem punktet $(0,1)$. 
\end{enumerate}