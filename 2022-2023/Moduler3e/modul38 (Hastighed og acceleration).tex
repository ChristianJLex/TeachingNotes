\begin{center}
	\Huge
	Hastighed og acceleration
\end{center}

\section*{Vektorfunktioner som stedfunktioner}
\stepcounter{section}

Da vi introducerede vektorfunktioner bemærkede vi, at vektorfunktioner kan udgøre stedfunktioner for partikler i planen. Præcist som i kender det fra fysik, så er hastigheden af et objekt givet som den afledede af stedfunktionen.
\begin{defn}[Hastighed]
	Lad $\vv{r}: \mathbb{R}\to \mathbb{R}^2$ være en stedfunktion for en partikel givet ved
	\begin{align*}
		\vv{r}(t) = 
		\begin{pmatrix}
			x(t) \\
			y(t)
		\end{pmatrix},
\end{align*}		
hvor koordinatfunktionerne $x$ og $y$ begge er differentiable. 
Så er \textit{hastigheden} af partiklen til tiden $t$ givet ved
\begin{align*}
	\vv{v}(t) =  \vv{r}'(t) = 
	\begin{pmatrix}
		x'(t) \\
		y'(t)
	\end{pmatrix}.
\end{align*}
\end{defn}

\begin{exa}\label{exa:exa1}
	Vi betragter en partikel, hvis position er beskrevet ved vektorfunktionen $\vv{r}$ givet ved
	\begin{align*}
		\vv{r}(t) = 
		\begin{pmatrix}
			t^2-2t \\
			\frac{1}{5}t^5+3t^3
		\end{pmatrix}.
	\end{align*}
	Hastighedsfunktionen for denne vektorfunktion er så givet ved
	\begin{align*}
		\vv{v}(t) = \vv{r}'(t) = 
		\begin{pmatrix}
			2t-2 \\
			t^4+9t^2
		\end{pmatrix}.
	\end{align*}
	Skal vi bestemme hastigheden til tidspunktet $t=2$, så indsættes $2$ i hastighedsfunktionen. 
	\begin{align*}
		\vv{v}(2) = 
		\begin{pmatrix}
			2\cdot 2 - 2 \\
			2^4+9\cdot 2^2
		\end{pmatrix} = 
		\begin{pmatrix}
			2 \\
			52
		\end{pmatrix}.
	\end{align*}
	I $x$-aksens retning er hastigheden derfor $2$ og i $y$-aksens retning er hastigheden $52$. 
\end{exa}
Som I også ved fra jeres fysikundervisning, så beskriver den afledede af en hastighedsfunktion accelerationen af partiklen. 
\begin{defn}[Acceleration]
Lad $\vv{r}:\mathbb{R} \to \mathbb{R}^2$ være en stedfunktion til en partikel givet ved
\begin{align*}
	\vv{r}(t) = 
	\begin{pmatrix}
		x(t) \\
		y(t)
	\end{pmatrix},
\end{align*}
hvor koordinatfunktionerne $x$ og $y$ begge er to gange differentiable funktioner. Så er \textit{accelerationen} af partiklen til tiden $t$ givet ved
\begin{align*}
	\vv{a}(t) = \vv{v}'(t) = \vv{r}''(t) = 
	\begin{pmatrix}
		x''(t) \\
		y''(t)
	\end{pmatrix}.
\end{align*}
\end{defn}

\begin{exa}
	Vi betragter igen vektorfunktionen $\vv{r}$ fra Eksempel \ref{exa:exa1} givet ved
	\begin{align*}
		\vv{r}(t) = 
		\begin{pmatrix}
			t^2-2t \\
			\frac{1}{5}t^5+3t^3
		\end{pmatrix}.	
	\end{align*}
	For at bestemme accelerationen for partiklen bestemmer vi $\vv{r}'':$
	\begin{align*}
		\vv{a}(t) = \vv{r}''(t) = 
		\begin{pmatrix}
			2 \\
			4t^3 + 18t
		\end{pmatrix}.
	\end{align*}
	Accelerationen af partiklen til tidspunktet $t=2$ er derfor
	\begin{align*}
		\vv{a}(2) = 
		\begin{pmatrix}
			2 \\
			4\cdot 2^3 + 18\cdot 2
		\end{pmatrix} =
		\begin{pmatrix}
			2 \\
			68
		\end{pmatrix}.
	\end{align*}
\end{exa}

\section*{Opgave 1}

Bestem hastigheds- og accelerationsvektorfunktionerne $\vv{v}$ og $\vv{a}$ for følgende vektorfunktioner.
\begin{align*}
	&1) \ \vv{r}(t) =
	\begin{pmatrix}
		\cos(t)  \\
		\sin(t)
	\end{pmatrix}
	&&2) \ \vv{r}(t) =
	\begin{pmatrix}
		t^4  \\
		t^2
	\end{pmatrix}   \\
	&3) \ \vv{r}(t) =
	\begin{pmatrix}
		t^3-t^2  \\
		t^5-15t^3
	\end{pmatrix}
	&&4) \ \vv{r}(t) =
	\begin{pmatrix}
		t^3+t^2+t+1  \\
		t^6-10t^3
	\end{pmatrix}   \\
\end{align*}

\section*{Opgave 2}

\begin{enumerate}[label=\roman*)]
	\item En partikels stedfunktion er givet ved
	\begin{align*}
		\vv{r}(t) = 
		\begin{pmatrix}
			e^{2t} \\
			t^3-3t
		\end{pmatrix}.
	\end{align*}
	Bestem hastighedsvektoren til tidspunktet $t=0$ og accelerationsvektoren til tidspunktet $t=1$. 
	\item En partikels stedfunktion er givet ved
	\begin{align*}
		\vv{r}(t) = 
		\begin{pmatrix}
			\sin(2t) \\
			\cos(4t)
		\end{pmatrix}.
	\end{align*}
	Bestem hastighedsvektoren til tidspunktet $t=\pi$ og accelerationsvektoren til tidspunktet $t=\pi/2$.
\end{enumerate}

\section*{Opgave 3}

\begin{enumerate}[label=\roman*)]
	\item En partikels stedfunktion er givet ved
	\begin{align*}
		\vv{r}(t) = 
		\begin{pmatrix}
			\frac{1}{3}t^3 + \frac{1}{2}t^2+t+2 \\
			3t^2+2t+1
		\end{pmatrix}.
	\end{align*}
	Bestem $t$, så hastighedsvektoren for partiklen er givet ved
	\begin{align*}
		\begin{pmatrix}
			13 \\
			20
		\end{pmatrix}
	\end{align*}
	\item En partikels stedfunktion er givet ved
	\begin{align*}
		\vv{r}(t) = 
		\begin{pmatrix}
			t^4-4t^2 \\
			2t^2-2t
		\end{pmatrix}.
	\end{align*}
	Bestem de værdier for $t$, så accelerationen for partiklen er givet ved
	\begin{align*}
		\begin{pmatrix}
			56 \\
			4
		\end{pmatrix}.
	\end{align*}
\end{enumerate}

\section*{Opgave 4}
\begin{enumerate}[label=\roman*)]
	\item En partikels stedfunktion er givet ved
	\begin{align*}
		\vv{r}(t) = 
		\begin{pmatrix}
			t^2-8t \\
			\frac{1}{3}t^3-4t
		\end{pmatrix}.
	\end{align*}
	Bestem de værdier for $t$, så hastighedsvektoren er lodret, og bestem de værdier for $t$, så hastighedsvektoren er vandret. 
\end{enumerate}
\textsl{•}