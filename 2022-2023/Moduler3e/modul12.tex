\begin{center}
\Huge
Stationære punkter
\end{center}

\section*{Stationære punkter i én variabel}
\stepcounter{section}

Vi så sidste år på hvordan vi kunne bruge differentialregning til at bestemme ekstremumspunkter for funktioner. Sådanne punkter kaldes også for \textit{stationære punkter}. Vi skal bruge de partielle afledede til at afgøre, hvilken type stationært punkt et stationært punkt er. Vi starter med at betragte funktioner af én variabel.

\begin{setn}[Stationære punkter]
	\label{setn:1}
	Lad $f$ være differentiabel i $x_0$ og antag, at $f'(x_0) = 0$. 
	\begin{enumerate}[label=\roman*)]
		\item Hvis $f''(x_0) >0$, så er $x_0$ et minimumspunkt. 
		\item Hvis $f''(x_0)<0$, så er $x_0$ et maksimumspunkt. 
		\item Hvis $f''(x_0)=0$, så kan intet konkluderes.
	\end{enumerate}
\end{setn}
Vi vil ikke bevise denne sætning, men en intuitiv forklaring kan ses af Fig. \ref{fig:fmm}.
\begin{figure}[H]
	\centering
	\begin{tikzpicture}
		\node at (3,1) {$f(x)$};
		\node at (6,1) {$f'(x)$};
		\node at (9,1) {$f''(x)$};
		\draw (0,0) node[anchor = north west]  
		{
		\begin{tikzpicture}
			\begin{axis}[axis lines = middle, xmin = -3, xmax = 3,
			ticks = none,
			xmin = -2, xmax = 2,
			ymin = -1, ymax = 6,
			yscale = 0.3,xscale = 0.3]
				\addplot[color = blue!40, samples = 100, thick]  {x^2};
			\end{axis}
		\end{tikzpicture}
		};
		\draw (3,0) node[anchor = north west]
		{
		\begin{tikzpicture}
			\begin{axis}[axis lines = middle, xmin = -3, xmax = 3,
			ticks = none,
			xmin = -2, xmax = 2,
			ymin = -1, ymax = 6,
			yscale = 0.3,xscale = 0.3]
				\addplot[color = blue!40, samples = 100, thick]  {2*x};
			\end{axis}
		\end{tikzpicture}
		};
		
		\draw (6,0) node[anchor = north west]
		{
		\begin{tikzpicture}
			\begin{axis}[axis lines = middle, xmin = -3, xmax = 3,
			ticks = none,
			xmin = -2, xmax = 2,
			ymin = -1, ymax = 6,
			yscale = 0.3,xscale = 0.3]
				\addplot[color = blue!40, samples = 100, thick]  {2};
			\end{axis}
		\end{tikzpicture}
		};
		\draw (0,-3) node[anchor = north west]  
		{
		\begin{tikzpicture}
			\begin{axis}[axis lines = middle, xmin = -3, xmax = 3,
			ticks = none,
			xmin = -2, xmax = 2,
			ymin = -6, ymax = 1,
			yscale = 0.3,xscale = 0.3]
				\addplot[color = blue!40, samples = 100, thick]  {-x^2};
			\end{axis}
		\end{tikzpicture}
		};
		\draw (3,-3) node[anchor = north west]  
		{
		\begin{tikzpicture}
			\begin{axis}[axis lines = middle, xmin = -3, xmax = 3,
			ticks = none,
			xmin = -2, xmax = 2,
			ymin = -6, ymax = 1,
			yscale = 0.3,xscale = 0.3]
				\addplot[color = blue!40, samples = 100, thick]  {-2*x};
			\end{axis}
		\end{tikzpicture}
		};
		\draw (6,-3) node[anchor = north west]  
		{
		\begin{tikzpicture}
			\begin{axis}[axis lines = middle, xmin = -3, xmax = 3,
			ticks = none,
			xmin = -2, xmax = 2,
			ymin = -6, ymax = 1,
			yscale = 0.3,xscale = 0.3]
				\addplot[color = blue!40, samples = 100, thick]  {-2};
			\end{axis}
		\end{tikzpicture}
		};
		\draw (0,-6) node[anchor = north west]  
		{
		\begin{tikzpicture}
			\begin{axis}[axis lines = middle, xmin = -3, xmax = 3,
			ticks = none,
			xmin = -2, xmax = 2,
			ymin = -3, ymax = 4,
			yscale = 0.3,xscale = 0.3]
				\addplot[color = blue!40, samples = 100, thick]  {x^3};
			\end{axis}
		\end{tikzpicture}
		};
		\draw (3,-6) node[anchor = north west]  
		{
		\begin{tikzpicture}
			\begin{axis}[axis lines = middle, xmin = -3, xmax = 3,
			ticks = none,
			xmin = -2, xmax = 2,
			ymin = -3, ymax = 4,
			yscale = 0.3,xscale = 0.3]
				\addplot[color = blue!40, samples = 100, thick]  {3*x^2};
			\end{axis}
		\end{tikzpicture}
		};
		\draw (6,-6) node[anchor = north west]  
		{
		\begin{tikzpicture}
			\begin{axis}[axis lines = middle, xmin = -3, xmax = 3,
			ticks = none,
			xmin = -2, xmax = 2,
			ymin = -3, ymax = 4,
			yscale = 0.3,xscale = 0.3]
				\addplot[color = blue!40, samples = 100, thick]  {3*x};
			\end{axis}
		\end{tikzpicture}
		};
	\end{tikzpicture}
	\caption{Tre funktioner, der alle har et stationært punkt i $(0,0)$.}
	\label{fig:fmm}
\end{figure}

\begin{exa}
	Lad os bestemme stationære punkter for funktionen $f$ givet ved
	\begin{align*}
		f(x) = 3x^2+2x+1.
	\end{align*}
	Vi differentierer og sætter den afledede lig nul:
	\begin{align*}
		f'(x) = 6x+2.
	\end{align*}
	Dette sættes lig nul:
	\begin{align*}
		6x+2=0,
	\end{align*}
	så $f$ har et stationært punkt i $x=-1/3$. Vi differentierer igen:
	\begin{align*}
		f''(x) = 6, 
	\end{align*}
	Vi indsætter nu vores stationære punkt:
	\begin{align*}
		f''\left( -\frac{1}{3} \right) = 6, 
	\end{align*}
	og da dette tal er positivt ved vi fra Sætning \ref{setn:1}, at $f$ har et minimum i $-1/3$. 
\end{exa}

\section*{Stationære punkter i to variable}
\stepcounter{section}

Vi definerer stationære punkter i to variable nøjagtigt som vi gjorde i én variabel:
\begin{defn}[Stationære punkter]
Lad $f:\mathbb{R}^2 \to \mathbb{R}$ være givet. Hvis et punkt $P(x_0,y_0,f(x_0,y_0))$ opfylder, at 
\begin{align*}
	\nabla f(x_0,y_0) = \vv{0},
\end{align*} 
så siges $P$ at være et stationært punkt. 
\end{defn}
Maksima og minima for funktioner af to variable vil også findes i stationære punkter, men ikke alle stationære punkter er maksima eller minima. Vi har et tilsvarende resultat som i tilfældet med én variabel. 

\begin{setn}\label{setn:2}
	Lad $P(x_0,y_0,f(x_0,y_0))$ være et stationært punkt for en funktion $f:\mathbb{R}^2 \to \mathbb{R}$. 
	\begin{enumerate}[label=\roman*)]
		\item Hvis 
			$$f_{xx}(x_0,y_0)f_{yy}(x_0,y_0)>(f_{xy}(x_0,y_0))^2,$$
		så er $P$ et ekstremumspunkt. Hvis $f_{xx}(x_0,y_0)>0$, så er $P$ et minimum, og hvis 
		$f_{xx}(x_0,y_0) <0$, så er $P$ et maksimum. 
		\item Hvis 
			$$f_{xx}(x_0,y_0)f_{yy}(x_0,y_0)<(f_{xy}(x_0,y_0))^2,$$
		så er $P$ et saddelpunkt.
		\item Hvis 
			$$f_{xx}(x_0,y_0)f_{yy}(x_0,y_0)=(f_{xy}(x_0,y_0))^2,$$	
		så kan intet konkluderes.
	\end{enumerate}
\end{setn}

\begin{exa}
Vi skal bestemme, om funktionen $f$ givet ved
\begin{align*}
	f(x,y) = x^2+2x+y^2-4y
\end{align*}
har stationære punkter og i givet fald typen af dem. Vi starter med at bestemme gradienten.
\begin{align*}
	\nabla f(x,y) =
	\begin{pmatrix}
		2x+2 \\ 2y - 4		
	\end{pmatrix}.
\end{align*}
Gradienten er kun $\vv{0}$ i punktet $(-1,2)$. Vi bestemmer $f_{xx}$, $f_{yy}$ og $f_(xy)$:
\begin{align*}
	f_{xx}(-1,2) &= 2,
	f_{yy}(-1,2) &= 2,
	f_{xy}(-1,2) &= 0.
\end{align*} 
Da $2\cdot 2>0$ er det klart, at dette er et ekstremumspunkt ifølge Sætning \ref{setn:2}. Da $f_{xx}(x_0,y_0) = 2>0,$ så er det et minimumspunkt. 
\end{exa}

\section*{Opgave 1}
Bestem stationære punkter for følgende funktioner af én variabel, og afgør typen af de stationære punkter.
\begin{align*}
	&1) \  -x^2+4x+1  &&2) \ x^2-4   \\
	&3) \ \frac{1}{3}x^3+\frac{1}{2}x^2 + x  &&4) \ \sqrt{x^2+1}
\end{align*}

\section*{Opgave 2}
Bestem stationære punkter for følgende funktioner af to variable, og afgør typen af de stationære punkter. 
\begin{align*}
	&1)  \ x^2-y^2  &&2) \ x^2+yx+y^2   \\
	&3)  \ e^{3x^2+y^2+2x} &&4) \ \sqrt{x(y^2+3y)}  \\
	&5)  \ \ln(3x^2+4y^2+2xy)  &&6) \  \cos(x^2+yx+y^2)   \\
\end{align*}
