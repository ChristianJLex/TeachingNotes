
\begin{center}
\Huge
Tangentplaner og kugler
\end{center}

\stepcounter{section}
Har vi en funktion $f:\mathbb{R}^2 \to \mathbb{R}$, så kan vi i et punkt $(x_0,y_0,f(x_0,y_0))$ finde en tangentplan for $f$. 
\begin{setn}[Tangentplan]
	Lad $f:\mathbb{R}^2 \to \mathbb{R}$ samt et punkt $P(x_0,y_0,f(x_0,y_0))$ være givet. Så er ligningen for tangentplanen $T$ for $f$ i $P$ givet ved
	\begin{align*}
		z = f(x_0,y_0) + f_x'(x_0,y_0)(x-x_0) + f_y'(x_0,y_0)(y-y_0).
	\end{align*}
\end{setn}
\begin{proof}
Det skal I selv.
\end{proof}
\begin{exa}
	Lad $f$ være givet ved
	\begin{align*}
		f(x) = x^2 + y^2.
	\end{align*}
	Vi finder så tangentplanen i et punkt $P(2,2)$ ved
	\begin{align*}
		z &= f(x_0,y_0) + f_x(x_0,y_0)(x-x_0) + f_y(x_0,y_0)(y-y_0)\\
		&= 8 + 4(x-2) + 4(y-2).
	\end{align*}
\end{exa}

\begin{exa}
	Lad en kugle $K$ være givet ved ligningen
	\begin{align*}
		(x-3)^2 + (y+1)^2 + (z-1)^2 = 16.
	\end{align*}
	Vi vil gerne bestemme tangentplanen i punktet $P(2,-1,1)$. Vi bestemmer derfor en vektor til centrum $C(3,-1,1)$ fra $P$:
	\begin{align*}
		\vv{PC} = 
		\begin{pmatrix}
			3-2 \\ -1-(-1) \\ 1-(1+\sqrt{15}) 
		\end{pmatrix}=
		\begin{pmatrix}
			1 \\ 0 \\ -\sqrt{15}. 
		\end{pmatrix}
	\end{align*}
	Dette er en normalvektor til tangentplanen. Vi kan derfor bestemme tangentplanens ligning ud fra planens ligning.
	\begin{align*}
		1(x-2) + 0(y+1) - \sqrt{15}(z-1-\sqrt{15}) = 0 \ \Leftrightarrow	\\
		x-2-\sqrt{15}(z-1-\sqrt{15}) = 0
	\end{align*}
\end{exa}

\section*{Opgave 1}
\begin{enumerate}[label=\roman*)]
	\item Bestem en tangentplan for funktionen $f$ givet ved
	\begin{align*}
		f(x,y) = \sin(\frac{xy}{4}) 
	\end{align*}
	i $(0,2\pi)$. 
	\item Bestem en tangentplan for funktionen $f$ givet ved
	\begin{align*}
		f(x,y) = (xy)^4
	\end{align*}
	i $(-2,-3)$.
	\item Bestem en tangentplan for funktionen $f$ givet ved
	\begin{align*}
		f(x,y) = (x^2+y^2)^4
	\end{align*}
	i $(-1,2)$.  
\end{enumerate}

\section*{Opgave 2}

\begin{enumerate}[label=\roman*)]
	\item Bestem en tangentplan for kuglen $K$ med ligningen
	\begin{align*}
		x^2+y^2+z^2=9
	\end{align*}
	i punktet $(0,0,3)$.
	\item Bestem en tangentplan for kuglen $K$ med ligningen 
	\begin{align*}
		(x-4)^2+(y-4)^2 + (z-1)^2 = 25
	\end{align*}
	i punktet $(2,2,1+\sqrt{17})$.
	\item Bestem en tangentplan for kuglen $K$ med ligningen
	\begin{align*}
		x^2-2x+y^2-4x+z^2-6z=-10
	\end{align*}
	i punktet $(1,2,3+\sqrt{3})$. 
\end{enumerate}

\section*{Opgave 3}
Dette er en guidet tour gennem beviset for Sætning 1.1.

\begin{enumerate}[label=\roman*)]
	\item Argumentér for, at vektoren $\vv{v_x}$ givet ved
	\begin{align*}
		\vv{v_x} = 
		\begin{pmatrix}
			1 \\ 0 \\ f_x'(x_0,y_0)
		\end{pmatrix}
	\end{align*}
	er en retningsvektor for tangentlinjen i $(x_0,y_0)$ for snitfunktionen langs $y$-aksen.
	\item Argumentér for, at vektoren $\vv{v_x}$ givet ved
	\begin{align*}
		\vv{v_y} = 
		\begin{pmatrix}
			0 \\ 1 \\ f_y'(x_0,y_0)
		\end{pmatrix}
	\end{align*}
	er en retningsvektor for tangentlinjen i $(x_0,y_0)$ for snitfunktionen langs $x$-aksen.
	\item Givet to vektorer i rummet $\vv{a}$ og $\vv{b}$ kan vi bestemme deres krydsprodukt $\vv{a}\times \vv{b}$ ved
	\begin{align*}
		\vv{a} \times \vv{b} = 
		\begin{pmatrix}
			a_1 \\ a_2 \\ a_3 
		\end{pmatrix} \times 
		\begin{pmatrix}
			b_1 \\ b_2 \\ b_3
		\end{pmatrix} =
		\begin{pmatrix}
			a_2b_3-a_3b_2 \\
			-a_1b_3+a_3b_1 \\
			a_1b_2-a_2b_1
		\end{pmatrix}.
	\end{align*}
	Krydsproduktet af to vektorer er orthogonalt til begge vektorerne $\vv{a}$ og $\vv{b}$. 
	Argumenter for, at $\vv{v_x}\times \vv{v_y}$ må være en normalvektor til tangentplanen og brug dette til at udlede ligningen for tangentplanen. 
\end{enumerate}

\section*{Opgave 4}
Aflevering!