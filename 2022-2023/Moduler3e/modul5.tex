
\begin{center}
\Huge
Funktioner af to variable
\end{center}
\section*{Vektorer i rummet}
\stepcounter{section}

I har tidligere stiftet bekendtskab med funktionen $f: \mathbb{Z} \times \mathbb{Z} \to \mathbb{Q}$, givet ved
\begin{align*}
	f(x,y) = \frac{x}{y}.
\end{align*}
Denne funktion har ikke én, men to variable $x$ og $y$, og bestemmer så brøken mellem de to brøker. Hvis et punkt lægger på grafen for $f$, så skal punktet have tre koordinater 
\begin{align*}
P\left(x,y,\frac{x}{y}\right).
\end{align*}
Dette punkt ligger i et koordinatsystem med tre akser, $x$, $y$ og $z$. Et sådant koordinatsystem kan ses af Fig. \ref{fig:3d}.
\begin{figure}[H]
	\centering
	\begin{tikzpicture}
		\begin{axis}
		[
		axis lines =middle,
		xmin = -1, xmax = 3,
		ymin = -1, ymax = 3, 
		zmin = -1, zmax = 3,
		ticks = none,
		xlabel = {$x$},
		ylabel = {$y$},
		zlabel = {$z$}
		]		
		\end{axis}
	\end{tikzpicture}
	\caption{Koordinatsystemer i rummet.}
	\label{fig:3d}
\end{figure}

Da grafer for funktioner af to variable tilsvarer punktmængder af punkter i rummet, vil vi tilsvarende introducere vektorer i rummet. 
\begin{defn}[Vektorer i rummet]
	En vektor i rummet $\vv{v}\in \mathbb{R}^{3}$ defineres som et objekt
	\begin{align*}
		\vv{v} =
		\begin{pmatrix}
			x \\ y \\ z
		\end{pmatrix},
	\end{align*}
	hvor $x,y,z\in \mathbb{R}$. 
\end{defn}
\begin{exa}
To vektorer i rummet $\vv{u}$ og $\vv{v}$ er givet ved henholdsvist
\begin{align*}
	\vv{u} =
	\begin{pmatrix}
		2 \\ -2 \\ -3
	\end{pmatrix},
\end{align*}
og
\begin{align*}
	\vv{v} = 
	\begin{pmatrix}
		-1 \\ 4 \\ 2
	\end{pmatrix}.
\end{align*}

Disse vektorer kan ses på Fig. \ref{fig:vec3d}.
\begin{figure}[H]
	\centering
	\begin{tikzpicture}
		\begin{axis}
		[
		axis lines = middle, 
		xmin = -4, xmax = 4,
		ymin = -4, ymax = 4,
		zmin = -4, zmax = 4,		
		xlabel = {$x$},
		ylabel = {$y$},
		zlabel = {$z$},
		xticklabels = empty,
		yticklabels = empty,
		zticklabels = empty,
		]
		\draw[-{Stealth[scale = 1.3]}] (axis cs: 0,0,0) -- (axis cs: 2,-2,-3);
		\draw[-{Stealth[scale = 1.3]}] (axis cs: 0,0,0) -- (axis cs: -1,4,2);
		\draw[color = white, thick] (axis cs: -1,0,0) -- (axis cs: 2,0,0);
		\draw[color = white, thick] (axis cs: 0,0,-3) -- (axis cs: 0,0,2);
		\draw[color = white, thick] (axis cs: 0,-2,0) -- (axis cs: 0,4,0);
		
		\draw[dashed, color = blue!40, thick] (axis cs: 0,0,0) -- (axis cs: 2,0,0);
		\draw[dashed, color = blue!40, thick] (axis cs: 2,0,0) -- (axis cs: 2,-2,0);
		\draw[dashed, color = blue!40, thick] (axis cs: 2,-2,0) -- (axis cs: 2,-2,-3);
		\draw[dashed, color = blue!40, thick] (axis cs: 0,0,0) -- (axis cs: 0,-2,0);
		\draw[dashed, color = blue!40, thick] (axis cs: 0,-2,0) -- (axis cs: 0,-2,-3);
		\draw[dashed, color = blue!40, thick] (axis cs: 0,0,0) -- (axis cs: 0,0,-3);
		\draw[dashed, color = blue!40, thick] (axis cs: 0,-2,0) -- (axis cs: 2,-2,0);
		\draw[dashed, color = blue!40, thick] (axis cs: 0,-2,-3) -- (axis cs: 0,-2,-3);
		\draw[dashed, color = blue!40, thick] (axis cs: 0,-2,-3) -- (axis cs: 2,-2,-3);
		\draw[dashed, color = blue!40, thick] (axis cs: 0,0,-3) -- (axis cs: 2,0,-3);
		\draw[dashed, color = blue!40, thick] (axis cs: 2,0,0) -- (axis cs: 2,0,-3);
		\draw[dashed, color = blue!40, thick] (axis cs: 2,0,-3) -- (axis cs: 2,-2,-3);
		\draw[dashed, color = blue!40, thick] (axis cs: 0,0,-3) -- (axis cs: 0,-2,-3);
		
		\draw[dashed, color = red!40, thick] (axis cs: 0,0,0) -- (axis cs: 0,4,0);
		\draw[dashed, color = red!40, thick] (axis cs: 0,0,0) -- (axis cs: -1,0,0);
		\draw[dashed, color = red!40, thick] (axis cs: 0,0,0) -- (axis cs: 0,0,2);
		\draw[dashed, color = red!40, thick] (axis cs: 0,0,2) -- (axis cs: -1,0,2);
		\draw[dashed, color = red!40, thick] (axis cs: 0,0,2) -- (axis cs: 0,4,2);
		\draw[dashed, color = red!40, thick] (axis cs: 0,4,0) -- (axis cs: 0,4,2);
		\draw[dashed, color = red!40, thick] (axis cs: 0,4,0) -- (axis cs: -1,4,0);
		\draw[dashed, color = red!40, thick] (axis cs: -1,0,0) -- (axis cs: -1,0,2);
		\draw[dashed, color = red!40, thick] (axis cs: -1,0,0) -- (axis cs: -1,4,0);
		\draw[dashed, color = red!40, thick] (axis cs: -1,4,0) -- (axis cs: -1,4,2);
		\draw[dashed, color = red!40, thick] (axis cs: -1,4,2) -- (axis cs: -1,0,2);
		\draw[dashed, color = red!40, thick] (axis cs: -1,4,2) -- (axis cs: -1,0,2);
		\draw[dashed, color = red!40, thick] (axis cs: -1,4,2) -- (axis cs: 0,4,2);
		\end{axis}
	\end{tikzpicture}
	\caption{To vektorer i rummet.}
	\label{fig:vec3d}
\end{figure}
\end{exa}
Vektorer i rummet opfører sig nøjagtigt som vektorer i planen. Følgende definition beskriver vektorsummation, differens, skalarmultiplikation og prikprodukt. 
\begin{defn}[Regneoperationer for vektorer]
	Lad $\vv{u}$ og $\vv{v}$ være vektorer givet ved
	\begin{align*}
		\vv{u} =
		\begin{pmatrix}
			x_1 \\ y_1 \\ z_1	
		\end{pmatrix}, \ \ \vv{u} =
		\begin{pmatrix}
			x_2 \\ y_2 \\ z_2
		\end{pmatrix}.
	\end{align*}
	Lad desuden $k\in \mathbb{R}$ være et vilkårligt tal. 
	
	Så defineres summen af $\vv{u}$ og $\vv{v}$ som 
	\begin{align*}
		\vv{u} + \vv{v} =
		\begin{pmatrix}
			x_1 \\ y_1 \\ z_1
		\end{pmatrix} +
		\begin{pmatrix}
			x_2 \\ y_2 \\ z_2
		\end{pmatrix} =
		\begin{pmatrix}
			x_1 + x_2 \\
			y_1 + y_2 \\
			z_1 + z_2
		\end{pmatrix}.
	\end{align*}	 
	Differensen af $\vv{u}$ og $\vv{v}$ defineres
	\begin{align*}
		\vv{u}-\vv{v} =
		\begin{pmatrix}
			x_1 \\ y_1 \\ z_1
		\end{pmatrix} -
		\begin{pmatrix}
			x_2 \\ y_2 \\ z_2
		\end{pmatrix} =
		\begin{pmatrix}
			x_1 - x_2 \\
			y_1 - y_2 \\
			z_1 - z_2
		\end{pmatrix}.
	\end{align*}
	Skalarmultiplikation med en konstant $k$ defineres som
	\begin{align*}
		k\vv{u} = k
		\begin{pmatrix}
			x_1 \\ y_1 \\ z_1
		\end{pmatrix}=
		\begin{pmatrix}
			kx_1 \\ ky_1 \\ kz_1
		\end{pmatrix}
	\end{align*}
	Til slut defineres prikproduktet mellem $\vv{u}$ og $\vv{v}$ som 
	\begin{align*}
		\vv{u}\cdot \vv{v} = 
		\begin{pmatrix}
			x_1 \\ y_1 \\ z_1
		\end{pmatrix}
		\cdot
		\begin{pmatrix}
			x_2 \\ y_2 \\ z_2
		\end{pmatrix}
		= 
		x_1x_2+y_1y_2+z_1z_2.
	\end{align*}
	To vektorer i rummet siges at være orthogonale, hvis deres prikprodukt er lig nul. 
\end{defn}

\begin{exa}
	Lad $\vv{u}$ og $\vv{v}$ være givet ved
		\begin{align*}
				\vv{u} =
				\begin{pmatrix}
					5 \\ 2 \\ -4
				\end{pmatrix}, \  \  \vv{v} = 
				\begin{pmatrix}
					2 \\ -3 \\ 1
				\end{pmatrix}
		\end{align*}
		Så er summen af $\vv{u}$ og $\vv{v}$ givet ved
		\begin{align*}
			\vv{u} + \vv{v} = 
			\begin{pmatrix}
				5+2 \\
				2-3 \\
				-4+1
			\end{pmatrix} = 
			\begin{pmatrix}
				7\\ -1 \\ -3
			\end{pmatrix}
		\end{align*}
		Differensen $\vv{u}-\vv{v}$ er
		\begin{align*}
			\vv{u} - \vv{v} =
			\begin{pmatrix}
				5-2\\
				2+3\\
				-4-1
			\end{pmatrix}=
			\begin{pmatrix}
				3 \\ 5 \\ -5
			\end{pmatrix}.
		\end{align*}
		Vi kan gange $\vv{v}$ med $-3$ og få
		\begin{align*}
			-3\vv{v}=
			\begin{pmatrix}
				-3\cdot 2\\
				-3 \cdot -3\\
				-3 \cdot 1
			\end{pmatrix} =
			\begin{pmatrix}
				-6 \\ 9 \\ -3
			\end{pmatrix}.
		\end{align*}
		Prikker vi vektorerne sammen så fås
		\begin{align*}
			\vv{u} \cdot \vv{v} = 
			\begin{pmatrix}
				5 \\ 2 \\ -4
			\end{pmatrix} \cdot
			\begin{pmatrix}
				2 \\ -3 \\ 1
			\end{pmatrix} =
			5\cdot 2 + 2\cdot(-3) + -4\cdot 1 = 0.
		\end{align*}
		Derfor er vektorerne $\vv{u}$ og $\vv{v}$ orthogonale.
\end{exa}
Vi definerer også længden af en vektor i rummet.
\begin{defn}[Længde af vektor]
	Lad $\vv{u}$ være en vektor i rummet givet ved
	\begin{align*}
		\vv{u} = 
		\begin{pmatrix}
			x \\ y \\ z
		\end{pmatrix}
	\end{align*}
	Så defineres \textit{længden} af $\vv{u}$ som 
	\begin{align*}
		|\vv{u}| = \sqrt{x^2+y^2+z^2}.
	\end{align*}
\end{defn}
\begin{exa}
	Lad $\vv{u}$ være givet ved
	\begin{align*}
		\vv{u} = 
		\begin{pmatrix}
			2 \\ 3 \\ 6
		\end{pmatrix}.
	\end{align*}
	Længden af $\vv{u}$ er så
	\begin{align*}
		|\vv{u}| = \sqrt{2^2+3^2+6^2} = \sqrt{49} =7. 
	\end{align*}
\end{exa}

\section*{Opgave 1}

Udregn følgende:
\begin{align*}
	&1) \ 
	\begin{pmatrix}
		1 \\ 2 \\ 4
	\end{pmatrix} + 
	\begin{pmatrix}
		5 \\ -3 \\ -7	
	\end{pmatrix}	 
	&&2) \  
	\begin{pmatrix}
		-10 \\ 4 \\ 6
	\end{pmatrix} - 
	\begin{pmatrix}
		11 \\ -15 \\ 1	
	\end{pmatrix}	 \\
	&3) \ 
	\begin{pmatrix}
		\frac{3}{2} \\ 1 \\ -\frac{1}{2}
	\end{pmatrix} + 
	\begin{pmatrix}
		-1 \\ -7 \\ -9	
	\end{pmatrix} +
	\begin{pmatrix}
		\frac{7}{3} \\ \frac{3}{4} \\ \frac{5}{2}	
	\end{pmatrix}	 
	&&4) \  
	\begin{pmatrix}
		-1 \\ 2 \\ 3
	\end{pmatrix} - 
	\begin{pmatrix}
		4 \\ -5 \\ 6	
	\end{pmatrix}+ \frac{1}{3}
	\begin{pmatrix}
		-2 \\ 6 \\ 9
	\end{pmatrix}
\end{align*}

\section*{Opgave 2}
Bestem følgende prikprodukter og afgør, om vektorerne er orthogonale.
\begin{align*}
	&1) \ 	
	\begin{pmatrix}
		1 \\ 4 \\ 7
	\end{pmatrix} \cdot
	\begin{pmatrix}
		-2 \\ 0 \\ 3
	\end{pmatrix} 
	&&2) \ 	
	\begin{pmatrix}
		1 \\ 0 \\ 0
	\end{pmatrix} \cdot
	\begin{pmatrix}
		0 \\ 10 \\ 3
	\end{pmatrix}
	\\
	&3) \ 
	\left(	
	\begin{pmatrix}
		0 \\ 2 \\ 5
	\end{pmatrix} +
	\begin{pmatrix}
		-4 \\ -6 \\ 6
	\end{pmatrix}
	\right)	
	\cdot 	
	\left(
	\begin{pmatrix}
		-2 \\ 1 \\ -11
	\end{pmatrix}
	+
	\begin{pmatrix}
		-4 \\ 5 \\ 7
	\end{pmatrix}
	\right)
	&&4) \ 	
	7
	\begin{pmatrix}
		0 \\ 0 \\ 4
	\end{pmatrix} 
	\cdot
	\frac{1}{2}
	\begin{pmatrix}
		-2 \\ 0 \\ 10
	\end{pmatrix}
\end{align*}
\section*{Opgave 3}
Bestem længden af følgende vektorer
\begin{align*}
	&1) \ 
	\begin{pmatrix}
		1 \\ 2 \\ 2
	\end{pmatrix}
	&&2) \ 
	\begin{pmatrix}
		4 \\ 1 \\ 8
	\end{pmatrix}
	\\
	&3) \ 
	\begin{pmatrix}
		2 \\ 6 \\ 9
	\end{pmatrix}
	&&4) \ 
	\begin{pmatrix}
		6 \\ 6 \\ 7
	\end{pmatrix}
	\\
	&5) \ 
	\begin{pmatrix}
		10 \\ 11 \\ 2
	\end{pmatrix}
	&&6) \ 
	\begin{pmatrix}
		4 \\ 4 \\ 7
	\end{pmatrix}
\end{align*}

\section*{Opgave 4}
\begin{enumerate}[label=\roman*)]
	\item Løs ligningen
	\begin{align*}
		\begin{pmatrix}
			2 \\ 3 \\ 4
		\end{pmatrix} =
		\begin{pmatrix}
			x \\ 3 \\ -9 
		\end{pmatrix}+
		\begin{pmatrix}
			-4 \\ y \\ z
		\end{pmatrix}.
	\end{align*}
	\item Løs ligningen
	\begin{align*}
		\begin{pmatrix}
			4 \\ 7 \\ -12
		\end{pmatrix} =
		3		
		\begin{pmatrix}
			x \\ y \\ z 
		\end{pmatrix} + 
		\begin{pmatrix}
			4 \\ 7 \\ 10
		\end{pmatrix}.
	\end{align*}
\end{enumerate}

\section*{Opgave 5}
Der gælder en række regneregler for vektorer i planen som også gælder for vektorer i rummet. Et udpluk af dem er følgende: For vektorer $\vv{u}$, $\vv{v}$ og $\vv{w}$ samt konstanter $k,c\in \mathbb{R}$ gælder der, at 
\begin{enumerate}[label=\roman*)]
	\item $\vv{u}+\vv{v} = \vv{v}+\vv{u}$.
	\item $(\vv{u}+\vv{v})+\vv{w} = \vv{u} + (\vv{v}+\vv{w})$.
	\item $k(\vv{u}+\vv{v}) = k\vv{u}+k\vv{v}$.
	\item $(k+c)\vv{u} = k\vv{u}+k\vv{v}$.
	\item $(kc)\vv{u} = k(c\vv{u})$.
	\item Der findes en vektor $\vv{0}$, så $\vv{0}+\vv{u}$ for alle vektorer $\vv{u}$. 
	\item For enhver vektor $\vv{u}$ findes der en vektor $-\vv{u}$, så $\vv{u}+(-\vv{u}) = \vv{0}$.
	\item $1\vv{u} = \vv{u}$.
\end{enumerate}
Vis, at disse regneregler er korrekte.