\begin{center}
\Huge
Flere separable differentialligninger
\end{center}

\section*{Separation af variable}
\stepcounter{section}
Vi så sidste gang, hvordan vi kunne bruge separation af variable til at løse separable differentialligninger.
\begin{setn}[Separation af variable]
	Lad $f$ og $g$ være kontinuerte funktioner, samt $g \neq 0$. 
	Så har differentialligningen
	\begin{align*}
		y' = h(x)g(y)
	\end{align*}
	den fuldstændige løsning $y = f(x)$, der opfylder, at 
	\begin{align*}
		\int \frac{1}{g(y)}dy = \int h(x) dx.
	\end{align*}
\end{setn}


\section*{Opgave 1}
\begin{enumerate}[label=\roman*)]
	\item Bestem den fuldstændige løsning til differentialligningen
	\begin{align*}
		\frac{dy}{dx} = \frac{x^2}{y^2}.
	\end{align*}
	\item Bestem den fuldstændige løsning til differentialligningen
	\begin{align*}
		y' = (x+1)\frac{1}{x^2+2x-1}(-y^2).
	\end{align*}
\end{enumerate}


\section*{Opgave 2}
\begin{enumerate}[label=\roman*)]
	\item Bestem den partikulære løsning til differentialligningen
	\begin{align*}
		y' = 2xe^y,
	\end{align*}
	der går gennem punktet $(0,1)$.
	\item Bestem den partikulære løsning til differentialligningen 
	\begin{align*}
		y' = (3x+1)\sin(x^3+x)y,
	\end{align*}
	der går gennem punktet $(0,1)$. 
\end{enumerate}

\section*{Opgave 3}
\begin{enumerate}[label=\roman*)]
	\item Bestem den fuldstændige løsning til differentialligningen
	\begin{align*}
		\frac{y'}{y^3} = x^2+x+1.
	\end{align*}
	\item Bestem den fuldstændige løsning til differentialligningen
	\begin{align*}
		\frac{y'}{\cos(x)e^{\sin(x)}} = y.
	\end{align*}
	\item Bestem den fuldstændige løsning til differentialligningen
	\begin{align*}
		\frac{y'}{y^5} = x^5
	\end{align*}
	\item Bestem den partikulære løsning til differentialligningen
	\begin{align*}
		\frac{y'}{\cos(x)\cos(\sin(x))} = \frac{1}{y},
	\end{align*}
	der opfylder, at $y(0)=4$.
\end{enumerate}

\section*{Opgave 4}
Et bestemt vejrfænomen kan beskrives ved differentialligningen
\begin{align}\label{eq:opg3}
	y'\cdot y^4 = 12x^3\cos(3x^4) + x^2\sin\left(\frac{1}{3}x^3\right).
\end{align}
\begin{enumerate}[label=\roman*)]
	\item Bestem en fuldstændig løsning til differentialligningen \eqref{eq:opg3}.
	\item Bestem den partikulære løsning til differentialligningen, der går gennem punktet $(0,2)$.
	\item Brug dit svar til at bestemme væksten af $y$, når $y = 3$. (Brug Maple).
\end{enumerate}

\section*{Opgave 5}
Brug separation af variable til at vise, at differentialligningen
\begin{align*}
	y' = \frac{x^n}{y^m}
\end{align*}
har den fuldstændige løsning
\begin{align*}
	y(x) = \sqrt[m+1]{\frac{m+1}{n+1}x^{n+1}+k}.
\end{align*}