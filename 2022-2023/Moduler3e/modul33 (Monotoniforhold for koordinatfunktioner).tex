\begin{center}
	\Huge
	Monotoniforhold for koordinatfunktionerne
\end{center}
\section*{Monotoniforhold}
\stepcounter{section}

Vi kan bestemme monotoniforholdene for koordinatfunktionerne $x$ og $y$ for en vektorfunktion
\begin{align*}
	\vv{r}(t) = 
	\begin{pmatrix}
		x(t) \\
		y(t).
	\end{pmatrix}
\end{align*}
Dette gøres nøjagtigt som da vi bestemte monotoniforhold for differentiable funktioner af én variabel i 2.g. Fortolkningen er også mere eller mindre analog. 
\begin{align*}
	&\textnormal{Hvis $y'(t)>0$, så bevæger partiklen sig opad på grafen.}\\
	&\textnormal{Hvis $y'(t)<0$, så bevæger partiklen sig nedad på grafen.}\\
	&\textnormal{Hvis $x'(t)>0$, så bevæger partiklen sig til højre på grafen.}\\
	&\textnormal{Hvis $x'(t)<0$, så bevæger partiklen sig til venstre på grafen.}\\
\end{align*}

\begin{exa}
Vi ønsker at bestemme monotoniforholdene for vektorfunktionen 
\begin{align*}
	\vv{r}(t) = 
	\begin{pmatrix}
		t^2-t+6 \\
		t^4-2t^2
	\end{pmatrix}
\end{align*}
Vi bestemmer de afledede af koordinatfunktionerne og sætter dem lig 0.
\begin{align*}
	x'(t) = 2t-1 = 0\\
\end{align*},
så $t=0.5$.
Tilsvarende for $y(t)$.
\begin{align*}
	y'(t) = 4t^3-4t = 0,
\end{align*}
så $t=0$, $t=1$ eller $t=-1$. 
Vi bestemmer desuden hældningerne mellem toppunkterne:
\begin{align*}
	x'(0) = 2\cdot 0 -1 = -1,
\end{align*}
og
\begin{align*}
	x'(1) = 2\cdot 1-1 = 1.
\end{align*}
Vi har derfor monotoniforholdene for $x$:
\begin{align*}
	&\textnormal{$x(t)$ er aftagende for $t\in (-\infty,0.5]$},\\
	&\textnormal{$x(t)$ er voksende for $t \in [0.5,\infty)$}.
\end{align*}
Tilsvarende bestemmer vi hældningerne for $y(t)$:
\begin{align*}
	y(-2) &= 4(-2)^3 -4(-2) = -24,\\
	y(-0.5) &= 4(-0.5)^3 - 4(-0.5) = 1.5,\\
	y(0.5) &= 4(0.5)^3 - 4(0.5) = -1.5,\\
	y(2) &= 4(2)^2 - 4(2) = 24.
\end{align*}
Derfor lyder monotoniforholdene for $y$:
\begin{align*}
	&\textnormal{$y(t)$ er aftagende for $t\in (-\infty,-1]$},\\
	&\textnormal{$y(t)$ er voksende for $t \in [-1,0]$},\\
	&\textnormal{$y(t)$ er aftagende for $t\in [0,1]$},\\
	&\textnormal{$y(t)$ er voksende for $t \in [1,\infty)$}.
\end{align*}

\end{exa}

\section*{Opgave 1}

Bestem monotoniforholdene for koordinatfunktionerne for følgende funktioner. Tegn parameterkurverne i Maple og undersøg, at du har fundet de korrekte monotoniforhold. 
\begin{align*}
	&1) \ \vv{r} = 
	\begin{pmatrix}
		t-4\\
		t^2-2
	\end{pmatrix}   
	&&2) \ \vv{r} = 
	\begin{pmatrix}
		2t^3\\
		t-7t^4
	\end{pmatrix}   \\
	&3) \ \vv{r} = 
	\begin{pmatrix}
		t^2-t-1\\
		t^4-t^3-t^2-t-1
	\end{pmatrix}   
	 &&4) \ \vv{r} = 
	\begin{pmatrix}
		t^6-t\\
		t^3+t
	\end{pmatrix}   \\
	&5) \ 
	\vv{r} = 
	\begin{pmatrix}
		3t^2-2t\\
		4t^3-3t^2
	\end{pmatrix}  
	  &&6) \
	  \vv{r} = 
	\begin{pmatrix}
		2t-1\\
		3t-2
	\end{pmatrix}    
\end{align*}