\begin{center}
\Huge
Planen og kuglens ligning
\end{center}
\section*{Planens ligning}
\stepcounter{section}

Vi har tidligere set, at linjens ligning er givet ved
\begin{align*}
	a(x-x_0)+b(y-y_0) = 0,
\end{align*}
hvor vektoren $\vv{n}$ givet ved
\begin{align*}
	\vv{n} =
	\begin{pmatrix}
		a \\ b
	\end{pmatrix}
\end{align*}
er en normalvektor til linjen og punktet $P(x_0,y_0)$ ligger på linjen. Udledningen af denne ligning består i at betragte alle vektorer, der er orthogonale til $\vv{n}$ og som starter i punktet $P$ må beskrive punktmængden for linjen. Vi kan gøre noget tilsvarende i rummet.

Vi lader $\vv{n}$ være en normalvektor til en plan, hvor 
\begin{align*}
	\vv{n} = 
	\begin{pmatrix}
		a \\ b \\ c
	\end{pmatrix}.
\end{align*}
Desuden lader vi $P(x_0,y_0,z_0)$ være et punkt på planen. Vi konstruerer nu planens ligning som følgende: Ethvert punkt $(x,y,z)$ på planen giver os en vektor langs planen, der  er orthogonal til normalvektoren $\vv{n}$ givet ved
\begin{align*}
	\vv{v} = 
	\begin{pmatrix}
		x-x_0 \\ y-y_0 \\ z-z_0.
	\end{pmatrix}
\end{align*}
Der må derfor gælde, at $\vv{v}\cdot \vv{n} = 0$. Skrives dette ud fås 
\begin{align*}
	\vv{v} \cdot \vv{n}\  &\Leftrightarrow 
	\begin{pmatrix}
		x-x_0 \\ y-y_0 \\ z-z_0
	\end{pmatrix} \cdot 
	\begin{pmatrix}
		a \\ b \\ c
	\end{pmatrix}\ \\
	 &\Leftrightarrow  a(x-x_0) + b(y-y_0) + c(z-z_0) = 0
\end{align*}
Denne ligning kaldes for \textit{planens ligning}.
\begin{setn}[Planens ligning]
	Lad $P(x_0,y_0,z_0)$ være et punkt på en plan og lad 
	\begin{align*}
		\vv{n} = 
		\begin{pmatrix}
			a \\ b \\ c
		\end{pmatrix}
	\end{align*}
	være en normalvektor til planen. Så er planens ligning givet ved
	\begin{align*}
		a(x-x_0)+b(y-y_0)+c(z-z_0)=0.
	\end{align*}
\end{setn}
\begin{exa}
Lad $P(1,2,3)$ være et punkt på en plan, og lad
\begin{align*}
	\vv{n} = 
	\begin{pmatrix}
		-2\\ 4\\ -3
	\end{pmatrix}
\end{align*}
være en normalvektor til planen. Så har planen ligningen
\begin{align*}
	-2(x-1)+4(y-2)-3(z-3) = 0.
\end{align*}
\end{exa}
\section*{Kuglens ligning}
\stepcounter{section}
Som det er tilfældet med cirklens ligning i planen kan vi også bestemme kuglens ligning i rummet. Vi lader $K$ være en kugle med centrum i $C(x_0,y_0,z_0)$ og med radius $r$. Ethvert punkt $P(x,y,z)$ på kuglens overflade vil så opfylde, at afstanden fra centrum til $(x,y,z)$ vil være $r$. Vektoren, der går fra $C$ til $P$ skal derfor have længde $r$. Skrives dette ud fås
\begin{align*}
	\left|
	\begin{pmatrix}
		x-x_0 \\ y-y_0 \\ z-z_0
	\end{pmatrix}
	\right| = r \ &\Leftrightarrow 
	\sqrt{(x-x_0)^2+(y-y_0)^2+(z-z_0)^2} = r \\\ &\Leftrightarrow \ (x-x_0)^2+(y-y_0)^2+(z-z_0)^2 = r^2.
\end{align*} 
Vi kan nu konkludere med en sætning:
\begin{setn}[Kuglens ligning]
	Lad $K$ være en kugle med centrum i $C(x_0,y_0,z_0)$ og radius $r$. Så er \textit{kuglens ligning} givet ved
	\begin{align*}
		(x-x_0)^2 + (y-y_0)^2 + (z-z_0)^2 = r^2.
	\end{align*}
\end{setn}

\begin{exa}
	Kuglen med radius i $C(4,2,-3)$ og radius $3$ har ligningen
	\begin{align*}
		(x-4)^2 + (y-2)^2 + (z+3)^2 = 9
	\end{align*}
\end{exa}

\begin{exa}
	En kugle har ligningen 
	\begin{align}\label{eq:ligning1}
		x^2-2x+y^2-4y+z^2-6z = 2.
	\end{align}
	Vi skal bestemme centrum og radius for kuglen, og skal derfor kvadratkomplettere som i tilfældet med cirklens ligning. Da vi har 
	\begin{align*}
		-2xx_0 &= -2x, \\
		-2yy_0 &= -4y, \\
		-2zz_0 &= -6z,
	\end{align*}
	så må $x_0=1$, $y_0 = 2$ og $z_0 = 3$. Vi lægger derfor $1^2+2^2+3^2$ til på begge sider af lighedstegnet i \eqref{eq:ligning1} og får
	\begin{align*}
		x^2-2x+y^2-4y+z^2-6z+14 = 2+14 = 16, .
	\end{align*}
	og derfor at cirklens radius er $\sqrt{16}=4$.
\end{exa}

\section*{Opgave 1}
\begin{enumerate}[label=\roman*)]
	\item På en plan $L$ ligger punktet $P(5,4,-2)$ og den har 
	\begin{align*}
		\vv{n} = 
		\begin{pmatrix}
			7 \\ 2 \\ -4
		\end{pmatrix} 
	\end{align*}
	som normalvektor. Bestem en ligning for $K$. 
	\item På en plan $L$ ligger punktet $P(1,10,5)$ og den har 
	\begin{align*}
		\vv{n} = 
		\begin{pmatrix}
			-11 \\ -12 \\ 13
		\end{pmatrix} 
	\end{align*}
	som normalvektor. Bestem en ligning for $K$. 
\end{enumerate}

\section*{Opgave 2}
\begin{enumerate}[label=\roman*)]
	\item En plan $L$ har ligningen
	\begin{align*}
		2(x-2) + 3(y+3) +5(z-1) = 0.
	\end{align*}
	Afgør om punkterne $(1,1,1)$ og $(1,6,-4)$ ligger på $L$. 
	\item En plan $L$ har ligningen 
	\begin{align*}
		z=0.
	\end{align*}
	Afgør, om punkterne $(10000,4,2)$ og $(\pi, e,0)$ ligger på $L$.
\end{enumerate}

\section*{Opgave 3}
\begin{enumerate}[label=\roman*)]
	\item En kugle $K$ har centrum i $(0,0,0)$ og radius $1$. Bestem ligningen for $K$. 
	\item En kugle $K$ har centrum i $(-2,4,8)$ og radius $5$. Bestem ligningen for $K$.
\end{enumerate}
\section*{Opgave 4}
\begin{enumerate}[label=\roman*)]
	\item En kugle $K$ har ligningen 
	\begin{align*}
		x^2 -4x+y^2+4y+z^2-8z=1.
	\end{align*}
	Bestem centrum og radius for $K$.
	\item En kugle $K$ har ligningen 
	\begin{align*}
		x^2-6x+y^2-6y+z^2-10z=21.
	\end{align*}
	Bestem centrum og radius for $K$.
	\item Tjek i GeoGebra, at du har fundet de rigtige kugler. (-: 
\end{enumerate}
\section*{Opgave 5}
\begin{enumerate}[label=\roman*)]
	\item En plan $L$ er givet ved ligningen 
	\begin{align*}
		2x+3y+6z = 45.
	\end{align*}
	Omskriv ligningen for $L$ til formen 
	\begin{align*}
		a(x-x_0)+b(y-y_0)+c(z-z_0) = 0.
	\end{align*}
	\item En plan $L$ er givet ved ligningen 
	\begin{align*}
		-4x+9y+7z = 9.
	\end{align*}
	Omskriv ligningen for $L$ til formen 
	\begin{align*}
		a(x-x_0)+b(y-y_0)+c(z-z_0) = 0.
	\end{align*}
\end{enumerate}
