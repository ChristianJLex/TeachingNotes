\begin{center}
	\Huge
	Differentialkvotient for vektorfunktioner
\end{center}
\section*{Differentiation af vektorfunktioner}
\stepcounter{section}

Da vi bestemte monotoniforhold for vektorfunktioner udnyttede vi, at vi kunne differentiere koordinatfunktionerne $x$ og $y$ for vektorfunktionen
\begin{align*}
	\vv{r}(t) = 
	\begin{pmatrix}
		x(t) \\
		y(t)
	\end{pmatrix}.
\end{align*}
Vi vil definere hvad vi mener med differentialkvotienten for en vektorfunktion lidt mere præcist. Den opfører sig (heldigvist) præcist som vi vil forvente. 
\begin{defn}[Differentialkvotient for vektorfunktion]
	Lad $\vv{r}: \mathbb{R} \to \mathbb{R}^2$ være en vektorfunktion.
	Så er differentialkvotienten for $\vv{r}$ defineret ved grænseværdien
	\begin{align*}
		\vv{r}'(t) = \lim_{h\to 0}\frac{\vv{r}(t+h)-\vv{r}(t)}{h},
	\end{align*}
	hvis grænsen eksisterer. 	
\end{defn} 
Følgende sætning sikrer os i, at differentiation af vektorfunktioner blot er differentiation af koordinatfunktioner. 
\begin{setn}
	Lad $\vv{r}(t):\mathbb{R} \to \mathbb{R}^2$ være givet ved
	\begin{align*}
		\vv{r}(t) = 
		\begin{pmatrix}
			x(t) \\
			y(t)
		\end{pmatrix},
	\end{align*}
	hvor både $x$ og $y$ er differentiable funktioner. Så gælder der, at 
	\begin{align*}
		\vv{r}'(t) = 
		\begin{pmatrix}
			x'(t) \\
			y'(t)
		\end{pmatrix}.
	\end{align*}
\end{setn}
\begin{proof}
	Vi betragter definitionen af differentialkvotienten for $\vv{r}$:
	\begin{align*}
		r'(t) &= \lim_{h\to 0}\frac{\vv{r}(t+h)-\vv{r}(t)}{h}\\
		&= \lim_{h\to 0 }
		\frac{
		\begin{pmatrix}
			x(t+h) \\
			y(t+h)
		\end{pmatrix} -
		\begin{pmatrix}
			x(t) \\
			y(t)
		\end{pmatrix}
		}{	
		h
		} \\
		&=
		\lim_{h\to 0 }
		\frac{
		\begin{pmatrix}
			x(t+h)-x(t) \\
			y(t+h)-y(t)
		\end{pmatrix}
		}{
		h
		}
		\\
		&=
		\lim_{h\to 0}
		\begin{pmatrix}
			\frac{x(t+h)-x(t)}{h} \\
			\frac{y(t+h)-y(t)}{h}
		\end{pmatrix} \\
		&=
		\begin{pmatrix}
			x'(t) \\
			y'(t)
		\end{pmatrix},
	\end{align*}
	hvor vi i sidste lighed udnytter, at $x$ og $y$ begge er differentiable funktioner.
\end{proof}

\begin{exa}
	Differentialkvotienten af vektorfunktionen
	\begin{align*}
		\vv{r}(t) = 
		\begin{pmatrix}
			t^2+2t+1 \\
			\ln(t) + 3e^{2t}
		\end{pmatrix}
	\end{align*}
	er givet ved
	\begin{align*}
		\vv{r}'(t) = 
		\begin{pmatrix}
			(t^2+2t+1)' \\
			(\ln(t) + 3e^{2t})'
		\end{pmatrix} =
		\begin{pmatrix}
			2t+2 \\
			\frac{1}{t} + 6e^{2t}
		\end{pmatrix}.
	\end{align*}
\end{exa}

\section*{Tangentvektorer}
\stepcounter{section}

Vi vil bruge differentialkvotienten for en vektorfunktion til at definere tangentvektoren til en parameterkurve for en vektorfunktion. 
\begin{defn}
	Lad $\vv{r}:\mathbb{R} \to \mathbb{R}^2$ være en vektorfunktion. Så er tangentvektoren til parameterkurven for $\vv{r}$ i punktet $P_t(x(t),y(t))$ givet ved $\vv{r}'(t)$. 
\end{defn}
\begin{exa}
	Betragt vektorfunktionen
	\begin{align*}
		\vv{r}(t) =
		\begin{pmatrix}
			t^4-2t^2 \\
			t^3-t
		\end{pmatrix}
	\end{align*}
	Vi ønsker at bestemme en parameterfremstilling for tangentlinjen for parameterkurven for $\vv{r}$ gennem punktet 
	\begin{align*}
		P_2(x(2),y(2)) = P_2(2^4-2\cdot 2^2,2^3-2) = P_2(8,6).
	\end{align*}
	Tangentvektoren for $\vv{r}$ er givet ved
	\begin{align*}
		\vv{r}'(t) =
		\begin{pmatrix}
			4t^3-4t \\
			3t^2-1
		\end{pmatrix}.
	\end{align*}
	Dermed er retningsvektoren for tangenten gennem $P_2$ givet ved
	\begin{align*}
		\vv{r}'(2) = 
		\begin{pmatrix}
			4\cdot 2^3 - 4\cdot 2 \\
			3\cdot 2^2-1
		\end{pmatrix}
		=
		\begin{pmatrix}
			24 \\
			11
		\end{pmatrix}.				
	\end{align*}
	Da er parameterfremstillingen for tangentlinjen givet ved
	\begin{align*}
		\begin{pmatrix}
			x \\
			y
		\end{pmatrix} =
		\begin{pmatrix}
			8 \\
			6
		\end{pmatrix} +
		t \begin{pmatrix}
			24 \\
			11
		\end{pmatrix}.
	\end{align*}
	
	Vi kan også bestemme en ligning for tangenten. Da 
	\begin{align*}
		\begin{pmatrix}
			24 \\
			11
		\end{pmatrix}
	\end{align*}
	er en retningsvektor for tangenten, så vil
	\begin{align*}
		\begin{pmatrix}
			11 \\
			-24
		\end{pmatrix}
	\end{align*}
	være en normalvektor til tangenten. Derfor er en ligning for tangenten givet ved
	\begin{align*}
		11(x-8) -24(y-6) = 0.
	\end{align*}
\end{exa}

\section*{Opgave 1}
Bestem den afledede af følgende vektorfunktioner
\begin{align*}
	&1) \ \vv{r}(t) = 
	\begin{pmatrix}
		6t^2+2t+1\\
		5t^2+10t
	\end{pmatrix}	
	  &&2) \ \vv{r}(t) = 
	\begin{pmatrix}
		\ln(t^2)\\
		4t^4
	\end{pmatrix}	   \\
	&3) \ \vv{r}(t) = 
	\begin{pmatrix}
		\cos(t) \sin(t)\\
		\cos(t)
	\end{pmatrix}	  &&4) \  \vv{r}(t) = 
	\begin{pmatrix}
		\sqrt{t}\ln(t)\\
		\sin(t^4+2t^2)
	\end{pmatrix}	  \\
\end{align*}

\section*{Opgave 2}
\begin{enumerate}[label=\roman*)]
	\item Bestem en parameterfremstilling for tangenten til parameterkurven for
	\begin{align*}
		\vv{r}(t) = 
		\begin{pmatrix}
			\ln(t) \\
			t^2
		\end{pmatrix}
	\end{align*}
	i punktet $P_1$.
	\item Bestem en parameterfremstilling for tangenten til parameterkurven for
	\begin{align*}
		\vv{r}(t) = 
		\begin{pmatrix}
			\frac{1}{3}t^3 - \frac{1}{2}t^2 \\
			t
		\end{pmatrix}
	\end{align*}
	i punktet $P_4$.
	\item Bestem en parameterfremstilling for tangenten til parameterkurven
	\begin{align*}
		\vv{r}(t) = 
		\begin{pmatrix}
			\cos(t) \\
			\sin(t)
		\end{pmatrix}
	\end{align*}
	i punktet $P_{\pi}$.
\end{enumerate}

\section*{Opgave 3}

\begin{enumerate}[label=\roman*)]
	\item Bestem en ligning for tangenten til parameterkurven for 
	\begin{align*}
		\vv{r}(t) = 
		\begin{pmatrix}
			t(t^2-1) \\
			2\sqrt{t}
		\end{pmatrix}
	\end{align*}
	i punktet $P_4$.
	\item Bestem en ligning for tangenten til parameterkurven for 
	\begin{align*}
		\vv{r}(t) = 
		\begin{pmatrix}
			\cos(t)\sin(t)
			1
		\end{pmatrix}
	\end{align*}
	i punktet $P_{\pi/2}$.
\end{enumerate}

\section*{Opgave 4}

\begin{enumerate}[label=\roman*)]
	\item Brug definitionen af differentialkvotienten for vektorfunktioner til at bestemme differentialkvotienten for 
	\begin{align*}
		\vv{r}(t) = 
		\begin{pmatrix}
			t^2 \\
			t^3
		\end{pmatrix}
	\end{align*}
	\item Brug definitionen af differentialkvotienten for vektorfunktioner til at bestemme differentialkvotienten for 
	\begin{align*}
		\vv{r}(t) = 
		\begin{pmatrix}
			\frac{1}{t} \\
			4t^2+2t+1
		\end{pmatrix}
	\end{align*}
\end{enumerate}