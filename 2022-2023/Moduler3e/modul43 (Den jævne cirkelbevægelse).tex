
\begin{center}
	\Huge
	Den jævne cirkelbevægelse
\end{center}

\section*{Den jævne cirkelbevægelse}
\stepcounter{section}

Vi kan parametrisere en cirkel ved hjælp af de trigonometriske funktioner $\cos$ og $\sin$.
\begin{defn}[Jævn cirkelbevægelse]
	\textit{Den jævne cirkelbevægelse} defineres som partiklen, hvis stedfunktion er vektorfunktionen $\vv{r}$ givet ved
	\begin{align*}
		\vv{r}(t) = 
		\begin{pmatrix}
			r\cos(\omega t) \\
			r\sin(\omega t )
		\end{pmatrix}.
	\end{align*}
\end{defn}
Vi vil gerne vise, at hastigheden af partiklen er konstant, og at partiklens bane udgør en cirkel med radius $r$. 
\begin{setn}
	Den jævne cirkelbevægelse beskrevet ved vektorfunktionen
	\begin{align*}
		\vv{r}(t) = 
		\begin{pmatrix}
			r\cos(\omega t)\\
			r\sin(\omega t)
		\end{pmatrix}		
	\end{align*}
	har konstant hastighed $\omega r$ og banekurven for partiklen udgør en cirkel med radius $r$ og centrum i $(0,0)$.
\end{setn}
\begin{proof}
	Vi betragter først hastighedsfunktionen for $\vv{r}$.
	\begin{align*}
		\vv{v}(t) = \vv{r}'(t) = 
		\begin{pmatrix}
			-r \omega \sin(\omega t) \\
			r \omega \cos(\omega t)
		\end{pmatrix}.
	\end{align*}
	Vi bestemmer nu farten $|\vv{v}|$.
	\begin{align*}
		|\vv{v}(t)| &= \left|
		\begin{pmatrix}
			-r\omega \sin(\omega t) \\	
			r \omega \cos( \omega t)
		\end{pmatrix}				
		\right| \\
		&= \sqrt{(-r\omega \sin(\omega t))^2 + (r \omega\cos(\omega t))^2} \\
		&= \sqrt{r^2\omega^2(\sin(\omega t)^2+\cos(\omega t)^2)}\\
		&= r\omega \sqrt{\sin(\omega t)^2 + \cos(\omega t )^2}\\
		&= \omega r,
	\end{align*}
	hvor den sidste lighed følger af idiotformlen
	\begin{align*}
		\cos(x)^2 + \sin(x)^2 = 1.
	\end{align*}
	Vi definerede en cirkel som alle punkter, der har afstand $r$ fra centrum. Vi undersøger derfor længden af stedfunktionen $\vv{r}$, da dette må tilsvare afstanden fra 
	$(0,0)$. 
	\begin{align*}
		|\vv{r}(t)| &= 
		\left| 
		\begin{pmatrix}
			r\cos(\omega t) \\
			r\sin(\omega t)		
		\end{pmatrix}
		\right| \\
		&= \sqrt{(r\cos(\omega t))^2 + (r\sin(\omega t))^2} \\
		&= \sqrt{r^2(\cos(\omega t)^2 + \sin(\omega t)^2)} \\
		&= r\sqrt{\cos(\omega t)^2 + \sin(\omega t)^2} \\
		&= r,
	\end{align*}
	hvor vi igen udnytter idiotformlen. Da stedvektoren har konstant afstand $r$ fra origo, er kontinuert og har konstant fart må banekurven for $\vv{r}$ udgøre en cirkel med 
	radius $r$.
\end{proof}


\section*{Opgave 1}
Lad en jævn cirkelbevægelse være givet ved
	\begin{align*}
		\vv{r}(t) = 
		\begin{pmatrix}
			2\cos(\pi t) \\
			2\sin(\pi t)
		\end{pmatrix}.
	\end{align*}	
\begin{enumerate}[label=\roman*)]
	\item Bestem $r(0)$ og $r(1)$.
	\item Bestem de værdier for $t$, så 
	\begin{align*}
		\vv{r}(t) = 
		\begin{pmatrix}
			0 \\ 1
		\end{pmatrix}
	\end{align*}
\end{enumerate}

\section*{Opgave 2}
Lad en jævn cirkelbevægelse være givet ved
	\begin{align*}
		\vv{r}(t) = 
		\begin{pmatrix}
			2\cos(\omega t) \\
			2\sin(\omega t)
		\end{pmatrix}.
	\end{align*}	
\begin{enumerate}[label=\roman*)]
	\item Det oplyses, at $P_0(1,0)$, $P_1(1,0)$ og $P_2(-1,0)$. Bestem ud fra dette vinkelhastigheden $\omega$, så $\omega \leq 2\pi$.
	\item Hvis det oplyses, at omløbstiden er givet ved $T$, hvordan findes vinkelhastigheden $\omega$ så?
\end{enumerate}

\section*{Opgave 3}

\begin{enumerate}[label=\roman*)]
	\item Vis, at hastighedsvektoren for den jævne cirkelbevægelse er parallel med stedvektoren $\vv{r}$.
	\item Vis, at accelerationsvektoren for den jævne cirkelbevægelse peger i modsat retning af $\vv{r}$.
\end{enumerate}



