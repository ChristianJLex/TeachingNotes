\begin{center}
\LARGE
Delprøve uden hjælpemidler 
\end{center}
\stepcounter{section}

\begin{opgavetekst}{Opgave 1}
	En funktion $f$ er givet ved 
	\begin{align*}
		f(x) = 3x^2+2x+1
	\end{align*}
\end{opgavetekst}
	\begin{delopgave}{}{1}
		Bestem $f(7)$.
	\end{delopgave}
	\begin{delopgave}{}{2}
		Bestem det punkt, hvor grafen for $f$ skærer $y$-aksen.
	\end{delopgave}
\begin{opgavetekst}{Opgave 2}
	Punkterne $A(0,1)$, $B(-4,2)$ og  $C(5,3)$ er givet. 
\end{opgavetekst}
	\begin{delopgave}{}{1}
		Bestem vektoren $\vv{BC}$.
	\end{delopgave}
	\begin{delopgave}{}{2}
		Bestem en parameterfremstilling for linjen $l$, der går gennem $A$ og $B$.
	\end{delopgave}
	\begin{delopgave}{}{3}
		Bestem en ligning for linjen $m$, der går gennem $A$ og $C$. 
	\end{delopgave}
	\begin{delopgave}{}{4}
		Bestem skæringspunktet mellem $l$ og $m$. 
	\end{delopgave}
\begin{opgavetekst}{Opgave 3}
	En linje $l$ er givet ved ligningen 
	\begin{align*}
		2(x-1) + 3(y+4) = 0
	\end{align*}
\end{opgavetekst}
\begin{delopgave}{}{1}
	Bestem en ligning for linjen $m$, der er orthogonal med $l$ og som skærer gennem $P(-2,-4)$.
\end{delopgave}
\begin{opgavetekst}{Opgave 4}
	En ligning er givet ved
	\begin{align*}
		2(s+K) = 24s
	\end{align*}
\end{opgavetekst}
\begin{delopgave}{}{1}
	Isolér $K$ i ligningen. 
\end{delopgave}
\begin{opgavetekst}{Opgave 5}
	To punkter $P(1,2)$ og $Q(3,32)$ er givet. Funktionen $f$ givet ved
	\begin{align*}
		f(x) = b\cdot a^x
	\end{align*}
	skærer gennem $P$ og $Q$.
\end{opgavetekst}
\begin{delopgave}{}{1}
	Bestem konstanterne $a$ og $b$
\end{delopgave}
\begin{delopgave}{}{2}
	Udregn funktionsværdien $f(2)$. 
\end{delopgave}

\newpage
\begin{center}
\LARGE
Delprøve med hjælpemidler 
\end{center}
\stepcounter{section}

\begin{opgavetekst}{Opgave 6}
	I en bakteriekoloni kan bakterieantallet $B$ i de første 24 timer beskrives ved en eksponentiel sammenhæng. Et datasæt hvori bakterieantallet $B$ til tiden $t$ er givet i 
	Tab. \ref{tab:bakterie}.
	\begin{table}[H]
		\centering
		\begin{tabular}{c|c|c|c|c|c|c|c}
		$t$ (timer) &1 & 2 & 3 & 4 & 5 & 6 & 7 \\
		\hline
		$B$ (bakterier i mio.) & 19.1 & 22.6 & 29.1 & 32.9 & 44.0 & 50.4 & 65.1
		\end{tabular}
		\caption{Antallet af bakterier $(B)$ i mio. som funktion af tiden $(t)$ i timer. }
		\label{tab:bakterie}
	\end{table}\phantom{h}
\end{opgavetekst}
\begin{delopgave}{}{1}
	Brug datasættet fra Tab. \ref{tab:bakterie} til at bestemme en forskrift for $B(t)$.
\end{delopgave}
\begin{delopgave}{}{2}
	Bestem, hvor mange bakterier, der er efter 10 timer. 
\end{delopgave}
\begin{delopgave}{}{3}
	Afgør, hvornår antallet af bakterier overstiger 1 mia.
\end{delopgave}
\begin{opgavetekst}{Opgave 7}
	To linjer $l$ og $m$ er givet ved følgende ligning og parameterfremstilling henholdsvist:
	\begin{align*}
		&l: \ 2(x-5)-6(y+7) = 0 \\
		&m: \ 
		\begin{pmatrix}
			x \\ y
		\end{pmatrix}=
		\begin{pmatrix}
			1 \\ 1
		\end{pmatrix} + t
		\begin{pmatrix}
			-2 \\ -3
		\end{pmatrix}
	\end{align*}
\end{opgavetekst}
\begin{delopgave}{}{1}
	Bestem en retningsvektor for $l$ og $m$ og bestem derefter vinklen mellem linjerne. 
\end{delopgave}
\begin{delopgave}{}{2}
	Projicér retningsvektoren for $l$ ned på retningsvektoren for $m$.  
\end{delopgave}
\begin{delopgave}{}{3}
	Bestem skæringspunktet mellem $l$ og $m$.
\end{delopgave}
\begin{opgavetekst}{Opgave 9}
	Et polynomium $f$ er givet ved
	\begin{align*}
		f(x) = x^5+3x^4+x^2-4.
	\end{align*}
\end{opgavetekst}
\begin{delopgave}{}{1}
	Bestem graden for polynomiet og afgør det minimale og maksimale antal rødder for $f$. 
\end{delopgave}
\begin{delopgave}{}{2}
	Tegn grafen for $f$ på intervallet $[-4,2]$.
\end{delopgave}
\begin{delopgave}{}{3}
	Bestem rødderne for $f$ på intervallet $[-4,2]$.
\end{delopgave}
