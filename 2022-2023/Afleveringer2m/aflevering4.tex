\begin{center}
\LARGE
Delprøve uden hjælpemidler 
\end{center}
\stepcounter{section}
%%%%%%%%%%%%%%%%%%%%%%%%%%%%%%%%%%%%%%%%%%%%%%%%%%%%%%%%%%%%%%%%%%%%%%%
%							Ny Opgave!!!!!							%
%%%%%%%%%%%%%%%%%%%%%%%%%%%%%%%%%%%%%%%%%%%%%%%%%%%%%%%%%%%%%%%%%%%%%%%
\begin{opgavetekst}{Opgave 1}
En kvinde skal vælge tre atletikdiscipliner blandt 7.
\end{opgavetekst}
	\begin{delopgave}{}{1}
		På hvor mange måder kan hun vælge disse, hvis rækkefølgen af disciplinerne ikke er ligegyldig?
	\end{delopgave}
	\begin{delopgave}{}{2}
		På hvor mange måder kan hun vælge disse, hvis rækkefølgen af disciplinerne er ligegyldig?
	\end{delopgave}
%%%%%%%%%%%%%%%%%%%%%%%%%%%%%%%%%%%%%%%%%%%%%%%%%%%%%%%%%%%%%%%%%%%%%%%
%							Ny Opgave!!!!!							%
%%%%%%%%%%%%%%%%%%%%%%%%%%%%%%%%%%%%%%%%%%%%%%%%%%%%%%%%%%%%%%%%%%%%%%%
\begin{opgavetekst}{Opgave 2}
	En lineær funktion $f$ går gennem punkterne $P(2,5)$ og $Q(4,15)$.
\end{opgavetekst}
\begin{delopgave}{}{1}
	Bestem en forskrift for $f$. 
\end{delopgave}
\begin{delopgave}{}{2}
	Bestem skæringspunktet mellem grafen for $f$ og $x$-aksen. 
\end{delopgave}
\begin{meretekst}
	En anden funktion $g$ har forskriften 
	\begin{align*}
		g(x) = x^2+5x-21.
	\end{align*}
\end{meretekst}
\begin{delopgave}{}{1}
	Bestem koordinaterne til skæringspunkterne mellem graferne for $f$ og $g$.
\end{delopgave}
%%%%%%%%%%%%%%%%%%%%%%%%%%%%%%%%%%%%%%%%%%%%%%%%%%%%%%%%%%%%%%%%%%%%%%%
%							Ny Opgave!!!!!							%
%%%%%%%%%%%%%%%%%%%%%%%%%%%%%%%%%%%%%%%%%%%%%%%%%%%%%%%%%%%%%%%%%%%%%%%
\begin{opgavetekst}{Opgave 3}
	En linje $l$ går gennem punktet $(-5,5)$ og har retningsvektoren
	\begin{align*}
		\vv{r} = 
		\begin{pmatrix}
			1 \\ -9
		\end{pmatrix}.
	\end{align*}	
\end{opgavetekst}
\begin{delopgave}{}{1}
	Bestem en parameterfremstilling for $l$.
\end{delopgave}
\newpage
%%%%%%%%%%%%%%%%%%%%%%%%%%%%%%%%%%%%%%%%%%%%%%%%%%%%%%%%%%%%%%%%%%%%%%%
%							Ny Opgave!!!!!							%
%%%%%%%%%%%%%%%%%%%%%%%%%%%%%%%%%%%%%%%%%%%%%%%%%%%%%%%%%%%%%%%%%%%%%%%
\begin{opgavetekst}{Opgave 4}
	To vektorer $\vv{u}$ og $\vv{v}$ er givet ved
	\begin{align*}
		&\vv{u} =
		\begin{pmatrix}
			-2 \\ 5
		\end{pmatrix},
		&& \vv{v} =
		\begin{pmatrix}
			10 \\ 1
		\end{pmatrix}
	\end{align*}
\end{opgavetekst}
\begin{delopgave}{}{1}
	Bestem arealet af parallellogrammet, der udspændes af $\vv{u}$ og $\vv{v}$.
\end{delopgave}
\begin{meretekst}
En vektor $\vv{a}$ er givet ved
\begin{align*}
	\vv{a} =
	\begin{pmatrix}
		s \\ s^2
	\end{pmatrix}.
\end{align*}
\end{meretekst}
\begin{delopgave}{}{1}
	Bestem de værdier for $s$, så $\vv{a}$ og $\vv{v}$ er orthogonale.
\end{delopgave}
%%%%%%%%%%%%%%%%%%%%%%%%%%%%%%%%%%%%%%%%%%%%%%%%%%%%%%%%%%%%%%%%%%%%%%%
%							Ny Opgave!!!!!							%
%%%%%%%%%%%%%%%%%%%%%%%%%%%%%%%%%%%%%%%%%%%%%%%%%%%%%%%%%%%%%%%%%%%%%%%

\begin{opgavetekst}{Opgave 5}
	En eksponentialfunktion $f$ er givet ved
	\begin{align*}
		f(x) = 1.76\cdot 1.04^x.
	\end{align*}
\end{opgavetekst}
\begin{delopgave}{}{1}
	Bestem vækstraten for $f$ og forklar, hvad den fortæller om udviklingen for $f$.
\end{delopgave}
%%%%%%%%%%%%%%%%%%%%%%%%%%%%%%%%%%%%%%%%%%%%%%%%%%%%%%%%%%%%%%%%%%%%%%%
%							Ny Opgave!!!!!							%
%%%%%%%%%%%%%%%%%%%%%%%%%%%%%%%%%%%%%%%%%%%%%%%%%%%%%%%%%%%%%%%%%%%%%%%
\begin{opgavetekst}{Opgave 6}
	En cirkel har centrum i punktet $P(1,1)$ og radius $6$.
\end{opgavetekst}
\begin{delopgave}{}{1}
	Opskriv en ligning for cirklen.
\end{delopgave}
\begin{meretekst}
	Et punkt $Q(-2,-3)$ er givet.
\end{meretekst}
\begin{delopgave}{}{2}
	Bestem $|\vv{PQ}|$.
\end{delopgave}
\begin{delopgave}{}{3}
	Brug dit svar på b) til at afgøre, om $Q$ ligger inden i cirklen, på cirklen eller uden for cirklen. 
\end{delopgave}
\newpage
\begin{center}
\LARGE
Delprøve med hjælpemidler 
\end{center}
\stepcounter{section}
%%%%%%%%%%%%%%%%%%%%%%%%%%%%%%%%%%%%%%%%%%%%%%%%%%%%%%%%%%%%%%%%%%%%%%%
%							Ny Opgave!!!!!							%
%%%%%%%%%%%%%%%%%%%%%%%%%%%%%%%%%%%%%%%%%%%%%%%%%%%%%%%%%%%%%%%%%%%%%%%
\begin{opgavetekst}{Opgave 7}
	To punkter er givet ved $P(2,2)$ og $Q(4,-6)$. 
\end{opgavetekst}
\begin{delopgave}{}{1}
	Bestem $\vv{PQ}$.
\end{delopgave}
\begin{meretekst}
	En linje $l$ er givet ved ligningen
	\begin{align*}
		l:\ 464x+116y-703=0
	\end{align*}
\end{meretekst}
\begin{delopgave}{}{2}
	Afgør, om $l$ er parallel med $\vv{PQ}$. 
\end{delopgave}

%%%%%%%%%%%%%%%%%%%%%%%%%%%%%%%%%%%%%%%%%%%%%%%%%%%%%%%%%%%%%%%%%%%%%%%
%							Ny Opgave!!!!!							%
%%%%%%%%%%%%%%%%%%%%%%%%%%%%%%%%%%%%%%%%%%%%%%%%%%%%%%%%%%%%%%%%%%%%%%%

\begin{opgavetekst}{Opgave 8}
	Et polynomium $f$ er givet ved
	\begin{align*}
		f(x) = 3x^5-30x^2+14x+10.
	\end{align*}
\end{opgavetekst}
\begin{delopgave}{}{1}
	Tegn grafen for $f$ på intervallet $[-1,2.5]$.
\end{delopgave}
\begin{delopgave}{}{2}
	Bestem rødderne for $f$. 
\end{delopgave}
\begin{meretekst}
	En lineær funktion $g$ opfylder, at $f(-1) = g(-1)$ og at $f(2) = g(2)$.
\end{meretekst}
\begin{delopgave}{}{3}
	Bestem det tredje skæringspunkt mellem graferne for $f$ og $g$. 
\end{delopgave}

