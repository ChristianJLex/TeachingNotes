\documentclass[12pt]{article}
\input{preamble}

\pagestyle{fancy}
\fancyhf{}

\rhead{Nørre Gymnasium\\
3.e
}


%Husk at rette modul og dato!
\lhead{Matematik A\\
Modul ??
}
\chead{Februar 2023
}

\cfoot{Side \thepage \hspace{1pt} af \pageref{LastPage}}

\begin{document}

%Udfyld afsnit herunder og lav til egen Latex-fil

%Kopier følgende til overskrift:

%\begin{center}
%\Huge
%Aflevering 1
%\end{center}
%\section*{Opgave 1}
%\stepcounter{section}

\begin{center}
	\Huge
	Forberedelse til terminsprøve
\end{center}

\section*{Uden hjælpemidler}
\section*{Opgave 1}
\stepcounter{section}

	En vektorfunktion $\vv{r}$ er givet ved
	\begin{align*}
		\vv{r}(t) = 
		\begin{pmatrix}
			t^2 \\
			4t-16
		\end{pmatrix}.
	\end{align*}
\begin{enumerate}[label=\roman*)]
	\item Bestem skæringspunktet mellem parameterkurven for $\vv{r}$ og $x$-aksen. 
\end{enumerate}

\section*{Opgave 2}

	En funktion $f$ af to variable er givet ved
	\begin{align*}
		f(x,y) = x y-2y-2x-10
	\end{align*}
\begin{enumerate}[label=\roman*)]
	\item Bestem $f(-4,5)$.
	\item Bestem koordinaterne til det stationære punkt for $f$
\end{enumerate}

\section*{Opgave 3}


	En funktion $f$ er givet ved
	\begin{align*}
		f(x) = \sqrt{3x^4+\sin(x)}.
	\end{align*}

\begin{enumerate}[label=\roman*)]
	\item	Bestem $f'(x)$.
\end{enumerate}

\section*{Opgave 4}

En stokastisk variabel $X$ er normalfordelt med middelværdi 102.3 og spredning 11.3


\begin{enumerate}[label=\roman*)]
	\item	Angiv intervallet for de udfald, der ikke er exceptionelle udfald. 
	\item	Lad $f$ angive tæthedsfunktionen for $X$. Opskriv et integral der angiver sandsynligheden for at få et normalt udfald.
\end{enumerate}

\section*{Opgave 5}

\begin{enumerate}[label=\roman*)]
	\item Løs ligningen $(x^2-4)(x^2+2x-3)=0$.
\end{enumerate}

\section*{Opgave 6}

En funktion $f$ er givet ved
\begin{align*}
	f(x) = \frac{\cos(x)+4x^3}{\sin(x)+x^4}
\end{align*}
\begin{enumerate}[label=\roman*)]
	\item Løs integralet
	\begin{align*}
		I = \int f(x) dx
	\end{align*}
\end{enumerate}

\section*{Opgave 7}

To funktioner $f$ og $g$ er givet ved henholdsvist 
\begin{align*}
	f(x) = e^{x^3-12}
\end{align*}
og 
\begin{align*}
	g(x) = \ln(x)+4.
\end{align*}

\begin{enumerate}[label=\roman*)]
	\item Løs ligningen $g(f(x)) = 0$.
\end{enumerate}

\section*{Opgave 8}
En differentialligning er givet ved
	\begin{align*}
		y' = x^2\cdot y + y
	\end{align*}
	Det oplyses, at funktionen $f$ er en løsning til differentialligningen.
\begin{enumerate}[label=\roman*)]
	\item Bestem en ligning for tangenten til grafen for $f$ gennem punktet $P(2,3)$.
	\item Vis, at funktionen $g$ givet ved 
	\begin{align*}
		g(x) = e^{\frac{1}{3}(x^3-3x)}
	\end{align*}
	også er en løsning til differentialligningen.
\end{enumerate}

\newpage
\section*{Med hjælpemidler}
\section*{Opgave 9}

En funktion $f$ af to variable er givet ved
\begin{align*}
	f(x,y) = \ln(x)\cdot y - x^2 + 5
\end{align*}
\begin{enumerate}[label=\roman*)]
	\item Tegn grafen for $f$ så $(x,y,z)\in [0,5]\times [0,5]\times [-20,5]$.
	\item Bestem $k$, så snitkurven $g(x) = f(x,k)$ har et maksimum i $x=1$. 
\end{enumerate}

\section*{Opgave 10}

En funktion $f$ er givet ved
	\begin{align*}
		f(x) = \ln(x)+10x^2
	\end{align*}
	
\begin{enumerate}[label=\roman*)]
	\item Bestem en stamfunktion til $f$, der går gennem punktet $P(2,-13)$. 
\end{enumerate}

\section*{Opgave 11}

En bestemt væksttype kan beskrives ved differentialligningen
\begin{align*}
	y' = 1002.17\cdot y \cdot (0.56-y).
\end{align*}
Det oplyses desuden, at en løsning $y=f(x)$ opfylder, at $f(2.1) = 0.22$.

\begin{enumerate}[label=\roman*)]
	\item Bestem en forskrift for løsningen $f$.
	\item Bestem tidspunktet, hvor væksten er størst. 
\end{enumerate}

\section*{Opgave 12}

En funktion $h$ er givet ved
\begin{align*}
	h(x) = 2*cos(x)*x + x^2 - 10.
\end{align*}
\begin{enumerate}[label=\roman*)]
	\item Tegn grafen for $h$ på intervallet $[-5,5]$. 
	\item Bestem rødderne for $h$. 
	\item Bestem rumfanget af det omdregningslegeme der dannes, når området afgrænset af $x$-aksen og grafen for $h$ roteres om $x$-aksen.
\end{enumerate}

\section*{Opgave 13}

En vektorfunktion $\vv{r}$ er givet ved
\begin{align*}
	\vv{r}(t) = 
	\begin{pmatrix}
		t^4 - 2t^3 - 28t^2 + 40t + 180\\
		t^2 - 2t - 4
	\end{pmatrix}.
\end{align*}

\begin{enumerate}[label=\roman*)]
	\item Tegn parameterkurven for $\vv{r}$.
	\item Bestem de værdier for $t$, så parameterkurven for $\vv{r}$ har en lodret tangent.
	\item Bestem dobbeltpunktet for parameterkurven for $\vv{r}$. 
\end{enumerate}

\section*{Opgave 14}
Et \href{https://github.com/ChristianJLex/TeachingNotes/raw/master/2022-2023/Data%20og%20lign/tr%C3%A6ningTilTermin.xlsx}{\color{blue!60} datasæt} antages at være tilnærmelsesvist normalfordelt. 
\begin{enumerate}[label=\roman*)]
	\item Vis, at datasættet er tilnærmelsesvist normalfordelt
	\item Bestem sandsynligheden for at få et udfald på mindre end 160. 
\end{enumerate}

\end{document}
