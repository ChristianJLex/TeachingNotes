\begin{center}
\Huge
Kombinatorik og binomialkoefficienten
\end{center}

\section*{Binomialkoefficienten}
\stepcounter{section}
Vi diskuterede sidste gang permutationer og antallet af permutationer af $k$ elementer blandt $n$ elementer. I den bearbejdning tog vi ikke højde for, at permutationers rækkefølge ofte er ligegyldig. Trækker vi fem kort i et kortspil, så er vi ligeglade med, om vi har trukket spar to før klør fem eller omvendt. Dette vil vi tage højde for nu. D

\begin{defn}
Vi betegner antallet af måder, vi kan udvælge $k$ elementer blandt $n$ elementer, hvor rækkefølgen ikke har betydning som
\begin{align*}
K(n,k) = \binom{n}{k},
\end{align*}
hvoraf det er den sidste skrivemåde, der er klart mest anvendt (men jeres bog bruger $K(n,k)$). Dette symbol kaldes for binomialkoefficienten. En måde at udvælge $k$ elementer uden betydning af rækkefølge kaldes for en \textit{kombination}.
\end{defn}
\begin{setn}
Binomialkoefficienten $\binom{n}{k}$ er givet ved
\begin{align*}
\binom{n}{k} = \frac{n!}{k!(n-k)!}.
\end{align*}
\end{setn}
\begin{proof}
Antallet af måder, vi kan udvælge $k$ elementer blandt $n$ elementer, hvor rækkefølge har betydning så vi sidst var $P(n,k) = n!/(n-k)!$. Vi skal nu tage højde for alle de permutationer, der består af de samme elementer i forskellige rækkefølge. Men for ethvert valg af $k$ elementer, så er der jo lige præcis $k!$ måder at permutere dem. Derfor må vi have $k!$ gange for mange permutationer med. Altså er $\binom{n}{k}$ givet ved
\begin{align*}
 \binom{n}{k} = \frac{P(n,k)}{k!} = \frac{\frac{n!}{(n-k)!}}{k!} = \frac{n!}{k!(n-k)!}.
\end{align*}
\end{proof}

\begin{exa}
Vi skal bestemme på hvor mange måder, vi kan udvælge 3 kort i et kortspil. Da rækkefølge ikke betyder noget, og da der er 52 kort i et kortspil, så er det givet som
\begin{align*}
\binom{52}{3} = \frac{52!}{3!(49)!} = 21000.
\end{align*}
\end{exa}

\section*{Opgave 1}
Vi har en pose med fem forskellige bolde. 
\begin{enumerate}[label=\roman*)]
	\item På hvor mange forskellige måder kan vi vælge to bolde i posen, hvis rækkefølgen ikke betyder noget?
	\item På hvor mange forskellige måder kan vi vælge 4 bolde i posen, hvis rækkefølgen ikke betyder noget?
\end{enumerate}

\section*{Opgave 2}
En isbutik har 12 forskellige typer is, og vi er ligeglade med rækkefølgen af kugler
\begin{enumerate}[label=\roman*)]
	\item På hvor mange forskellige måder kan vi bestille 3 kugler is?
	\item Hvis vi vil have chokolade som den sidste kugle, på hvor mange forskellige måder kan vi så bestille is?
	\item Hvis vi vil have chokolade, men er ligeglade med placeringen, på hvor mange forskellige måder kan vi så
	 bestille en is?
\end{enumerate}



\section*{Opgave 3}
En mand har i sit klædeskab syv skjorter, fem par bukser og 3 jakker. Han skal have tre skjorter, to par bukser og en jakke med på ferie. På hvor mange måder kan han pakke sin kuffert?

\section*{Opgave 4}
Bestem følgende binomialkoefficienter:
\begin{align*}
&1) \ \binom{7}{2}  &&2) \ \binom{10}{3}  \\
&3) \ \binom{5}{4}  &&4) \ \binom{200}{1}   \\
\end{align*}

\section*{Opgave 5}
En pokerhånd består af fem kort fra et kortspil på 52 kort.
\begin{enumerate}[label=\roman*)]
\item Hvor mange hænder er der i poker?
\item En flush består af fem kort i samme kulør. På hvor mange forskellige måder kan man få flush med hjerter? På hvor mange måder kan man få flush i alt?
\item En straight består af 5 kort i følge; e.g. 2,3,4,5,6. På hvor mange forskellige måder kan man få straight?
\end{enumerate} 

\section*{Opgave 6}
Du skal i biografen med klassen. I en biograf med 200 sæder, på hvor mange forskellige måder kan I så vælge jeres sæder?

\section*{Opgave 7}
I skal lave netværksgrupper i klassen. Disse består af 5 personer.
\begin{enumerate}[label=\roman*)]
	\item På hvor mange forskellige måder kan man lave en netværksgruppe.
\end{enumerate}

\section*{Opgave 8}
Vi er til dimission og alle får serveret et glas champagne (eller hvad end, de nu ønsker). Alle gæster skåler med alle andre gæster, og vi hører i alt 435 klir fra glas, der støder sammen.
\begin{enumerate}[label=\roman*)]
	\item Hvor mange er der til dimissionen?
	\item Gæsterne skåler nu tre og tre på alle mulige måder. Hvor mange klir hører vi nu?
\end{enumerate}