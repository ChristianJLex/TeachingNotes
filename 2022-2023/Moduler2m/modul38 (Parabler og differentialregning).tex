
\begin{center}
\Huge
Differentialregning og poylnomier
\end{center}

\section*{Toppunkt og hældning}
\stepcounter{section}

Vi kan bruge differentialregning til at bestemme toppunktet for et andetgradspolynomium.
\begin{setn}[Toppunktsformlen]
	Lad $f$ være et andetgradspolynomium givet ved
	\begin{align*}
		f(x) = ax^2+bx+c.
	\end{align*}
	Så er toppunktet for parablen for $f$ givet ved
	\begin{align*}
		\left(\frac{-b}{2a},\frac{-d}{4a} \right),
	\end{align*}
	hvor $d$ er givet ved $b^2-4ac$.
\end{setn}
\begin{proof}
	Vi differentierer $f$ og sætter funktionen lig $0$. 
	\begin{align*}
		f'(x) = 2ax+b=0 \ &\Leftrightarrow \ 2ax = -b \\
		&\Leftrightarrow \ x = \frac{-b}{2a}.
	\end{align*}
	Dette udtryk indsættes nu i forskriften for $f$.
	\begin{align*}
		f\left( \frac{-b}{2a}\right) &= a\left(\frac{-b}{2a}\right)^2 + b\frac{-b}{2a} + c\\
		&= a \frac{b^2}{4a^2} - \frac{b^2}{2a} + c \\
		&= \frac{b^2}{4a} - \frac{b^2}{2a} + c \\
		&= \frac{b^2}{4a} - \frac{2b^2}{4a} + \frac{4ac}{4a} \\
		&= \frac{b^2 - 2b^2 + 4ac}{4a} \\
		&= \frac{-b^2 + 4ac}{4a} \\
		&= \frac{-d}{4a},
	\end{align*}
	hvor $d=b^2-4ac$.
\end{proof}
Vi kan også bruge differentialregning til at afgøre, hvorfor hældningen af en parabel i skæringspunktet med $y$-aksen er lig $b$.
\begin{setn}
	Lad $f$ være et andetgradspolynomium givet ved
	\begin{align*}
		f(x) = ax^2+bx+c.
	\end{align*}
	Så tilsvarer $b$ hældningen af parablen for $f$ i skæringen med $y$-aksen. 
\end{setn}
\begin{proof}
	Hældningen af grafen i skæringspunktet med $y$-aksen må være givet ved $f'(0)$. Dette bestemmes.
	\begin{align*}
		f'(0) = 2a\cdot 0 + b = b.
	\end{align*}
\end{proof}

Følgende er det sidste bevis, vi skal se i differentialregning.
\begin{setn}
	Der gælder, at 
	\begin{align*}
		\left(\frac{1}{x}\right)' = -\frac{1}{x^2}
	\end{align*}
	for $x\neq 0$. 
\end{setn}
\begin{proof}
	Vi anvender definitionen af differentialkvotienten for $f$ givet ved
	\begin{align*}
		f(x) = \frac{1}{x}.
	\end{align*}
	Denne er givet som
	\begin{align*}
		f'(x) &= \lim_{h \to 0} \frac{f(x+h)-f(x)}{h} \\
		&= \lim_{h \to 0} \frac{\frac{1}{x+h} - \frac{1}{x}}{h} \\
		&= \lim_{h \to 0} \frac{\frac{x}{x(x+h)}-\frac{x+h}{x(x+h)}}{h} \\
		&= \lim_{h \to 0} \frac{\frac{x-x-h}{x(x+h)}}{h} \\
		&= \lim_{h \to 0} \frac{-h}{x(x+h)h} \\
		&= \lim_{h \to 0} \frac{-1}{x(x+h)} \\
		&= \lim_{h \to 0} \frac{-1}{x^2+xh} \\
		&= -\frac{1}{x^2}.
	\end{align*}
\end{proof}

\section*{Opgave 1}

Del jer ind i 6 grupper. Bevis én af følgende differentialregningsregler i jeres gruppe.
\begin{align*}
	&(f(x)+g(x))' = f'(x)+ g'(x) \\
	&(x^2)' = 2x  \\
	&(x^3)' = 3x^2 \\
	&\textnormal{Tangentligningen er givet ved } y = f'(x_0)(x-x_0) + f(x_0) \\
	&\left( \frac{1}{x}\right)' = -\frac{1}{x^2}\\
	&\textnormal{Toppunkt for parabel er givet ved } \left(\frac{-b}{2a},\frac{-d}{4a}\right)
\end{align*}
Fremlæggelse for hinanden i slutningen af modulet. 