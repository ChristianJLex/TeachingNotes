\begin{center}
\Huge
Behandling af solcelledata
\end{center}

\section*{Opgaver}
\stepcounter{section}

\begin{enumerate}[label=\roman*)]
	\item Følgende \href{https://github.com/ChristianJLex/TeachingNotes/raw/master/2022-2023/Data%20og%20lign/Solcelledata.xlsx}{\color{blue!60} datasæt} er produktionsdata fra skolens solceller d. 24 juni 2022. 
	Overfør dette datasæt fra Excel til Maple
	\item Prøv at fitte et andengradspolynomium til datasættet. 
	\item Bestem toppunktet til dette andengradspolynomium. Hvornår på døgnet er elproduktionen maksimal? Hvad er den maksimale produktion. 
	\item Prøv i stedet at fitte polynomier af højere grad på datasættet. Bliver resultatet bedre? Sammenlign residualerne. 
	\item Bestem også for din nye model toppunktet, og sammenlign med toppunktet fra før.
	\item Hvornår på dagen er elproduktionen halv så stor som når produktionen er maksimal? Hvad med kvart så stor?
	\item På \href{https://www.suninfo.dk/solhojde/solhojde.php}{\color{blue!60} følgende side} kan du se solens højde på himlen (sørg for at vælge den rigtige dato). Vælg 8-10 datapunkter og prøv at bestemme en model, der beskriver sammenhængen mellem
	højden af solen på himlen og	elproduktionen. 
\end{enumerate}
