\begin{center}
\Huge
Opgaver i Analytisk Geometri
\end{center}

\section*{Opgave 1}
Udregn følgende:
\begin{align*}
	&1) 
	\begin{pmatrix}
		2 \\ -3		
	\end{pmatrix} \cdot 
	\begin{pmatrix}
		5 \\- 8
	\end{pmatrix}	
	&&2)  
	\det \left(\begin{pmatrix}
		6 \\ 1		
	\end{pmatrix},
	\begin{pmatrix}
		-8 \\ 10
	\end{pmatrix}
	 \right)
\end{align*}
\section*{Opgave 2}
\begin{enumerate}[label=\roman*)]
	\item Bestem en ligning for linjen $l$, der går gennem punktet $(4,3)$ og har 
	\begin{align*}
		\vv{n} = 
		\begin{pmatrix}
			6 \\ -20
		\end{pmatrix}
	\end{align*}
	som normalvektor.
	\item 
	Bestem en parameterfremstilling for linjen $m$, der går gennem punktet $(\sqrt{2},9)$ og som har
	\begin{align*}
		\vv{r} = 
		\begin{pmatrix}
			-7 \\ 5
		\end{pmatrix}
	\end{align*}
	som retningsvektor.
\end{enumerate}
\section*{Opgave 3}
\begin{enumerate}[label=\roman*)]
	\item Bestem skæringen mellem linjerne givet ved ligningerne
	\begin{align*}
		2(x-1)+7(y-2) = 0
	\end{align*}
	og 
	\begin{align*}
		-4(x-8) -7(y+3) = 0.
	\end{align*}
	\item Bestem skæringspunktet mellem linjerne givet ved ligningerne
	\begin{align*}
		6x+7y=0
	\end{align*}
	og 
	\begin{align*}
		y = 14x-16.	
	\end{align*}
\end{enumerate}

\section*{Opgave 4}
\begin{enumerate}[label=\roman*)]
	\item Bestem afstanden mellem punktet $(2,3)$ og linjen givet ved ligningen
	\begin{align*}
		4x+7y-3=0
	\end{align*}
	\item Bestem afstanden mellem punktet $(-1,6)$ og linjen givet ved ligningen
	\begin{align*}
		3(x-1) + 4(y-2) = 0.
	\end{align*}
\end{enumerate}
\section*{Opgave 5}
\begin{enumerate}[label=\roman*)]
	\item Bestem projektionen $\vv{a_{\vv{b}}}$ af følgende vektorer:
	\begin{align*}
		\vv{a} = 
		\begin{pmatrix}
			6  \\ 4
		\end{pmatrix},
		\ \vv{b} =
		\begin{pmatrix}
			3 \\ -4
		\end{pmatrix}.
	\end{align*}	
	\item Bestem projektionen $\vv{a_{\vv{b}}}$ af følgende vektorer:
	\begin{align*}
		\vv{a} = 
		\begin{pmatrix}
			7  \\ -5
		\end{pmatrix},
		\ \vv{b} =
		\begin{pmatrix}
			1 \\ 1
		\end{pmatrix}.
	\end{align*}	
	\item Bestem projektionen $\vv{a_{\vv{b}}}$ af følgende vektorer:
	\begin{align*}
		\vv{a} = 
		\begin{pmatrix}
			2  \\ 11
		\end{pmatrix},
		\ \vv{b} =
		\begin{pmatrix}
			-20 \\ 16
		\end{pmatrix}.
	\end{align*}	
\end{enumerate}
\section*{Opgave 6}
\begin{enumerate}[label=\roman*)]
	\item En cirkel er givet ved ligningen
	\begin{align*}
		(x-2)^2+(y-5)^2 = 256
	\end{align*}
	og en linje er givet ved
	\begin{align*}
		y=2x+1.
	\end{align*}
	Bestem skæringen mellem linjen og cirklen.
	\item En cirkel er givet ved ligningen
	\begin{align*}
		x^2+y^2 = 4
	\end{align*}
	og en linje er givet ved
	\begin{align*}
		y=600x-2.
	\end{align*}
	Bestem skæringen mellem linjen og cirklen.
\end{enumerate}
\section*{Opgave 7}
\begin{enumerate}[label=\roman*)]
	\item En cirkel har centrum i $(11,2)$ og punktet $(10,-77)$ ligger på cirklen. Bestem en ligning for tangenten til 
	cirklen i punktet $(10,-77)$.
	\item En cirkel har centrum i $(0.5,7)$ og punktet $(4,\sqrt{2})$ ligger på cirklen. Bestem en ligning for
	tangenten til cirklen i punktet $(4,\sqrt{2})$.
	\item En cirkel har centrum i $(6,5)$ og punktet $(7,-9)$ ligger på cirklen. Bestem en ligning for tangenten til 
	cirklen i punktet $(7,-9)$.
\end{enumerate}