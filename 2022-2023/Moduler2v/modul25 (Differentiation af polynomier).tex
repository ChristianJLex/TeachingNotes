\begin{center}
\Huge
Differentiation af polynomier og potensfunktioner
\end{center}
\section*{Polynomier og potensfunktioner}
\stepcounter{section}

Det er særligt nemt at differentiere potensfunktioner og polynomier. Følgende sætning fortæller os, hvordan vi gør.

\begin{setn}
	Lad $f$ være givet ved
	\begin{align*}
		f(x) = x^a.
	\end{align*}
	Så gælder, at 
	\begin{align*}
		f'(x) = ax^{a-1}.
	\end{align*}
\end{setn}

Sidste gang så vi specialtilfældende 
\begin{align*}
	(x^3)' = 3x^2
\end{align*}
og 
\begin{align*}
	(x^2)' = 2x.
\end{align*}

\begin{exa}
	Vi skal differentiere polynomiet 
	\begin{align*}
		f(x) = 5x^7-12x^3-2x^2-x-1.
	\end{align*}
	Dette gøres ledvist og konstanter ganget på lades stå som sædvanligt.
	\begin{align*}
		f'(x) &= 5\cdot 7x^6 - 12\cdot 3x^2-2\cdot 2x-1\\
		&= 35x^6-36x^2-4x-1
	\end{align*}
\end{exa}

\begin{exa}
	Vi skal differentiere funktionen 
	\begin{align*}
		f(x) = 2x^{-\frac{1}{7}}.
	\end{align*}
	Vi bruger samme regel og får
	\begin{align*}
	 	f'(x) &= -\frac{2}{7}x^{-\frac{1}{7}-1} \\
	 	&= -\frac{2}{7}x^{-\frac{8}{7}}
	\end{align*}
\end{exa}

Vi tilføjer denne regel til vores tabel.
\begin{setn}
Vi har følgende sammenhæng mellem funktioner $f$ og afledede funktion $f'$.
\begin{center}
\begin{tabular}{c|c}
$f(x)$& $f'(x)$\\
\hline
\textnormal{konstant}&$0$\\
\hline
x&$1$\\
\hline
$ax+b$&$a$\\
\hline
$x^2$&$2x$\\
\hline
$x^3$&$3x^2$\\
\hline
$\frac{1}{x}$&$-\frac{1}{x^2}$\\
\hline
$\sqrt{x}$&$\frac{1}{2\sqrt{x}}$\\
\hline
$x^a$ & $ax^{a-1}$
\end{tabular}
\end{center}
\end{setn}

Vi skal se vores første bevis for en differentialkvotient. Vi viser, at $(x^2)' = 2x$. 

\begin{setn}[Differentiation af $x^2$]
	For funktionen $f$ givet ved
	\begin{align*}
		f(x) = x^2
	\end{align*}
	gælder der, at
	\begin{align*}
		f'(x) = 2x.
	\end{align*}
\end{setn}
\begin{proof}
	Vi betragter definitionen af differentialkvotienten.
	\begin{align*}
		f'(x) &= \lim_{h\to 0}\frac{f(x+h)-f(x)}{h}\\
		&= \lim_{h\to 0} \frac{(x+h)^2-x^2}{h}\\
		&= \lim_{h\to 0} \frac{x^2+h^2+2xh-x^2}{h}\\
		&= \lim_{h \to 0} \frac{h^2+2xh}{h}\\
		&=\lim_{h\to 0} h+2x\\
		&= 2x.
	\end{align*}
\end{proof}

\section*{Opgave 1}
Differentiér følgende funktioner
\begin{align*}
	&1) \ x^5   &&2) \ 2x^4-6x^2+10x-11     \\
	&3) \ \frac{1}{10}x^{10}+7x^5-11x^2+20  &&4) \ x^4-3x^2+2x+1     \\ 
	&5) \ \frac{1}{4}x^4 + \frac{1}{3}x^3+\frac{1}{2}x^2+x + 1  &&6) \  x^\frac{1}{2}    \\ 
	&7) \ 6x^{\frac{-5}{2}}  &&8) \  0.7x^{1.2}    \\ 
\end{align*}

\section*{Opgave 2}
Bestem hældningen af tangenterne punkterne $(2,f(2))$ og $ (3,f(3))$ af følgende funktioner. Tegn først funktionerne i Geogebra og prøv at bestemme hældningen der først. 
\begin{align*}
&1) \ 5x^2+10   &&2) \ \sqrt{x}    \\
&3) \ \frac{\sqrt{x}}{3} + x^3  &&4) \ \frac{2}{x}   \\
&5) \ 7   &&6) \ \frac{7\sqrt{x}}{2} + \frac{1}{x}   \\
&7) \ x^2+x^3+1  &&8) \ \frac{1}{3}x^3+\frac{1}{2}x^2 + x    \\
&9) \ \frac{10}{3} + x+3\sqrt{x}   &&10) \ 7+\frac{1}{3} + x^3 + \frac{2}{x}    \\
\end{align*} 
