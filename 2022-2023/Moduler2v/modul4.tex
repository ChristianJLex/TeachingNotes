\begin{center}
\Huge
Vinkler mellem linjer + hængeparti
\end{center}

\section*{Skæring givet parametrisering}
\stepcounter{section}
Tilsvarende kan vi også bestemme et skæringspunkt mellem to linjer $l$ og $m$, hvis deres parametrisering er givet. 
\begin{exa}
	Lad $l$ og $m$ være linjer med følgende parametriseringer:
	\begin{align*}
		&l: \ 
		\begin{pmatrix}
			x \\ y
		\end{pmatrix} = 
		\begin{pmatrix}
			4 \\ 4
		\end{pmatrix} + t
		\begin{pmatrix}
			2 \\ -2
		\end{pmatrix},\\
		&m: \ 
		\begin{pmatrix}
			x \\ y 
		\end{pmatrix} = 
		\begin{pmatrix}
			-4 \\ 0
		\end{pmatrix} + t
		\begin{pmatrix}
			5 \\ 1
		\end{pmatrix}.
	\end{align*}
	Vi skal bestemme skæringen mellem disse linjer. Vi sætter dem derfor lig hinanden:
	\begin{align*}
		\begin{pmatrix}
			4 \\ 4
		\end{pmatrix} + t_1
		\begin{pmatrix}
			2 \\ -2
		\end{pmatrix} =
		\begin{pmatrix}
			-4 \\ 0
		\end{pmatrix} + t_2
		\begin{pmatrix}
			5 \\ 1
		\end{pmatrix}.
	\end{align*}
	Dette giver os to lineære ligninger med to ubekendte, som vi enten kan løse med 
	substitution eller ved lige store koefficienters metode. Vi bruger substitutionsmetoden. 
	Ligningerne lyder:
	\begin{align*}
		4 + 2t_1 = -4+5t_2
	\end{align*} og
	\begin{align*}
		4-2t_1 = t_2.
	\end{align*}
	Anden ligning indsættes i første ligning:
	\begin{align*}
		&4 + 2t_1 = -4+5t_2\\
		\Leftrightarrow \ &4 + 2t_1 = -4+5(4-2t_1)\\
		\Leftrightarrow \ &4+2t_1 = -4+20-10t_1\\
		\Leftrightarrow \ &8+2t_1 = 20-10t_1\\
		\Leftrightarrow \ &12t_1 = 12\\
		\Leftrightarrow \ &t_1 = 1.
	\end{align*}
	Dette indsættes i første parameterfremstilling:
	\begin{align*}
		\begin{pmatrix}
			x \\ y
		\end{pmatrix} = 
		\begin{pmatrix}
			4 \\ 4 
		\end{pmatrix} 
		+ 1
		\begin{pmatrix}
			2 \\ -2
		\end{pmatrix} = 
		\begin{pmatrix}
			6 \\ 2
		\end{pmatrix}.
	\end{align*}
	Derfor skærer linjerne $l$ og $m$ hinanden i punktet $(6,2)$. 
\end{exa}

\section*{Vinkler mellem linjer}
\stepcounter{section}


I 1.g lærte vi, hvordan vi bestemte vinklen mellem to vektorer. Givet to vektorer $\vv{a}$ og $\vv{b}$, så gælder der følgende forhold mellem vinklen $v$ mellem $\vv{a}$ og $\vv{b}$ og deres indbyrdes prikprodukt:
\begin{align*}
	\vv{a}\cdot \vv{b} = \cos(v)\cdot |\vv{a}|\cdot |\vv{b}|.
\end{align*}
Vi kan derfor finde vinklen $v$ ved 
\begin{align*}
	\cos(v) = \frac{\vv{a}\cdot \vv{b}}{|\vv{a}|\cdot |\vv{b}|} \ \Leftrightarrow \ v
	= \cos^{-1}\left(\frac{\vv{a}\cdot \vv{b}}{|\vv{a}|\cdot |\vv{b}|}\right).
\end{align*}
Typisk vil vi blot bruge ligningen til venstre og så løse med \texttt{solve}-kommandoen i Maple. 
Lad os betragte et par eksempler.
\begin{exa}
	En linje $l$ er givet ved ligningen
	\begin{align*}
		l: \ 2(x-1) + 3(y-2) = 0,
	\end{align*}
	og en linje $m$ er givet ved ligningen
	\begin{align*}
		m: \ -1(x+2) -5(y-1) = 0,
	\end{align*}
	Linjen $l$ har derfor normalvektoren $\vv{n_l}$ givet ved
	\begin{align*}
		\vv{n_l} = \begin{pmatrix}
			2 \\ 3	
		\end{pmatrix},
	\end{align*}
	og linjen $m$ har normalvektoren $\vv{n_m}$ givet ved
	\begin{align*}
		\vv{n_m} = \begin{pmatrix}
			-1\\-5
		\end{pmatrix}.
	\end{align*}
	Vinklen mellem linjerne $l$ og $m$ må være lig vinklen mellem deres normalvektorer. 
	Vinklen $v$ mellem linjerne bestemmes derfor ved at løse ligningen
	\begin{align*}
		\cos(v) = \frac{2(-1)+3(-5)}{\sqrt{2^2+3^2}\cdot\sqrt{(-1)^2+(-5)^2}}. 
	\end{align*}
	Dette løses i Maple, og vinklen mellem linjerne fås til at være $157,62^{\circ}$. Dette er 
	klart den stumpe vinkel. Den spidse vinkel mellem linjerne er derfor 
	\begin{align*}
		180-157,62 = 22,38^{\circ}.
	\end{align*}
\end{exa}
\begin{exa}
	To linjer $l$ og $m$ er givet ved følgende to parameterfremstillinger henholdsvist.
	\begin{align*}
		&l: \ 
		\begin{pmatrix}
			x \\ y
		\end{pmatrix} = 
		\begin{pmatrix}
			1 \\ -1
		\end{pmatrix} + t
		\begin{pmatrix}
			-7 \\ 4
		\end{pmatrix}
		\\
		&m: \ 
		\begin{pmatrix}
			x \\ y
		\end{pmatrix} =
		\begin{pmatrix}
			-2 \\ -4
		\end{pmatrix} + t
		\begin{pmatrix}
			3 \\ 5
		\end{pmatrix}
	\end{align*}
	Tilsvarende eksemplet med linjens ligning må vinklen mellem to linjer være lig vinklen
	mellem linjernes retningsvektorer $\vv{r_l}$ og $\vv{r_m}$. Derfor findes vinklen mellem
	$l$ og $m$ som vinklen mellem 
	\begin{align*}
		\vv{r_l} = 
		\begin{pmatrix}
			-7 \\ 4
		\end{pmatrix}\\
	\end{align*}
	og 
	\begin{align*}
		\vv{r_m} = 
		\begin{pmatrix}
			3 \\ 5
		\end{pmatrix}
	\end{align*}
	
	Dette gøres igen i Maple, og vi får, at vinklen mellem linjerne $l$ og $m$ er givet ved
	$v = 91.22^{\circ}$. 
\end{exa}


\section*{Opgave 1}
\begin{enumerate}[label=\roman*)]

\item Bestem skæringen mellem linjerne $l$ og $m$, der har følgende parametriseringer henholdsvist:
\begin{align*}
\begin{pmatrix}
x \\ y
\end{pmatrix}
= 
\begin{pmatrix}
1 \\ -1
\end{pmatrix}
+
t
\begin{pmatrix}
2 \\ 1
\end{pmatrix}
\end{align*}
og 
\begin{align*}
\begin{pmatrix}
x \\ y
\end{pmatrix}
= 
\begin{pmatrix}
-3 \\ 3
\end{pmatrix}
+
t
\begin{pmatrix}
-2 \\ 1
\end{pmatrix}
\end{align*}

\item Bestem skæringen mellem linjerne $l$ og $m$, der har følgende parametriseringer henholdsvist:
\begin{align*}
\begin{pmatrix}
x \\ y
\end{pmatrix}
= 
\begin{pmatrix}
5 \\ 2
\end{pmatrix}
+
t
\begin{pmatrix}
3 \\ 4
\end{pmatrix}
\end{align*}
og 
\begin{align*}
\begin{pmatrix}
x \\ y
\end{pmatrix}
= 
\begin{pmatrix}
-1 \\ 6
\end{pmatrix}
+
t
\begin{pmatrix}
-3 \\ 6
\end{pmatrix}
\end{align*}
	\item Undersøg i Maple om du har fundet de rigtige skæringspunkter.
\end{enumerate}

\section*{Opgave 2}
\begin{enumerate}[label=\roman*)]
	\item En linje $l$ går gennem punkterne $(1,1)$ og $(2,3)$. Bestem en parametrisering for 
	$l$. 
	\item En linje $m$ går gennem punkterne $(-2,-4)$ og $(3,5)$. Bestem en ligning for $m$.
\end{enumerate}

\section*{Opgave 3}
\begin{enumerate}[label=\roman*)]
	\item En linje $l$ har parameterfremstillingen 
	\begin{align*}
		\begin{pmatrix}
			x \\ y
		\end{pmatrix} = 
		\begin{pmatrix}
			1 \\ -2
		\end{pmatrix} + t
		\begin{pmatrix}
			2 \\ 1
		\end{pmatrix}
	\end{align*}
	og en anden linje $m$ har ligningen
	\begin{align*}
		(x-3) + 7(y+2) = 0
	\end{align*}
	Bestem skæringspunktet mellem $l$ og $m$. 
	\item En linje $l$ har parameterfremstillingen 
	\begin{align*}
		\begin{pmatrix}
			x \\ y
		\end{pmatrix} = 
		\begin{pmatrix}
			0 \\ 5
		\end{pmatrix} + t
		\begin{pmatrix}
			5 \\ 3
		\end{pmatrix}
	\end{align*}
	og en anden linje $m$ har ligningen
	\begin{align*}
		-5(x+3) + 2(y-2) = 0
	\end{align*}
	Bestem skæringspunktet mellem $l$ og $m$. 
	\item Undersøg i Maple om du har fundet de rigtige skæringspunkter.
\end{enumerate}

\section*{Opgave 4}

\begin{enumerate}[label=\roman*)]
	\item To linjer $l$ og $m$ har følgende ligninger:
	\begin{align*}
		&l: \	4(x-1) + 4(y-1) =0,\\
		&m: \ 2(x+1) -2(y-7) = 0.
	\end{align*}	 
	Bestem den spidse vinkel mellem $l$ og $m$. 
	\item To linjer $l$ og $m$ har følgende ligninger:
	\begin{align*}
		&l: \ -7(x+13) - 20(y-1) = 0,\\
		&m: \ \frac{1}{3}(x+4) + \sqrt{2}(y-2) = 0.
	\end{align*}
	Bestem den stumpe vinkel mellem $l$ og $m$. 
\end{enumerate}

\section*{Opgave 5}
\begin{enumerate}[label=\roman*)]
	\item To linjer $l$ og $m$ har følgende parameterfremstillinger:
	\begin{align*}
		\begin{pmatrix}
			x \\ y
		\end{pmatrix}=
		\begin{pmatrix}
			0 \\ 1
		\end{pmatrix}+ t
		\begin{pmatrix}
			-4 \\ 5
		\end{pmatrix}.
	\end{align*}
	Bestem vinklen $v$ mellem $l$ og $m$. 
	
	\item To linjer $l$ og $m$ har følgende parameterfremstillinger:
	\begin{align*}
		\begin{pmatrix}
			x \\ y
		\end{pmatrix}=
		\begin{pmatrix}
			0 \\ 1
		\end{pmatrix}+ t
		\begin{pmatrix}
			-4 \\ 5
		\end{pmatrix}.
	\end{align*}
	Bestem den spidse vinkel mellem $l$ og $m$. 
\end{enumerate}

\section*{Opgave 6}
\begin{enumerate}[label=\roman*)]
	\item En linje $l$ er givet ved ligningen
	\begin{align*}
		6(x-6) + 8(y-8) = 0,
	\end{align*}
	og en linje $m$ er givet ved parametriseringen
	\begin{align*}
		\begin{pmatrix}
			x \\ y
		\end{pmatrix} = 
		\begin{pmatrix}
			\frac{1}{2} \\ \frac{2}{3}
		\end{pmatrix} + t
		\begin{pmatrix}
			\frac{3}{2} \\ \frac{4}{5}
		\end{pmatrix}
	\end{align*}
	Bestem vinklen mellem $l$ og $m$
\end{enumerate}
