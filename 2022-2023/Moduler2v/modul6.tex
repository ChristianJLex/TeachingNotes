\begin{center}
\Huge
Projektioner af vektorer
\end{center}

\section*{Projektioner}
\stepcounter{section}

Har vi to vektorer $\vv{a}$ og $\vv{b}$, så kan vi være interesserede i at bestemme den vektor $\vv{a_{\vv{b}}}$, der peger i samme retning som $\vv{b}$ og som i en forstand er så tæt på $\vv{a}$ som muligt. Vi kalder i et sådant tilfælde vektoren $\vv{a_{\vv{b}}}$ for projektionen af $\vv{a}$ på $\vv{b}$. Vi skriver også 
\begin{align*}
\textnormal{proj}_{\vv{b}}(\vv{a}) = \vv{a}_{\vv{b}}.
\end{align*}
\begin{figure}[H]
	\centering
	\begin{tikzpicture}
		\begin{axis}[axis lines = middle,
		xlabel = $x$,
		ylabel = $y$,
		ticks = none, 
		xmin = -1, xmax = 6,
		ymin = -1, ymax = 6,
		]
			\draw[-{Stealth[scale = 1.3]}, color = blue!40, thick] (axis cs: 0,0) -- (axis cs: 5,1);
			\draw[-{Stealth[scale = 1.3]}, color = red!40, thick] (axis cs: 0,0) -- (axis cs: 3,3);
			\draw[dashed, thick, color = red!40] (axis cs:  45/13, 9/13) -- (axis cs: 3,3);
			\draw[dashed, red!40, thick] (axis cs: 45/13-0.4-0.4*0.2, 9/13-0.4*0.2+0.4) -- (axis cs: 45/13-0.4, 9/13-0.4*0.2);
			\draw[dashed, red!40, thick] (axis cs: 45/13-0.4-0.4*0.2, 9/13-0.4*0.2+0.4) -- (axis cs: 45/13-0.4*0.2, 9/13+0.4);
			\node[color = blue!40] at (axis cs: 5,1-0.5) {$\vv{b}$};
			\node[color = red!40] at (axis cs: 3,3+0.5) {$\vv{a}$};
			\node[color = purple] at (axis cs:  45/13, 9/13-0.4) {$\vv{a_{\vv{b}}}$};
			\draw[-{Stealth[scale = 1.3]}, color = purple, thick] (axis cs: 0,0) -- (axis cs:  45/13, 9/13);
		\end{axis}
	\end{tikzpicture}
\end{figure}

Vi starter med at vise, hvordan vi finder projektionen af en vektor på en anden vektor. 
\begin{setn}[Projektionssætningen]
For to vektorer $\vv{a}$ og $\vv{b}$ er projektionen af $\vv{a}$ på $\vv{b}$, som vi betegner
\begin{align*}
\textnormal{proj}_{\vv{b}}(\vv{a}) = \vv{a}_{\vv{b}}, 
\end{align*}
givet ved
\begin{align*}
\textnormal{proj}_{\vv{b}}(\vv{a}) = \frac{\vv{a}\cdot \vv{b}}{|\vv{b}|^2} \vv{b}.
\end{align*}
Længden af  $\vv{a}_{\vv{b}}$ er givet ved
\begin{align*}
\left|\vv{a}_{\vv{b}}\right| = \frac{|\vv{a}\cdot \vv{b}|}{|\vv{b}|}
\end{align*}
\end{setn}
\begin{proof}
Af konstruktionen af projektionen $\vv{a_{\vv{b}}}$ så findes der et tal $k$, så 
\begin{align*}
k\vv{b} = \vv{a_{\vv{b}}}.
\end{align*}
Lad $n$ være en normalvektor til $\vv{u}$, der opfylder, at 
\begin{align*}
\vv{a_{\vv{b}}} + \vv{n} = \vv{a}.
\end{align*}
Vi har så, at 
\begin{align*}
\vv{n} = \vv{a} - \vv{a_{\vv{b}}}.
\end{align*}
Da $\vv{n}$ er en normalvektor til $\vv{b}$, så får vi følgende prikprodukt.

\begin{align*}
\vv{n} \cdot \vv{b}=0 &\Leftrightarrow (\vv{a}-\vv{a_{\vv{b}}}) \cdot \vv{b} = 0 \\
&\Leftrightarrow \vv{a}\cdot \vv{b} - \vv{a_{\vv{b}}}\cdot \vv{b} = 0 \\
&\Leftrightarrow \vv{a}\cdot \vv{b} = \vv{a_{\vv{b}}}\cdot \vv{b} = k \vv{b} \cdot \vv{b} \\
&\Leftrightarrow k  = \frac{\vv{a}\cdot \vv{b}}{\vv{b} \cdot \vv{b}} = \frac{\vv{a}\cdot \vv{b}}{|\vv{b}|^2}
\end{align*}

Derfor får vi, at 
\begin{align*}
\vv{a_{\vv{b}}} &= k \vv{b}\\
&= \frac{\vv{a}\cdot \vv{b}}{|\vv{b}|^2} \vv{b}.
\end{align*} 

Vi kan nu bestemme længden af vektoren:
\begin{align*}
|\vv{a_{\vv{b}}}| &= \left|\frac{\vv{a}\cdot \vv{b}}{|\vv{b}|^2}\vv{b}\right|\\
&= \left|\frac{\vv{a}\cdot \vv{b}}{|\vv{b}|^2}\right||b| \\
&= \frac{|\vv{a} \cdot \vv{b}|}{|b|}
\end{align*}
\end{proof}

\begin{exa}
Vi skal bestemme projektionen af vektoren 
\begin{align*}
\vv{a} = \begin{pmatrix}
1 \\ 2
\end{pmatrix}
\end{align*}
på vektoren 
\begin{align*}
\vv{b} = \begin{pmatrix}
3 \\ 4
\end{pmatrix}.
\end{align*}
Vi bestemmer først
\begin{align*}
\vv{a} \cdot \vv{b} = \begin{pmatrix}
1 \\ 2
\end{pmatrix}
\cdot \begin{pmatrix}
3 \\ 4
\end{pmatrix}
= 1\cdot 2 + 3\cdot 4 = 14
\end{align*}
Vi bestemmer så
\begin{align*}
|b|^2 = (\sqrt{5^2+7^2})^2 = 25
\end{align*}
Vi kan nu bestemme projektionen $\vv{a_{\vv{b}}}$ som
\begin{align*}
\vv{a_{\vv{b}}} &= \frac{\vv{a}\cdot \vv{b}}{|\vv{b}|^2}\vv{b} \\&= \frac{14}{25}\begin{pmatrix}
3 \\ 4
\end{pmatrix} = \begin{pmatrix}
\frac{42}{25} \\ \frac{58}{25}
\end{pmatrix}
\end{align*}
\end{exa}

\begin{exa}
Vi skal projicere vektoren $\vv{u}$ givet ved
\begin{align*}
	\vv{u}=
	\begin{pmatrix}
		2 \\ 3
	\end{pmatrix}
\end{align*}
ned på linjen med parameterfremstillingen 
\begin{align*}
	\begin{pmatrix}
		x \\ y
	\end{pmatrix} =
	\begin{pmatrix}
		-1 \\ -1
	\end{pmatrix}+ t
	\begin{pmatrix}
		5 \\ 1
	\end{pmatrix}.
\end{align*}
Vi skal først projicere ned på retningsvektoren
\begin{align*}
	\vv{r} = 
	\begin{pmatrix}
		5 \\ 1
	\end{pmatrix}.
\end{align*}
Dette gøres:
\begin{align*}
	\vv{v}_{\vv{r}} = \frac{\vv{v}\cdot \vv{r}}{|\vv{r}|^2}\vv{r} = 
	\begin{pmatrix}
		\frac{5}{2} \\ \frac{1}{2}	
	\end{pmatrix}.
\end{align*}

\end{exa}

\section*{Opgave 1}
\begin{enumerate}[label=\roman*)]
	\item Projicér vektoren $\vv{v}$ givet ved
	\begin{align*}
		\vv{v} = 
		\begin{pmatrix}
			1 \\ 1
		\end{pmatrix}
	\end{align*} ned på vektoren
	\begin{align*}
		\begin{pmatrix}
			2 \\ 3
		\end{pmatrix}.
	\end{align*}
	Tegn vektorerne i GeoGebra og undersøg, om du har fundet den korrekte projektionsvektor. 
	\item Projicér vektoren 
	\begin{align*}
		\begin{pmatrix}
			3\\5
		\end{pmatrix}
	\end{align*}
	ned på vektoren 
	\begin{align*}
		\begin{pmatrix}
			-2 \\ 4
		\end{pmatrix}.
	\end{align*}
	Tegn vektorerne i GeoGebra og undersøg, om du har fundet den korrekte projektionsvektor.
\end{enumerate}


\section*{Opgave 2}

\begin{enumerate}[label=\roman*)]
	\item Projicér vektoren 
	\begin{align*}
		\begin{pmatrix}
			-5 \\ -3
		\end{pmatrix}
	\end{align*}
	ned på linjen givet ved parameterfremstillingen
	\begin{align*}
		\begin{pmatrix}
			x \\ y
		\end{pmatrix} = 
		\begin{pmatrix}
			-7 \\ 2
		\end{pmatrix} + t
		\begin{pmatrix}
			1 \\ 1
		\end{pmatrix}
	\end{align*}
	\item Projicér vektoren 
	\begin{align*}
		\begin{pmatrix}
			-1 \\ 6
		\end{pmatrix}
	\end{align*}
	ned på linjen givet ved parameterfremstillingen
	\begin{align*}
		\begin{pmatrix}
			x \\ y
		\end{pmatrix} = 
		\begin{pmatrix}
			2 \\ -9
		\end{pmatrix} + t
		\begin{pmatrix}
			6 \\ -1
		\end{pmatrix}
	\end{align*}
	\item Undersøg dit resultat i GeoGebra.
\end{enumerate}

\section*{Opgave 3}

\begin{enumerate}[label=\roman*)]
	\item Projicér vektoren 
	\begin{align*}
		\begin{pmatrix}
			1 \\ -2
		\end{pmatrix}
	\end{align*}
	ned på linjen givet ved ligningen
	\begin{align*}
		-1(x-12) + 9(y-1) = 0.
	\end{align*}
	\item Projicér vektoren 
	\begin{align*}
		\begin{pmatrix}
			6 \\  7
		\end{pmatrix}
	\end{align*}
	ned på linjen givet ved ligningen
	\begin{align*}
		-7(x+1) + 2(y-2) = 0.
	\end{align*}
	\item Undersøg dit resultat i GeoGebra.
\end{enumerate}

\section*{Opgave 4}
Argumentér både geometrisk og ved hjælp af projektionsformlen for, at projektionen af en vektor ned på en ortogonal vektor giver nulvektoren. 