\begin{center}
\Huge
Linjens parameterfremstilling
\end{center}

\section*{Parameterfremstilling}
\stepcounter{section}

I stedet for at repræsentere en linje ved hjælp af et punkt $P(x_0,y_0)$ og en normalvektor $\vv{n}$, så kan vi repræsentere linjen ved hjælp af et punkt $P(x_0,y_0)$ og en \textit{retningsvektor} $\vv{r}$ givet ved 
\begin{align*}
	\vv{r} = 	
	\begin{pmatrix}
 		r_1 \\ r_2
	\end{pmatrix}.
\end{align*}

så må punktet $(x,y)$, der opfylder, at
\begin{align*}
\begin{pmatrix}
x \\ y
\end{pmatrix}
= \begin{pmatrix}
x_0 \\ y_0
\end{pmatrix} + \begin{pmatrix}
r_1 \\ r_2
\end{pmatrix}
\end{align*}
ligge på $l$. Men for $t\in \mathbb{R}$ ved vi, at alle vektorer $t\vv{r}$ er parallelle med $\vv{r}.$ Derfor må der desuden gælde, at alle punkter $(x,y)$, der opfylder, at 
\begin{align*}
\begin{pmatrix}
x \\ y
\end{pmatrix}
=
\begin{pmatrix}
x_0 \\ y_0
\end{pmatrix} 
+
t
\begin{pmatrix}
r_1 \\ r_2
\end{pmatrix}
 \end{align*}
må være præcist de punkter, der ligger på linjen $l$. Dette kan ses af Fig. \ref{fig:linjeligning}
\begin{figure}[H]
	\centering
	\begin{tikzpicture}
		\begin{axis}[axis lines = middle,
		xmin = -1, 
		ymin = -1,
		ticks = none		
		]
			\addplot[color = blue!40, thick] {0.6*x+1};
			
			\draw[-{Stealth[scale=1.5]}, thick, red!60] (axis cs:1,1.6) -- (axis cs:4,3.4);
			\draw[-{Stealth[scale=1.5]}, thick] (axis cs:1,1.6) -- (axis cs:3,2.8);
			\node[color = blue!40] at (axis cs:4.7,3.5) {$l$};
			\node at (axis cs:2,2.4) {$\vv{r}$};
			
			
			\node[circle, fill = red!60, inner sep = 0pt, minimum size = 5pt] at (axis cs: 1,1.6) {};
			\node[color = red!60] at (axis cs: 0.7,1.9) {$(x_0,y_0)$};
			\node[circle, fill = red!60, inner sep = 0pt, minimum size = 5pt] at (axis cs:4,3.4) {};
			\node[color = red!60] at (axis cs: 3.8,3.7) {$(x,y)$};
			
			
		\end{axis}
	\end{tikzpicture}
	\caption{Retningsvektor til linjen $l$.}
	\label{fig:linjeligning}
\end{figure}

Vi kan nu konkludere med en sætning.
\begin{setn}[Parametrisering af linjen]
For en linje $l$ gælder der, at alle punkter $(x,y)$, der ligger på $l$ opfylder, at
\begin{align*}
\begin{pmatrix}
x \\ y
\end{pmatrix}
=
\begin{pmatrix}
x_0 \\ y_0
\end{pmatrix} 
+
t
\begin{pmatrix}
r_1 \\ r_2
\end{pmatrix},
\end{align*} 
hvor $P(x_0,y_0)$, $\vv{r} = \begin{pmatrix}
r_1 \\ r_2
\end{pmatrix}$
er en retningsvektor for linjen og $t\in \mathbb{R}$ er et vilkårligt reelt tal. Vi kalder dette for en \textit{parametrisering af l}.
\end{setn}
\begin{defn}[Tværvektor]
	Lad $\vv{v}$ være givet ved
	\begin{align*}
		\vv{v} = 
		\begin{pmatrix}
			a \\ b
		\end{pmatrix}.
	\end{align*}
	Så defineres \textit{tværvektoren} $\widehat{\vv{v}}$ til $\vv{v}$ som 
	\begin{align*}
		\widehat{\vv{v}} = 
		\begin{pmatrix}
			-b \\ a
		\end{pmatrix}.
	\end{align*}
\end{defn}
Det er ikke svært at overbevise sig selv om at en vektor $\vv{v}$ og dens tværvektor $\widehat{\vv{v}}$ er orthogonale. Tjekker vi efter, får vi, at 
\begin{align*}
	\vv{v} \cdot \widehat{\vv{v}} = 
	\begin{pmatrix}
		a \\ b
	\end{pmatrix} \cdot
	\begin{pmatrix}
		-b \\ a
	\end{pmatrix} = 
	a\cdot b - b\cdot a = 0. 
\end{align*}
\begin{exa}
Har vi et punkt $P = (-1,3)$ og en retningsvektor $\vv{r} = \begin{pmatrix}
5 \\ 1
\end{pmatrix}$ kan vi bestemme parametriseringen for linjen $l$, der går gennem $P$ og har retningsvektor $\vv{r}$:
\begin{align*}
\begin{pmatrix}
x \\ y
\end{pmatrix}
= 
\begin{pmatrix}
-1 \\ 3
\end{pmatrix} +  
t
\begin{pmatrix}
5 \\ 1
\end{pmatrix}.
\end{align*}
Da $\begin{pmatrix}
-1 \\ 5
\end{pmatrix}$
er orthogonal til $\vv{r}$ vil denne vektor være en normalvektor til $l$. Vi kan derfor også repræsentere $l$ ved ligningen
\begin{align*}
-1(x+1) + 5(y-3) = 0.
\end{align*}
\end{exa}

\begin{exa}
	En linje $l$ har parametriseringen 
	\begin{align*}
		l: \  \begin{pmatrix} x \\ y \end{pmatrix} = 
		\begin{pmatrix} 1 \\ -2 \end{pmatrix} + t
		\begin{pmatrix}
			4\\2
		\end{pmatrix}.
	\end{align*}
	Vi skal afgøre, om punkterne $(1,1)$ og $(5,0)$ ligger på $l$. Vi indsætter første punkt i 
	parametriseringen:
	\begin{align*}
		\begin{pmatrix}
			1 \\ 1
		\end{pmatrix} =
		\begin{pmatrix}
			1 \\ -2
		\end{pmatrix} + t
		\begin{pmatrix}
			4 \\ 2
		\end{pmatrix}.
	\end{align*}
	I første linje har vi ligningen
	\begin{align*}
		1 = 1 + t\cdot 4,
	\end{align*}
	så $t = 0$. Det indsættes i den nederste ligning:
	\begin{align*}
		1 = -2 + 0\cdot  2 = -2,
	\end{align*}
	hvilket ikke er korrekt. Derfor ligger punktet $(1,1)$ ikke på linjen $l$. 
	Tilsvarende indsættes punktet $(5,0)$, og vi får
	\begin{align*}
		\begin{pmatrix}
			5 \\ 0
		\end{pmatrix} = 
		\begin{pmatrix}
			1 \\ -2
		\end{pmatrix} + t
		\begin{pmatrix}
			4 \\ 2
		\end{pmatrix}.
	\end{align*}
	Første ligning lyder så
	\begin{align*}
		5 = 1 + t\cdot 4,
	\end{align*}
	så $t = 1$. Dette indsættes i anden ligning:
	\begin{align*}
		0 = -2 + 1\cdot 2 = 0,
	\end{align*}
	hvilket er sandt, så punktet $(5,0)$ ligger på linjen $l$. 
\end{exa}

\section*{Opgave 1}
Bestem en parametrisering for følgende linjer, der har retningsvektorer $$\vv{r} = \begin{pmatrix}
r_1 \\ r_2
\end{pmatrix}$$ og hvor punkterne $P(x_0,y_0)$ ligger på linjen.
\begin{align*}
&1) \ P(1,1), \ \vv{r} = \begin{pmatrix}
-2 \\ 3
\end{pmatrix}.   &&2) \ P(-5,-3), \ \vv{r} = \begin{pmatrix}
2 \\ 7
\end{pmatrix}.    \\
&3) \ P\left(\frac{-2}{5},13\right), \ \vv{r} = \begin{pmatrix}
-10 \\ 20
\end{pmatrix}.   &&4) \ P(\sqrt{2},\sqrt{3}), \ \vv{r} = \begin{pmatrix}
\frac{1}{5} \\ \frac{7}{10}
\end{pmatrix}.   \\
&5) \ P(0,0), \ \vv{r} = \begin{pmatrix}
1 \\ 1
\end{pmatrix}.   &&6) \ P(-100,5), \ \vv{r} = \begin{pmatrix}
-\frac{\sqrt{2}}{2} \\ -9
\end{pmatrix}.   \\
\end{align*}

\section*{Opgave 2}
\begin{enumerate}[label=\roman*)]
	\item En linje $l$ har parametriseringen
	\begin{align*}
		l: \ 
		\begin{pmatrix}
			x \\ y
		\end{pmatrix}=
		\begin{pmatrix}
			3 \\5
		\end{pmatrix} + t
		\begin{pmatrix}
			-1 \\ 1
		\end{pmatrix}.
	\end{align*}
	Afgør, om punkterne $(3,-2)$ og $(2,2)$ ligger på $l$. 
	\item En linje $l$ har parametriseringen
	\begin{align*}
		l: \ 
		\begin{pmatrix}
			x \\ y
		\end{pmatrix}=
		\begin{pmatrix}
			-2 \\2
		\end{pmatrix} + t
		\begin{pmatrix}
			6 \\ -7  
		\end{pmatrix}.
	\end{align*}
	Afgør, om punkterne $(0,0)$ og $(16,23)$ ligger på $l$. 
\end{enumerate}

\section*{Opgave 3}

\begin{enumerate}[label=\roman*)]
\item Bestem en parametrisering for linjen $l$ givet ved ligningen 
\begin{align*}
2(x-2) + 3(y+1) = 0.
\end{align*}
\item Bestem en parametrisering for linjen $l$ givet ved ligningen 
\begin{align*}
-10(x-10) + 10(y+10) = 0.
\end{align*}
\item Bestem en ligning for linjen $l$ givet ved parametriseringen 
\begin{align*}
\begin{pmatrix}
x \\ y
\end{pmatrix}
= 
\begin{pmatrix}
2 \\ 3
\end{pmatrix}
+
t
\begin{pmatrix}
-4 \\ 6
\end{pmatrix}.
\end{align*}
\item Bestem en ligning for linjen $l$ givet ved parametriseringen 
\begin{align*}
\begin{pmatrix}
x \\ y
\end{pmatrix}
= 
\begin{pmatrix}
\sqrt{5} \\ \frac{4}{3}
\end{pmatrix}
+
t
\begin{pmatrix}
\frac{8}{3} \\ \sqrt{10}
\end{pmatrix}
\end{align*}
\end{enumerate}

\section*{Opgave 4}
I følgende koordinatsystemer er tegnet en linje $l$ samt en retningsvektor $\vv{r}$ til linjen. Brug koordinatsystemerne til at bestemme linjens ligning for hver af linjerne.
\begin{center}
\resizebox{0.45\textwidth}{0.45\textwidth}{
\begin{tikzpicture}
	\begin{axis}[axis lines = middle, grid, xmin = -0.5, xmax = 5.5, ymin = -0.5, ymax = 5.5]
		\addplot[color = blue!60] {-0.5*x+3};
		\draw[-{Stealth[scale = 1.5]},thick, color = red!60] (axis cs:2,2) -- (axis cs: 4, 1);
	\end{axis}
\end{tikzpicture}
}
\resizebox{0.45\textwidth}{0.45\textwidth}{
\begin{tikzpicture}
	\begin{axis}[axis lines = middle, grid, xmin = -3.5, xmax = 3.5, ymin = -1.5,ymax = 5.5]
		\addplot[color = blue!60] {2*x+6};
		\draw[-{Stealth[scale = 1.5]},thick, color = red!60] (axis cs:-1,4) -- 
		(axis cs: -3, 0);
	\end{axis}
\end{tikzpicture}
}
\end{center}
\begin{center}
\resizebox{0.45\textwidth}{0.45\textwidth}{
\begin{tikzpicture}
	\begin{axis}[axis lines = middle, grid, xmin = -1.5, xmax = 3.5, ymin = -1.5,ymax = 3.5]
		\addplot[color = blue!60] {-x+1};
		\draw[-{Stealth[scale = 1.5]}, thick, color = red!60] (axis cs:1,0) --
		 (axis cs: 2, -1);
	\end{axis}
\end{tikzpicture}
}
\resizebox{0.45\textwidth}{0.45\textwidth}{
\begin{tikzpicture}
	\begin{axis}[axis lines = middle, grid,xmin = -1.5, xmax = 3.5, ymin = -1.5,ymax = 3.5]
		\addplot[color = blue!60] {2};
		\draw[-{Stealth[scale = 1.5]},thick, color = red!60] (axis cs:2,2) -- 
		(axis cs: -1, 2);
	\end{axis}
\end{tikzpicture}
}
\end{center}