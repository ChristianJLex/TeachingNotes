\begin{center}
\Huge
Opgaver om linjer
\end{center}

\section*{Tværvektorer}
\stepcounter{section}

Vi starter ud med en definition
\begin{defn}[Tværvektor]
	Lad $\vv{v}$ være givet ved
	\begin{align*}
		\vv{v} = 
		\begin{pmatrix}
			a\\b
		\end{pmatrix}.	
	\end{align*}
	Så definerer vi \textit{tværvektoren} til $\vv{v}$ som
	\begin{align*}
		\widehat{\vv{v}} = 
		\begin{pmatrix}
			-b \\ a
		\end{pmatrix}.
	\end{align*}
	$\widehat{\vv{v}}$ skal læses som hat-v.
\end{defn}

Det er ikke svært at overbevise sig om, at tværvektoren $ \widehat{\vv{v}}$ og $\vv{v}$ er orthogonale, siden
\begin{align*}
	\widehat{\vv{v}}\cdot \vv{v} = 
	\begin{pmatrix}
		a \\ b
	\end{pmatrix} \cdot 
	\begin{pmatrix}
		-b \\ a
	\end{pmatrix} =
	a(-b) + b\cdot a	 = 0	.
\end{align*}

\section*{Opgave 1}
\begin{enumerate}[label=\roman*)]
	\item En linje $l$ er givet ved ligningen 
	\begin{align*}
		l: \ -7(x-4) + 2(y+5) =0.
	\end{align*}
	Bestem en parameterfremstilling for $l$. 
	\item En linje $l$ er givet ved parameterfremstillingen'
	\begin{align*}
		l: \ 
		\begin{pmatrix}
			x \\ y
		\end{pmatrix}=
		\begin{pmatrix}
			-1 \\ 10
		\end{pmatrix}+ t
		\begin{pmatrix}
			5  \\ 6
		\end{pmatrix}.
	\end{align*}
	Bestem en ligning for $l$ på formen 
	\begin{align*}
		a(x-x_0) + b(y-y_0) = 0.
	\end{align*}
\end{enumerate}

\section*{Opgave 2}
\begin{enumerate}[label=\roman*)]
	\item En linje $l$ er givet ved ligningen
	\begin{align*}
		l: \ 2(x-2) + 3(y-4) = 0.
	\end{align*}
	Afgør om følgende to punkter ligger på $l$:
	\begin{align*}
	&1) \ (5,2)  &&2) \ (7,1).
	\end{align*}
\end{enumerate}

\section*{Opgave 3}

\begin{enumerate}[label=\roman*)]
	\item For punktet $P(2,4)$ og vektoren $\vv{v}$ givet ved
	\begin{align*}
		\vv{v} = 
		\begin{pmatrix}
			-1\\1
		\end{pmatrix}
	\end{align*}
	bestem så både en ligning for linjen $l$ der har $\vv{v}$ som normalvektor og skærer
	gennem $P$ samt en parameterfremstilling for linjen $m$, der har $\vv{v}$ som 
	retningsvektor og skærer gennem $P$.
	\item Bestem en parameterfremstilling for linjen, der skærer gennem punkterne $(1,1)$ og
	$ (2,-4)$.
	\item En linje $l$ skærer gennem punktet $(2,-1)$ og har $\vv{n}$ givet ved
	\begin{align*}
		\vv{n} = 
		\begin{pmatrix}
			-6\\7
		\end{pmatrix}	
	\end{align*}
	som normalvektor. Bestem en ligning for linjen $m$, der går gennem $(2,-1)$ og står 
	vinkelret på $l$. 
\end{enumerate}

\section*{Opgave 4}
\begin{enumerate}[label=\roman*)]
	\item En linje $l$ er givet ved ligningen 
	\begin{align*}
		l: \ 1(x-2) -2(y+-2) = 0
	\end{align*}
	og en linje $m$ er givet ved parameterfremstillingen
	\begin{align*}
		m: \
		\begin{pmatrix}
			x \\ y
		\end{pmatrix} =
		\begin{pmatrix}
			8\\ 4
		\end{pmatrix}
		+ t
		\begin{pmatrix}
			2\\6
		\end{pmatrix}.	
	\end{align*}
	Bestem skæringspunktet mellem $l$ og $m$. 
	\item En linje $l$ er givet ved ligningen 
	\begin{align*}
		l: \ y = 4x+1
	\end{align*}
	og en linje $m$ er givet ved parameterfremstillingen
	\begin{align*}
		m: \
		\begin{pmatrix}
			x \\ y
		\end{pmatrix} =
		\begin{pmatrix}
			5\\ 1
		\end{pmatrix}
		+ t
		\begin{pmatrix}
			4\\-4
		\end{pmatrix}.	
	\end{align*}
	Bestem skæringspunktet mellem $l$ og $m$. 
	\item Undersøg, om I har fundet det rigtige skæringspunkt i Maple
\end{enumerate}

\section*{Opgave 5}
\begin{enumerate}[label=\roman*)]
	\item En linje $l$ har ligningen 
	\begin{align*}
		l: \ 6(x+2) -4(y-1) = 0,
	\end{align*}
	og en linje $m$ har ligningen 
	\begin{align*}
		m: \ 10(x-11) +11(y-12) =0.
	\end{align*}
	Bestem vinklen $v$ mellem $l$ og $m$ og bekræft dit resultat ved at tegne linjerne i GeoGebra.
	\item En linje $l$ har parameterfremstillingen 
	\begin{align*}
		l: \ 
		\begin{pmatrix}
			x \\ y
		\end{pmatrix} = 
		\begin{pmatrix}
			2 \\ 2
		\end{pmatrix} + t
		\begin{pmatrix}
			5 \\ -11
		\end{pmatrix},
	\end{align*}
	og en linje $m$ har parameterfremstillingen 
	\begin{align*}
		m: \ 
		\begin{pmatrix}
			x \\ y
		\end{pmatrix} = 
		\begin{pmatrix}
			4 \\ -3
		\end{pmatrix} + t
		\begin{pmatrix}
			-6 \\-13
		\end{pmatrix}.
	\end{align*}
	Bestem vinklen $v$ mellem $l$ og $m$ og bekræft dit resultat ved at tegne linjerne i GeoGebra.
	\item En linje $l$ er givet ved ligningen 
	\begin{align*}
		l: \ 12(x-1) -7(y-\frac{1}{2}) = 0,
	\end{align*}
	og en linje $m$ er givet ved parameterfremstillingen
	\begin{align*}
		m: \ 
		\begin{pmatrix}
			x \\ y
		\end{pmatrix} = 
		\begin{pmatrix}
			-\frac{3}{2} \\ \frac{2}{3}
		\end{pmatrix} + t
		\begin{pmatrix}
			1 \\ 1
		\end{pmatrix}.
	\end{align*}
	Bestem vinklen mellem $l$ og $m$. Bekræft dit resultat ved at tegne linjerne i GeoGebra.
\end{enumerate}
