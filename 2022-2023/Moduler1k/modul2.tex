
\begin{center}
\Huge
Brøker
\end{center}

\section*{Brøkregneregler}
\stepcounter{section}

Vi skal i dag arbejde med brøkregneregler.
\begin{enumerate}[label=\roman*)]
\item Addition og subtraktion af brøker. For to brøker $\frac{a}{b}$ og $\frac{c}{d}$ er summen og differensen givet ved
\begin{align*}
\frac{a}{b} \pm \frac{c}{d} = \frac{ad\pm cb}{bd}.
\end{align*}
\item Multiplikation af brøker. Produktet mellem $\frac{a}{b}$ og $\frac{c}{d}$ er givet ved
\begin{align*}
\frac{a}{b}\cdot \frac{c}{d} = \frac{ac}{bd}.
\end{align*}
Specielt er produktet mellem en brøk $\frac{a}{b}$ og et tal $c$ givet ved
\begin{align*}
c\cdot \frac{a}{b} = \frac{ca}{b}.
\end{align*}
\item Division af brøker. Forholdet mellem to brøker $\frac{a}{b}$ og $\frac{c}{d}$ er givet ved
\begin{align*}
\frac{\frac{a}{b}}{\frac{c}{d}} = \frac{ad}{bc}.
\end{align*}
(Vi ganger med den omvendte brøk.) Specielt er en brøk $\frac{a}{b}$ divideret med et tal $c$ givet ved
\begin{align*}
\frac{\frac{a}{b}}{c} = \frac{a}{bc},
\end{align*}
og et tal $c$ divideret med en brøk $\frac{a}{b}$ givet ved
\begin{align*}
\frac{c}{\frac{a}{b}} = \frac{ac}{b}
\end{align*}
\item Brøker og potenser/rødder. En brøk $\frac{a}{b}$ opløftet i et tal $c$ er givet ved 
\begin{align*}
\left(\frac{a}{b}\right)^{c} = \frac{a^c}{b^c}.
\end{align*}
$n$'te roden af en brøk $\frac{a}{b}$ er givet ved 
\begin{align*}
\sqrt[n]{\frac{a}{b}} = \frac{\sqrt[n]{a}}{\sqrt[n]{b}}.
\end{align*}
\end{enumerate}
\begin{exa}
Lad os se på et eksempel, hvor vi anvender nogle af disse regler:
\begin{align*}
\frac{\frac{2}{5}+\frac{5}{7}}{\frac{10}{3}} \overset{\textnormal{i)}}{=} \frac{\frac{14+25}{35}}{\frac{10}{3}} = \frac{\frac{39}{35}}{\frac{10}{3}} \overset{\textnormal{iii)}}{=} \frac{39\cdot 3}{35 \cdot 10} = \frac{117}{350}.
\end{align*}
\end{exa}

\section*{Opgave 1}
Udregn følgende brøker (forkort så meget som muligt). Brug Maple til at tjekke jeres svar
\begin{align*}
	&1)\  \frac{6}{7} + 3              &&2)\ \frac{7}{22} + \frac{9}{10}\\
 	&3)\ \frac{4}{5} + \frac{3}{2}      &&4)\ \frac{1}{2} - \frac{3}{4}\\
 	&5)\  2\frac{2}{3} + \frac{7+4}{2}              &&6)\ \frac{4}{\frac{5}{7}}\\
 	&7)\  \frac{\frac{2}{3}}{6} - \frac{2}{18}           &&8)\ \frac{-7+\frac{2}{6}}{8} + \frac{9}{5}\\
	&9)\ \frac{\frac{4}{3}}{\frac{2}{3}} &&10)\ \frac{\frac{10}{3}-\frac{2}{4}}{\frac{4}{3}+\frac{5}{3}} \\
	&11)\  \frac{\frac{1}{2}-\frac{7}{10}}{\frac{2}{5}+\frac{11}{3}} -  \frac{\frac{22}{3}+ \frac{-23}{6}}{\frac{1}{2}}        &&12)\ \sqrt{\frac{16}{25}} + \left(\frac{3}{2+4}\right)^2\\
	&13)\ \sqrt{\frac{\frac{100}{36}}{\frac{25}{49}}} && 14) \   \left(\frac{\sqrt{\frac{5}{7}}}{\sqrt{\frac{2}{9}}}\right)^2 + \sqrt[3]{\frac{\left(\frac{2+5}{3+11}\right)^3}{\left( \frac{5}{7-6}\right)^3}}
\end{align*}

\section*{Opgave 2}
Udregn følgende brøker og forkort så meget som muligt. 
\begin{align*}
	&1) \ \frac{ab}{a}   &2) \  \frac{a+b}{b} + \frac{a-c}{a}    \\
	&3) \ \frac{(a+b)b-ab}{b}   &4) \  \left(\frac{a}{b}\right)^2 - \frac{a^3}{b^3}   \\
\end{align*}

\section*{Opgave 3}
Løs følgende ligninger.
\begin{align*}
	&1) \ \frac{x}{9} = \frac{4}{x}   &&2) \ \frac{x}{4+\frac{2}{5}} = 2     \\
	&3) \ \sqrt{\frac{x}{4}} = \frac{\frac{2}{5}+\frac{16}{10}}{\frac{1}{2}}    &&4) \ \frac{\frac{1}{4}}{\frac{1}{8}} + \frac{x+\frac{4}{2}x}{2} = \frac{7}{\frac{1}{2}}    \\
\end{align*}
