\begin{center}
\Huge
Den naturlige logaritme
\end{center}

\section*{Bevis for regneregler for titalslogaritmen}
\stepcounter{section}
Vi så sidste gang følgende regneregler for titalslogaritmen. Vi vil her give et bevis.
\begin{setn}[Logaritmeregneregler]
	For $a,b>0$ gælder der, at
	\begin{enumerate}[label=\roman*)]
		\item $\log(a\cdot b) = \log(a)+ \log(b)$,
		\item $\log\left(\frac{a}{b}\right) = \log(a)-\log(b)$,
		\item $\log(a^x) = x\log(a).$
	\end{enumerate}		
\end{setn}
\begin{proof}
	Vi vil løbende udnytte, at $\log(10^a) = a$ og $10^{\log(a)} = a$. Vi betragter udtrykkene.
	\\
	i)
	\begin{align*}
		\log(a\cdot b) &= \log(10^{\log(a)}10^{\log(b)}) \\
		&= \log(10^{\log(a)+\log(b)})\\
		&\log(a) + \log(b).
	\end{align*}
	ii)
	\begin{align*}
		\log\left(\frac{a}{b}\right) &= \log\left(\frac{10^a}{10^b}\right)\\
		&= \log(10^{\log(a)-\log(b)})\\
		&= \log(a)-\log(b).
	\end{align*}
	iii)
	\begin{align*}
		\log(a^x) &= \log\left( \left(10^{\log(a)}\right)^x\right)\\
		&= \log\left(10^{\log(a)x}\right)\\
		&= x\log(a),
    \end{align*}		
    og vi er færdige med beviset. 
\end{proof}
\section*{Den naturlige logaritme}
\stepcounter{section}
\begin{defn}
Den naturlige logaritme er den entydige funktion $\ln$, der opfylder, at
\begin{align*}
\ln(e^x) = x, 
\end{align*}
og
\begin{align*}
e^{\ln(x)} = x,
\end{align*}
hvor $e$ er Euler's tal. ($e \approx 2.7182$)
\end{defn}
Funktionen $e^x$ kaldes for den naturlige eksponentialfunktion, og vi vil senere beskrive den nærmere.

\begin{setn}[Regneregler for $\ln$]
For den naturlige logaritme $\ln$ gælder der for $a,b>0$, at
\begin{enumerate}[label=\roman*)]
\item $\ln(a\cdot b) = \ln(a) + \ln(b)$,
\item $\ln(\frac{a}{b}) = \ln(a)-\ln(b)$,
\item $\ln(a^x) = x\ln(a)$.
\end{enumerate}
\end{setn}



\section*{Opgave 1}
Løs følgende ligninger
\begin{align*}
&1) \ \ln(x)=1   &&2) \ \ln(x)=e    \\
&3) \ \ln(3x+7) = 3   &&4) \  \ln(x^2) = e^4   \\
\end{align*}
\section*{Opgave 2}
Bestem følgende:
\begin{align*}
&1) \  \ln(e)  &&2) \  \ln(e^3)    \\
&3) \  \ln(\sqrt{e})  &&4) \ \ln(\sqrt[5]{e^4})      \\
\end{align*}


\section*{Opgave 3}
\begin{enumerate}[label=\roman*)]
\item Bevis, at  $\ln(ab) = \ln(a)+\ln(b).$
\item Bevis, at  $\ln(\frac{a}{b}) = \ln(a)-\ln(b)$.
\item Bevis, at  $\ln(a^x) = x\ln(a)$.
\end{enumerate}
(Vink: Brug beviset for regnereglerne for titalslogaritmen som skabelon.)
