
\begin{center}
\Huge
Logaritmer
\end{center}
\stepcounter{section}

\section*{Titalslogaritmen}
Har vi en ligning af typen $x^2 = k$, så kan vi bestemme $x$ ved at tage kvadratroden på begge sider af lighedstegnet og bestemme (en af) løsningerne til ligningen. I forbindelse med eksponentiel vækst
vil vil gerne kunne løse ligninger af typen $10^x = k$ og $e^x=k$  (hvor $e$ betegner Eulers tal, $e \approx 2.71)$. Til dette vil vi introducere logaritmefunktionerne. 

\begin{defn}[Titalslogaritmen]
	Titalslogaritmen $\log$ er den entydige funktion, der opfylder, at 
	\begin{align*}
		\log(10^x) = x
	\end{align*}
	og 
	\begin{align*}
		10^{\log(x)} = x.
	\end{align*}
\end{defn}

\begin{exa}
	Vi har, at 
	\begin{align*}
		\log(100) = \log(10^2) = 2.
	\end{align*}
\end{exa}

For titalslogaritmen gælder der en række regneregler. 
\begin{setn}[Logaritmeregneregler]
	For $a,b>0$ gælder der, at
	\begin{enumerate}[label=\roman*)]
		\item $\log(a\cdot b) = \log(a)+ \log(b)$,
		\item $\log\left(\frac{a}{b}\right) = \log(a)-\log(b)$,
		\item $\log(a^x) = x\log(a).$
	\end{enumerate}		
\end{setn}
Vi vil bevise denne sætning næste gang. 

\begin{exa}
	Vi ønsker at løse ligningen $10^{x+5} = 1000$. Vi tager derfor logaritmen på begge sider af lighedstegnet:
	\begin{align*}
		\log\left(10^{x+5}\right) = \log(1000) \ \Leftrightarrow \ x+5 = \log(1000) = 3 \ \Leftrightarrow	\ x = -2.
	\end{align*}	 
\end{exa}
\begin{exa}
	Vi ønsker at løse ligningen 
	\begin{align*}
		\log(4x) = 4. 
	\end{align*}
	Vi opløfter derfor $10$ i begge sider af lighedstegnet.
	\begin{align*}
		10^{\log(4x)} = 10^4 \ \Leftrightarrow \ 4x = 10000 \ \Leftrightarrow	\ x = 2500.
	\end{align*}
\end{exa}

\section*{Opgave 1}
Løs følgende udtryk
\begin{align*}
	&1) \ \log(10^7)    &&2) \ \log(10000)   \\  
	&3) \ \log(10^{1.5})   &&4) \ \log(10^{\sqrt{2}})     \\  
	&5) \ \log(10000000)   &&6) \ \log(1)   \\  
	&7) \  \log(10)  &&8) \ \log(20)+ \log(5)   \\   
\end{align*}

\section*{Opgave 2}
\begin{enumerate}[label=\roman*)]
	\item Løs ligningen
	\begin{align*}
		10^x = 100.
	\end{align*}
	\item Løs ligningen 
	\begin{align*}
		10^{x^2} = 10000.
	\end{align*}
	\item Løs ligningen 
	\begin{align*}
		10^{5x+9} = 10
	\end{align*}
	\item Løs ligningen 
	\begin{align*}
		10^{\sqrt{x}} = 1000
	\end{align*}
\end{enumerate}

\section*{Opgave 1}
Løs følgende ligninger
\begin{align*}
&1) \  \log(x) = 1  &&2) \ \log(x) = 2.5    \\
&3) \ \log(2x) = 4   &&4) \ \log(3x+10)=3    \\
&5) \ \log(x^2) = 10   &&6) \  \log(5x) = 5    \\
\end{align*}
\section*{Opgave 2}
Bestem følgende 
\begin{align*}
&1) \ \log(\sqrt{10})    &&2) \  \log(\sqrt[3]{100})  \\
&3) \ \log(\sqrt[n]{1000})   &&4) \ \log(2) + \log(50)    \\
&5) \ \log(200)-\log(20)   &&6) \ \log(2\cdot 10^5)   
\end{align*}
