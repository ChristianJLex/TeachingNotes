\begin{center}
\Huge
To ligninger med to ubekendte
\end{center}
\section*{To lineære ligninger med to ubekendte}
\stepcounter{section}

Hvis vi har to linjer $l$ og $m$ i planen, så har disse netop et skæringspunkt, hvis de ikke er parallelle. Vi husker på, at rette linjer kan beskrives ved følgende ligninger:

\begin{align*}
	&l:\ y=ax + b,\\
	&m:\ y=cx + d.
\end{align*}
Skal vi finde skæringen mellem sådan to linjer, så tilsvarer det at løse to lineære ligninger med to ubekendte. Vi skal i dag lære to metoder til at løse sådanne ligningssystemer.
\begin{exa}[Lige store koefficienters metode]
	Vi skal bestemme løsningen til ligningerne
	\begin{align}\label{eq:1}
		4x+6y = 12
	\end{align}
	og 
	\begin{align}\label{eq:2}
		x+2y=6.
	\end{align}
	Vi vil bruge lige store koefficienters metode. Vi starter med at gange \eqref{eq:2} med 4 og får
	\begin{align*}
		4x+8y=24.
	\end{align*}
	Vi trækker nu \eqref{eq:2} fra \eqref{eq:1} og får
	\begin{align*}
		4x-4x+6y-8y = 12-24 \ &\Leftrightarrow\ -2y = -12\\
		&\Leftrightarrow y = 6.
	\end{align*}
	Vi indsætter nu 	$y=6$ i \eqref{eq:2}.
	\begin{align*}
		x + 2y = 6 \ &\Leftrightarrow \ x+2\cdot 6 = 6\\
		&\Leftrightarrow	\ x = 6-12 = -6.
	\end{align*}
	Løsningen til ligningssystemet er derfor $(x,y) = (-6,6)$.
\end{exa}

\begin{exa}[Substitutionsmetoden]
	Vi skal bestemme løsningen til samme ligninger som før.
		\begin{align}\label{eq:3}
		4x+6y = 12
	\end{align}
	og 
	\begin{align}\label{eq:4}
		x+2y=6.
	\end{align}
	Vi vil nu benytte os af substitutionsmetoden, som I har set i grundforløbet. 
	Vi isolerer $x$ i \eqref{eq:4}.
	\begin{align*}
		x+2y = 6 \ &\Leftrightarrow \ x = 6-2y 
	\end{align*}
	Dette indsættes nu i \eqref{eq:3}.
	\begin{align*}
		4x+6y = 12 \ &\Leftrightarrow \ 4(6-2y)+6y = 12	\\
		&\Leftrightarrow \ 24-8y+6y = 12\\
		&\Leftrightarrow	\ -2y = -12\\
		&\Leftrightarrow	\ y = 6.
	\end{align*}
	Dette indsættes i ligningen for $x$, vi brugte før.
	\begin{align*}
		x&= 6-2y\\
		&=6-2 \cdot 6\\
		&=6-12\\
		&= -6.
	\end{align*}
	Også med denne metode finder vi altså løsningen $(x,y) = (-6,6)$. 
\end{exa}

\section*{Opgave 1}
Løs følgende ligningssystemer ved brug af både substitution og lige store koefficienters metode:
\begin{enumerate}[label=\roman*)]
	\item 
	\begin{align*}
		y = 4x+2,\\
		x+y=2.		
	\end{align*}
\item \begin{align*}
x+y&=0,\\
-3x+6y&=0.
\end{align*}
\item 
\begin{align*}
2x-10y&=8,\\
-x+6y&=1.
\end{align*}
\item
\begin{align*}
1x-2y&=3,\\
4x-5y&=6.
\end{align*}
\item
\begin{align*}
\frac{1}{6}x-\frac{1}{3}y&=3,\\
3x-9y&=3.
\end{align*}
\item 
\begin{align*}
2x+4y&=8,\\
3x-9y&=27.
\end{align*}
\end{enumerate}

\section*{Opgave 2}
\begin{enumerate}[label=\roman*)]
	\item Løs følgende ikke-lineære ligningssystem ved substitutionsmetoden.
	\begin{align*}
		x^2 -x = y,\\
		x+y = 4
	\end{align*}
\end{enumerate}

\section*{Opgave 3}
Løs følgende ligningssystem ved lige store koefficienters metode.
\begin{align*}
x+y+z&=4,\\
x-y+2z&=8,\\
2x-2y+8z&=8.
\end{align*}