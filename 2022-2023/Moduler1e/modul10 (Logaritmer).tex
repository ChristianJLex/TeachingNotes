
\begin{center}
\Huge
Logaritmer
\end{center}
\stepcounter{section}

\section*{Titalslogaritmen}
Har vi en ligning af typen $x^2 = k$, så kan vi bestemme $x$ ved at tage kvadratroden på begge sider af lighedstegnet og bestemme (en af) løsningerne til ligningen. I forbindelse med eksponentiel vækst
vil vil gerne kunne løse ligninger af typen $10^x = k$ og $e^x=k$  (hvor $e$ betegner Eulers tal, $e \approx 2.71)$. Til dette vil vi introducere logaritmefunktionerne. 

\begin{defn}[Titalslogaritmen]
	Titalslogaritmen $\log$ er den entydige funktion, der opfylder, at 
	\begin{align*}
		\log(10^x) = x
	\end{align*}
	og 
	\begin{align*}
		10^{\log(x)} = x.
	\end{align*}
\end{defn}

\begin{exa}
	Vi har, at 
	\begin{align*}
		\log(100) = \log(10^2) = 2.
	\end{align*}
\end{exa}

For titalslogaritmen gælder der en række regneregler. 
\begin{setn}[Logaritmeregneregler]
	For $a,b>0$ gælder der, at
	\begin{enumerate}[label=\roman*)]
		\item $\log(a\cdot b) = \log(a)+ \log(b)$,
		\item $\log\left(\frac{a}{b}\right) = \log(a)-\log(b)$,
		\item $\log(a^x) = x\log(a).$
	\end{enumerate}		
\end{setn}
\begin{proof}
	Vi vil løbende udnytte, at $\log(10^a) = a$ og $10^{\log(a)} = a$. Vi betragter udtrykkene.
	\\
	i)
	\begin{align*}
		\log(a\cdot b) &= \log(10^{\log(a)}10^{\log(b)}) \\
		&= \log(10^{\log(a)+\log(b)})\\
		&\log(a) + \log(b).
	\end{align*}
	ii)
	\begin{align*}
		\log\left(\frac{a}{b}\right) &= \log\left(\frac{10^a}{10^b}\right)\\
		&= \log(10^{\log(a)-\log(b)})\\
		&= \log(a)-\log(b).
	\end{align*}
	iii)
	\begin{align*}
		\log(a^x) &= \log\left( \left(10^{\log(a)}\right)^x\right)\\
		&= \log\left(10^{\log(a)x}\right)\\
		&= x\log(a),
    \end{align*}		
    og vi er færdige med beviset. 
\end{proof}
Vi vil bevise denne sætning næste gang. 

\begin{exa}
	Vi ønsker at løse ligningen $10^{x+5} = 1000$. Vi tager derfor logaritmen på begge sider af lighedstegnet:
	\begin{align*}
		\log\left(10^{x+5}\right) = \log(1000) \ \Leftrightarrow \ x+5 = \log(1000) = 3 \ \Leftrightarrow	\ x = -2.
	\end{align*}	 
\end{exa}
\begin{exa}
	Vi ønsker at løse ligningen 
	\begin{align*}
		\log(4x) = 4. 
	\end{align*}
	Vi opløfter derfor $10$ i begge sider af lighedstegnet.
	\begin{align*}
		10^{\log(4x)} = 10^4 \ \Leftrightarrow \ 4x = 10000 \ \Leftrightarrow	\ x = 2500.
	\end{align*}
\end{exa}


\section*{Den naturlige logaritme}
\stepcounter{section}
\begin{defn}
Den naturlige logaritme er den entydige funktion $\ln$, der opfylder, at
\begin{align*}
\ln(e^x) = x, 
\end{align*}
og
\begin{align*}
e^{\ln(x)} = x,
\end{align*}
hvor $e$ er Euler's tal. ($e \approx 2.7182$)
\end{defn}
Funktionen $e^x$ kaldes for den naturlige eksponentialfunktion, og vi vil senere beskrive den nærmere.

\begin{setn}[Regneregler for $\ln$]
For den naturlige logaritme $\ln$ gælder der for $a,b>0$, at
\begin{enumerate}[label=\roman*)]
\item $\ln(a\cdot b) = \ln(a) + \ln(b)$,
\item $\ln(\frac{a}{b}) = \ln(a)-\ln(b)$,
\item $\ln(a^x) = x\ln(a)$.
\end{enumerate}
\end{setn}


\section*{Opgave 1}
Løs følgende udtryk
\begin{align*}
	&1) \ \log(10^7)    &&2) \ \log(10000)   \\  
	&3) \ \log(10^{1.5})   &&4) \ \log(10^{\sqrt{2}})     \\  
	&5) \ \log(10000000)   &&6) \ \log(1)   \\  
	&7) \  \log(10)  &&8) \ \log(20)+ \log(5)   \\   
\end{align*}

\section*{Opgave 2}
Bestem følgende:
\begin{align*}
&1) \  \ln(e)  &&2) \  \ln(e^3)    \\
&3) \  \ln(\sqrt{e})  &&4) \ \ln(\sqrt[5]{e^4})      \\
\end{align*}

\section*{Opgave 3}
Bestem følgende 
\begin{align*}
&1) \ \log(\sqrt{10})    &&2) \  \log(\sqrt[3]{100})  \\
&3) \ \log(\sqrt[n]{1000})   &&4) \ \log(2) + \log(50)    \\
&5) \ \log(200)-\log(20)   &&6) \ \log(2\cdot 10^5)   
\end{align*}

\section*{Opgave 4}
Løs følgende ligninger
\begin{align*}
&1) \ \ln(x)=1   &&2) \ \ln(x)=0    \\
&3) \ \ln(3x+7) = 3   &&4) \  \ln(x^2) = e^4   \\
\end{align*}

\section*{Opgave 5}
\begin{enumerate}[label=\roman*)]
	\item Løs ligningen
	\begin{align*}
		10^x = 100.
	\end{align*}
	\item Løs ligningen 
	\begin{align*}
		10^{x^2} = 10000.
	\end{align*}
	\item Løs ligningen 
	\begin{align*}
		10^{5x+9} = 10
	\end{align*}
	\item Løs ligningen 
	\begin{align*}
		10^{\sqrt{x}} = 1000
	\end{align*}
\end{enumerate}

\section*{Opgave 6}
\begin{enumerate}[label=\roman*)]
\item Bevis, at  $\ln(ab) = \ln(a)+\ln(b).$
\item Bevis, at  $\ln(\frac{a}{b}) = \ln(a)-\ln(b)$.
\item Bevis, at  $\ln(a^x) = x\ln(a)$.
\end{enumerate}
(Vink: Brug beviset for regnereglerne for titalslogaritmen som skabelon.)
