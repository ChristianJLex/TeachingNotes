
\begin{center}
\Huge
Vi regner flere andengradsligninger!
\end{center}

Vi husker på, at for en andengradsligning $ax^2+bx+c=0$ er diskriminantformlen er givet ved
\begin{align*}
x = \frac{-b\pm \sqrt{d}}{2a},
\end{align*}
hvor $d= b^2-4ac$. Hvis $d<0$, så har ligningen ingen reelle løsninger. Hvis $d>0$, så har ligningen præcist to løsninger, og hvis $d=0$, så har ligningen præcist én løsning. 
\section*{Opgave 1}
Løs følgende andengradsligninger med diskriminantformlen:
\begin{align*}
&1) \ x^2+x-5 =0   &&2) \  x^2+8x+16 = 0  \\
&3) \ x^2-1 = 0  &&4) \ x^2+1=0  \\
&5) \ 2x+12x-14 &&6) \  4x^2-16x+64   \\
&7) \  6x^2=6x+12  &&8) \  x^2=4  \\
&9) \  \frac{1}{2}x^2 -\frac{7}{2}x=-6  &&10) \ x^4 - 8x^2+16 \textnormal{ Hint: Sæt } y=x^2   \\
&11) \  x = \frac{4}{x}  && 12) \    x-2 = \frac{15}{x}   
\end{align*}

\section*{Opgave 2}
Stil endelig jeres spørgsmål til afleveringen.