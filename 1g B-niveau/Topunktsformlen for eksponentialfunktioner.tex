
\begin{center}
\Huge
Eksponentiel regression/interpolation
\end{center}
\section*{Topunktsformlen for ekspontialfunktioner}
\stepcounter{section}
Har vi to punkter $(x_1,y_1)$ og $(x_2,y_2)$, så kan vi finde den entydige rette linje med ligning $y = ax + b$, der skærer gennem disse punkter ved brug af topunktsformlen. Den fortæller os, at hældningskoefficienter $a$ er givet som
\begin{align*}
a = \frac{y_2-y_1}{x_2-x_1},
\end{align*}
og vi kan så finde skæringen med $y$-aksen $b$ ved
\begin{align*}
b = y_1 - ax_1.
\end{align*}
Vi har tidligere set, at vi kan linearisere eksponentiel vækst ved at tage den naturlige logaritme (eller en hvilken som helst logaritme). Derfor vil det også give mening, at der er en entydig eksponentiel funktion, der skærer gennem punkterne $(x_1,y_1)$ og $(x_2,y_2)$. Denne funktion kan findes ved topunktsformlen for eksponentialfunktioner.
\begin{setn}
Givet to punkter $(x_1,y_1)$ og $(x_2,y_2)$ er der en entydig eksponentialfunktion $f$ givet ved
\begin{align*}
f(x) = b\cdot a^x,
\end{align*}
hvis graf går gennem disse punkter. Fremskrivningsfaktoren $a$ er givet ved
\begin{align*}
a=\sqrt[(x_2-x_1)]{\frac{y_2}{y_1}}.
\end{align*}
Skæringen med $y$-aksen $b$ er givet ved
\begin{align*}
b = \frac{y_1}{a^{x_1}}.
\end{align*}
\end{setn}
\begin{proof}
Vi skal bestemme en eksponentiel funktion
\begin{align*}
f(x) = b\cdot a^x, 
\end{align*}
der går gennem punkterne $(x_1,y_1)$ og $(x_2,y_2)$. Vi må derfor have, at $y_1 = b\cdot a^{x_1}$ og $y_2 = b\cdot a^{x_2}$. Vi finder nu forholdet mellem $y_2$ og $y_1$ som
\begin{align*}
\frac{y_2}{y_1} &= \frac{b\cdot a^{x_2}}{b\cdot a^{x_1}}\\
				&= \frac{a^{x_2}}{a^{x_1}}\\
				&=a^{x_2-x_1}.
\end{align*}
Vi tager nu $x_2-x_1$'te roden på begge sider af lighedstegnet.
\begin{align*}
\sqrt[(x_2-x_1)]{\frac{y_2}{y_1}} = \sqrt[(x_2-x_1)]{a^{x_2-x_1}} = a,
\end{align*}
og vi har altså bestemt $a$, siden $a>0$. Da vi ved, at $y_1 = b\cdot a^{x_1}$, så får vi ved at dividere igennem med $a^{x_1}$, at 
\begin{align*}
\frac{y_1}{a^{x_1}} = \frac{b\cdot a^{x_1}}{a^{x_1}} = b, 
\end{align*}
og beviset er færdigt. 
\end{proof}
Bemærk, at vi gerne vil have, at punkterne ligger over x-aksen.
\begin{exa}
Vi skal bestemme den eksponentialfunktion, der går gennem $(1,2)$ og $(3,8)$. Vi bruger topunktsformlen for eksponentialfunktioner, og får
\begin{align*}
a = \sqrt[3-1]{\frac{8}{2}} = \sqrt[2]{4} = 2, 
\end{align*}
og 
\begin{align*}
b = \frac{2}{2^{1}} = 1.
\end{align*}
Den eksponentialfunktion, der går gennem disse punkter er derfor 
\begin{align*}
f(x) = 1\cdot 2^x = 2^x.
\end{align*}
\end{exa}

\section*{Opgave 1}
Bestem de eksponentialfunktioner, der går gennem følgende par af punkter (I må gerne bruge Maple):
\begin{align*}
&1) \ (2,8),\  (4,32)    &&2) \ (1,e), \ (3,e^3)     \\
&3) \ (2.3,4.5), \  (2.5,9.7)   &&4) \  (2,9), \ (3,27)    \\
&5) \ (\frac{1}{2},\frac{1}{3}), \ (\frac{4}{3},\frac{5}{3})   &&6) \ (4.25,4.50), \ (4.50,4.76)     \\
\end{align*}

\section*{Opgave 2}
\begin{enumerate}[label=\roman*)]
\item Lav eksponentiel regression på følgende data:
\begin{center}
\begin{tabular}{c|cccccccc}
$x$ & 0 & 1 & 2 & 3 & 4 & 5 & 6 & 7\\
\hline
$y$ & 2.0 & 1.9 & 3.2 & 3.4 & 4.2 & 4.7 & 4.6 & 5.9
\end{tabular}
\end{center}
\end{enumerate}

\section*{Opgave 3}
Vi vil give et alternativt bevis for topunktsformlen. Vi husker på, at vi kan linearisere eksponentiel data ved at tage logaritmen af det. Dette tilsvarer for to punkter punkterne $(x_1,\ln(y_1))$ og $(x_2,\ln(y_2))$.
\begin{enumerate}[label=\roman*)]
\item Brug topunktsformlen for lineære funktioner på punkterne $(x_1,\ln(y_1))$ og $(x_2,\ln(y_2))$ til at bestemme hældningskoefficienten $a$. 
\item Bestem nu $e^a$. Denne vil være fremskrivningsfaktoren for den eksponentielle vækst. Lad os kalde den $\alpha = e^a$.  
\item Brug nu, at $y_1 = \beta \cdot \alpha ^x$ for at bestemme $\beta$. 
\item Du har nu bevist topunktsformlen for eksponentiel vækst med forskrift
\begin{align*}
f(x) = \beta \cdot \alpha^x.
\end{align*}
\end{enumerate}