\begin{center}
\Huge
Enhedsvektorer og prikprodukter
\end{center}

\section*{Enhedsvektorer}
\stepcounter{section}
 
Vi kalder alle vektorer, der har længde $1$ for enhedsvektorer. Givet en vektor $\vv{v}$ kan vi bestemme en enhedsvektor, der peger i samme retning som $\vv{v}$.
\begin{setn}
Lad $\vv{v}$ være en ikke-nulvektor. Så er vektoren
\begin{align*}
\vv{e} = \frac{1}{|\vv{v}|}\vv{v}
\end{align*}
en enhedsvektor ensrettet og parallel med $\vv{v}$.
\end{setn}
\begin{proof}
Længden af $\vv{e}$ er givet ved
\begin{align*}
|\vv{e}| = \left|\frac{1}{|\vv{v}|}\vv{v}\right| = \frac{1}{|\vv{v}|}|\vv{v}| = 1.
\end{align*}
Vektorerne er klart parallelle, og de peger i samme retning, da $\frac{1}{\vv{v}}>0$. 
\end{proof}

\begin{exa}
Lad $\vv{v} = \begin{pmatrix}
1\\ 1
\end{pmatrix}$. Så kan vi bestemme en enhedsvektor parallel med $\vv{v}$ som
\begin{align*}
\vv{e} = \frac{1}{|\vv{v}|}\vv{v} = \frac{1}{\sqrt{1^2+1^2}}\begin{pmatrix}
1\\ 1
\end{pmatrix} = \begin{pmatrix}
\frac{1}{\sqrt{2}}\\ \frac{1}{\sqrt{2}}
\end{pmatrix}.
\end{align*}
\end{exa}

\section*{Prikprodukt}
\stepcounter{section}

Vi vil ikke definere nogen måde at gange vektorer sammen. Vi vil i stedet definere det såkaldte prikprodukt. Dette kan blandt andet bruges til at bestemme vinklen mellem to vektorer. 
\begin{defn}
Lad $\vv{v}$ og $\vv{w}$ være defineret som
\begin{align*}
\vv{v} = \begin{pmatrix}
v_1\\ v_2
\end{pmatrix}, \textnormal{ og }\vv{w} = \begin{pmatrix}w_1 \\ w_2
\end{pmatrix}.
\end{align*}
Så defineres \textit{prikproduktet}, \textit{skalarproduktet}, eller \textit{det indre produkt}  mellem $\vv{v}$ og $\vv{w}$ som
\begin{align*}
v\cdot w = v_1w_1 + v_2w_2.
\end{align*}
Dette skrives også til tider $\langle \vv{v}, \vv{w} \rangle$. 
\end{defn}
\begin{exa}
Lad $\vv{v} = \begin{pmatrix}
2 \\ 2
\end{pmatrix}$ og lad $\vv{w} = \begin{pmatrix}
-2 \\ 5
\end{pmatrix}$. Så kan vi bestemme prikproduktet mellem $\vv{v}$ og $\vv{w}$ som
\begin{align*}
\vv{v} \cdot \vv{w} = 2\cdot(-2)+2\cdot 5 = 6.
\end{align*}
\end{exa}

\begin{setn}\label{setn:1}
To vektorer $\vv{v}$ og $\vv{w}$ er orthogonale hvis og kun hvis $\vv{v}\cdot \vv{w} = 0$.
\end{setn}
Vi vil senere se mere præcist hvordan sammenhængen mellem vinklen mellem vektorer og prikproduktet er. 
\begin{exa}
Vi vil afgøre, om $\vv{v} = \begin{pmatrix}
2\\2
\end{pmatrix}$ og $\vv{w} = \begin{pmatrix}
-1\\1
\end{pmatrix}$ er orthogonale. Vi bestemmer derfor prikproduktet
\begin{align*}
\vv{v}\cdot \vv{w} = 2\cdot(-1)+2\cdot 1 = 0. 
\end{align*}
Derfor ved vi, at vinklen mellem de to vektorer er $0^\circ$, og at de derfor er orthogonale eller vinkelrette.
\end{exa}

\begin{setn}[Regneregler for prikproduktet]
For vektorer $\vv{u}$, $\vv{v}$ og $\vv{w}$ samt konstanter $k$ gælder der, at 
\begin{enumerate}[label=\roman*)]
\item $\vv{u}\cdot \vv{v} = \vv{v}\cdot \vv{u}$,
\item $\vv{u} \cdot (\vv{v}+\vv{w}) = \vv{u} \cdot \vv{v}+ \vv{u}\cdot \vv{w})$,
\item $(k\vv{u})\cdot \vv{v} = k(\vv{u}\cdot \vv{v}) = \vv{u}\cdot (k\vv{v})$,
\item $|\vv{u}|^2 = \vv{u}\cdot \vv{u}$,
\item $|\vv{u}\pm \vv{v}|^2 = |\vv{u}|^2 \pm 2\vv{u}\cdot\vv{v} + |\vv{v}|^2 $.
\end{enumerate}
\end{setn}
\section{Opgave 1}
\begin{enumerate}[label=\roman*)]
\item Bestem en enhedsvektor, der peger i samme retning som følgende vektorer:
\begin{align*}
&1)  \ \begin{pmatrix}3\\ 4 \end{pmatrix}   &&2) \ \begin{pmatrix}10\\ 0 \end{pmatrix}     \\
&3)  \  \begin{pmatrix}-1\\ -2 \end{pmatrix}   &&4) \  \begin{pmatrix}\sqrt{2}\\ -\sqrt{2} \end{pmatrix}    \\
\end{align*}
\end{enumerate}
\section*{Opgave 2}
\begin{enumerate}[label=\roman*)]
\item Bestem prikproduktet mellem følgende vektorer
\begin{align*}
&1) \ \begin{pmatrix}1 \\ 0\end{pmatrix} \textnormal{ og } \begin{pmatrix}0 \\ 1\end{pmatrix}    &&2) \  \begin{pmatrix}\sqrt{2} \\ \sqrt{5}\end{pmatrix} \textnormal{ og } \begin{pmatrix}-\sqrt{2} \\ -\sqrt{5}\end{pmatrix}    \\
&3) \ \begin{pmatrix}12 \\ 15\end{pmatrix} \textnormal{ og } \begin{pmatrix}-3 \\ 14 \end{pmatrix}   &&4) \ \begin{pmatrix}\frac{1}{5} \\ -\frac{2}{7}\end{pmatrix} \textnormal{ og } \begin{pmatrix} 0.1 \\ \frac{7}{3}\end{pmatrix}     \\
\end{align*}
\end{enumerate}

\section*{Opgave 3}
Bevis Sætning \ref{setn:1}.
