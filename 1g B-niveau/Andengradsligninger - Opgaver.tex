\begin{center}
\Huge
Andengradsligninger - Opgaver
\end{center}

Vi fortsætter med at finde rødder for andengradspolynomier.

\begin{exa}
Vi ønsker at bestemme rødderne for polynomiet $f$ givet ved
\begin{align*}
f(x) = 8x^2+12x-8.
\end{align*}
Vi har derfor, at koefficienterne er $a = 8$, $b = 12$ og $c= -8$. Vi bruger diskriminantformlen og får
\begin{align*}
\frac{-b\pm \sqrt{b^2-4ac}}{2a} &= \frac{-12 \pm \sqrt{144-4\cdot 8 \cdot (-8)}}{2\cdot 8}\\
&= \frac{-12 \pm \sqrt{144+256}}{16}\\
&= \frac{-12\pm 20}{16}\\
\end{align*}
Vi får derfor rødderne 
\begin{align*}
x_1 =\frac{-12+20}{16} = \frac{1}{2} 
\end{align*}
og 
\begin{align*}
	x_2 = \frac{-12-20}{16} = -2
\end{align*}
\end{exa}



\section*{Opgave 1}
Brug diskriminantformlen til at bestemme rødderne for følgende polynomier. I skal løse opgaverne i hånden. 
\begin{align*}
&1) \ x^2-1   &&2) \ 2x^2-2x-4    \\
&3) \  3x^2+x+7   &&4) \ x^2-x-2   \\
&5) \  x^2+2x+1  &&6) \  4x^2-8x+4  \\
&7) \ 2x^2-4x-6   &&8) \ 3x^2+3x-6   \\
&9) \ x^2-4   &&10) \ x^2+4x+4   \\
&11) \  x^2-4x+4  &&12) \ 6x^2+18x+12   
\end{align*}

