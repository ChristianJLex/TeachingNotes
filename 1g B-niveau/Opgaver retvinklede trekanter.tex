
\begin{center}
\Huge
Trigonometriske funktioner og retvinklede trekanter
\end{center}

\section*{Opgave 1}
Vi står foran en høj bygning og vil gerne bestemme, hvor høj den er. Vi står 300m fra bygningen og måler, at vinklen mellem jorden og sigtelinjen fra jorden til toppen af bygningen er $25^\circ$.
\begin{enumerate}[label=\roman*)]
\item Hvor langt er der i lige linje fra os til toppen af bygningen?
\item Hvor høj er bygningen?
\end{enumerate}

\section*{Opgave 2}
Vi vil godt bestemme højden på det skæve tårn i Pisa. Vi måler, at vinklen mellem jorden og tårnet er $86^\circ$. Vi går fra tårnets fod ud til der, hvor vi har tårnets spids lige under os. Fra foden af tårnet og hertil er der 3.9 meter. 
\begin{enumerate}[label=\roman*)]
\item Hvor langt er tårnet fra foden til toppen?
\item Tårnet er tilnærmelsesvist cylindrisk. Hvor højt er tårnet, der hvor det er lavest?
\item Vi måler, at tårnet er 12 meter bredt. Hvor højt er tårnet, der hvor det er højest?
\end{enumerate}

\section*{Opgave 3}
En kanon skal skyde på et mål 30 meter oppe i luften, men den skyder altid i en vinkel på 30 grader. Hvis vi ikke tager højde for, at projektilen daler, når det affyres, hvor langt skal kanonen så stå fra målet?

\section*{Opgave 4}
Bestem vinklen mellem følgende par af vektorer:
\begin{align*}
&1) \begin{pmatrix}   
1 \\ 2
\end{pmatrix}  \textnormal{ og } \begin{pmatrix}
5\\-7
\end{pmatrix} &&2)  \begin{pmatrix}   
3 \\ 4
\end{pmatrix}  \textnormal{ og } \begin{pmatrix}
-1\\-1
\end{pmatrix}  \\
&3) \begin{pmatrix}   
-10 \\ 7
\end{pmatrix}  \textnormal{ og } \begin{pmatrix}
-6\\ 4
\end{pmatrix} &&4)  \begin{pmatrix}   
-1 \\ 1
\end{pmatrix}  \textnormal{ og } \begin{pmatrix}
1\\1
\end{pmatrix} 
\end{align*}
