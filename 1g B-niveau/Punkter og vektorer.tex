\begin{center}
\Huge
Punkter og vektorer
\end{center}

\section*{Vektorer mellem punkter}
\stepcounter{section}
\begin{defn}
Har vi to punkter $P = (p_1,p_2)$ og $Q = (q_1,q_2)$, så betegner vi vektoren mellem punkterne som
\begin{align*}
\vv{PQ}.
\end{align*}
Som et vigtigt eksempel har vi vektoren fra punktet $O = (0,0)$ til et punkt $P$, der betegnes 
\begin{align*}
\vv{OP},
\end{align*}
og kaldes for \textit{stedvektoren} for $P$. 
Koordinaterne til vektoren $\vv{OP}$ er givet ved $\vv{OP} = \begin{pmatrix}
p_1\\p_2
\end{pmatrix}.$ Vi definerer derfor koordinaterne til vektoren $\vv{PQ}$ ved 
\begin{align*}
\vv{PQ} = \begin{pmatrix}
q_1 - p_1\\q_2-p_2
\end{pmatrix}.
\end{align*}
\end{defn}
\begin{exa}
Stedvektoren til punktet $P = (2,3)$ er givet ved 
\begin{align*}
\vv{OP} = \begin{pmatrix}
2\\3
\end{pmatrix}.
\end{align*}
Har vi også punktet $Q = (-3,5)$, så er vektoren $\vv{PQ}$ givet ved
\begin{align*}
\vv{PQ} = \begin{pmatrix}
-3-2\\
5-3
\end{pmatrix} = \begin{pmatrix}
-5\\2
\end{pmatrix}.
\end{align*}
\end{exa}

Da vi definerede vektoren mellem to vektorer benyttede vi følgende kendsgerning:
\begin{setn}[Indskudssætningen]
For tre punkter $A,B,C$ gælder der, at 
\begin{align*}
\vv{AC} = \vv{AB} + \vv{BC}.
\end{align*}
\end{setn}
\begin{exa}
Indskudssætningen medfører desuden, at vi kan indskyde et vilkårligt antal punkter. Har vi eksempelvis punkter $A,B,C,D,E$, så gælder der, at
\begin{align*}
\vv{AE} = \vv{AB} + \vv{BC}+ \vv{CD} + \vv{DE}. 
\end{align*}
\end{exa}

\begin{setn}[Afstand mellem to punkter]
Skal vi bestemme afstanden mellem et punkt $P = (p_1,p_2)$ og et punkt $Q = (q_1,q_2)$, så er det givet ved
\begin{align*}
\texttt{dist}(P,Q) = |\vv{PQ}| = \sqrt{(p_1-q_1)^2+(p_2-q_2)^2}
\end{align*}
\end{setn}
\begin{proof}
Vektoren $\vv{PQ}$ er givet ved
\begin{align*}
\vv{PQ}=\begin{pmatrix}
q_1-p_1\\q_2-p_2
\end{pmatrix}.
\end{align*}
Vi anvender nu blot formlen for længden af en vektor og får
\begin{align*}
|\vv{PQ}| = \sqrt{(q_1-p_1)^2+(q_2-p_2)^2}
\end{align*}
\end{proof}
\begin{exa}
Afstanden fra punktet $P = (0,1)$ og punktet $Q=(-1,2)$ er givet ved
\begin{align*}
\texttt{dist}(P,Q) = \sqrt{(-1-1)^2+(2-1)^2} = \sqrt{5}.
\end{align*}
\end{exa}
\section{Opgave 1}
\begin{enumerate}[label=\roman*)]
\item Bestem stedvektoren til følgende punkter 
\begin{align*}
&1) \ (1,2)   &&2) \  (0,0)  \\
&3) \  (\sqrt{2},\sqrt{5})  &&4) \ (-2,7)    \\
\end{align*}
\item Bestem vektorerne i begge retninger mellem følgende punkter
\begin{align*}
&1) \ (4,5) \textnormal{ og } (-5,4) &&2) \ (0,0) \textnormal{ og } (1,1)  \\
&3) \  (1,2)\textnormal{ og }(-3,-4)   &&4) \ (9,7)\textnormal{ og }(10,-4)   \\
\end{align*}
\item For punkterne $A = (-2,-4)$, $B = (-8,10)$ og $C = (-2,3)$ bestem
\begin{align*}
&1) \ \vv{AC} + \vv{BC}  &&2) \ \vv{AB} + \vv{BC}   \\
&3) \ \vv{CA} - \vv{BC}  &&4) \ \vv{BA} + \vv{AC} + \vv{AB}   \\
\end{align*}
\end{enumerate}
\section*{Opgave 2}
\begin{enumerate}[label=\roman*)]
\item Lad $P = (0,3)$ og $Q = (4,0)$. En trekant har $\vv{OP}$ og $\vv{OQ}$ som to af siderne. Bestem længden af den sidste side. 
\item For $A = (4,0)$, $B = (12,1)$ og $C = (-3,-4)$ bestem
\begin{align*}
&1) \ |\vv{OA}|   &&2) \ |\vv{BC}|     \\
&3) \  |\vv{AB} + \vv{BC}|  &&4) \  \texttt{dist}(A,B)   \\
\end{align*} 
\end{enumerate}

\section*{Opgave 3}

Vis $i)$, $ii)$, $iv)$ og $v)$ i Sætning \ref{setn:regneregler} fra sidste gang.
