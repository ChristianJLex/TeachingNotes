\begin{center}
\Huge
Vinkler og prikproduktet
\end{center}

\section*{Vinkel mellem vektorer}
\stepcounter{section}

Vi kan bruge prikproduktet til at bestemme vinklen mellem to vektorer.
\begin{setn}\label{setn:1}
Vinklen $v = \angle (\vv{u},\vv{v})$ mellem to vektorer $\vv{u}$ og $\vv{v}$ kan bestemmes ved følgende forhold:
\begin{align*}
\cos(v) = \frac{\vv{u}\cdot \vv{v}}{|\vv{u}|\cdot |\vv{v}|}
\end{align*}
\end{setn}

Vi kan også se, hvorfor to vektorer er orthogonale, hvis deres prikprodukt er $0$, siden $\cos(90) = 0$.
\begin{exa}
Lad os bestemme vinklen mellem følgende vektorer:
\begin{align*}
\vv{v} = \begin{pmatrix}
1\\ 2
\end{pmatrix}\textnormal{ og }\vv{w}=\begin{pmatrix}
3\\ 4
\end{pmatrix}.
\end{align*}
Prikproduktet mellem vektorerne er 
\begin{align*}
\begin{pmatrix}
1\\ 2
\end{pmatrix} \cdot \begin{pmatrix}
3\\ 4
\end{pmatrix} = 3+8 = 11.
\end{align*}
Længderne af vektorerne er
\begin{align*}
|\vv{v}| = \sqrt{5} \approx 2.23, \textnormal{ og } |\vv{w}| = 5.
\end{align*}
Dette samler vi og får
\begin{align*}
\cos(v) = \frac{11}{2.23\cdot 5} \approx 0.98.
\end{align*}
Vi skal så løse ligningen $\cos(v) = 0.98$. Dette kan gøres med funktionen $\arccos$, der er en "lokal" invers funktion til $\cos$. Dette giver $\arccos(0.98) = 11.4^{\circ}$, og vinklen $v$ mellem $\vv{v}$ og $\vv{w}$ er derfor $11.4^\circ.$
\end{exa}

\section*{Opgave 1}

Bestem følgende prikprodukter:
\begin{align*}
&1) \  \begin{pmatrix} 4\\ 4\end{pmatrix}\cdot \begin{pmatrix} -1\\ 1\end{pmatrix} &&2) \ \begin{pmatrix} -2\\ 3\end{pmatrix}\cdot \begin{pmatrix} \frac{1}{3}\\ \frac{3}{4}\end{pmatrix}   \\
&3) \  \begin{pmatrix} -1\\ -2\end{pmatrix}\cdot \begin{pmatrix} \sqrt{2}\\ \sqrt{3}\end{pmatrix}  &&4) \ \begin{pmatrix} 3\\ \frac{2}{3} \end{pmatrix}\cdot \begin{pmatrix} \frac{1}{3}\\ \frac{3}{2}\end{pmatrix}   \\
\end{align*}

\section*{Opgave 2}
Bestem vinklen mellem følgende vektorer:
\begin{align*}
&1) \ \begin{pmatrix} 2\\ 2\end{pmatrix} \textnormal{ og } \begin{pmatrix} -2\\ -2\end{pmatrix}  &&2) \ \begin{pmatrix} 1\\ 2\end{pmatrix} \textnormal{ og } \begin{pmatrix} 5\\ -6\end{pmatrix}    \\
&3) \  \begin{pmatrix} 10\\ -5\end{pmatrix} \textnormal{ og } \begin{pmatrix} 4\\ 1\end{pmatrix} &&4) \ \begin{pmatrix} 0.5\\ 0\end{pmatrix} \textnormal{ og } \begin{pmatrix} 20\\ 1\end{pmatrix}    \\
\end{align*}

\section*{Opgave 3}
En trekant $ABC$ har hjørner i punkterne $A = (5,-5)$, $B = (4,1)$ og $(-2,2)$. 
\begin{enumerate}[label=\roman*)]
\item Skitsér trekanten $ABC$.
\item Bestem vektorerne $\vv{AB}$ og $\vv{AC}$ samt vektorerne $\vv{BA}$ og $\vv{BC}$. 
\item Brug resultatet fra ii) til at bestemme hjørnevinklerne i $A$ og $B$.
\item Bestem vinklen i det sidste hjørne af $ABC$.
\end{enumerate}

\section*{Opgave 4}
Brug Sætning \ref{setn:1} til at afgøre, hvad fortegnet af prikproduktet har at gøre med vinklen mellem to vektorer.