
\begin{center}
\Huge
Nyt forløb: Vækst!
\end{center}

\section*{Procentregning}
\stepcounter{section}

Procentregning handler om forhold mellem tal. Hvis vi eksempelvis vil bestemme hvor stor en del af en klasse, der er højere en $180cm$, så udregner vi forholdet $\frac{\textnormal{Antal personer højere end }180cm}{\textnormal{Det samlede antal personer}}.$ Hvis vi vil have det i procent, så ganger vi med $100$ (da procent betyder pr. hunrede), og får
\begin{align*}
\frac{\textnormal{Antal personer højere end }180cm}{\textnormal{Det samlede antal personer}}\cdot 100.
\end{align*}
Lad os sige, at 4 personer i 1.v er højere end $180cm$. Inklusiv undertegnede er vi i alt $26$ personer. Det vil altså sige, at procentdelen af personer i 1v, der er højere end 180cm er 
\begin{align*}
\frac{4}{26}\cdot 100 = 15.62\%.
\end{align*}
\section*{Procentregneregler}
\stepcounter{section}
Vi har følgende procentregneregler
\begin{regel}[Procentregneregel 1]
Skal vi bestemme $p\ \%$ af en størrelse $S$, så bestemmes den ved 
\begin{align*}
\frac{p}{100}\cdot S.
\end{align*} 
\end{regel}
\begin{exa}
Vi skal bestemme $p=10\%$ af $S=200$. Vi bruger Procentregneregel 1 og bestemmer
\begin{align*}
\frac{10}{100}\cdot 200 = 20. 
\end{align*} 
Normalt vil vi omskrive brøken $10/100$ til $0.1$, så regnestykket lyder
\begin{align*}
0.1\cdot 200 = 20.
\end{align*} 
\end{exa}
\begin{regel}[Procentregneregel 2]
Vi bestemmer den procentvise andel en størrelse $S$ udgør af en størrelse $T$ ved 
\begin{align*}
\frac{S}{T}\cdot 100
\end{align*}
\end{regel}
\begin{regel}[Procentregneregel 3]
Vi øger en størrelse $S$ med $p$ procent ved at bestemme
\begin{align*}
S\cdot(1+\frac{p}{100}). 
\end{align*}
$(1+\frac{p}{100})$ er det, der kaldes fremskrivningsfaktoren. 
Tilsvarende mindsker vi en størrelse $S$ med $p$ procent ved at bestemme
\begin{align*}
S \cdot (1-\frac{p}{100}).
\end{align*}
\end{regel}
\begin{exa}
Vi skal øge 130 med $15 \%$. Vi bestemmer derfor 
\begin{align*}
130 (1+\frac{15}{100}) = 130\cdot 1,15 = 149,5.
\end{align*}
Vi skal mindske $149,5$ med $15 \%$. Vi bestemmer
\begin{align*}
149,5(1-\frac{15}{100}) = 149,5\cdot 0,85 = 127,075.
\end{align*}
\end{exa}
Det er klart nemmest altid at bruge fremskrivningsfaktoren, når vi skal øge med en bestemt procentdel. 

\begin{regel}[Procentregneregel 4]
Hvis vi skal bestemme den procentvise ændring af en størrelse $S$ til en størrelse $T$, så beregnes dette 
\begin{align*}
\frac{T-S}{S}\cdot 100.
\end{align*}
\end{regel}
\begin{exa}
Prisen på 1000 liter fyringsolie var d. 24 august 2021 $9318,5$ kr. D. 24 november koster 1000 liter fyringsolie $10328,5$ kr. Vi bestemmer den procentvise ændring
\begin{align*}
\frac{10328,5-9318,5}{9318,5}\cdot 100 = 10.83\%. 
\end{align*}
\end{exa}
\section*{Opgave 1}
\begin{enumerate}[label=\roman*)]
\item Bestem $10\%$, $50\%$ og $150\%$ af $300$.
\item Bestem $93\%$, $107\%$ og $1.2\%$ af 45.
\item Hvor mange procent er $7$, $11.5$ og $13$ af $100$?
\item Hvor stor en procentdel er $500$ af $100000$?
\item Øg 50 med $80\%$. 
\item Øg 20 med $12,5\%$.
\item Gør 166 65$\%$ mindre
\item Gør 10005 99$\%$ mindre.
\end{enumerate}
\section*{Opgave 2}
\begin{enumerate}[label=\roman*)]
\item Der er udsalg, og priserne på en bestemt trøje er gået fra 300kr til $170$kr. Hvad er det procentvise fald i prisen?
\item En mand er gået op i vægt. Han vejede tidligere for et år siden $92$kg og vejer nu $165$kg. Hvor stor er hans procentvise stigning i vægt?
\item  To huse er  hhv. 5 og 6 meter høje. Hvor stor procentvis forskel er der på deres højder? (Både det lille hus i forhold til det store, og det store i forhold til det lille.)
\end{enumerate}