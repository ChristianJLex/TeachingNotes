
\begin{center}
\Huge
Logaritmer
\end{center}
\stepcounter{section}
\section*{Titalslogaritmen}
Logaritmer er en slags omvendte funktioner til eksponentialfunktioner. Vi ser primært på logaritmer for to eksponentialfunktioner $f(x) = 10^x$ og $g(x) = e^x$. Princippet er, at vi gerne vil have en funktion, der - når vi stopper $f(x)$ ind i funktionen - så giver den $x$. Vi definerer derfor følgende.
\begin{defn}
Vi definerer titalslogaritmefunktionen $\log_{10}(x)$, som den entydige funktion, der opfylder, at 
\begin{align*}
\log_{10}(10^x) = x
\end{align*}
og 
\begin{align*}
10^{log_{10}(x)} = x.
\end{align*}
\end{defn}
Vi vil typisk udelade $10$-tallet og blot skrive $\log(x)$.

\begin{exa}
Der gælder, at $\log(100) = 2$, da
\begin{align*}
\log(100) = \log(10^2) = 2.
\end{align*}
\end{exa}
\begin{exa}
Der gælder, at $\log(10000) = 4$, da
\begin{align*}
\log(10000) = \log(10^4) = 4.
\end{align*}
\end{exa}

\section*{Den naturlige logaritme}
\stepcounter{section}
\begin{defn}
Den naturlige logaritme er den entydige funktion $\ln$, der opfylder, at
\begin{align*}
\ln(e^x) = x, 
\end{align*}
og
\begin{align*}
e^{\ln(x)} = x,
\end{align*}
hvor $e$ er Euler's tal. ($e \approx 2.7182$)
\end{defn}
Funktionen $e^x$ kaldes for den naturlige eksponentialfunktion.

\begin{setn}[Regneregler for $\log$.]
For titalslogaritmen $\log_{10} =\log$ gælder der for $a,b>0$, at 
\begin{enumerate}
\item $\log(a\cdot b) = \log(a) + \log(b)$. 
\item $\log\left(\frac{a}{b}\right) = \log(a)-\log(b)$.
\item $\log(a^x) = x\log(a)$.
\end{enumerate} 
\end{setn}
\begin{proof}
Ad $i)$:
\begin{align*}
\log(a\cdot b) &= \log(10^{\log(a)}\cdot 10^{\log(b)})\\
&= \log(10^{\log(a)+\log(b)})\\
&= \log(a)+\log(b).
\end{align*}
Ad $ii)$:
\begin{align*}
\log\left(\frac{a}{b}\right) &= \log \left(\frac{10^{\log(a)}}{10^{\log(b)}}\right)\\
&= \log(10^{\log(a)-\log(b)})\\
&= \log(a)-\log(b).
\end{align*}
Ad $iii)$:
\begin{align*}
\log(a^x) &= \log((10^{\log(a)})^x)\\
&= \log(10^{x\log(a)})\\
&= x\log(a).
\end{align*}
\end{proof}
\begin{setn}[Regneregler for $\ln$]
For den naturlige logaritme $\ln$ gælder der for $a,b>0$, at
\begin{enumerate}[label=\roman*)]
\item $\ln(a\cdot b) = \ln(a) + \ln(b)$,
\item $\ln(\frac{a}{b}) = \ln(a)-\ln(b)$,
\item $\ln(a^x) = x\ln(a)$.
\end{enumerate}
\end{setn}
\section*{Opgave 1}
Løs følgende ligninger
\begin{align*}
&1) \  \log(x) = 1  &&2) \ \log(x) = 2.5    \\
&3) \ \log(2x) = 4   &&4) \ \log(3x+10)=3    \\
&5) \ \log(x^2) = 10   &&6) \  \log(5x) = 5    \\
\end{align*}
\section*{Opgave 2}
Bestem følgende 
\begin{align*}
&1) \ \log(\sqrt{10})    &&2) \  \log(\sqrt[3]{100})  \\
&3) \ \log(\sqrt[n]{1000})   &&4) \ \log(2) + \log(50)    \\
&5) \ \log(200)-\log(20)   &&6) \ \log(2\cdot 10^5)   
\end{align*}

\section*{Opgave 3}
Løs følgende ligninger
\begin{align*}
&1) \ \ln(x)=1   &&2) \ \ln(x)=e    \\
&3) \ \ln(3x+7) = 3   &&4) \  \ln(x^2) = e^4   \\
\end{align*}
\section*{Opgave 4}
Bestem følgende:
\begin{align*}
&1) \  \ln(e)  &&2) \  \ln(e^3)    \\
&3) \  \ln(\sqrt{e})  &&4) \ \ln(\sqrt[5]{e^4})      \\
\end{align*}


\section*{Opgave 5}
\begin{enumerate}[label=\roman*)]
\item Bevis, at  $\ln(ab) = \ln(a)+\ln(b).$
\item Bevis, at  $\ln(\frac{a}{b}) = \ln(a)-\ln(b)$.
\item Bevis, at  $\ln(a^x) = x\ln(a)$.
\end{enumerate}

