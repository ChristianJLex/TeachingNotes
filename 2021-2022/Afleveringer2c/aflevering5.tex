\begin{center}
\Huge
Aflevering 5
\end{center}

\section*{Opgave 1}
\stepcounter{section}
Et taxafirma \textit{TaxA} har følgende prissætning på sine rejser: 
\begin{itemize}
\item Det koster 50kr at påbegynde en rejse.
\item De første 5 km koster 15kr per km.
\item Fra 5 til 10 km koster rejsen 10kr per km.
\item Resten af rejsen koster 8kr per km.   
\end{itemize}
\begin{enumerate}[label=\roman*)]
\item Lad $f$, $g$ og $h$ være givet ved følgende stykvist definerede funktionsforskrifter
\begin{align*}
f(x) &= \begin{cases} 15\cdot x, \ &0\leq x < 5,\\
10 \cdot x, \ &5\leq x < 10,\\
8\cdot x, \ &10 \leq x,
\end{cases}\\
g(x) &= \begin{cases}
15\cdot x + 50, \ &0\leq x < 5,\\
10 \cdot x, \ &5\leq x < 10,\\
8\cdot x, \ &10 \leq x,
\end{cases}\\
h(x) &= \begin{cases}
15\cdot x + 50, \ &0\leq x < 5,\\
10 \cdot x + 75, \ &5\leq x < 10,\\
8\cdot x +95, \ &10 \leq x.
\end{cases}
\end{align*}
Bestem hvilken af funktionerne $f$, $g$ og $h$, der bedst beskriver prisen af en rejse med taxafirmaet. Begrund dit svar.
\item Brug din valgte funktion til at bestemme prisen på en rejse på $2$km, 7km og 25km.
\item Et andet taxafirma \textit{TaxB} har prissætning bestemt af følgende stykvist defineret funktion
\begin{align*}
p(x) = 
\begin{cases}
15\cdot x \ &0\leq x < 5,\\
10 \cdot x + 25 \ &5 \leq x,
\end{cases}
\end{align*}
hvor $x$ er antallet af kørte km og $p(x)$ er prisen i kr. Hvor langt skal du køre, før prisen på TaxB er højere end prisen for TaxA?
\end{enumerate}

\section*{Opgave 2}
To funktioner $f$ og $g$ er givet ved 
\begin{align*}
f(x) = 2\cdot x^2 \textnormal{ og } g(x) = 3\cdot x - 4.
\end{align*}
\begin{enumerate}[label=\roman*)]
\item Bestem de sammensatte funktioner $f(g(x))$ og $g(f(x))$. 
\item Bestem rødderne til funktionerne $f(g(x))$ og $g(f(x))$. (Husk, at rødderne til en funktion er der, hvor funktionen skærer $x$-aksen.)
\end{enumerate}
\section*{Opgave 3}
\stepcounter{section}
Vi har fået opgivet et datasæt, der beskriver befolkningstallet i tusinder i en by som funktion af tid i årtier efter år 1800. Dette kan ses af Tabel \ref{tab:datatab}.
\begin{table}[H]
\centering
\begin{tabular}{c|c|c|c|c|c|c|c|c|c|c}
$x$ (Tid i årtier) & 0&1&2&3&4&5&6&7&8&9 \\ \hline
$y$ (Befolkning i tusinder)& 12,3 & 0,1 & 1,3 & 1,9 & 7,2 & 18,4 & 15,0 & 23,1 & 55,4 & 73,5
\end{tabular}
\caption{Datasæt}
\label{tab:datatab}
\end{table}
\begin{enumerate}[label=\roman*)]
\item Lav både lineær regression og eksponentiel regression på datasættet, og kommentér på, hvad der ser ud til at passe bedst.
\item Sammenlign dit kvalitative svar med forklaringsgraderne for regressionerne. Hvilken regression er bedst i følge forklaringsgraden?
\item Bestem et residualplot for hver af de to regressioner. Hvilken af disse passer bedst?
\item Brug både den lineære model og den eksponentielle model til at bestemme antallet af personer i år 1900.
\item Det rigtige antal personer i år 1900 er 110,7 tusinde. Hvilken model rammer dette bedst?
\end{enumerate}

\section*{Opgave 4}
Lad $P_1 = (1,3)$ og $Q_1 = (4,6)$, og lad $P_2 = (0,1)$ og $Q_2 = (1,e)$. 
\begin{enumerate}[label=\roman*)]
\item Brug topunktsformlen for potensfunktioner til at bestemme den potensfunktion $f$, der går gennem punkterne $P_1$ og $Q_1$.
\item Brug topunktsformlen for eksponentialfunktioner til at bestemme den eksponentialfunktion $g$, der går gennem punkterne $P_2$ og $Q_2$. 
\item Bestem skæringenspunkterne $A$ og $B$ mellen graferne for $f$ og $g$. 
\item Brug topunktsformlen for lineære funktioner til at bestemme den lineære funktion, der går gennem punkterne $A$ og $B$. Hvad er hældningen af denne funktion?
\end{enumerate}

\section*{Opgave 5}
På en konto indsætter vi hvert år $3000$kr, og på kontoen er der en årlig rente på $3\%$. 
\begin{enumerate}[label=\roman*)]
\item Hvor meget står der på kontoen efter 15 år?
\item Hvad har vi i alt tjent på renter på de 15 år?
\item Hvis vi i stedet havde indsat $45000$kr på en gang ind på kontoen og så ikke indsat mere efterfølgende, hvor meget ville der så stå på kontoen efter 15 år?
\end{enumerate}