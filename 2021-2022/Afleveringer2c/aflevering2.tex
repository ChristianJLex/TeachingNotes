
\begin{center}
\Huge
Aflevering 2
\end{center}
\section*{Opgave 1 \large (med hjælpemidler)}
En funktion $f$ er givet ved 
\begin{align*}
f(x) = x^3-2x+1.
\end{align*}
\begin{enumerate}[label=\roman*)]
\item Bestem hældningen af tangenten til funktionen $f$ i punktet $(1,f(1))$.
\item Bestem ligningen for tangenten til funktionen $f$ i punktet $(1,f(1))$.
\item Bestem de punkter, hvor tangenthældningen er $0$.
\item Bestem ligningen for disse tangenter.
\end{enumerate}

\section*{Opgave 2 \large (med hjælpemidler)}
En funktion $f$ er givet ved 
\begin{align*}
f(x) = x^5 -3x^2+102.
\end{align*}
\begin{enumerate}[label=\roman*)]
\item Bestem hældningen for tangenten til $f$ i punktet $(0,f(0))$. 
\item Bestem ligningen for tangenten til $f$ i punktet $(0,f(0))$. 
\item Er der andre tangenter til $f$, der har samme hældning? I så fald bestem deres ligninger. 
\end{enumerate}

\section*{Opgave 3 \large (med hjælpemidler)}
Vi har to funktioner $f(x) = x^2+1$ og $g(x) = -2x^2-7x+10$. 
\begin{enumerate}[label=\roman*)]
\item Bestem skæringspunkterne mellem $f$ og $g$. 
\item Bestem ligningen for den linje $l$, der går gennem disse skæringspunkter.
\item Bestem vinklen mellem $l$ og linjen $m$ bestemt ved
\begin{align*}
m: \ y = x.
\end{align*}
\item Linjerne $l$ og $m$ afgrænser sammen med $x$-aksen et trekantet område. Hvad er arealet af dette område?
\end{enumerate}
\section*{Opgave 4 \large (med hjælpemidler)}

En cirkel $C$ med centrum i $(2,2)$ og radius $5$ er givet.
\begin{enumerate}[label=\roman*)]
\item Hvad er ligningen for $C$?
\item I hvilke punkter skærer $C$ $x$- og $y$-aksen?
\item En anden cirkel $D$ har radius $3$ og centrum i $(3,8)$. Opskriv cirklens ligning for $D$.
\item Hæv parenteserne i ligningerne for $C$ og $D$ og træk ligningen for $C$ fra ligningen for $D$. Dette er en lineær ligning, og vi kalder denne ligning $l$.
\item Isolér $x$ i ligningen $l$ og indsæt dette i enten ligningen for $C$ eller ligningen for $D$. Dette giver os en andengradsligning i $y$. Løs denne for at bestemme $y$. Gør nu det samme for $x$. Dette giver til sammen de to skæringspunkter mellem cirklen $C$ og cirklen $D$. 
\item Hvad er ligningen for den rette linje, der går gennem disse punkter? Hvordan relaterer denne til ligningen $l$?
\end{enumerate}

\section*{Opgave 5 \large (med hjælpemidler)}
Vi antager, at antallet af mennesker i et bestemt land kan beskrives ved modellen
\begin{align*}
M(t) = 10,1e^{0.04t},
\end{align*}
hvor $M$ beskriver antallet af mennesker i mio. og $t$ er antal år efter år $2000$. 
\begin{enumerate}[label=\roman*)]
\item Hvor mange mennesker var der i år $2000$? Hvad med år $2010$?
\item Hvornår er der i følge modellen 12 mio. mennesker?
\item Bestem $M'(t)$. Hvad er væksthastigheden i år $2005$? 
\item Hvornår vil antallet af mennesker stige med $100000$ om året?
\end{enumerate}
