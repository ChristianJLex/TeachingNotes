

%Overskrift
\begin{center}
\Huge
Facitliste
\end{center}
\section*{Opgave 1}
\stepcounter{section}
\begin{setn}
Produktet mellem et lige tal og et ulige tal er et lige tal.
\end{setn}
\begin{proof}
Lad $n$ være et lige tal og lad $y$ være et ulige tal. Det betyder, at vi kan skrive $n = 2m$ og $y = 2x+1$ for andre heltal $m,x$. Hvis vi ganger $n$ og $y$ sammen, får vi
\begin{align*}
n\cdot y = (\underbrace{2m}_{=n})\cdot (\underbrace{2x+1}_{=y}) = 4mx+2m = 2\cdot(\underbrace{2mx+m}_{\textnormal{heltal}}).
\end{align*}
Da $n\cdot y$ kan skrives som $2$ gange et heltal, så er $n\cdot y$ et lige tal.
\end{proof}

\section*{Opgave 2}
\begin{setn}
$n$ er lige hvis og kun hvis $n^2$ er lige.
\end{setn}
\begin{proof}
Antag, at $n$ er lige. Vi kan så skrive $n = 2m$ for et andet heltal $m$. Derfor kan vi skrive
\begin{align*}
n^2 = (2m)^2 = 2^2m^2 = 2\cdot(2m^2).
\end{align*}
Det betyder, at $n^2$ er et heltal, da det kan skrives som $2$ gange et heltal. 
Antag, at et heltal $n^2$ er lige, og antag, at $n$ er ulige. Vi kan derfor skrive $n = 2m + 1$ for et andet heltal $m$. Dette samler vi, og får 
\begin{align*}
n^2 = (2m+1)^2 = 4m^2+4m+1,
\end{align*}
men da $4m^2$ er lige, og $4m$ er lige, så må  $n^2 = 4m^2+4m+1$ være ulige. Det er i modstrid med antagelsen om, at $n^2$ er lige, altså må $n$ også være lige.
\end{proof}

\section*{Opgave 3}
\begin{enumerate}[label=\roman*)]
\item $\{0,2,4,6,8,10\}$.
\item $\{1,3,5,7,9\}$.
\item $\{\textnormal{Rød},\textnormal{Hvid},\textnormal{Blå},\textnormal{Gul}\}$.
\item $\emptyset$.
\end{enumerate}

\section*{Opgave 4}
Vi skal i denne opgave bruge, at betingelser i mængdebyggernotation opdeles af et logisk "og" også kaldet en konjunktion $\wedge$.
\begin{enumerate}[label=\roman*)]
\item $\left\{a \in \mathbb{Z} \ \middle| \ 0 \leq a \leq 10 \wedge \textnormal{der findes } m\in \mathbb{Z} \textnormal{ så } a=2m  \right\}$.
\item $\left\{a \in \mathbb{Z} \ \middle| \ 0 \leq a \leq 10 \wedge \textnormal{der findes } m\in \mathbb{Z} \textnormal{ så } a=2m+1  \right\}$.
\item $\left\{a \in \mathbb{R} \ \middle| \ a \neq \frac{b}{c} \textnormal{ for } b,c \in \mathbb{Z}\right\}$.
\item Med mængdebygger-notation: 
\begin{align*}
\left\{x,y \in \mathbb{Q} \ \middle| y = 3x+7 \wedge y = -4x+2\right\}.
\end{align*}
Uden mængebygger-notation: $\left\{\frac{-5}{7}\right\}$.
\item Med mængdebygger-notation:
\begin{align*}
\left\{x,y \in \mathbb{Q} \ \middle| y = 5x-2 \wedge y = 5x+13\right\}.
\end{align*}
Uden mængdebygger-notation: $\emptyset$.
\end{enumerate}