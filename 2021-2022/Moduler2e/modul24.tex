\begin{center}
\Huge
Anvendelser af integraler
\end{center}

\section*{Integration ved substitution}
\stepcounter{section}

\begin{setn}
For en differentiabel funktion $g$ og en kontinuert funktion $f$ gælder der, at 
\begin{align*}
\int f(g(x))\cdot g'(x) \intd x = F(g(x)) + k
\end{align*}
\end{setn}
\begin{proof}
Det er blot at differentiere højresiden
\begin{align*}
\left(F(g(x)) + k \right)' = f(g(x))\cdot g'(x),
\end{align*}
hvoraf resultatet følger direkte.
\end{proof}

\section*{Kurvelængder}
\stepcounter{section}
\begin{setn}
Kurvelængde $s$ af grafen for en differentiabel funktion $f$ på intervallet $[a,b]$ er givet ved
\begin{align*}
s = \int_a^b\sqrt{1+(f'(x))^2} \intd x
\end{align*}
\end{setn}

\begin{exa}
Lad $f$ være givet ved
\begin{align*}
f(x) = x^2.
\end{align*}
Så kan vi finde længden af kurven på intervallet $[0,2]$ ved følgende:
\begin{align*}
s = \int_0^2\sqrt{1+4x^2}\intd x.
\end{align*}
Med et CAS-værktøj bestemmer vi derfor $s$ til at være
\begin{align*}
s \approx 4.64
\end{align*}
\end{exa}

\section*{Gennemsnit af funktioner}
\stepcounter{section}
Skal vi bestemme et gennemsnit af højden af alle i en klasse, vil vi måle højden på alle i klassen, summere det og dele med antallet af studerende i klassen. Vi kan generalisere dette til kontinuerte funktioner ved hjælp af integraler.

\begin{defn}
Gennemsnittet af en kontinuert funktion $f$ på intervallet $[a,b]$ defineres som integralet
\begin{align*}
I = \frac{1}{b-a}\int_a^bf(x)\intd x.
\end{align*}
\end{defn}

\begin{exa}
Koncentrationen af en bestemt type partikel i en by kan beskrives ved eksponentialfunktionen
\begin{align*}
f(x) = 22.145\cdot (1.32)^x,
\end{align*}
hvor $f$ er i ppm og $x$ er i år efter år 2000. Ønsker vi at bestemme den gennemsnitlige koncentration fra år 2002 til år 2010, så kan vi bestemme det som
\begin{align*}
I = \frac{1}{10-2}\int_2^{10}22.145\cdot (1.32)^x \intd x, 
\end{align*}
som med et CAS-værktøj bestemmes til at være
$$I = 142.751 ppm.$$
\end{exa}

\section*{Opgave 1}
\begin{enumerate}[label=\roman*)]
\item Bestem kurvelængden af funktionen $f(x) = 2x+3$ på intervallet $[-5,5]$.
\item Bestem kurvelængden af funktionen $f(x) = \sqrt{x}$ på intervallet $[0,10]$.
\item Bestem kurvelængden af funktionen $f(x) = 3x^3+2x^2+x+1$ på intervallet $[-1,2]$.
\end{enumerate}

\section*{Opgave 2}
Et reb hænger mellem to punkter $A = (-10,20)$ og $B = (10,20)$. Den bue, rebet danner er tilnærmelsesvist en parabel for funktionen $f$ med forskriften
\begin{align*}
f(x) = \frac{3}{20}x^2+5.
\end{align*}
Bestem længden på rebet. 


\section*{Opgave 3}
\begin{enumerate}[label=\roman*)]
\item Bestem gennemsnittet af funktionen $x^2$ på intervallet $[0,10]$.
\item Bestem gennemsnittet af funktionen $\cos(x)$ på intervallet $[-\pi,\pi]$. 
\item Bestem gennemsnittet af funktionen $3x^2+10x+1$ på intervallet $[-1,3]$. 
\end{enumerate}

\section*{Opgave 4}
Vi skal bevise rumfangsformlen for en kugle med radius $r$. Formlen lyder som bekendt $V = \frac{4\pi r^3}{3}$.
\begin{enumerate}[label=\roman*)]
\item ligningen for en cirkel med radius $r$ og centrum i $(0,0)$ er $x^2+y^2 = r^2$. Isolér $y$ i dette udtryk. 
\item Brug dette udtryk for $y$ til at bestemme rumfanget af omdregningslegemet dette udtryk danner på intervallet $[-r,r]$. Dette burde være rumfanget af kuglen. (Hint: Omdregningslegemets rumfang er givet ved $\pi \int_a^b(f(x))^2\intd x)$.
\end{enumerate}
