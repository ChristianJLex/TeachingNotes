
\begin{center}
\Huge
Integration ved substitution + hængeparti. 
\end{center}

\section*{Grafer under x-aksen}
\stepcounter{section}

Ønsker vi at bestemme arealet afgrænset af en negativ funktion $f$ og $x$-aksen på et interval $[a,b]$ kan vi bruge sætningen om arealer mellem funktioner. Dette giver os følgende sætning.
\begin{setn}
Lad $f$ være en kontinuert funktion, der opfylder, at $f(x)<0$ for alle $x\in [a,b]$. Så er arealet af området afgrænset af $f$ og $x$-aksen på intervallet $[a,b]$ givet ved 
\begin{align*}
A = -\int_a^bf(x)\intd x.
\end{align*}
\end{setn}
\begin{proof}
Vi betragter arealet mellem funktionen $g(x) = 0$ og $f(x)$. Dette er givet ved
\begin{align*}
A &= \int_a^b g(x)-f(x)\intd x\\
&= \int_a^b0-f(x)\intd x \\
&= -\int_a^bf(x)\intd x,
\end{align*}
hvilket konkluderer beviset.
\end{proof}

\begin{exa}
Vi skal bestemme arealet mellem $x$-aksen og funktionen $\sin(x)$ på intervallet $[0,4\pi]$. Vi starter med at plotte funktionen. Dette kan ses af Fig. \ref{fig:sin}
\begin{figure}[H]
\centering
\begin{tikzpicture}
\begin{axis}[axis lines = middle, xmin = -0.5,xmax =8.5,ymin = -1.3,ymax = 1.3]
\addplot[samples = 1000, color = blue, domain = -0.5:8.5] {sin(deg(x))};
\end{axis}
\end{tikzpicture}
\caption{Grafen for $\sin(x)$}
\label{fig:sin}
\end{figure}
Af figuren kan vi se, at $\sin(x)$ er positiv på intervallet $[0,\pi]$ og negativ på intervallet $[\pi,2\pi]$. Derfor skal vi dele integralet op i to for at bestemme integralet. Dette gøres:

\begin{align*}
A &= \int_0^\pi \sin(x) \intd x - \int_\pi^{2\pi}\sin(x)\intd x \\
&= \left[-\cos(x)\right]_0^\pi - \left[-\cos(x)\right]_\pi^{2\pi}\\
&= -\cos(\pi)-(-\cos(0)) - (-\cos(2\pi) -(-\cos(\pi)))\\
&= 1 + 1 + 1 + 1\\
&= 4
\end{align*}
\end{exa}

\section*{Integration ved substitution}
\stepcounter{section}

Skal vi integrere en sammensat funktion, skal vi bruge en slags omvendt kæderegel. Dette kaldes ofte integration ved substitution. 
Vi starter med at bevise, at metoden virker.
\begin{setn}
Lad $f(g(x))$ være en kontinuert funktion i $x$, og antag, at $g(x)$ er differentiabel. Så gælder der, at 
\begin{align*}
\int f(g(x))\cdot g'(x) \intd x = F(g(x))+k.
\end{align*}
\end{setn}
\begin{proof}
Det er sådan set bare at gøre prøve. Vi tester, om $F(g(x))$ er en stamfunktion ved at differentiere:
\begin{align*}
\left(F(g(x))\right)' = F'(g(x))\cdot g'(x) = f(g(x))\cdot g'(x),
\end{align*}
hvilket var hvad vi skulle vise. 
\end{proof}

\begin{exa}
Vi skal løse
\begin{align*}
\int \frac{2x}{x^2+1}\intd x.
\end{align*}
Vi har en indre funktion $x^2+1$, som vi betegner med $u = x^2+1$. Vi skal i følge sætningen bruge den differentierede til den indre funktion. Derfor bestemmes 
\begin{align*}
\frac{\intd u}{\intd x} = 2x.
\end{align*}
Vi betragter $\frac{\intd u}{\intd x}$ som en brøk (Det er det ikke!), og isolerer $dx$
\begin{align*}
\intd x = \frac{\intd u}{2x}.
\end{align*}
Vi substituerer nu $\intd x$ og $u$ ind i integralet:
\begin{align*}
\int \frac{2x}{x^2+1} \intd x &= \int \frac{2x}{u} \frac{\intd u}{2x}\\
&= \int \frac{1}{u} \intd u \\
&= \ln(u) + k.
\end{align*}
Vi substituerer nu tilbage:
\begin{align*}
\int \frac{2x}{x^2+1} \intd x  = \ln(x^2+1)+k.
\end{align*}
Det er let at tjekke, at dette udtryk differentieret giver 
\begin{align*}
(\ln(x^2+1)+k)' = \frac{2x}{x^2+1}
\end{align*}
\end{exa}



\section*{Opgave 1}
Bestem arealet afgrænset af funktionen $f(x) = 4x^3-9x^2$ og $x$-aksen. Start med at tegne grafen.

\section*{Opgave 2}
Bevis $ii)$ og $iii)$ fra Sætning 1.2 fra sidst.

\section*{Opgave 3}
Løs følgende integraler ved integration ved substitution
\begin{align*}
&1) \ \int x\sin(x^2)\intd x   &&2) \ \int \frac{1}{\sqrt{2x}} \intd x   \\
&3) \  \int \frac{12x}{3x^2-1} \intd x   &&4) \ \int 2\sin(2x) -2x\cos(x^2)\intd x   \\
\end{align*}

\section*{Opgave 4}
Opgaver fra sidst