

\section*{Opgave 1}
\begin{enumerate}[label=\roman*)]
\item En producent af et lægemiddel påstår, at $12\%$ af anvendelserne af lægemiddelet er succesfulde. Ud af 100 personer bliver kun 5 helbredt. Vores nulhypotese er, at producenten taler sandt. Med et signifikansniveau på $5\%$, vil vi så forkaste nulhypotesen? Hvad er den kritiske mængde for forsøget?
\item Til seneste valg spurgte vi 300 personer om, hvad de stemte. 30 stemte på liste Q. Vi har nu spurgt 300 nye personer og denne gang har partiet fået 25 stemmer. Er det rimeligt at antage, at partiets vælgertilslutning er faldet? Hvis vores nulhypotese er, at tilslutningen ikke er faldet, kan vi så med et signifikansniveau på $5\%$ forkaste denne nulhypotese? Bestem desuden den kritiske mængde
\item En mand finder 3 sten i sit glas med oliven og synes, at det er lidt for mange. Han kigger på glasset, og der er det lovet, at glasset er $99\%$ stenfri. I glasset er der 70 oliven. Har han grund til sin skepsis? Opstil nulhypotese og se, om den kan forkastes med et signifikansniveau på $5\%$. Bestem desuden den kritiske mængde. 
\item Vi kaster med en mønt, og vil se, om plat og krone er lige sandsynlige. Vi slår 150 gange og får plat 72 gange. Nulhypotesen er $H_0$: Plat og krone er lige sandsynlige. Bestem med et signifikansniveau på $5\%$ om vi vil forkaste nulhypotesen. Bestem desuden den kritiske mængde. 
\end{enumerate}

\section*{Opgave 2}
En butik har et klientel, der består af $30\%$ mænd. 
\begin{enumerate}[label=\roman*)]
\item Opstil en binomialmodel, der beskriver antallet af mænd i en stikprøve på $50$ personer.
\item Hvad er sandsynligheden for, at 12 af personerne i stikprøven er mænd?
\item En dag havde butikken $220$ kunder. 79 af disse var mænd. Opstil en nulhypotese, der beskriver undersøgelsen, og brug en binomialtest med et signifikansniveau på $5\%$ til at undersøge nulhypotesen. 
\end{enumerate}

\section*{Opgave 3}
Laver vi krydsninger med en bestemt type røde og violette blomster siger Mendels love, at $75\%$ af blomsterne vil være røde og $75\%$ af blomsterne vil være hvide. I et krydsningsforsøg har vi 705 violette blomster og 224 hvide blomster. 
\begin{enumerate}[label=\roman*)]
\item Opstil en nulhypotese, der kan benyttes til at afgøre, om blomsternes farve følger Mendels love
\item Bestem de forventede antal violette og hvide blomster, hvis nulhypotesen er sand
\item Benyt binomialtest for at afgøre, om vi skal forkaste nulhypotesen. Signifikansniveauet er som altid $5\%$.
\end{enumerate}

\section*{Opgave 4}
I en skov er fordelingen af løvtræer og nåletræer fordelt omtrent ligeligt. En gruppe biologistuderende antager, at spættearten \textit{den store flagspætte} er ligeglad med, om den laver bo i et udgået løvtræ eller et udgået nåletræ. I skoven finder de 201 spætteboer i nåletræer og 167 i løvtræer. 
\begin{enumerate}[label=\roman*)]
\item Opstil en passende nulhypotese, der kan teste gruppen af biologistuderendes påstand. 
\item Benyt en binomialtest for at afgøre, om de skal forkaste deres nulhypotese
\item Hvad er den kritiske mængde for denne nulhypotese?
\end{enumerate}

\section*{Opgave 5}
Meyer er et terningespil, hvor hver spiller skiftes til at slå med to terninger. Det bedste slag er Meyer, der er en toer og en etter. Du spiller Meyer med en gruppe af dine venner, og du synes, at en af dem får Meyer lidt for ofte. Du tæller, at han ud af 93 slag har fået Meyer 11 gange.
\begin{enumerate}[label=\roman*)]
\item Hvad er sandsynligheden for at slå Meyer med to terninger?
\item Hvilken fordeling vil du forvente, at antallet af Meyere i 93 slag vil følge? Hvad er parametrene for fordelingen?
\item Opstil en nulhypotese, der undersøger din formodning.
\item Undersøg din nulhypotese med en passende test. 
\item Du spiller med 10 venner. Kan du med et signifikansniveau på $5\%$ sige, at ingen af disse bør få 11 eller flere gange Meyer?
\end{enumerate}

\section*{Opgave 6}
En medicinalvirksomhed med dårlig moral tester et nyt præparat op mod placebo. De giver $500$ personer placebo, og det viser sig, at $15\%$ af disse personer opnår fremgang i henseende A. De tester nu deres nye præparat på en tilsvarende gruppe af $500$ personer, og oplever, at $90$ personer opnår fremgang i henseende $A$. 
\begin{enumerate}[label=\roman*)]
\item Hvis præparatet hverken er bedre eller værre end placebo, hvor mange forventer vi så at helbrede?
\item Under nulhypotesen, at præparatet ikke er bedre end placebo, test så med binomialtest om denne nulhypotese kan forkastes med et signifikansniveau på $5\%$.
\end{enumerate}
I virksomheden har de glemt at slette deres indbyrdes sms'er, og det kommer frem, at virksomheden udover at teste for henseende A også har testet for forbedring inden for 26 andre henseender uden at kunne se en effekt. Det kaldes en type I fejl at forkaste en sand nulhypotese, og med et signifikansniveau på $5\%$ er sandsynligheden for at lave en type I fejl generelt mindre end $5\%$. 
\begin{enumerate}[label=\roman*)]
\setcounter{enumi}{2}
\item Hvad er sandsynligheden for at lave en type I fejl, hvis man tester for forbedring i 27 forskellige henseender op mod placebo hver med et signifikansniveau på $5\%$?
\end{enumerate}
