
\begin{center}
\Huge
Gensyn med linjens ligning og parametrisering
\end{center}

\section*{Repræsentationer for linjen}
\stepcounter{section}

Vi husker på, at vi kan beskrive en linje $l$ ud fra et punkt på linjen $P=(x_0,y_0)$ samt en normalvektor til linjen $\vv{n} = \begin{pmatrix}a\\ b\end{pmatrix}.$ Dette gøres ud fra følgende kendsgærning. Vi kan danne en vektor fra $(x_0,y_0)$ til ethvert andet punkt $(x,y)$ som \begin{align}\label{eq:eq1}
\begin{pmatrix}
x-x_0\\y-y_0
\end{pmatrix}.
\end{align}
Vi husker på, at to vektorer er orthogonale hvis og kun hvis deres prikprodukt er 0. Derfor må alle punkter $(x,y)$ opfylde, at vektoren \eqref{eq:eq1} og $\vv{n}$ er orthogonale, altså
\begin{align*}
\begin{pmatrix}
x-x_0\\ y-y_0
\end{pmatrix} \cdot 
\begin{pmatrix}
a\\ b
\end{pmatrix} = 0,
\end{align*}
hvilket medfører, at 
\begin{align*}
a(x-x_0) + b(y-y_0) = 0.
\end{align*}
Vi kan nu konkludere med en sætning.
\begin{setn}[Linjens ligning]
Alle punkter $(x,y)$, der ligger på en linje $l$ opfylder,  at
\begin{align*}
a(x-x_0) + b(y-y_0) = 0,
\end{align*}
hvor $P = (x_0,y_0)$ er et punkt på linjen og $\vv{n} = \begin{pmatrix}
a\\ b
\end{pmatrix}$
er en normalvektor til $l$. 
\end{setn}

\begin{exa}
På linjen $l$ kender vi punktet $P = (-1,3)$ og normalvektoren $\vv{n} = \begin{pmatrix}
2 \\ 4
\end{pmatrix}$. Ligningen for $l$ er derfor givet
\begin{align*}
2(x+1) + 4(y-3) = 0.
\end{align*}
\end{exa}

Vi kan repræsentere linjen ved hjælp af en retningsvektor i stedet. Har vi en linje $l$, der går gennem et punkt $P = (x_0,y_0)$, samt har en retningsvektor $\vv{r} = \begin{pmatrix}
r_1 \\ r_2
\end{pmatrix}$, så må punktet $(x,y)$, der opfylder, at
\begin{align*}
\begin{pmatrix}
x \\ y
\end{pmatrix}
= \begin{pmatrix}
x_0 \\ y_0
\end{pmatrix} + \begin{pmatrix}
r_1 \\ r_2
\end{pmatrix}
\end{align*}
ligge på $l$. Men for $t\in \mathbb{R}$ ved vi, at alle vektorer $t\vv{r}$ er parallelle med $\vv{r}.$ Derfor må der desuden gælde, at alle punkter $(x,y)$, der opfylder, at 
\begin{align*}
\begin{pmatrix}
x \\ y
\end{pmatrix}
=
\begin{pmatrix}
x_0 \\ y_0
\end{pmatrix} 
+
t
\begin{pmatrix}
r_1 \\ r_2
\end{pmatrix}
 \end{align*}
Må være præcist de punkter, der ligger på linjen $l$. Dette leder os til følgende sætning.
\begin{setn}[Parametrisering af linjen]
For en linje $l$ gælder der, at alle punkter $(x,y)$, der ligger på $l$ opfylder, at
\begin{align*}
\begin{pmatrix}
x \\ y
\end{pmatrix}
=
\begin{pmatrix}
x_0 \\ y_0
\end{pmatrix} 
+
t
\begin{pmatrix}
r_1 \\ r_2
\end{pmatrix},
\end{align*} 
hvor $P=(x_0,y_0)$ er et punkt på linjen og $\vv{r} = \begin{pmatrix}
r_1 \\ r_2
\end{pmatrix}$
er en retningsvektor for linjen. Vi kalder dette for en \textit{parametrisering af l}.
\end{setn}

\begin{exa}
Har vi et punkt $P = (-1,3)$ og en retningsvektor $\vv{r} = \begin{pmatrix}
5 \\ 1
\end{pmatrix}$ kan vi bestemme parametriseringen for linjen $l$, der går gennem $P$ og har retningsvektor $\vv{r}$:
\begin{align*}
\begin{pmatrix}
x \\ y
\end{pmatrix}
= 
\begin{pmatrix}
-1 \\ 3
\end{pmatrix} +  
t
\begin{pmatrix}
5 \\ 1
\end{pmatrix}.
\end{align*}
Da $\begin{pmatrix}
-1 \\ 5
\end{pmatrix}$
er orthogonal til $\vv{r}$ vil denne vektor være en normalvektor til $l$. Vi kan derfor også repræsentere $l$ ved ligningen
\begin{align*}
-1(x+1) + 5(y-3) = 0.
\end{align*}
\end{exa}

\section*{Opgave 1}
Bestem linjens ligning for følgende punkter $P = (x_0,y_0)$ og normalvektorer\\ $\vv{n} = \begin{pmatrix}
a \\ b
\end{pmatrix}$
\begin{align*}
&1) \ P = (1,1), \ \vv{n} = \begin{pmatrix}
-2 \\ 3
\end{pmatrix}.   &&2) \ P = (-5,-3), \ \vv{n} = \begin{pmatrix}
2 \\ 7
\end{pmatrix}.    \\
&3) \ P = (\frac{-2}{5},13), \ \vv{n} = \begin{pmatrix}
-10 \\ 20
\end{pmatrix}.   &&4) \ P = (\sqrt{2},\sqrt{3}), \ \vv{n} = \begin{pmatrix}
\frac{1}{5} \\ \frac{7}{10}
\end{pmatrix}.   \\
&5) \ P = (0,0), \ \vv{n} = \begin{pmatrix}
1 \\ 1
\end{pmatrix}.   &&6) \ P = (-100,5), \ \vv{n} = \begin{pmatrix}
-\frac{\sqrt{2}}{2} \\ -9
\end{pmatrix}.   \\
\end{align*}

\section*{Opgave 2}
Bestem parametriseringen for en linje, der står vinkelret på hver af linjerne fra Opgave 1.

\section*{Opgave 3}

\begin{enumerate}[label=\roman*)]
\item Bestem en parametrisering for linjen $l$ givet ved ligningen 
\begin{align*}
2(x-2) + 3(y+1) = 0.
\end{align*}
\item Bestem en parametrisering for linjen $l$ givet ved ligningen 
\begin{align*}
-10(x-10) + 10(y+10) = 0.
\end{align*}
\item Bestem en ligning for linjen $l$ givet ved parametriseringen 
\begin{align*}
\begin{pmatrix}
x \\ y
\end{pmatrix}
= 
\begin{pmatrix}
2 \\ 3
\end{pmatrix}
+
t
\begin{pmatrix}
-4 \\ 6
\end{pmatrix}.
\end{align*}
\item Bestem en ligning for linjen $l$ givet ved parametriseringen 
\begin{align*}
\begin{pmatrix}
x \\ y
\end{pmatrix}
= 
\begin{pmatrix}
\sqrt{5} \\ \frac{4}{3}
\end{pmatrix}
+
t
\begin{pmatrix}
\frac{8}{3} \\ \sqrt{10}
\end{pmatrix}
\end{align*}
\end{enumerate}

\section*{Opgave 4}
\begin{enumerate}[label=\roman*)]
\item Bestem skæringen mellem linjerne $l$ og $m$, der har følgende ligninger henholdsvist:
\begin{align*}
(x-3) + (y-4) = 0
\end{align*}
og 
\begin{align*}
4(x+2) + 2(y-1) = 0
\end{align*}

\item Bestem skæringen mellem linjerne $l$ og $m$, der har følgende ligninger henholdsvist:
\begin{align*}
2(x+1) + 3(y-1) = 0
\end{align*}
og 
\begin{align*}
4(x-4) + 1(y-2) = 0
\end{align*}

\item Bestem skæringen mellem linjerne $l$ og $m$, der har følgende parametriseringer henholdsvist:
\begin{align*}
\begin{pmatrix}
x \\ y
\end{pmatrix}
= 
\begin{pmatrix}
1 \\ -1
\end{pmatrix}
+
t
\begin{pmatrix}
2 \\ 1
\end{pmatrix}
\end{align*}
og 
\begin{align*}
\begin{pmatrix}
x \\ y
\end{pmatrix}
= 
\begin{pmatrix}
-3 \\ 3
\end{pmatrix}
+
t
\begin{pmatrix}
-2 \\ 1
\end{pmatrix}
\end{align*}

\item Bestem skæringen mellem linjerne $l$ og $m$, der har følgende parametriseringer henholdsvist:
\begin{align*}
\begin{pmatrix}
x \\ y
\end{pmatrix}
= 
\begin{pmatrix}
5 \\ 2
\end{pmatrix}
+
t
\begin{pmatrix}
3 \\ 4
\end{pmatrix}
\end{align*}
og 
\begin{align*}
\begin{pmatrix}
x \\ y
\end{pmatrix}
= 
\begin{pmatrix}
-1 \\ 6
\end{pmatrix}
+
t
\begin{pmatrix}
-3 \\ 6
\end{pmatrix}
\end{align*}

\end{enumerate}