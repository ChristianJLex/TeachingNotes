
\begin{center}
\Huge
Regneregler for differentiation
\end{center}
\section*{Regneregler}
\stepcounter{section}

\begin{setn}
\begin{enumerate}[label=\roman*)]
\item For den naturlige logaritme $\ln$ har vi
\begin{align*}
(\ln(x))' = \frac{1}{x}.
\end{align*}
\item For den naturlige eksponentialfunktion $e^x$ har vi
\begin{align*}
(e^x)' = e^x.
\end{align*}
\item For en given eksponentialfunktion $a^x$ har vi
\begin{align*}
(a^x)' = a^x\ln(a).
\end{align*}
\item For potensfunktioner (og specielt polynomier) $x^a$ har vi
\begin{align*}
(x^a)' = ax^{a-1}
\end{align*}
\end{enumerate}
\end{setn}
Vi vil bevise iv) i tilfældet, at $a$ er et heltal. Vi vil bevise iv) ved induktion. Princippet i induktionsbeviser er som følgende: Lad os sige, at vi ønsker at bevise noget for ethvert naturligt tal $n$. Vi kan så starte med at give et bevis i tilfældet at $n=0$ eller $n=1$ og derefter vise, at hvis det er sandt for et vilkårligt naturligt tal $k$, så må det også være sandt for $k+1$.
\begin{exa}
Det arketypiske eksempel på et induktionsbevis er følgende: Bevis, at summen af de første $n$ heltal er givet ved $n(n+1)/2$, altså
\begin{align}\label{eq:sumofints}
\sum_{i=0}^n i = \frac{n(n+1)}{2}.
\end{align}
\begin{proof}
Basisskridt: Antag, at $n=1$. Så er $n(n+1)/2 = 1(2)/2=1$, altså er \eqref{eq:sumofints} sand for $n=1$. 


Induktionsskridt: Antag nu, at \eqref{eq:sumofints} er sandt for et vilkårligt $n>1$. Vi lægger $n+1$ til på begge sider af lighedstegnet i \eqref{eq:sumofints} og får
\begin{align*}
\sum_{i=0}^{n+1} i &= \frac{n(n+1)}{2} +n+1 \\
&= \frac{n(n+1)}{2}+\frac{2(n+1)}{2}\\
&= \frac{n(n+1)+2(n+1)}{2}\\
&= \frac{(n+1)(n+2)}{2},
\end{align*}
og vi er færdige.
\end{proof}
\end{exa} 
\begin{proof}[Bevis for iv)]
Basisskridt: Antag, at $a=1$. Så har vi $x^a = x^1$ og $(x^1)' = 1 = 1x^0$ af en tidligere sætning. 

Induktionsskridt: Antag nu, at $(x^a)' = ax^{a-1}$ for et vilkårligt $a>1$, hvor $a$ er et heltal. Vi ser på
\begin{align*}
x^{a+1} &= \lim_{h\to 0} \frac{(x+h)^{a+1}-x^{a+1}}{h}\\
&= \lim_{h\to 0} \frac{(x+h)^a(x+h)-x^ax}{h}\\
&= \lim_{h\to 0} \frac{(x+h)^ax+(x+h)^ah-x^ax}{h}\\
&= \lim_{h\to 0} \frac{x[(x+h)^a-x^a] + (x+h)^ah}{h}\\
&=x ax^{a-1}+x^a\\
&= (a+1)x^a,
\end{align*}
og vi er færdige.
\end{proof}
\section*{Lidt om eksponentialfunktionen}
Når vi har med eksponentiel vækst at gøre, vil vi ofte gerne sige noget om væksten til et bestemt tidspunkt $t$, altså tangenthældningen af funktionen i punktet tilsvarende tid $t$. Eksponentiel vækst er på formen
\begin{align*}
f(x) = a^x.
\end{align*}
Enhver eksponentiel funktion kan omskrives til den naturlige eksponentialfunktion som
\begin{align*}
f(x) = a^x = e^{\ln(a)x} = e^{kx},
\end{align*}
hvor $k=\ln(a)$. Dette er smart, da den afledede af $f$ så blot bliver en konstant gange funktionen:
\begin{align*}
f'(x) = ke^{kx}
\end{align*}
af kædereglen. Tilsvarende vil fordoblings og halveringskonstanten for $f$ så være henholdsvist givet ved
\begin{align*}
&T_2 = \frac{\ln(2)}{k}, &&\textnormal{og }T_{1/2} = \frac{\ln(1/2)}{k}.
\end{align*}

\section*{Opgave 1}
Differentiér følgende funktioner
\begin{align*}
&1) \ x^\frac{27}{2}   &&2) \  e^{10x}   \\
&1) \ \frac{e^{2x+3}}{x^2}   &&2) \ \frac{3}{x^{x^{\frac{3}{2}}+1}}    \\
&1) \  4e^{2x^2}-\ln(3x)  &&2) \ \ln(25x^2)      \\
&1) \ 10^x   &&2) \  \frac{ln(x)}{\ln(x^2)}   \\
&1) \  ln(e^{2x})  &&2) \  e^{2ln(x^2)}   \\
\end{align*}
\section*{Opgave 2}
\begin{enumerate}[label=\roman*)]
\item
Antallet af mennesker i en given befolkning kan beskrives med
\begin{align*}
N(t) = 2.3e^{0.05t},
\end{align*}
hvor $t$ er tid målt i år og $N(t)$ er befolkningens størrelse målt i mio. mennesker. Hvilken væksttype vokser befolkningen med? Bestem den hastighed befolkningen vokser med både som funktion af $t$ og til tiden $t = 10$ og $t=50$. Hvad er fordoblingskonstanten for befolkningen?
\item Temperaturen $H$ i et glas vand kan modelleres i et rum med modellen
\begin{align*}
H(t) = 10e^{-0.023t}.
\end{align*}
$t$ er tid i minutter og $H$ er grader celsius. Hvilken væksttype er dette? Hvis modellen er korrekt, hvor varmt er der så i det omkringliggende miljø? Hvad er temperaturen af vandet til tid $t=0$? Hvor hurtigt aftager vandets temperatur efter 10min? Hvad er halveringstiden for temperaturen? Hvornår er vandet 2 grader? Hvad med 0 grader?
\end{enumerate}
