\begin{center}
\Huge
Afstand mellem linje og punkt
\end{center}

Vi kan bruge projektioner af vektorer til at bestemme afstanden mellem et punkt $P$ og en linje $l$ i planen. Det er ikke umiddelbart klart ud fra formlen at det er projektioner, vi bruger, men det vil ses af beviset. Afstanden findes ved følgende sætning. 

\begin{setn}[Afstand mellem punkt og linje]
Har vi et punkt $Q(x_1,y_1)$ og en linje $l$ givet ved ligningen
\begin{align*}
ax+by+c = 0, 
\end{align*}
så kan vi bestemme afstanden mellem $l$ og $Q$ ved
\begin{align*}
\textnormal{dist}(Q,l) = \frac{|ax_1+by_1+c|}{\sqrt{a^2+b^2}}
\end{align*}
\end{setn}
\begin{proof}
Vi vælger et punkt  $P(x_0,y_0)$ på $l$. Da $l$ har ligningen
\begin{align*}
ax+by+c = 0, 
\end{align*}
må der gælde, at 
\begin{align*}
ax_0+by_0+c = 0, 
\end{align*}'
og derfor har vi et udtryk for $c$ givet ved
\begin{align*}
c = -ax_0+by_0.
\end{align*}
Vektoren $\vv{PQ}$ er givet ved
\begin{align*}
\vv{PQ} = \begin{pmatrix}
x_1 - x_0 \\ y_1 - y_0.
\end{pmatrix}
\end{align*}
Vi har en normalvektor til $l$ givet ved 
\begin{align*}
\vv{n}=
\begin{pmatrix}
a \\ b
\end{pmatrix}
\end{align*}
Vi bemærker nu, at længden af projektionen $\vv{PQ_{\vv{n}}}$ må være afstanden fra $l$ til $Q$. Vi bestemmer derfor længden af denne projektion:
\begin{align*}
\left|\vv{PQ_{\vv{n}}}\right| &= \frac{|\vv{PQ_{\vv{n}}}\cdot \vv{n}|}{|\vv{n}|} \\ 
&= \frac{\left|\begin{pmatrix}
x_1-x_0 \\ y_1 - y_0
\end{pmatrix}\cdot \begin{pmatrix}
a \\ b
\end{pmatrix} \right|}{\left|\begin{pmatrix}
a \\ b
\end{pmatrix}\right|}\\
&=\frac{|a(x_1-x_0)+b(y_1-y_0)|}{\sqrt{a^2+b^2}}\\
&= \frac{|ax_1+by_1 -ax_0-by_0|}{\sqrt{a^2+b^2}}\\
&= \frac{|ax_1+by_1+c|}{a^2+b^2}.
\end{align*}
\end{proof}


\begin{exa}
Vi skal bestemme afstanden fra punktet $P(1,1)$ til linjen med ligningen $4(x-1) + 3(y+1) = 0$. Vi starter med at hæve parenteserne i ligningen:
\begin{align*}
4(x-1) + 3(y+1) = 0 \Leftrightarrow 4x+3y-1 = 0.
\end{align*}
Vi kan nu bruge formlen for afstand mellem punkt og linje og får:
\begin{align*}
\textnormal{dist}(P,l) &= \frac{|4\cdot 1 + 3\cdot 1-1|}{\sqrt{3^2+4^2}}\\
&=\frac{6}{5},
\end{align*}
hvilket er afstanden fra punktet $P$ til linjen $l$. 
\end{exa}

\begin{exa}
Vi skal bestemme $k$, så punktet $P(4,k)$ og linjen $l$ med ligningen 
\begin{align*}
2x-4y-3 = 0
\end{align*}
har afstand 1. Vi bruger afstandsformlen og får
\begin{align*}
\textnormal{dist}(P,l) = \frac{|2\cdot 4-4\cdot k-3|}{\sqrt{2^2+4^2}|} = \frac{|5-4k|}{\sqrt{20}} =  1.
\end{align*}
Vi løser denne ligning og får, at $k \approx 0.63$. 
\end{exa}

\section*{Opgave 1}
Bestem afstanden mellem linjen $l$ med ligningen 
\begin{align*}
-1(x+4) + 3(y-1) = 0
\end{align*}
og følgende punkter:
\begin{align*}
&1) \ (-1,1)  &&2) \ (3,4)    \\
&3) \ (\sqrt{2},\sqrt{2})  &&4) \ (-5,6)   \\
&5) \ (0,0)  &&6) \  (-\frac{1}{2},2)   \\
\end{align*}
Tegn desuden situationen i Geogebra og tjek, at du har fundet den korrekte afstand. 

\section*{Opgave 2}
\begin{enumerate}[label=\roman*)]
\item Bestem $k$, så afstanden mellem linjen givet ved 
\begin{align*}
x-5y+10 = 0
\end{align*}
og punktet $P(k,1)$ har afstand $4$. Start med at tegne det i Geogebra.
\item Bestem $b$, så $l$ med ligningen
\begin{align*}
y = 2x+b,
\end{align*}
 og punktet $P(1,1)$ har afstand 5. Start med at tegne i Geogebra.
\end{enumerate}

\section*{Opgave 3}
Aflevering