\begin{center}
\Huge
Trigonometriske funktioner
\end{center}

\section*{Cos og Sin}
\stepcounter{section}

Vi har tidligere betragtet de trigonometriske funktioner $\cos(x)$ og $\sin(x)$. Vi husker på, at for et punkt $P$ på enhedscirklen med vinkel $x$ mellem førsteaksen og stedvektoren $\vv{OP}$, så er $\cos(x)$ og $\sin(x)$ defineret som henholdsvis første og andenkoordinaten til $P$. 

Da omkredsen på en enhedscirkel er $2\pi$, så kan vi i stedet for at bruge vinklen ved origo bruge omkredsen på enhedscirklen. Omkredsen tilsvarer en vinkel, og denne omkreds måles i \textit{radianer}. En vinkel på $90^\circ$ tilsvarer en kvart omgang på enhedscirklen og derfor $\frac{\pi}{2}$ radianer. Vi skriver typisk ikke radianer, men blot en vinkel på $\frac{\pi}{2}$
\section*{Opgave 1}
Omregn følgende vinkler fra grader til radianer:
\begin{align*}
&1) \  90   &&2) \  180 \\
&3) \  130   &&4) \ 360  \\
&5) \  270   &&6) \ 200  \\  
\end{align*} 

\section*{Opgave 2}
Omregn følgende vinkler fra radianer til grader:
\begin{align*}
&1) \  2\pi  &&2) \  \frac{\pi}{4}    \\
&3) \  \frac{3\pi}{2}  &&4) \  \frac{\pi}{3}    \\
&5) \  \pi  &&6) \  3\pi   \\
\end{align*}

\section*{Opgave 3}
Bestem følgende 
\begin{align*}
&1) \ \cos(\pi)    &&2) \ \sin(\pi/2)    \\
&3) \ \sin(45^\circ)    &&4) \ \cos(2\pi)    \\
&5) \ \cos(120^\circ)    &&6) \  \cos(\pi/4)   \\
\end{align*}