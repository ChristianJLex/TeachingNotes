\begin{center}
\Huge
Sidste aflevering 2.g
\end{center}
\section*{Opgave 1 (uden hjælpemidler)}
Forkort udtrykket
\begin{align*}
\frac{a^2-b^2}{a-b} -b.
\end{align*}

\section*{Opgave 2 (uden hjælpemidler)}

Differentier følgende funktioner:

\begin{enumerate}[label=\roman*)]
\item $\sin(x)\cdot x^3$
\item $\ln(3x^2+2x-1)$
\end{enumerate}


\section*{Opgave 3 (uden hjælpemidler)}
\stepcounter{section}
Bestem følgende bestemte integral:
\begin{align*}
\int_{-2}^2 3x^2 -\frac{1}{x^2} \intd x
\end{align*}

\section*{Opgave 4 (uden hjælpemidler)}
Et polynomium $f$ er givet ved
\begin{align*}
f(x) = \frac{1}{3}x^3+\frac{1}{2}x^2-2x
\end{align*} 
\begin{enumerate}[label=\roman*)]
\item Løs ligningen $f'(x) = 0$
\item Bestem monotoniforholdene for $f$.
\end{enumerate}

\section*{Opgave 5 (uden hjælpemidler)}

Bestem følgende ubestemte integral:
\begin{align*}
\int \frac{e^{\ln(x)+7}}{x} \intd x
\end{align*}

\section*{Opgave 6 (uden hjælpemidler)}
En cirkel har ligningen 
\begin{align*}
x^2-6x+y^2-4y-3=0.
\end{align*}
\begin{enumerate}[label=\roman*)]
\item Bestem centrum og radius for cirklen. 
\item Punktet (3,6) ligger på cirklen. Bestem tangenten til cirklen i dette punkt. 
\end{enumerate}
\section*{Opgave 7 (uden hjælpemidler)}
To funktioner er givet ved 
\begin{align*}
f(x) = x^2+10
\end{align*}
og 
\begin{align*}
g(x) = \sqrt{x-10}
\end{align*}

\begin{enumerate}[label=\roman*)]
\item Bestem $f(g(x))$ og $g(f(x))$.
\end{enumerate}
\section*{Opgave 8 (uden hjælpemidler)}

En binomialfordelt stokastisk variabel $X \sim \textnormal{bin}(25,0.5)$ har antalsparameter $n=25$ og sandsynlighedsparameter $p=0.5$.
\begin{enumerate}[label=\roman*)]
\item Bestem middelværdien $\mathbb{E}[X] = \mu$.
\item Bestem spredningen $\sigma = \sqrt{\textnormal{Var}[X]}$.
\end{enumerate} 

\section*{Opgave 9 (med hjælpemidler)}

Antallet af en bestemt type bakterie i en opløsning kan beskrives ved funktionen $f$ givet ved
\begin{align*}
f(t) = \frac{517.2}{1+13.74 \cdot 0.751^t},
\end{align*}
hvor $t$ er antallet af forløbne timer og $f$ er antallet af bakterier i opløsningen i mio. 
\begin{enumerate}[label=\roman*)]
\item Tegn grafen for $f$ på intervallet $[0,25]$.
\item Afgør, hvornår antallet af bakterier i opløsningen overstiger 400 mio. 
\item Bestem tidspunktet, hvor antallet af bakterier tiltager mest. 
\end{enumerate}


\section*{Opgave 10 (med hjælpemidler)}
En appelsinbonde ønsker at bestemme sammenhængen mellem radius på hans appelsiner og mængden af juice, han får per appelsin. Han prøver derfor at presse appelsiner med forskellig radius, og måler mængden af juice per appelsin. Hans opsamlede data kan ses af Tab. \ref{tab:juice}.
\begin{table}[H]
\begin{tabular}{c|c|c|c|c|c|c|c|c|c|c}
Radius (cm) & 2 & 2.1 & 2.15 & 2.3 & 2.4 & 2.6 & 2.9 & 3.05 & 3.4 & 3.5\\ \hline
Juice (dL) & 0.2 & 0.22 & 0.19 &0.24 & 0.31 & 0.45 & 0.55 & 0.59 & 0.85 & 0.97
\end{tabular}
\caption{Juice fra appelsiner}
\label{tab:juice}
\end{table}

\begin{enumerate}[label=\roman*)]
\item Bestem den lineære funktion samt den potensfunktion, der bedst beskriver mængden af juice som funktion af radius af appelsinerne. 
\item Plot residualerne for de to modeller og afgør, hvilken model der beskriver sammenhængen bedst.
\end{enumerate}
En unavngiven virksomhed påstår, at der til 1.75L af deres juice skal bruges 16 søde appelsiner. 

\begin{enumerate}[label=\roman*)]
\setcounter{enumi}{2}
\item Hvor store skal disse appelsiner i følge din model være, hvis virksomhedens påstand skal være korrekt (og appelsinerne er lige store)?
\end{enumerate}

\section*{Opgave 11 (med hjælpemidler)}
To andengradspolynomier $f$ og $g$ er givet ved
\begin{align*}
	f(x) = -3x^2 + 2x + 42
\end{align*}
og 
\begin{align*}
	g(x) = 2x^2 - 3x + 12
\end{align*}
Disse funktioner afgrænser til sammen et område $A$. 
\begin{enumerate}[label=\roman*)]
\item Bestem arealet af $A$.
\item Bestem rumfanget af det omdregningslegeme, der dannes, når $A$ roteres omkring $x$-aksen.
\end{enumerate}


