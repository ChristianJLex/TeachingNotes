\lhead{Matematik B\\
Modul 37
}
\chead{ 21/3/2022
}

\begin{document}

%Udfyld afsnit herunder og lav til egen Latex-fil

%Kopier følgende til overskrift:

%\begin{center}
%\Huge
%Aflevering 1
%\end{center}
%\section*{Opgave 1}
%\stepcounter{section}


\begin{center}
\Huge
Faktorisering af polynomier
\end{center}
\section*{Faktorisering}
\stepcounter{section}

Vi har tidligere set på kvadratsætninger. Disse er relateret til faktorisering af polynomier. Med faktorisering af polynomier menes der, at polynomier opskrives som faktorer i stedet for led. Dette kan dog kun gøres, hvis polynomiet har rødder nok. 
\begin{exa}
Vi kan opskrive andengradspolynomiet $f(x) = x^2 -x -2$ som $f(x) = (x-2)(x+1)$. Dette kan ses ved at gange parentesen ud. Vi siger også, at vi har faktoriseret polynomiet i forhold til dets rødder, da det er let at se, at $f$ har rødder i $2$ og $-1$ på den faktoriserede form.
\end{exa}
\begin{setn}
Lad $f$ være et andengradspolynomium med to rødder $x_1$ og $x_2$ på formen
\begin{align*}
f(x) = ax^2+bx+c.
\end{align*}
Så kan vi faktorisere $f$ som
\begin{align}\label{eq:factor}
f(x) = a(x-x_1)(x-x_2)
\end{align}
\end{setn}
\begin{proof}
Vi ved, at $x_1 = \frac{-b+\sqrt{d}}{2a}$ og $x_2 = \frac{-b-\sqrt{d}}{2a}$. Dette indsætter vi i \eqref{eq:factor}, og får
\begin{align*}
f(x) &= a(x-x_1)(x-x_2)\\
	 &= a(x^2-x_2x-x_1x+x_1x_2)\\
	 &=ax^2-ax_2a-ax_1x+ax_1x_2\\
	 &=ax^2 -a\frac{-b-\sqrt{d}}{2a}x - a\frac{-b+\sqrt{d}}{2a}x + a\frac{(-b+\sqrt{d})}{2a}\frac{(-b-\sqrt{d})}{2a}\\
	 &=ax^2 + \frac{b}{2}x + \frac{b}{2}x + a\frac{b^2+b\sqrt{d}-b\sqrt{d}-d}{4a^2} \\
	 &= ax^2 + bx + \frac{b^2-b^2+4ac}{4a^2}\\
	 &= ax^2 + bx+c
\end{align*}
\end{proof}



\section*{Opgave 1}
Bestem rødderne til følgende faktoriserede polynomier:
\begin{align*}
&1) \  x-4  &&2) \ (x+1)(x+7)   \\
&3) \  4(x-6)(2x-2)  &&4) \ (x-1)(x+2)(x-3)(x+4)   \\
&5) \  (2x-4)(x-2)  &&6) \ (x+3)^{10}   \\
&7) \  3(x-1)\cdot 2(x-2)\cdot 5(x-5)  &&8) \ (x-\sqrt{2})(x+\sqrt{3})(10x-3)    \\
\end{align*}
\section*{Opgave 2}
Faktoriser følgende andengradspolynomier ved at bestemme deres rødder:
\begin{align*}
&1) \ 3x^2-15x+18  &&2) \ 2x^2-8x+8   \\
&3) \ x^2-4  &&4) \ 4x^2+4x-8   \\
&5) \ x^2-5x+6   &&6) \  2x^2-18  \\
\end{align*}

\section*{Opgave 3}
Et polynomium skærer punkterne $(-3,0)$, $(4,0)$ og $(0,-24)$. Bestem koefficienterne for dette polynomium. 
