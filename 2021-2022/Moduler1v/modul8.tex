
\begin{center}
\Huge
Logaritmer
\end{center}
\stepcounter{section}
\begin{exa}
pH-værdien af en opløsning er defineret som
\begin{align*}
pH = -\log(a_{H^+}),
\end{align*}
hvor $a_{H^+}$ betegner hydrogenionaktiviteten. $a_{H^{+}}$ er ca. stofmængdekoncentrationen af $H^{+}$-ioner i opløsningen. Hvis vi antager, at en opløsning har en stofmængdekoncentration på $10^{-3}mol/L$ $h^+$-ioner. Så vil 
pH-værdien af opløsningen være (ca.) $-\log(10^{-3}) = 3$.  
\end{exa}
\section*{Den naturlige logaritme}
\begin{defn}
Den naturlige logaritme er den entydige funktion $\ln$, der opfylder, at
\begin{align*}
\ln(e^x) = x, 
\end{align*}
og
\begin{align*}
e^{\ln(x)} = x,
\end{align*}
hvor $e$ er Euler's tal. ($e \approx 2.7182$)
\end{defn}
Funktionen $e^x$ kaldes for den naturlige eksponentialfunktion, og vi vil senere beskrive den nærmere.

\begin{setn}[Regneregler for $\ln$]
For den naturlige logaritme $\ln$ gælder der for $a,b>0$, at
\begin{enumerate}[label=\roman*)]
\item $\ln(a\cdot b) = \ln(a) + \ln(b)$,
\item $\ln(\frac{a}{b}) = \ln(a)-\ln(b)$,
\item $\ln(a^x) = x\ln(a)$.
\end{enumerate}
\end{setn}
\section*{Opgave 1}
Løs følgende ligninger
\begin{align*}
&1) \  \log(x) = 1  &&2) \ \log(x) = 2.5    \\
&3) \ \log(2x) = 4   &&4) \ \log(3x+10)=3    \\
&5) \ \log(x^2) = 10   &&6) \  \log(5x) = 5    \\
\end{align*}
\section*{Opgave 2}
Bestem følgende 
\begin{align*}
&1) \ \log(\sqrt{10})    &&2) \  \log(\sqrt[3]{100})  \\
&3) \ \log(\sqrt[n]{1000})   &&4) \ \log(2) + \log(50)    \\
&5) \ \log(200)-\log(20)   &&6) \ \log(2\cdot 10^5)   
\end{align*}
\section*{Opgave 3}
\begin{enumerate}[label=\roman*)]
\item Hvad er pH-værdien for en opløsning med en $H^+$-koncentration på $10^{-10}$?
\item Hvad er $H^+$-koncentrationen for en opløsning med pH-værdi $7,5$?
\item En opløsning har volumen $500L$ og indeholder $1$ mol $H^+$. Hvad er $pH$-værdien af opløsningen?
\end{enumerate}
\section*{Opgave 4}
\begin{enumerate}[label=\roman*)]
\item Bevis, at  $\ln(ab) = \ln(a)+\ln(b).$
\item Bevis, at  $\ln(\frac{a}{b}) = \ln(a)-\ln(b)$.
\item Bevis, at  $\ln(a^x) = x\ln(a)$.
\end{enumerate}