
\begin{center}
\Huge
Regneregler for vektorer
\end{center}

\section*{Nulvektoren}
\stepcounter{section}
Har vi en vektor $\vec{v}$, og den modsatrettede vektor $-\vec{v}$, så får vi, hvis vi lægger $\vec{v}$ og $-\vec{v}$ sammen
\begin{align*}
\vec{v}+(-\vec{v}) = \begin{pmatrix}
v_1\\v_2
\end{pmatrix} + \begin{pmatrix}
-v_1 \\-v_2
\end{pmatrix} = \begin{pmatrix}
0\\0
\end{pmatrix}.
\end{align*}
Vektoren $\begin{pmatrix}0\\0

\end{pmatrix}$ kalder vi for\textit{ nulvektoren} og betegner med $\vec{0}$. Om den gælder der for alle vektorer $\vec{v}$, at $$\vec{0}+\vec{v} = \vec{v}.$$

\section*{Regneregler for vektorer}
Hvis vi vil gøre en vektor $k$ gange længere, kan vi gange vektoren med et tal $k$. 
\begin{defn}[Vektorskalering]
For en vektor $\vec{v} = \begin{pmatrix}
v_1\\v_2
\end{pmatrix}$ og et tal $k$, kan vi skalere vektoren med $k$ som
\begin{align*}
k\vec{v} = k\begin{pmatrix}
v_1\\v_2 
\end{pmatrix} =  \begin{pmatrix}
kv_1\\kv_2
\end{pmatrix}.
\end{align*}
\end{defn}
Det gælder rent faktisk, at længden af $\vec{v}$ bliver ganget med $k$, når vi skalerer med $k$. Dette kan ses, da
\begin{align*}
|k\vec{v}| &= \sqrt{(kv_1)^2+(kv_2)^2}\\
 &= \sqrt{k^2v_1^2 + k^2v_2^2}\\
 &= \sqrt{k^2(v_1^2+v_2^2)}\\
  &= \sqrt{k^2}\sqrt{v_1^2+v_2^2} = k|\vec{v}|.
\end{align*}
\begin{exa}
Lad $\vec{v} = \begin{pmatrix}
1\\2\end{pmatrix}$ og lad $k = 3$. Vi bestemmer så
\begin{align*}
|3\vec{v}| &= \left| \begin{pmatrix}
3 \cdot 1\\ 3\cdot 2
\end{pmatrix} \right|\\
&= \left| \begin{pmatrix}
3\\ 6
\end{pmatrix}\right|\\
 &=\sqrt{3^2+6^2} \\ &= \sqrt{45} = 3\sqrt{5}
\end{align*}
Men vi har lige argumenteret for at dette må være det samme som $3\cdot |\vec{v}|$. Vi verificerer:
\begin{align*}
3 |\vec{v}| = 3\sqrt{1^2+2^2} = 3\sqrt{5}.
\end{align*}
\end{exa}
Vi har følgende regneregler for vektorer.
\begin{setn}[Regneregler for vektorer]\label{setn:regneregler}
For tal $a,b$ og vektorer $\vec{u},\vec{v},\vec{w}$ gælder der, at 
\begin{enumerate}[label=\roman*)]
\item $\vec{u} + \vec{v} = \vec{v}+\vec{u}$,
\item $(\vec{u} + \vec{v}) +\vec{w} = \vec{u} + (\vec{v} + \vec{w})$,
\item $a(\vec{u}+\vec{v}) = a\vec{u}+a\vec{b}$,
\item $(a+b)\vec{u} = a\vec{u}+ v\vec{u}$,
\item $(ab)\vec{v} = a(b\vec{v})$.
\end{enumerate}
\end{setn}
\begin{proof}
Vi viser $iii)$, og resten overlades til læseren. Betragt
\begin{align*}
a(\vec{u}+\vec{v}) &= a\left(\begin{pmatrix}
u_1\\u_2
\end{pmatrix}+ \begin{pmatrix}v_1 + v_2
\end{pmatrix} \right) \\
 &= a\begin{pmatrix}
u_1+v_1\\ u_2+v_2
\end{pmatrix}\\
 &= \begin{pmatrix}
au_1+av_1 \\ au_2+av_2
\end{pmatrix}\\
&= \begin{pmatrix}
au_1\\ au_2
\end{pmatrix} + \begin{pmatrix}
av_1\\ av_2
\end{pmatrix} = a\vec{u} + a\vec{v}
\end{align*}
\end{proof}
Vi kan ikke gange vektorer sammen, og derfor kan vi heller ikke dividere to vektorer. 
\section*{Opgave 1}
Lad følgende vektorer være givet
\begin{align*}
&\vec{u} = \begin{pmatrix}
2\\3
\end{pmatrix}&&\vec{v} = \begin{pmatrix}
6 \\ -2
\end{pmatrix}\\
&\vec{w} = \begin{pmatrix}
-3\\-4 
\end{pmatrix}
&& \vec{a} = \begin{pmatrix}
\sqrt{2} \\ \sqrt{2}
\end{pmatrix}.
\end{align*}
\begin{enumerate}[label=\roman*)]
\item Bestem 
\begin{align*}
&1) \ |\vec{u}|  &&2) \  2|v|   \\
&3) \ |10\vec{w}|  &&4) \  |\vec{a}|    \\
\end{align*}
\item Bestem 
\begin{align*}
&1) \ 3\vec{u}+2\vec{v}   &&2) \ \vec{u}+ \vec{v} - \vec{w} +\sqrt{2}\vec{a}  \\
&3) \ |3\vec{w}+\vec{a}|  &&4) \ |\vec{a}| + |\vec{u}| + |\vec{a}+\vec{u}|  \\
\end{align*}
\end{enumerate}
\section*{Opgave 2}
Vis $i)$, $ii)$, $iv)$ og $v)$ i Sætning \ref{setn:regneregler}.
