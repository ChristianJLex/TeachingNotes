
\begin{center}
\Huge
Prøve 
\end{center}

\section*{Opgave 1}
\stepcounter{section}
Isolér $x$ i ligningen 
\begin{align*}
6x+16 = 12+6x.
\end{align*}
\section*{Opgave 2}
Udregn følgende:
\begin{enumerate}[label=\roman*)]
\item $\frac{3}{5} + \frac{9}{4} $
\item $(4^{5})^{\frac{1}{10}}$
\end{enumerate}

\section*{Opgave 3}
To punkter $P$ og $Q$ er givet ved $P = (2,4)$ og $Q = (3,8)$.
\begin{enumerate}[label=\roman*)]
\item Bestem en forskrift for den lineære funktion $f$, der går gennem $P$ og $Q$.
\item Bestem en forskrift for den eksponentialfunktion $g$, der går gennem $P$ og $Q$. 
\item Bestem de to skæringspunkter for funktionerne $f$ og $g$. (Hint: Tænk)
\end{enumerate} 

\section*{Opgave 4}
En eksponentialfunktion er givet ved 
\begin{align*}
f(x) = b\cdot 2^x.
\end{align*}
\begin{enumerate}[label=\roman*)]
\item Hvad er fordoblingskonstanten for $f$?
\item Hvis $f(3) = 640$, hvad er så $f(4)$?
\end{enumerate}

\section*{Opgave 5}
Vi indsætter 1000kr på en konto med en årlig rente på $10\%$.
\begin{enumerate}[label=roman*)]
\item Hvor meget står der på kontoen efter et år? 
\item Hvor meget står der på kontoen efter to år?
\item Opstil en ligning, der bestemmer, hvornår der står 10.000kr på kontoen. Du behøver ikke at løse ligningen
\end{enumerate} 