
\begin{center}
\Huge
Mere om Prikproduktet
\end{center}

\section*{Prikprodukt}
\stepcounter{section}

\begin{defn}
Lad $\vv{v}$ og $\vv{w}$ være defineret som
\begin{align*}
\vv{v} = \begin{pmatrix}
v_1\\ v_2
\end{pmatrix}, \textnormal{ og }\vv{w} = \begin{pmatrix}w_1 \\ w_2
\end{pmatrix}.
\end{align*}
Så defineres \textit{prikproduktet}, \textit{skalarproduktet}, eller \textit{det indre produkt}  mellem $\vv{v}$ og $\vv{w}$ som
\begin{align*}
v\cdot w = v_1w_1 + v_2w_2.
\end{align*}
Dette skrives også til tider $\langle \vv{v}, \vv{w} \rangle$. 
\end{defn}
\begin{exa}
Lad $\vv{v} = \begin{pmatrix}
2 \\ 2
\end{pmatrix}$ og lad $\vv{w} = \begin{pmatrix}
-2 \\ 5
\end{pmatrix}$. Så kan vi bestemme prikproduktet mellem $\vv{v}$ og $\vv{w}$ som
\begin{align*}
\vv{v} \cdot \vv{w} = 2\cdot(-2)+2\cdot 5 = 6.
\end{align*}
\end{exa}

\begin{setn}
To vektorer $\vv{v}$ og $\vv{w}$ er orthogonale hvis og kun hvis $\vv{v}\cdot \vv{w} = 0$.
\end{setn}
Vi vil senere se mere præcist hvordan sammenhængen mellem vinklen mellem vektorer og prikproduktet er. 
\begin{exa}
Vi vil afgøre, om $\vv{v} = \begin{pmatrix}
2\\2
\end{pmatrix}$ og $\vv{w} = \begin{pmatrix}
-1\\1
\end{pmatrix}$ er orthogonale. Vi bestemmer derfor prikproduktet
\begin{align*}
\vv{v}\cdot \vv{w} = 2\cdot(-1)+2\cdot 1 = 0. 
\end{align*}
Derfor ved vi, at vinklen mellem de to vektorer er $0^\circ$, og at de derfor er orthogonale eller vinkelrette.
\end{exa}

\begin{setn}[Regneregler for prikproduktet]
\label{setn:1}
For vektorer $\vv{u}$, $\vv{v}$ og $\vv{w}$ samt konstanter $k$ gælder der, at 
\begin{enumerate}[label=\roman*)]
\item $\vv{u}\cdot \vv{v} = \vv{v}\cdot \vv{u}$,
\item $\vv{u} \cdot (\vv{v}+\vv{w}) = \vv{u} \cdot \vv{v}+ \vv{u}\cdot \vv{w}$,
\item $(k\vv{u})\cdot \vv{v} = k(\vv{u}\cdot \vv{v}) = \vv{u}\cdot (k\vv{v})$,
\item $|\vv{u}|^2 = \vv{u}\cdot \vv{u}$,
\item $|\vv{u}\pm \vv{v}|^2 = |\vv{u}|^2 \pm 2\vv{u}\cdot\vv{v} + |\vv{v}|^2 $.
\end{enumerate}
\end{setn}
\section{Opgave 1}
Bestem følgende prikprodukter:
\begin{align*}
&1) \ \begin{pmatrix}1\\ 2\end{pmatrix} \cdot \begin{pmatrix}3\\4\end{pmatrix}  &&2) \ \begin{pmatrix}-7\\ 4\end{pmatrix} \cdot \begin{pmatrix}8\\14\end{pmatrix}   \\
&3) \ \begin{pmatrix}\frac{2}{7}\\ \frac{2}{7}\end{pmatrix} \cdot \begin{pmatrix}\frac{3}{4}\\\frac{4}{3}\end{pmatrix}  &&4) \ \begin{pmatrix}\sqrt{2}\\ \sqrt{5}\end{pmatrix} \cdot \begin{pmatrix}\sqrt{2}\\ \sqrt{5}\end{pmatrix}   \\
&5) \ \begin{pmatrix}-0.5\\ 0.7\end{pmatrix} \cdot \begin{pmatrix}14\\10\end{pmatrix}  &&6) \  \begin{pmatrix}1\\ 2\end{pmatrix} \cdot \begin{pmatrix}1\\2\end{pmatrix}   \\ 
\end{align*}

\section*{Opgave 2}
Løs følgende ligninger:
\begin{enumerate}[label=\roman*)]
\item $\begin{pmatrix} 2 \\ 4\end{pmatrix} = \begin{pmatrix} 1 \\ 2\end{pmatrix}x +\begin{pmatrix} 1 \\ 2\end{pmatrix}$.
\item $  \begin{pmatrix} -2 \\ 3\end{pmatrix} = \begin{pmatrix} 10\\ 7\end{pmatrix}x - \begin{pmatrix} 2 \\ -3\end{pmatrix}$.
\item $\begin{pmatrix} 1 \\ 2\end{pmatrix} \cdot \begin{pmatrix} 4 \\ x\end{pmatrix} = 0$.
\item $\begin{pmatrix} 5 \\ 6\end{pmatrix} \cdot \begin{pmatrix} x \\ 2\end{pmatrix} = 22$.
\end{enumerate}

\section*{Opgave 3}
Bevis Sætning \ref{setn:1}.