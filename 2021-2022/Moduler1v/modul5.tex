\begin{center}
\Huge
Andengradsligninger
\end{center}

\section*{Hvad er en andengradsligning?}
\stepcounter{section}

Vi har tidligere set, hvordan man kan udnytte nulreglen og kvadratsætninger til at gætte løsninger til andengradsligninger. Vi vil i dag se, hvordan man mere generelt kan udregne løsninger til andengradsligninger, men vi vil først definere helt præcist hvad en andengradsligning er.
\begin{defn}[andengradsligning]
Hvis en ligning med ubekendt variabel $x$ er på formen 
\begin{align*}
ax^2+bx+c=0,
\end{align*}
hvor $a,b,c\in \mathbb{R}, a\neq 0$, så kalder vi ligningen for en andengradsligning. 
\end{defn}

\begin{exa}
Vi har følgende eksempler på andengradsligninger
\begin{align}
2x^2+10x+7&=0\label{eq:lig1},\\
\sqrt{3}x^2+4=0\label{eq:lig2}.
\end{align}
I \eqref{eq:lig1} er $a=2,$ $b=10$ og $c=7$. I \eqref{eq:lig2} er $a=\sqrt{3},$ $b=0$ og $c=4$. 
\end{exa}
Hvis ligninger kan omskrives til 2.gradsligninger vil vi ofte også kalde dem for andengradsligninger.
\begin{exa}[Det gyldne snit]
Løsninger af andengradsligninger er blevet studeret siden før antikken, men det var først i 1600-tallet at algebraiske ligninger (brug af $a,b,c$ til at betegne kendte størrelser og $x,y,z$ til at betegne ubekendte størrelser) blev introduceret. (Formentlig første gang af René Descartes). Tidligere blev andengradsligninger beskrevet med ord som "retoriske ligninger". Et eksempel på dette er det gyldne snit:

Opdel et linjestykke i to stykker, så forholdet mellem det korte og det lange stykke er lig med forholdet mellem det lange stykke og hele linjestykket. Lad desuden det korte stykke have længde $1$. Lad os kalde længden på det lange stykke for $x$ og længden på det korte stykke for $y=1$. Så får vi algebraisk
\begin{align*}
 \frac{x}{y} = \frac{x+y}{x} &\Leftrightarrow\\
 x = \frac{x+1}{x}.
\end{align*}
Det er ikke umiddelbart klart, at dette er en andengradsligning, så vi ganger derfor igennem med $x$, og får
\begin{align*}
x^2 = x+1\  \Leftrightarrow\ x^2-x-1=0,
\end{align*}
som klart er en andengradsligning.
\end{exa}
\section*{Løsninger af andengradsligninger}
En andengradsligning har altid enten $0,$ $1$ eller $2$ løsninger. De kan findes ved brug af følgende sætning.
\begin{setn}
Løsningerne til en andengradsligning $ax^2+bx+c=0$ er givet ved formlen
\begin{align*}
x = \frac{-b\pm \sqrt{d}}{2a},
\end{align*}
hvor diskriminanten $d$ er givet ved
\begin{align*}
d=b^2-4ac.
\end{align*}
\end{setn}
\begin{proof}
Vi ser på ligningen
\begin{align*}
ax^2+bx+c = 0.
\end{align*}
Vi ganger igennem med $4a$.
\begin{align*}
ax^2+bx+c = 0\ &\Leftrightarrow \ 4a^2x^2+4abx+4ac=0.
\end{align*}
Vi lægger $d = b^2-4ac$ på begge sider af lighedstegnet:
\begin{align*}
4a^2x^2+4abx+4ac=0 \ &\Leftrightarrow \ 4a^2x^2+4abx+4ac + b^2-4ac=b^2-4ac\\
&\Leftrightarrow 4a^2x^2+4abx+b^2 = d\\
&\Leftrightarrow (b+2ax)^2 = d.
\end{align*}
Vi kan nu isolere $x$ i formlen.
\begin{align*}
(b+2ax)^2 = d \ &\Leftrightarrow b+2ax = \pm \sqrt{d}\\
&\Leftrightarrow 2ax = -b \pm \sqrt{d}\\
&\Leftrightarrow x = \frac{-b\pm \sqrt{d}}{2a}.
\end{align*}
\end{proof}
Det er klart, at hvis $d>0$, så må $ax^2+bx+c=0$ have to løsninger. Hvis $d=0$, så har ligningen netop én løsning. Og hvis $d<0$, så kan ligningen ingen reelle løsninger have. 
\begin{exa}
Vi kan nu løse ligningen $x^2-x-1=0$ fra tidligere eksempel. I denne ligning er $a=1$, $b=-1$, og $c = -1$. Dette indsætter vi i diskriminantformlen og får
\begin{align*}
x = \frac{-(-1)\pm \sqrt{(-1)^2-(4\cdot 1 \cdot (-1))}}{2\cdot 1} = \frac{1 \pm\sqrt{5}}{2}.
\end{align*}
$x$ skulle tilsvare længden på et linjestykke, og derfor må det være den positive løsning, vi har interesse i. Derfor skal $x$ have længden $$x = \frac{1 +\sqrt{5}}{2}\approx 1.6, $$som er det, der nogle gange kaldes det gyldne snits forhold. 
\end{exa}

\section*{Opgave 1}
Omskriv følgende ligninger til formen $ax^2+bx+c = 0$
\begin{align*}
&1) \ ax^2+bc = -c   &&2) \  x^2=7  \\
&3) \ 9-5x=3x^2   &&4) \ x(10+2x)=5   \\
&5) \ (x-3)^2 =0  &&6) \  \sqrt{2}x^2+7^2=9   \\
&7) \  (x-2)(x-10)=10  &&8) \  (x-\sqrt{2})(x-\sqrt{3})=0  \\
&9) \  3x^2=x^2-2  &&10) \ 5x = \frac{1}{2x}+7   \\
\end{align*}

\section*{Opgave 2}
Brug diskriminantformlen til at løse andengradsligningerne fra Opgave 1.
