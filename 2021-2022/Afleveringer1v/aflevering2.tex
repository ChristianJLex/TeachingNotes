
\begin{center}
\Huge
Aflevering 2
\end{center}
Til opgaver uden hjælpemidler må I bruge jeres bøger og undervisningsnoterne.
\section*{Opgave 1 \large (uden hjælpemidler)}
\stepcounter{section}
Forkort følgende brøker mest muligt:
\begin{align*}
&1) \ 7\frac{6}{2}     &&2) \  5\frac{3-10}{7}   \\
&3) \ \frac{\frac{1}{7}}{\frac{14}{7}}    &&4) \  \frac{3}{4}+\frac{7}{11}  \\
&5) \ \frac{10}{11}-\frac{11}{12}  &&6) \  \frac{2}{8}\cdot \frac{10}{16}     \\
\end{align*}

\section*{Opgave 2 \large (uden hjælpemidler)}
Udregn følgende potenser eller rødder:
\begin{align*}
&1) \  \left(\frac{2}{4}\right)^3   &&2) \ \sqrt[5]{2^{10}}    \\
&3) \  4^{\frac{1}{2}}   &&4) \  \sqrt{2}\sqrt{32}   \\
&5) \ 2^{\frac{1}{2}}2^{(-\frac{1}{2})} &&4) (3^6)^\frac{1}{3}
\end{align*}

\section*{Opgave 3 \large (uden hjælpemidler)}
Løs følgende ligningssystemer (bestem $x$ og $y$):
\begin{enumerate}[label=\roman*)]
\item 
\begin{align*}
2x+4y &= 8,\\
4x+9y &= 9.
\end{align*}
\item \begin{align*}
x+10y &= -5,\\
3x-1y &= 12.\\
\end{align*}
\item
\begin{align*}
y &= 2x+3,\\
y &= -5x+7.\\
\end{align*}
\end{enumerate}

\section*{Opgave 4 \large (uden hjælpemidler)}
\begin{enumerate}[label=\roman*)]
\item Løs andengradsligningen $x^2-x-2=0$.
\item Brug løsningen fra i) til at forkorte brøken
\begin{align*}
\frac{x^2-x-2}{(x-2)}.
\end{align*}
\end{enumerate}

\section*{Opgave 5 \large (uden hjælpemidler)}
Forkort brøken
\begin{align*}
\frac{(3x+3y)(3x-3y)(x-i)}{9x^2-9y^2}.
\end{align*}
Hint: Anvend kvadratsætning. 

\section*{Opgave 6 \large (uden hjælpemidler)}
Prisen på en bestemt jakke var sidste uge 1500kr. Prisen i går var 1800kr, og prisen i dag er 1200kr. 
\begin{enumerate}[label=\roman*)]
\item Hvor stor en procentdel udgør prisen i dag af prisen i går?
\item Hvor mange procent er jakken faldet i pris fra sidste uge til i dag?
\item Hvor mange procent er jakken faldet i pris fra i går til i dag?
\end{enumerate}

\section*{Opgave 7 \large (uden hjælpemidler)}
En virksomhed er interesseret i at øge sin omsætning. Virksomheden omsætter i år for 8mio. 
\begin{enumerate}[label=\roman*)]
\item Hvor meget skal virksomheden omsætte for næste år, hvis de vil øge deres årlige omsætning med 25 procent?
\item Hvis virksomheden næste år kun øger deres omsætning med 10 procent, hvor mange penge omsætter de så mindre end deres ønskede omsætning?
\item Hvad er den procentvise forskel på den ønskede omsætning og den reelle omsætning næste år? (Du behøver ikke at udregne det, bare opstil regnestykket.)
\end{enumerate}