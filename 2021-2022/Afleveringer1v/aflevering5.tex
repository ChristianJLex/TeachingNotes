\begin{center}
\Huge
Aflevering 5 - uden hjælpemidler
\end{center}

\section*{Opgave 1}
Skitsér følgende sum eller differens af vektorer:
\begin{align*}
&1) \ \begin{pmatrix}1\\ 2\end{pmatrix} + \begin{pmatrix}2 \\ 3\end{pmatrix} &&2) \ \begin{pmatrix}2 \\ 3\end{pmatrix} + \begin{pmatrix}-4 \\ 2\end{pmatrix}  \\
&3) \ \begin{pmatrix} 4 \\ 4\end{pmatrix} - \begin{pmatrix} 2 \\ 2\end{pmatrix}&&4) \  \begin{pmatrix} -1 \\ 1\end{pmatrix} + \begin{pmatrix} 3 \\ 1\end{pmatrix} - \begin{pmatrix} 4 \\ 3 \end{pmatrix}\\
\end{align*}

\section*{Opgave 2}
Skitsér følgende par af vektorer og mål vinklen mellem dem:
\begin{align*}
&1) \ \begin{pmatrix} 1\\1\end{pmatrix},\ \begin{pmatrix}-1\\1 \end{pmatrix}  &&2) \begin{pmatrix}0\\ 4\end{pmatrix},\ \begin{pmatrix}2 \\ -2\end{pmatrix}
\end{align*}

\section*{Opgave 3}
Lad $\vv{v} = \begin{pmatrix} -2 \\ 4\end{pmatrix}$ og lad $\vv{w} = \begin{pmatrix}\frac{1}{2} \\ -1\end{pmatrix}$
\begin{enumerate}[label=\roman*)]
\item Bestem et tal $x$, så $\vv{v}x = \vv{w}$.
\item Løs ligningen
\begin{align*}
\vv{v} + \vv{w} + \begin{pmatrix}u_1 \\ u_2\end{pmatrix} = \begin{pmatrix}0 \\ 0\end{pmatrix}.
\end{align*}
\end{enumerate}

\section*{Opgave 4}
Lad punkterne $A = (0,2)$, $B = (-3,5)$, $C = (1,2)$ og $D = (-5,-5)$ være givet.
\begin{enumerate}[label=\roman*)]
\item Udregn følgende udtryk
\begin{align*}
&1) \ \vv{AB} + \vv{CD}   &&2) \  2\vv{AC} - 3\vv{DC}   \\
&3) \ \vv{BD} - \vv{DB}  &&4) \ \vv{AB} + \vv{CB} + \vv{CB} + \vv{DB}   \\
\end{align*}
\item Bestem længden af følgende vektorer:
\begin{align*}
&1) \ 3\vv{BA}   &&2) \  \vv{AC} + 2\vv{CD} 
\end{align*}
\item Vektorerne $\vv{AC}$ og $\vv{OC}$ afgrænser sammen med y-aksen et trekantet område. Bestem arealet af dette område.
\end{enumerate}
\section*{Opgave 5}
Lad punkterne $A = (8,3)$ og $B = (0,3)$ være givet. 
\begin{enumerate}[label=\roman*)]
\item Bestem midtpunktet $M$ mellem $A$ og $B$. 
\item Bestem nu midtpunktet $N$ mellem $B$ og $M$.
\item Bestem til sidst midtpunktet mellem $M$ og $N$.
\end{enumerate}

\section*{Opgave 6}
Bestem følgende udtryk og afgør, hvis nogle af vektorerne er orthogonale.
\begin{align*}
&1) \ \begin{pmatrix} 3 \\ 0\end{pmatrix}\cdot \begin{pmatrix}0 \\ -5 \end{pmatrix}  &&2) \  \begin{pmatrix} -2 \\ 3\end{pmatrix}\cdot \begin{pmatrix}4 \\ 5 \end{pmatrix}    \\
&3) \  4 \begin{pmatrix} 9 \\ 2\end{pmatrix}\cdot \begin{pmatrix}-2 \\ 3 \end{pmatrix}  &&4) \   \left(\begin{pmatrix} 1 \\ 2\end{pmatrix} + \begin{pmatrix} -3 \\ 4\end{pmatrix} \right)\cdot  \begin{pmatrix} 5 \\ 6\end{pmatrix}  \\
\end{align*}

\section*{Opgave 7}
Skitsér en enhedscirkel og bestem $\cos(v)$ og $\sin(v)$ for følgende vinkler ved aflæsning:
\begin{align*}
&1) \ 60^\circ  &&2) \ 270^\circ   \\
&3) \ 720^\circ &&4) \ 15^\circ   \\
&5) \ 270^\circ &&6) \ 310^\circ   \\
\end{align*}

\section*{Opgave 8}
\begin{enumerate}[label=\roman*)]
\item Det gælder, at $\sin(120^\circ) = \frac{\sqrt{3}}{2}$. Brug idioutformlen og enhedscirklen til at bestemme $\cos(120^\circ)$.
\item Brug idiotformlen og enhedscirklen til at bestemme $\cos(45^\circ)$. Hint: $\cos(45^\circ) = \sin(45^\circ)$.
\end{enumerate}