
\begin{center}
\Huge
Stokastiske variable
\end{center}
\section*{Definitioner}
\stepcounter{section}

Vi har tidligere set på sandsynligheder i stil med $P(\{\textit{en familie på tre har netop en datter}\})$, men det ville være smart, hvis vi i udgangspunktet havde kvantificeret alle udfaldene, så vi i stedet for at opskrive hændelser, så blot kunne opskrive et tal. Til dette vil vi definere begrebet stokastisk variabel.
\begin{defn}
En stokastisk variabel er en funktion, der sender en hændelse af et stokastisk eksperiment over i et reelt tal. 
\end{defn}
En stokastisk variabel vil typisk betegnes med $X$.
\begin{exa}
Vi kaster en mønt to gange og lader $X$ betegne antallet af gange, vi slår plat. Billedmængden (de værdier $X$ kan tage) er $\{0,1,2\}$ og vi har følgende sandsynligheder for udfaldene af $X$:
\begin{align*}
P(X=0) = \frac{1}{4}, \ P(X=1) = \frac{1}{4}, \ P(X=2) = \frac{1}{2}. 
\end{align*}
Som vi tidligere har nævnt, så kaldes disse sandsynligheder for fordelingen af $X$.
\end{exa}
\begin{exa}
Vi lader $X_1$ være den totale levetid for en amerikansk mand og $X_2$ er levetiden for en amerikansk kvinde. Så gælder der, at billedmængderne for de to variable er $[0,\infty)$ og
\begin{align*}
P(X_1\leq 30) \approx 0.16
\end{align*} 
samt
\begin{align*}
P(X_2\leq 30) \approx 0.11.
\end{align*}
\end{exa}
\begin{exa}
Vi slår med en terning og lader $X$ være antallet af øjne på terningen. Fordelingen af $X$ vil så være givet ved
\[P(X=1) = P(X=2) = \cdots = P(X=6) = \frac{1}{6}.\]
\end{exa}
Hvis fordelingen af en stokastisk variabel opfylder, at alle udfald af den stokastiske variabel er lige sandsynlige, så siger vi, at fordelingen af $X$ er uniform eller symmetrisk, og vi skriver at $X \sim \textnormal{Unif}$.

\begin{exa}\label{exa:exa1}
Vi ønsker at bestemme sandsynligheden for at få netop to seksere blandt seks kast. For at bestemme det, bruger vi multiplikationsprincippet til at bestemme, hvad sandsynligheden for at få to seksere i de to første kast og nul i resten. Det må være givet ved
\begin{align*}
P(\{\textit{seksere i netop de to første kast}\}) &= \left( \frac{1}{6}\right)\left( \frac{1}{6}\right)\left( \frac{5}{6}\right)\left( \frac{5}{6}\right)\left( \frac{5}{6}\right)\left( \frac{5}{6}\right) \\
&= \left( \frac{1}{6}\right)^2\left( \frac{5}{6}\right)^4
\end{align*}

Vi skal nu tage højde for, at sekserne ikke nødvendigvis behøver at optræde i de to første kast. Vi skal derfor bestemme på hvor mange forskellige måder, vi kan flytte positionen af sekserne rundt. Dette er netop givet af binomialkoefficienten 
\begin{align*}
\binom{6}{2},
\end{align*}
derfor er sandsynligheden for at få netop to seksere i seks forsøg givet ved
\begin{align*}
P(\{\textit{netop to seksere}\}) = \binom{6}{2}\left(\frac{1}{6}\right)^2\left(\frac{1}{6}\right)^4 \approx 0.2
\end{align*}
\end{exa}
Vi kan generalisere dette eksempel. Først skal vi bruge følgende definition:
\begin{defn}[Bernoulli-stokastisk variabel]
En stokastisk variabel $X$ kaldes en Bernoulli-stokastisk variabel, hvis den har to udfald (ofte betegnet $0$ og 1). Vi betegner $X=1$ som "succes" og $X=0$ som "fiasko". Sandsynligheden $p= P(X=1)$ kaldes for sandsynlighedsparametren og vi skriver, at $X \sim \textnormal{Ber}(p)$. 
\end{defn}
\begin{defn}[Binomialfordelt stokastisk variabel]
En stokastisk variabel, der beskriver antallet af successer af $n$ på hinanden følgende Bernoulli eksperimenter med sandsynlighedsparameter $p$ kaldes for en binomialfordelt stokastisk variabel, og vi skriver, at $X \sim B(n,p)$. $X$ er defineret som den stokastiske variabel med fordelingsfunktion
\begin{align*}
P(X=k) = \binom{n}{k}p^k(1-p)^{n-k}.
\end{align*}
\end{defn}
Argumentet går som i Eksempel \ref{exa:exa1}. Sandsynligheden for $k$ successer i de $k$ første eksperimenter er
\begin{align*}
 \underbrace{p\cdot p \cdots p}_{k\textnormal{ gange}}\cdot \underbrace{(1-p)\cdot (1-p)\cdots (1-p)}_{n-k\textnormal{ gange}}=p^k(1-p)^{n-k}.
\end{align*}
Antallet af måder, vi kan flytte rundt på de $k$ successer er $\binom{n}{k}$. Derfor fås sandsynligheden for nøjagtigt $k$ successer i $n$ forsøg som 
\begin{align*}
P(X=k) = \binom{n}{k}p^k(1-p)^{n-k}.
\end{align*}

\section*{Opgave 1}
\begin{enumerate}[label=\roman*)]
\item Bestem en passende stokastisk variabel for udfaldet af to terningekast og bestem sandsynligheden for hvert udfald af variablen. 
\item Bestem en passende stokastisk variabel for udfaldet af tre kast med en mønt og bestem sandsynlighederne for hvert udfald af variablen. 
\item Slå op på en tilfældig side i en bog på 400 sider. Bestem en stokastisk variabel, der beskriver udfaldet af dette eksperiment og bestem sandsynligheden for hvert udfald. 
\item Bestem en stokastisk variabel, der beskriver udfaldene af antal af døtre i en søskendeflok på tre og bestem sandsynligheden for hvert udfald. 
\end{enumerate}
\section*{Opgave 2}
\begin{enumerate}[label=\roman*)]
\item Hvad er sandsynlighedsparametren for den Bernoulli-fordelte stokastiske variabel, der beskriver et møntkast
\item Hvad er sandsynlighedsparametren for den Bernoulli-fordelte stokastiske variabel, der beskriver et slag på mere end 2 med en terning.
\end{enumerate}
\section*{Opgave 3}
\begin{enumerate}[label=\roman*)]
\item Hvad er sandsynligheden for at slå nøjagtigt fem seksere på seks slag med en terning?
\item Hvad er sandsynligheden for at få mindst 3 seksere?
\item Hvad er sandsynligheden for at få mindre en 4 nøjagtigt 2 gange?
\end{enumerate}
\section*{Opgave 4}
Et præparat bruges til en bestemt behandling, og gives til 10 personer. Sandsynligheden for helbredelse er 20$\%$. 
\begin{enumerate}[label=\roman*)]
\item Hvad er sandsynligheden for, at to personer helbredes?
\item Hvad er sandsynligheden for, at alle personer helbredes?
\item Hvad er sandsynligheden for, at mindst 4 helbredes? 
\end{enumerate}

